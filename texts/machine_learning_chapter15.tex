\documentclass[10pt]{article}
\usepackage[utf8]{inputenc}
\usepackage[T1]{fontenc}
\usepackage{amsmath}
\usepackage{amsfonts}
\usepackage{amssymb}
\usepackage[version=4]{mhchem}
\usepackage{stmaryrd}
\usepackage{graphicx}
\usepackage[export]{adjustbox}
\graphicspath{ {./images/} }

\begin{document}
指数分布族具有性质: 对数规范化因子 $A(\eta)$ 对自然参数 $\eta$ 的导数等于充分统计量 $T(x)$的数学期望。事实上,


\begin{align*}
\frac{\mathrm{d}}{\mathrm{d} \eta} A(\eta) & =\frac{\mathrm{d}}{\mathrm{d} \eta} \log \int h(x) \exp \left(\eta^{\mathrm{T}} T(x)\right) \mathrm{d} x \\
& =\frac{\int T(x) \exp \left(\eta^{\mathrm{T}} T(x)\right) h(x) \mathrm{d} x}{\int h(x) \exp \left(\eta^{\mathrm{T}} T(x)\right) \mathrm{d} x} \\
& =\int T(x) \exp \left(\eta^{\mathrm{T}} T(x)-A(\eta)\right) h(x) \mathrm{d} x \\
& =\int T(x) p(x \mid \eta) \mathrm{d} x \\
& =E[T(X)] \tag{E.5}
\end{align*}


狄利克雷分布属于指数分布族, 因为其密度函数可以写成指数分布族的密度函数形式:


\begin{align*}
p(\theta \mid \alpha) & =\frac{\Gamma\left(\sum_{l=1}^{K} \alpha_{l}\right)}{\prod_{k=1}^{K} \Gamma\left(\alpha_{k}\right)} \prod_{k=1}^{K} \theta_{k}^{\alpha_{k}-1} \\
& =\exp \left[\left(\sum_{k=1}^{K}\left(\alpha_{k}-1\right) \log \theta_{k}\right)+\log \Gamma\left(\sum_{l=1}^{K} \alpha_{l}\right)-\sum_{k=1}^{K} \log \Gamma\left(\alpha_{k}\right)\right] \tag{E.6}
\end{align*}


自然参数是 $\eta_{k}=\alpha_{k}-1$, 充分统计量是 $T\left(\theta_{k}\right)=\log \theta_{k}$, 对数规范化因子是 $A(\alpha)=$ $\sum_{k=1}^{K} \log \Gamma\left(\alpha_{k}\right)-\log \Gamma\left(\sum_{l=1}^{K} \alpha_{l}\right)$.

利用性质 (E.5), 对数规范化因子对自然参数的导数等于充分统计量的数学期望, 得到狄利克雷分布的数学期望 $E_{p(\theta \mid \alpha)}[\log \theta]$ 的计算式:


\begin{align*}
E_{p(\theta \mid \alpha)}\left[\log \theta_{k}\right] & =\frac{\mathrm{d}}{\mathrm{d} \alpha_{k}} A(\alpha)=\frac{\mathrm{d}}{\mathrm{d} \alpha_{k}}\left[\sum_{k=1}^{K} \log \Gamma\left(\alpha_{k}\right)-\log \Gamma\left(\sum_{l=1}^{K} \alpha_{l}\right)\right] \\
& =\Psi\left(\alpha_{k}\right)-\Psi\left(\sum_{l=1}^{K} \alpha_{l}\right), \quad k=1,2, \cdots, K \tag{E.7}
\end{align*}


其中, $\Psi$ 是 digamma 函数, 即对数伽马函数的一阶导数。

\section*{附录 $\mathrm{F}$ 软最大化函数的偏导数和交叉熵损失函数的偏导数}
\section*{1. 软最大化函数的偏导数}
软最大化函数的定义是


\begin{equation*}
p_{k}=\frac{\mathrm{e}^{z_{k}}}{\sum_{i=1}^{l} \mathrm{e}^{z_{i}}} \quad k=1,2, \cdots, l \tag{F.1}
\end{equation*}


其中, $z_{k} \in(-\infty,+\infty), p_{k} \in(0,1)$, 满足 $\sum_{k=1}^{l} p_{k}=1$ 。

求其偏导数

$$
\frac{\partial p_{k}}{\partial z_{j}}=\frac{\partial}{\partial z_{j}}\left(\frac{\mathrm{e}^{z_{k}}}{\sum_{i=1}^{l} \mathrm{e}^{z_{i}}}\right), \quad k, j=1,2, \cdots, l
$$

当 $j=k$ 时,

$$
\begin{aligned}
\frac{\partial p_{k}}{\partial z_{j}} & =\frac{\partial}{\partial z_{j}}\left(\frac{\mathrm{e}^{z_{j}}}{\sum_{i=1}^{l} \mathrm{e}^{z_{i}}}\right)=\frac{\mathrm{e}^{z_{j}}\left(\sum_{i=1}^{l} \mathrm{e}^{z_{i}}\right)-\mathrm{e}^{z_{j}} \mathrm{e}^{z_{j}}}{\left(\sum_{i=1}^{l} \mathrm{e}^{z_{i}}\right)^{2}} \\
& =\frac{\mathrm{e}^{z_{j}}}{\sum_{i=1}^{l} \mathrm{e}^{z_{i}}}\left(1-\frac{\mathrm{e}^{z_{j}}}{\sum_{i=1}^{l} \mathrm{e}^{z_{i}}}\right)=p_{j}\left(1-p_{j}\right)
\end{aligned}
$$

当 $j \neq k$ 时,

$$
\frac{\partial p_{k}}{\partial z_{j}}=\frac{\partial}{\partial z_{j}}\left(\frac{\mathrm{e}^{z_{k}}}{\sum_{i=1}^{l} \mathrm{e}^{z_{i}}}\right)=\frac{0-\mathrm{e}^{z_{j}} \mathrm{e}^{z_{k}}}{\left(\sum_{i=1}^{l} \mathrm{e}^{z_{i}}\right)^{2}}
$$

$$
=-\frac{\mathrm{e}^{z_{j}}}{\sum_{i=1}^{l} \mathrm{e}^{z_{i}} \sum_{i=1}^{l} \mathrm{e}^{z_{i}}}=-p_{j} p_{k}
$$

可统一表示为

\[
\frac{\partial p_{k}}{\partial z_{j}}=\left\{\begin{array}{ll}
p_{j}\left(1-p_{j}\right), & j=k,  \tag{F.2}\\
-p_{j} p_{k}, & j \neq k,
\end{array} \quad j, k=1,2, \cdots, l\right.
\]

\section*{2. 交叉嫡损失函数的偏导数}
交叉嫡损失函数的定义是


\begin{equation*}
L=-\sum_{k=1}^{l} y_{k} \log p_{k} \tag{F.3}
\end{equation*}


其中, $y_{k} \in\{0,1\}, k=1,2, \cdots, l$, 满足 $\sum_{k=1}^{l} y_{k}=1 ; p_{k} \in(0,1), k=1,2, \cdots, l$, 满足 $\sum_{k=1}^{l} p_{k}=1$ 。这里 $y_{k}$ 是常量, $p_{k}$ 是变量, 由式 (F.1) 定义。

求其对变量 $z_{j}$ 的偏导数:

$$
\begin{aligned}
\frac{\partial L}{\partial z_{j}} & =-\sum_{k=1}^{l} y_{k} \frac{\partial \log p_{k}}{\partial z_{j}} \\
& =-\sum_{k=1}^{l} \frac{y_{k}}{p_{k}} \frac{\partial p_{k}}{\partial z_{j}}, \quad j=1,2, \cdots, l
\end{aligned}
$$

代入式 (F.2), 得到:


\begin{align*}
\frac{\partial L}{\partial z_{j}} & =-y_{j}\left(1-p_{j}\right)+\sum_{k=1, k \neq j}^{l} y_{k} p_{j} \\
& =p_{j} \sum_{k=1}^{l} y_{k}-y_{j} \\
& =p_{j}-y_{j}, \quad j=1,2, \cdots, l \tag{F.4}
\end{align*}


\section*{索引}
\section*{数字和字母}
0-1 损失函数 (0-1 loss function), 14

AdaBoost 算法 (AdaBoost algorithm), 131

Adam (adaptive moment estimation algorithm) , 521

Bagging (Bagging) , 131

Baum-Welch 算法 (Baum-Welch algorithm), 175

BERT (bidirectional encoder representations from Transformers ) , 488, 495

BFGS (Broyden-Fletcher-Goldfarb-

Shanno)算法 (BFGS algorithm), 529

Broyden 类算法 (Broyden's algorithm), 530

DFP (Davidon-Fletcher-Powell) 算法

(DFP algorithm), 528

GPT (generative pre-training) , 488, 490

Gram 矩阵 (Gram matrix), 38

Jensen 不等式 (Jensen inequality), 151

$k$ 近邻法 ( $k$-nearest neighbor, $k$-NN), 41

$k$ 均值聚类 ( $k$-means clustering), 216

$k d$ 树 ( $k d$ tree) , 44

KKT 条件 (Karush-Kuhn-Tucker conditions), 121

KL 散度 (KL divergence, Kullback-Leibler divergence ) , 537

$L_{p}$ 距离 ( $L_{p}$ distance), 42

Mercer 核 (Mercer kernel), 118

Minkowski 距离 (Minkowski distance), 42

$n$ 元语言模型 ( $n$-gram model), 463

$Q$ 函数 ( $Q$ function), 150

RMSProp (root mean square propagation algorithm ), 521

$S$ 折交叉验证 ( $S$-fold cross validation ), 21

$\mathrm{S}$ 形曲线 (sigmoid curve), 77

$\mathrm{S}$ 型函数 (sigmoid function), 377\\
Transformer (Transformer) , 469, 475

\section*{A}
奥卡姆剃刀 (Occam's razor),20

B

板块表示 (plate notation), 331

半监督学习 (semi-supervised learning), 9

饱和函数 (saturating function), 379

贝塔分布 (beta distribution), 325

贝叶斯推理 (Bayesian inference), 12

贝叶斯学习 (Bayesian learning), 12

边 (edge), 181

边缘化 (marginalization), 311

变分 EM 算法 (variational EM algorithm), 324

变分分布(variational distribution), 338

变分推理 (variational inference), 333,338

变分自动编码器 (variational autoencoder), 517

标注 (tagging), 26

表示学习 (representation learning), 517

病态问题 (ill-formed problem), 15

伯努利分布(Bernoulli distribution), 327

不可约的 (irreducible), 306

不完全数据 (incomplete-data), 150

步幅 (stride), 419

\section*{C}
参数服务器(parameter server), 393

参数化模型 (parametric model), 11

参数空间 (parameter space), 14

残差单元 (residual unit), 438

残差连接(residual connection), 438

残差网络 (residual network, ResNet), 437

测试集(test set), 21

测试数据(test data), 6\\
测试误差 (test error), 17

策略(strategy), 9

层次聚类 (hierarchical clustering), 216

层归一化 (layer normalization), 405

查询 (query), 474

长短期记忆 (long short term memory, LSTM ), 447, 455

成对马尔可夫性 (pairwise Markov property), 181

重置门 (reset gate), 457

词嵌入 (word embedding), 460

词嵌入或单词嵌入 (word embedding), 460

词向量或单词向量 (word vector), 460

次最优 (sub-optimal), 59

簇(cluster), 219

\section*{D}
代价函数 (cost function), 14, 67

代理损失函数 (surrogate loss function), 112, 202

带符号的距离 (signed distance), 97

单纯形 (simplex), 81,289

单词-话题矩阵 (word-topic matrix), 274

单词-文本矩阵(word-document matrix), 272

单词频率-逆文本频率 (term frequencyinverse document frequency, TFIDF ) , 272

单词向量空间模型 (word vector space model ), 271

单 分 量 Metropolis-Hastings (singlecomponent Metropolis-Hastings), 315

单连接 (single linkage), 220

单向语言模型 (unidirectional language model ), 489

单元 (cell) $, 42,58$

单元 (unit), 455

狄利克雷分布(Dirichlet distribution), 325

动量算法 (momentum algorithm), 520

动态贝叶斯网络 (dynamic Bayesian network), 179

动态规划 (dynamic programming), 176

动态系统 (dynamical system), 449

动作 (action), 8

动作价值函数 (action value function), 9

独热向量 (one-hot vector), 380端到端(end-to-end), 517

对偶算法 (dual algorithm), 101

对偶问题 (dual problem), 101

对数几率 ( $\log$ odds), 78

对数几率 (logit), 33

对数似然损失函数 ( log-likelihood loss function), 15

对数损失函数(logarithmic loss function), 15

对数线性模型(logarithmic linear model), 87

多标签分类(multi-label classification), 380

多层感知机 (multilayer perceptron, MLP), 371

多词一义性 (synonymy),273

多类分类(multi-class classification), 380

多数表决规则 (majority voting rule), 44

多头注意力 (multi-head attention), 469

多头自注意力(multi-head self-attention), 469

多项分布(multinomial distribution), 324

多项逻辑斯谛回归模型 (multi-nominal logistic regression model), 79

多项式核函数 (polynomial kernel function), 118

\section*{E}
二项分布 (binomial distribution), 324

二项逻辑斯谛回归模型 (binomial logistic regression model) , 78

\section*{$\mathrm{F}$}
罚项 (penalty term), 16, 20

反卷积 (deconvolution), 509

反向传播 (backward propagation), 394

反向传播算法 (back propagation through time, BPTT ), 452

泛化能力 (generalization ability), 17, 21

泛化误差 (generalization error), 21

泛化误差上界 (generalization error bound), 22

非参数化模型(non-parametric model), 11

非负矩阵分解 (non-negative matrix factorization, NMF) , 271

非概率模型 (non-probabilistic model), 10

非线性模型(non-linear model), 11\\
非线性支持向量机 (non-linear support vector machine) , 94

非周期的 (aperiodic), 307

非周期性图 (aperiodic graph),350

非自回归过程 (non-autoregressive process), 495

分布式表示(distributed representation), 460

分类 (classification), 25

分类器 (classifier), 25

分类与回归树 (classification and regression tree, CART) , 68

分离超平面 (separating hyperplane), 31

分裂 (divisive),220

风险函数 (risk function), 15

弗罗贝尼斯范数 (Frobenius norm), 241

\section*{G}
改进的迭代尺度法 (improved iterative scaling, IIS) , 87

概率近似正确 (probably approximately correct, PAC) , 131

概率模型 (probabilistic model), 10

概率模型估计 (probability model estimation), 209

概率潜在语义分析 (probabilistic latent semantic analysis, PLSA), 286

概率潜在语义索引 (probabilistic latent semantic indexing, PLSI), 286

概率上下文无关文法 (probabilistic contextfree grammar) , 179

概率图模型 (probabilistic graphical model), $10,181,331$

概率无向图模型 (probabilistic undirected graphical model ) , 181, 182

感受野 (receptive field), 416

感知机 (perceptron), 30

高斯核函数 (Gaussian kernel function), 119

高斯混合模型(Gaussian mixture model), 154

根结点 (root node), 65

更新门 (update gate) , 457

工作服务器 (worker), 393

共轭分布 (conjugate distributions), 327

共轭先验 (conjugate prior), 327

估计误差 (estimation error), 44观测变量 (observable variable), 148

观测序列 (observation sequence), 163

广义拉格朗日函数 (generalized Lagrange function) , 531

广义期望极大 (generalized expectation maximization, GEM ), 158

归范化 (normalization), 311

规范化因子 (normalization factor), 183

过度参数化 (over-parameterized), 406

过拟合 (over-fitting) , 16, 18, 329

\section*{$\mathrm{H}$}
函数间隔(functional margin), 96

汉明距离 (Hamming distance), 461

合页损失函数 (hinge loss function), 111

核方法 (kernel method), 13, 94

核函数 (kernel function), 94

核技巧(kernel trick), 112

黑塞矩阵 (Hessian matrix), 526

后向(backward), 166

互相关 (cross correlation), 416

互信息 (mutual information), 63, 462

划分 (partition), 45,58

话题 (topic), 273

话题-文本矩阵(topic-document matrix), 275

话题分析 (topic modeling), 271

话题向量空间 (topic vector space), 274

话题向量空间模型 (topic vector space model) , 274

回归 (regression), 27

汇聚 (pooling), 415

汇聚层 (pooling layer), 512

\begin{center}
\includegraphics[max width=\textwidth]{2024_03_13_4dc28dfd971a29643fd3g-06}
\end{center}

机器学习 (machine learning), 3

迹 (trace), 256

基尼指数(Gini index), 69

基于上下文的表示 (contextualized representation) , 482

激活函数 (activation function), 372

吉布斯抽样 (Gibbs sampling), 296, 316, 321, 324,333

极大 (maximization), 148

极大-极大算法 (maximization-maximization algorithm ) , 158\\
极大似然估计 (maximum likelihood estimation) , 16

几何间隔 (geometric margin), 96

几率 (odds), 78

挤压函数 (squashing function), 387

计算图 (computation graph), 397

记忆元 (memory cell) 455

加法模型 (additive model), 137

价值函数 (value function), 9

假设空间 (hypothesis space), $4,6,13$

间隔 (margin), 100

监督学习 (supervised learning), 4,5

剪枝 (pruning), 67

简单循环神经网络 (simple recurrent neural network, S-RNN), 448

建议分布 (proposal distribution), 297, 312

键-值数据库 (key-value store), 472

奖励(reward), 8

奖励函数(reward function), 9

降维 (dimensionality reduction), 209

交叉熵(cross entropy), 390

交叉验证(cross validation), 20

接受-拒绝抽样法 (accept-reject sampling method), 297

接受分布 (acceptance distribution), 312

结点 (node) , 57, 181

结构风险最小化 (structural risk minimization, SRM), 16

截断奇异值分解 (truncated singular value decomposition), 234

解码 (decoding), 166

紧奇异值分解 (compact singular value decomposition) , 234

近似误差 (approximation error), 43

经验风险 (empirical risk), 15

经验风险最小化 (empirical risk minimization, ERM ), 15

经验熵 (empirical entropy), 62

经验损失 (empirical loss), 15

经验条件熵 (empirical conditional entropy), 62

精确率 (precision), 25

净输入 (net input), 372

径向基函数 (radial basis function), 119

局部马尔可夫性(local Markov property),\\
181

局部式表示 (local representation), 460

矩形对角矩阵 (rectangular diagonal matrix) , 229

距离 (distance), 217

聚合 (agglomerative), 220

聚类 (clustering), 208

卷积 (convolution), 509

卷积神经网络 (convolutional neural network, CNN ), 516

决策函数 (decision function), 6

决策树 (decision tree), 57

决策树桩 (decision stump), 140

绝对损失函数 (absolute loss function), 15

\section*{K}
可交换的 (exchangeable), 332

可逆马尔可夫链 (reversible Markov chain), 309

可约的 (reducible), 306

困惑度 (perplexity), 463

\begin{center}
\includegraphics[max width=\textwidth]{2024_03_13_4dc28dfd971a29643fd3g-07}
\end{center}

拉格朗日乘子 (Lagrange multiplier), 101

拉格朗日对偶性 (Lagrange duality), 531

拉格朗日函数 (Lagrange function), 101

拉普拉斯平滑 (Laplace smoothing), 54

类标记 (class label) , 42, 50

类别 (class), 25

类别分布 (categorical distribution), 325, 391

连接 (linkage), 220

链接分析 (link analysis) , 213, 349

零空间 (null space) , 535

零填充 (zero padding) , 419

流形 (manifold) , 209

留一交叉验证 (leave-one-out cross validation) , 21

滤波器 (filter), 417

路径(path), 350

逻辑斯谛分布 (logistic distribution), 77

逻辑斯谛函数 (logistic function), 377

逻辑斯谛回归 (logistic regression), 77

\section*{M}
马尔可夫过程 (Markov process), 300

马尔可夫决策过程 (Markov decision process ), 8\\
马尔可夫链 (Markov chain), 296

马尔可夫链蒙特卡罗法 (Markov Chain

Monte Carlo, MCMC) , 296

马尔可夫随机场 (Markov random field), 181, 182

马哈拉诺比斯距离 (Mahalanobis distance), 217

满条件分布(full conditional distribution), 314

曼哈顿距离 (Manhattan distance) , 43, 217

门控 (gated control), 455

门控循环单元 (gated recurrent unit, GRU), 447

门控循环神经网络 (gate controlled RNN), 455

蒙特卡罗法 (Monte Carlo method), 296

蒙特卡罗积分(Monte Carlo integration), 298

幂法 (power method) , 349, 357

闵可夫斯基距离(Minkowski distance), 217

模型 (model) , 4

模型选择 (model selection), 18

$\mathrm{N}$

内部结点 (internal node), 57

内部协变量偏移(internal covariate shift), 402

纳什均衡 (Nash equilibrium), 504

拟牛顿法 (quasi-Newton method), 526

逆暂退法 (inverted dropout), 409

牛顿法 (Newton method), 526

\section*{$\mathrm{O}$}
欧氏距离 (Euclidean distance), 43, 217

欧氏距离平方 (squared Euclidean distance), 223

\section*{P}
判别方法 (discriminative approach), 24

判别模型 (discriminative model), 24

判别网络 (discriminator), 504

批量归一化 (batch normalization), 402

批量学习 (batch learning), 11

偏置(bias), 30, 372

平方损失 (square loss), 390

平方损失函数 (quadratic loss function), 14

平均场 (mean filed), 339平均汇聚 (mean pooling), 424

评价准则 (evaluation criterion), 4

朴素贝叶斯 (naïve Bayes), 50

朴素贝叶斯算法 (naïve Bayes algorithm), 53

期望 (expectation), 148

期望极大算法 (expectation maximization algorithm) , 148

期望损失(expected loss), 15

奇异值(singular value), 230

奇异值分解 (singular value decomposition, SVD ), 229, 230, 271

恰当性 (adequacy), 522

前馈神经网络 (feedforward neural network), 371

前向 (forward), 166

前向-后向算法 (forward-backward algorithm), 167

前向分步算法 (forward stagewise algorithm) , 137

潜在变量 (latent variable), 148

潜在狄利克雷分配 (latent Dirichlet allocation, LDA), 324

潜在语义分析 (latent semantic analysis, LSA), 271

潜在语义索引 (latent semantic indexing, LSI) , 271

强化学习 (reinforcement learning), 4,8

强可学习 (strongly learnable), 131

强连通图 (strongly connected graph), 350

强制教学 (teacher forcing), 465, 471

切比雪夫距离 (Chebyshev distance), 217

切分变量 (splitting variable), 69

切分点 (splitting point), 69

区域(region), 58

去噪自动编码器 (denoising autoencoder), 494

全局马尔可夫性(global Markov property), 181

全填充 (full padding), 420

权值(weight), 30

权值向量(weight vector), 30

权重(weight), 372

权重衰减 (weight decay), 406\\
确定性模型(deterministic model), 10

\section*{$\mathrm{R}$}
人工神经网络 (artificial neural network), 371

人工神经元 (artificial neuron), 372

软间隔最大化(soft margin maximization), 94

软聚类 (soft clustering), 208, 219

软最大化函数(softmax function), 381

弱可学习 (weakly learnable), 131

\section*{$\mathrm{S}$}
散布矩阵 (scatter matrix), 220

散度 (divergence) , 280

熵 (entropy), 61

上采样 (upsampling), 425

上下文向量 (context vector), 474

深度卷积生成对抗网络 (deep convolutional generative adversarial networks, DCGAN ), 511

深度神经网络 (deep neural network, DNN), 371

深度学习 (deep learning), 11

神经网络 (neural network), 371

神经元 (neuron), 372

渗漏整流线性函数(leaky ReLu function), 512

生成对抗网络 (generative adversarial networks, GAN) , 519

生成方法 (generative approach), 24

生成模型 (generative model), 24

生成式预训练 (generative pre-training, GPT) , 490

生成网络 (generator), 504

时间齐次的马尔可夫链 (time homogenous Markov chain) , 300

实例 (instance), 5

势函数 (potential function), 183

试错(trial and error), 9

收缩的吉布斯抽样 (collapsed Gibbs sampling) , 334

输出空间 (output space), 5

输出门 (output gate), 455

输入空间 (input space), 5

输入门 (input gate) , 455

束搜索(beam search), 464,471数据 (data), 3

数学期望 (expectation), 311

数学期望估计 (estimation of mathematical expectation), 297

衰减系数 (discount factor), 9

双曲正切函数 (hyperbolic tangent function), 378

双向 Transformer 编码器表示 (bidirectional encoder representations from Transforers, BERT ) , 495

双向语言模型 (bidirectional language model) , 489

算法 (algorithm), 4

随机抽样 (random sampling), 296

随机过程 (stochastic process), 300

随机矩阵 (stochastic matrix), 301, 351

随机梯度下降法 (stochastic gradient descent), 33,392

随机游走 (random walk), 351

随时间的反向传播算法 (back propagation through time, BPTT) , 452

损失函数 (loss function), 14,67

\section*{T}
贪心搜索 (greedy search), 464

特异点 (outlier), 106

特征尺度变换(feature scaling transform), 402

特征函数 (feature function), 82

特征空间 (feature space), 6

特征图 (feature map), 421

特征向量 (feature vector), 5

梯度爆炸(exploding gradient), 401

梯度提升 (gradient boosting), 144

梯度下降法 (gradient descent), 524

梯度下降法 (gradient descent), 392, 524

梯度消失 (vanishing gradient), 401

提升 (boost) , 131

提升树 (boosting tree), 131, 140

填充 (padding), 419

条件熵 (conditional entropy), 62

条件随机场 (conditional random field, CRF ) , 181, 184

条件语言模型 (conditional language model), 470

跳元模型(skip-gram model), 461\\
跳元模型加负采样 (skip-gram model with negative sampling) , 461

通用近似定理 (universal approximation theorem) , 386

统计机器学习 (statistical machine learning), 3

统计模拟方法 (statistical simulation method) , 296

凸二次规划 (convex quadratic programming ) , 94, 98, 106

图 (graph), 181, 349

图分析 (graph analytics), 213

图卷积神经网络 (graph convolutional neural network), 516

图神经网络 (graph neural network), 516

团 (clique), 183

\section*{W}
完全连接 (complete linkage), 220

完全奇异值分解 (full singular value decomposition) , 233

完全数据 (complete-data), 150

微步卷积 (fractionally strided convolution), 509

微调 (fine tuning), 489

维特比算法 (Viterbi algorithm), 175

伪计数 (prior pseudo-counts), 328

位置嵌入 (position embedding), 476

稳健性 (robustness), 461, 522

无监督学习 (unsupervised learning), 4, 7

无限可交换(infinitely exchangeable), 332

误差反向传播算法 (error backpropagation algorithm) , 393

误差率 (error rate), 17

\section*{$\mathrm{X}$}
希尔伯特空间 (Hilbert space), 115

细致平衡方程 (detailed balance equation), 309

下采样 (down sampling), 424

下句预测 (next sentence prediction), 497

下游任务 (downstream task), 489

线性分类模型 (linear classification model), 30

线性分类器 (linear classifier), 30

线性可分数据集 (linearly separable data set ), 31线性可分支持向量机 (linear support vector machine in linearly separable case), 94

线性链 (linear chain), 181

线性链条件随机场 (linear chain conditional random field ), 184

线性模型 (linear model), 11

线性扫描 (linear scan), 44

线性支持向量机 (linear support vector ma-

chine), 94

相关系数 (correlation coefficient), 218

相似度 (similarity), 217

向量空间模型 (vector space model, VSM), 271

协方差矩阵 (covariance matrix), 220

信息增益(information gain), 60, 62

信息增益比(information gain ratio), 64

序列到序列学习 (sequence to sequence learn-

ing, Seq2Seq) , 469

序列最小最优化 (sequential minimal optimization, SMO), 121

学习率 (learning rate), 33

循环神经网络 (recurrent neural network,

RNN), 469, 516

训练集(training set), 21

训练数据 (training data), 6

训练误差 (training error), 17

\section*{Y}
掩码 (masking), 477

掩码语言模型化 (mask language modeling), 490, 497

验证集(validation set), 21

样本 (sample), 6

样本复杂度 (sample complexity), 389

叶结点 (leaf node), 57

一般化 Bregman 散度 (generalized Bregman divergence) , 367

一词多义性 (polysemy), 273

遗忘门 (forget gate) , 455

因子分解 (factorization), 183

因子负荷量 (factor loading), 256

隐变量 (hidden variable), 148

隐马尔可夫模型 (hidden Markov model, HMM ), 163

硬间隔最大化 (hard margin maximization), 94\\
硬聚类 (hard clustering) , 208, 219

有向边 (directed edge), 57

有向图 (directed graph), 288, 350

右饱和函数 (right saturating function), 379

右奇异向量 (right singular vector), 230

余弦 (cosine), 218

余弦相似度 (cosine similarity), 120

语言模型 (language model) , 302, 463

语言模型化 (language modeling) , 489

预训练 (pre-training) , 488

预训练语言模型 (pretrained language model) , 488

原始问题(primal problem), 101

\section*{Z}
再生核希尔伯特空间 (reproducing kernel Hilbert space, RKHS), 117

在线学习 (online learning), 11

暂退法 (dropout), 371, 406, 408

早停法 (early stopping) , 202, 406

早停止 (early stopping), 202

张成 (span), 534

张量 (tensor) , 248, 422

召回率 (recall), 25

整流线性函数 (rectified linear unit, ReLU), 379

正常返的 (positive recurrent), 307

正定核函数 (positive definite kernel function) , 115

正交矩阵 (orthogonal matrix), 229

正向传播 (forward propagation), 393

正则化 (regularization) , 16, 20, 406

正则化项 (regularizer), 16, 20

证据 (evidence), 339

证据下界(evidence lower bound, ELBO), 339

支持向量 (support vector), 100

支持向量机 (support vector machines, SVM), 94

直径 (diameter), 219

指示函数 (indicator function), 17

指数加权移动平均 (exponentially weighted moving average), 520

指数损失函数 (exponential loss function), 138

指针网络 (pointer network), 518

中位数 (median), 45

周期的 (periodic) , 307

逐元素积, 阿达玛积 (element-wiseproduct), 396

主动学习 (active learning), 10

主特征向量 (dominant eigenvector), 358

主特征值 (dominant eigenvalue), 357

注意力 (attention), 472

转换器 (transformer), 469,476

转移概率 (transition probability), 9

转移核 (transition kernel), 305

转置卷积 (transposed convolution), 509

状态 (state), 8

状态价值函数 (state value function), 9

状态序列 (state sequence), 163

准确率 (accuracy), 17

字符串核函数(string kernel function), 119

自动编码器 (auto encoder), 493

自回归过程 (autoregressive process), 495

自回归模型 (autoregressive model), 449

自上而下(top-down), 220

自下而上 (bottom-up), 220

自注意力 (self-attention), 479

阻尼因子 (damping factor), 354

组合性 (compositionality), 479

最大后验概率估计 (maximum posterior probability estimation, MAP) , 16

最大汇聚 (max pooling), 424

最大间隔法 (maximum margin method), 98

最大熵模型 (maximum entropy model), 77 , 80

最大团 (maximal clique), 183

最速下降法(steepest descent), 524

最小二乘法 (least squares), 28

最小二乘回归树 (least squares regression tree), 69

左饱和函数 (left saturating function), 379

左奇异向量 (left singular vector), 230

\begin{center}
\includegraphics[max width=\textwidth]{2024_03_13_4dc28dfd971a29643fd3g-12}
\end{center}

彩图 15.1 奇异值分解的几何解释

\begin{center}
\begin{tabular}{|c|c|c|c|c|}
\hline
 & doc 1 & $\operatorname{doc} 2$ & $\operatorname{doc} 3$ & $\operatorname{doc} 4$ \\
\hline
word 1 & 2 & 2 & 4 & 3 \\
\hline
word 2 & 2 & 1 & 5 & 3 \\
\hline
word 3 & 1 & 1 & 2 & 0 \\
\hline
word 4 & 0 & 1 & 2 & 1 \\
\hline
\end{tabular}
\end{center}

彩图 18.1 概率潜在语义分析的直观解释

\begin{center}
\includegraphics[max width=\textwidth]{2024_03_13_4dc28dfd971a29643fd3g-12(1)}
\end{center}

彩图 19.1 接受-拒绝抽样法

\begin{center}
\includegraphics[max width=\textwidth]{2024_03_13_4dc28dfd971a29643fd3g-13}
\end{center}

彩图 20.1 狄利克雷分布例

\begin{center}
\includegraphics[max width=\textwidth]{2024_03_13_4dc28dfd971a29643fd3g-13(1)}
\end{center}

彩图 20.3 LDA 的文本生成过程\\
\includegraphics[max width=\textwidth, center]{2024_03_13_4dc28dfd971a29643fd3g-14(1)}

彩图 23.3 神经元的三维图形\\
\includegraphics[max width=\textwidth, center]{2024_03_13_4dc28dfd971a29643fd3g-14}

彩图 23.9 神经元的三维图形

\begin{center}
\includegraphics[max width=\textwidth]{2024_03_13_4dc28dfd971a29643fd3g-14(2)}
\end{center}

彩图 23.10 前馈神经网络例的三维图形

\begin{center}
\includegraphics[max width=\textwidth]{2024_03_13_4dc28dfd971a29643fd3g-15(1)}
\end{center}

彩图 23.12 XOR 神经网络例的三维图形

\begin{center}
\includegraphics[max width=\textwidth]{2024_03_13_4dc28dfd971a29643fd3g-15}
\end{center}

彩图 23.18 非凸优化问题\\
\includegraphics[max width=\textwidth, center]{2024_03_13_4dc28dfd971a29643fd3g-15(2)}

深度

彩图 24.6 用张量表示的三通道数据和特征图\\
\includegraphics[max width=\textwidth, center]{2024_03_13_4dc28dfd971a29643fd3g-16}

彩图 24.15 用卷积层代替全连接层

图中显示的是一维卷积

\begin{center}
\includegraphics[max width=\textwidth]{2024_03_13_4dc28dfd971a29643fd3g-16(1)}
\end{center}

输入层

彩图 24.17 卷积神经网络中的感受野\\
\includegraphics[max width=\textwidth, center]{2024_03_13_4dc28dfd971a29643fd3g-16(2)}

彩图 24.21 ResNet-18 的架构

每一模块表示一层, 有色模块是残差网络的连接层, 每一种颜色成一组, 每一组有两个残差单元。数字是核的大小和步幅

\begin{center}
\includegraphics[max width=\textwidth]{2024_03_13_4dc28dfd971a29643fd3g-17(2)}
\end{center}

彩图 25.4 简单循环神经网络上的反向传播

\begin{center}
\includegraphics[max width=\textwidth]{2024_03_13_4dc28dfd971a29643fd3g-17(5)}
\end{center}

彩图 25.14 束搜索

Head 8-10

\begin{itemize}
  \item Direct objects attend to their verbs
  \item $86.8 \%$ accuracy at the dobj relation
\end{itemize}

\begin{center}
\includegraphics[max width=\textwidth]{2024_03_13_4dc28dfd971a29643fd3g-17}
\end{center}

\section*{Head 8-11}
\begin{itemize}
  \item Noun modifiers (e.g., determiners) attend to their noun
  \item $94.3 \%$ accuracy at the det relation
\end{itemize}

\begin{center}
\includegraphics[max width=\textwidth]{2024_03_13_4dc28dfd971a29643fd3g-17(4)}
\end{center}

\section*{Head 9-6}
\begin{itemize}
  \item Prepositions attend to their objects
  \item $76.3 \%$ accuracy at the pobj relation\\
\includegraphics[max width=\textwidth, center]{2024_03_13_4dc28dfd971a29643fd3g-17(3)}
\end{itemize}

\section*{Head 5-4}
\begin{itemize}
  \item Coreferent mentions attend to their antecedents - $65.1 \%$ accuracy at linking the head of a coreferent mention to the head of an antecedent
\end{itemize}

\begin{center}
\includegraphics[max width=\textwidth]{2024_03_13_4dc28dfd971a29643fd3g-17(1)}
\end{center}

彩图 27.8 BERT 模型中的注意力权重分布的例子

可以表示词汇、语法、语义关系

\begin{center}
\includegraphics[max width=\textwidth]{2024_03_13_4dc28dfd971a29643fd3g-18(1)}
\end{center}

(a)

\begin{center}
\includegraphics[max width=\textwidth]{2024_03_13_4dc28dfd971a29643fd3g-18(2)}
\end{center}

(b)

\begin{center}
\includegraphics[max width=\textwidth]{2024_03_13_4dc28dfd971a29643fd3g-18(3)}
\end{center}

(c)

\begin{center}
\includegraphics[max width=\textwidth]{2024_03_13_4dc28dfd971a29643fd3g-18}
\end{center}

(d)

彩图 28.2 GAN 的学习过程\\
\includegraphics[max width=\textwidth, center]{2024_03_13_4dc28dfd971a29643fd3g-18(4)}

彩图 28.3 卷积例\\
\includegraphics[max width=\textwidth, center]{2024_03_13_4dc28dfd971a29643fd3g-18(6)}

彩图 28.4 转置卷积例\\
\includegraphics[max width=\textwidth, center]{2024_03_13_4dc28dfd971a29643fd3g-18(5)}

彩图 28.5 转置卷积例

\begin{center}
\includegraphics[max width=\textwidth]{2024_03_13_4dc28dfd971a29643fd3g-19}
\end{center}

\section*{李 航}
字节跳动公司人工智能实验室总监。ACL会士, IEEE 会士, ACM 杰出科学家。京都大学毕业, 东京大学博士。主要研究方向为自然语言处理、信息检索、机器学习、数据挖掘。


\end{document}