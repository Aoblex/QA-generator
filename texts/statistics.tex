\documentclass[10pt]{article}
\usepackage[utf8]{inputenc}
\usepackage[T1]{fontenc}
\usepackage{CJKutf8}
\usepackage{graphicx}
\usepackage[export]{adjustbox}
\graphicspath{ {./images/} }
\usepackage{amsmath}
\usepackage{amsfonts}
\usepackage{amssymb}
\usepackage[version=4]{mhchem}
\usepackage{stmaryrd}
\usepackage{hyperref}
\hypersetup{colorlinks=true, linkcolor=blue, filecolor=magenta, urlcolor=cyan,}
\urlstyle{same}
\usepackage{multirow}

\title{(第8版) }


\author{标准正态分布表}
\date{}


%New command to display footnote whose markers will always be hidden
\let\svthefootnote\thefootnote
\newcommand\blfootnotetext[1]{%
  \let\thefootnote\relax\footnote{#1}%
  \addtocounter{footnote}{-1}%
  \let\thefootnote\svthefootnote%
}

%Overriding the \footnotetext command to hide the marker if its value is `0`
\let\svfootnotetext\footnotetext
\renewcommand\footnotetext[2][?]{%
  \if\relax#1\relax%
    \ifnum\value{footnote}=0\blfootnotetext{#2}\else\svfootnotetext{#2}\fi%
  \else%
    \if?#1\ifnum\value{footnote}=0\blfootnotetext{#2}\else\svfootnotetext{#2}\fi%
    \else\svfootnotetext[#1]{#2}\fi%
  \fi
}

\begin{document}
\begin{CJK*}{UTF8}{bsmi}
\maketitle
\begin{center}
\end{center}

贾俊平 何晓群 金勇进 编著

\section*{(第8版) }


\section*{图书在版编目 (CIP) 数据}
统计学/贾俊平, 何晓群, 金勇进编著. --8版

. -- 北京: 中国人民大学出版社, 2021.10

21 世纪统计学系列教材

ISBN 978-7-300-29310-3

I . (1)统… II . (1) 贾… (2)何… (3)金… III. (1)统计学

一高等学校-教材 IV. (1)C8

中国版本图书馆 CIP 数据核字 (2021) 第 069516 号

“十二五”普通高等教育本科国家级规划教材

国家统计局优秀统计教材

21 世纪统计学系列教材

\section*{统计学 (第 8 版)}
贾俊平 何晓群 金勇进 编著

Tongjixue

出版发行 中国人民大学出版社

社 址 北京中关村大街 31 号

邮政编码 100080

电话 010-62511242 (总编室)

$010-82501766$ (由沟部)

010-62511770 (质管部)

010-62515195 (发行公司)

$010-62514148$ (门市部)

$010-62515275$ (盗版举报)

网 址 \href{http://www}{http://www}. crup. \href{http://com.cn}{com.cn}

经 销 新华书店

印 刷 北京密兴印刷有限公司

版 次 2000 年 9 月第 1 版

规 格 $185 \mathrm{~mm} \times 260 \mathrm{~mm} 16$ 开本

印 张 21.5 插页 1

字 数 475000 2021 年 10 月第 8 版

印次 2021 年 10 月第 1 次印刷

定 价 49.00 元

\section*{版权所有 侵权必究 E装差错负责调换}
\section*{总 序}
教育是国之大计、党之大计。习近平总书记指出: “两个一百年, 奋斗目标的实现、中华民族伟大复兴中国梦的实现, 归根到底靠人才、靠教育。”

改革开放以来, 高等统计教育有了很大的发展。作为培养我国统计专门人才的摇篮, 中国人民大学统计学院自 1952 年创建以来, 始终坚持理论与应用相结合的办学方向, 着力培养能够理论联系实际、解决实际问题的高层次人才。为了更好地服务教学, 中国人民大学统计学院组织并与统计学界同仁共同编写, 于 2000 年首次出版了国内较早、成体系的一套丛书-21 世纪统计学系列教材。本系列教材, 历经 20 余年的建设, 适应统计教育的发展变化, 不断修订完善, 得到了社会的广泛认可, 已成为全国统计学高等教育最有影响力的系列教材之一。系列教材中,既有 “十一五” 或 “十二五” 普通高等教育本科国家级规划教材, 又有教育部高等学校统计学类专业教学指导委员会推荐用书。在 2021 年首届全国教材建设奖评选中, 《统计学 (第 7 版》获评全国优秀教材 (高等教育类)。\\
随着时代的发展和高等教育的变化, 系列教材也要与时俱进, 及时反映新时代的要求, 为此, 我们对系列教材不断改版更新。一方面, 深入学习贯彻习近平新时代中国特色社会主义思想, 紧扣立德树人的根本任务, 密切结合中国实际, 大量融入中国案例和数据, 使学生进一步坚定中国特色社会主义道路自信、理论自信、制度自信、文化自信。另一方面, 深刻把握统计学科的专业特点, 面对大数据时代的新形势, 突出统计理论方法与计算机实现技术的结合, 强调方法与技术在实际领域中的应用, 同时, 精心打造配套的新形态和立体化教学资源, 助力现代化教学实践。

感谢所有关注我们的同仁, 他们本着对统计学科的热情和提高统计教育水平的愿望,帮助我们不断改进这套教材。感谢参与教材编写的同行专家、统计学院的教师。愿大家的辛勤劳动能够结出丰硕的果实。我们期待着与统计学界的同仁共同创造统计学科辉煌的明天。

王晓军

中国人民大学统计学院

\section*{前 言}
统计学是一门数据分析科学, 它提供一套通用于所有学科领域的数据分析方法。作为高等院校经济管理类专业及其他相关专业的一门基础必修课, 其内容主要包括描述统计、推断统计和其他一些常用的数据分析方法。

本书自 2000 年初版以来, 得到社会各界的厚爱, 被全国近千所院校选作教材, 受到教师、学生和社会各界的广泛好评。获评国家统计局优秀统计教材, 先后入选普通高等教育“十五” 国家级规划教材、普通高等教育“十一五” 国家级规划教材和 “十二五”普通高等教育本科国家级规划教材, 并荣获首届全国教材建设奖全国优秀教材 (高等教育类)。

第 8 版是在第 7 版的基础上修订而成的。本次修订在广泛吸取读者意见的基础上, 对第 7 版中的部分内容及不当之处进行了修改。

第 8 版的主要变化体现在以下几个方面:

第一, 强调课程内容与思政建设的结合。书中例题和习题数据均以国内的社会经济等领域为背景, 紧密联系实际问题讲述统\\
计方法的应用, 以便学生能更好地利用统计方法解决中国的实际问题。

第二, 强调内容体系上的完整。除了数据和例题的更新外, 部分章节在内容和结构上均有较大变动, 更强调统计方法的实际应用。

第三, 强调软件的与时俱进。第 8 版使用了 Excel 2019 和 SPSS 26 (中文版) 两款软件, 书中软件的操作步骤也随之作了更新。

第 8 版的各章执笔人如下: 贾俊平(第 1 章、第 3 章、第 4 章、第 7 章、第 10 章、第 11 章、第 12 章、第 13 章)、何晓群 (第 5 章、第 6 章)、金勇进 (第 2 章、第 8 章、第 9 章、第 14 章)。

感谢为本次修订提出宝贵意见的教师和读者。感谢中国人民大学出版社对本书出版的大力支持。

由于作者水平所限, 书中的错误和疏漏之处在所难免。敬请读者提出宝贵意见, 以便进一步修订和改进。

\section*{目录}
第 / 章导论 ..... 001\\
第 $f$ 章 数据的概括性度量 ..... 057\\
1.1 统计及其应用领域 ..... 002\\
1.2 统计数据的类型 ..... 004\\
1.3 统计中的几个基本概念 ..... 006\\
思考与练习 ..... 008\\
第 章 数据的搜集 ..... 010\\
2. 1 数据的来源 ..... 011\\
2.2 调查方法 ..... 013\\
2.3 实验方法 ..... 022\\
2. 4 数据的误差 ..... 027\\
思考与练习 ..... 032\\
第 3 章 数据的图表展示 ..... 033\\
3.1 数据的预处理 ..... 034\\
3.2 分类数据的整理与展示 ..... 037\\
3.3 数值数据的整理与展示 ..... 043\\
3.4 合理使用图表 ..... 054\\
思考与练习 ..... 054\\
4.1 集中趋势的度量 ..... 058\\
4.2 离散程度的度量 ..... 064\\
4.3 分布形状的度量 ..... 069\\
思考与练习 ..... 071\\
第 $j$ 章 概率与概率分布 ..... 074\\
5.1 随机事件及其概率 ..... 075\\
5.2 离散型随机变量及其分布 ..... 078\\
5.3 连续型随机变量的概率分布 ..... 089\\
思考与练习 ..... 095\\
第 () 章 统计量及其抽样分布 ..... 097\\
6.1 统计量 ..... 098\\
6.2 由正态分布导出的几个重要分布 ..... 099\\
6. 3 样本均值的分布与中心极限定理 ..... 103\\
思考与练习 ..... 106\\
第 7 章 参数估计 ..... 108\\
7. 1 参数估计的基本原理 ..... 109\\
7.2 一个总体参数的区间估计 ..... 114\\
7.3 两个总体参数的区间估计 ..... 120\\
7. 4 样本量的确定 ..... 127\\
思考与练习 ..... 129\\
第 $\delta$ 章 假设检验 ..... 134\\
8.1 假设检验的基本问题 ..... 135\\
8.2 一个总体参数的检验 ..... 142\\
8.3 两个总体参数的检验 ..... 149\\
8. 4 检验问题的进一步说明 ..... 160\\
思考与练习 ..... 163\\
第 章 分类数据分析 ..... 165\\
9.1 分类数据与 $\chi^{2}$ 统计量 ..... 166\\
9.2 拟合优度检验 ..... 167\\
9. 3 列联分析: 独立性检验 ..... 169\\
9. 4 列联表中的相关测量 ..... 173\\
9.5 列联分析中应注意的问题 ..... 177\\
思考与练习 ..... 180\\
第 10 章 方差分析 ..... 183\\
10.1 方差分析引论 ..... 184\\
10.2 单因素方差分析 ..... 189\\
10.3 双因素方差分析 ..... 199\\
思考与练习 ..... 208\\
第 $/$ 章 一元线性回归 ..... 212\\
11.1 变量间关系的度量 ..... 213\\
11.2 一元线性回归 ..... 221\\
11.3 利用回归方程进行预测 ..... 234\\
11.4 残差分析 ..... 238\\
思考与练习 ..... 241\\
第 /2 章 多元线性回归 ..... 244\\
12.1 多元线性回归模型 ..... 245\\
12.2 回归方程的拟合优度 ..... 249\\
12.3 显著性检验 ..... 250\\
12.4 多重共线性 ..... 253\\
12.5 利用回归方程进行预测 ..... 256\\
12.6 变量选择与逐步回归 ..... 258\\
思考与练习 ..... 261\\
第 13 章 时间序列分析和预测 ..... 265\\
13.1 时间序列及其分解 ..... 266\\
13.2 时间序列的描述性分析 ..... 268\\
13.3 预测方法的选择 ..... 272\\
13.4 平稳序列的预测 ..... 275\\
13.5 趋势型序列的预测 ..... 280\\
13.6 复合型序列的分解预测 ..... 287\\
思考与练习 ..... 292\\
第 $1+$ 章指 数 ..... 295\\
14.1 基本问题 ..... 296\\
14.2 总指数编制方法 ..... 298\\
14.3 指数体系 ..... 305\\
14.4 几种典型的指数 ..... 309\\
14.5 综合评价指数 ..... 315\\
思考与练习 ..... 317\\
附录一 术语表 ..... 320\\
附录二 用 Excel 生成概率分布表 ..... 327\\
参考文献 ..... 337\\

统计在许多领域都有应用。在日常生活中, 我们也会经常接触到各种统计数据, 比如, 媒体报道中使用的一些统计数据、图表等。下面就是统计研究得到的一些结论: 吸烟对健康是有害的; 不结婚的男性会早逝 10 年; 身材高的父亲, 其子女的身材也较高;第二个出生的子女没有第一个聪明, 第三个出生的子女没有第二个聪明, 依此类推; 两天服一片阿司匹林会减少心脏病再次发作的概率; 如果每天摄取 500 毫升维生素 C, 生命可延长 6 年; 怕老婆的丈夫得心脏病的概率较大; 学生在听了莫扎特钢琴曲 10 分钟后的推理测试会比他们听 10 分钟娱乐节目或其他曲目做得更好。这些结论是正确的吗?你相信这些结论吗? 要正确阅读并理解这些数据, 就需要具备一些统计学知识。

理解并掌握一些统计学知识对普通大众是有必要的。每天我们都会关心生活中的一些事情, 其中就包含统计知识。比如, 在外出旅游时, 需要关心一段时间内的天气预报; 在投资股票时, 需要了解股票市场价格的信息, 了解某只特定股票的有关财务信息; 在观看世界杯足球赛时, 需要了解各支球队的技术统计等。

理解和掌握一些统计知识, 对政治家或制定政策的人来说更为重要, 在他们作决策时, 如果不懂统计可能会闹出笑话来。比如, 一个统计办公室的主管是一位行政事务官, 一次与统计学者开会, 统计学者抱忽从其他部门得到的一些估计值没有给出标准误差 (估计时的误差大小, 表示估计的精度), 这个主管马上问道: “误差也有标准吗?”一个统计顾问提交给茶叶委员会的报告中, 含有标题为 “饮茶人数的估计值 (含标准误差)”的附表。不久, 一封信被送到这个统计顾问手中, 问什么是人们喝红茶时的 “标准误差”。健康部门的一位官员看到一个统计学者提供的报告, 报告中提到去年由于某种疾病, 平均 1000 人中死亡 3.2 人, 这位官员对这个数字产生了兴趣。他问他的私人秘书: “3.2 人是如何死法?”他的秘书说: “先生, 当一个统计学家说死了 3.2 人时, 意味着三个人已经死了, 两个人正要死。”\\
本章将介绍统计学的一些基本问题, 包括统计学的含义、统计数据及其分类、统计中常用的基本概念等。

\section*{1.1 统计及其应用领域}
\subsection*{1.1.1 什么是统计学}
统计是处理数据的一门科学。人们给统计学下的定义很多, 比如, “统计学是收集、分析、表述和解释数据的科学” “统计是一组方法, 用来设计实验、获得数据, 然后在这些数据的基础上组织、概括、演示、分析、解释和得出结论”。综合地说, 统计学 (statistics) 是收集、处理、分析、解释数据并从数据中得出结论的科学。

统计学是关于数据的科学, 它所提供的是一套有关数据收集、处理、分析、解释并从数据中得出结论的方法, 统计研究的是来自各领域的数据。数据收集就是取得统计数据;数据处理是将数据用图表等形式展示出来; 数据分析则是选择适当的统计方法研究数据,并从数据中提取有用信息进而得出结论。

数据分析所用的方法可分为描述统计方法和推断统计方法。描述统计 (descriptive statistics) 研究的是数据收集、处理、汇总、图表描述、概括与分析等统计方法。推断统计 (inferential statistics) 是研究如何利用样本数据来推断总体特征的统计方法。比如,要了解一个地区的人口特征, 不可能对每个人的特征一一进行测量; 对产品的质量进行检验往往是破坏性的, 也不可能对每个产品进行测量。这就需要抽取部分个体即样本进行测量, 然后根据获得的样本数据对所研究的总体特征进行推断, 这就是推断统计要解决的问题。

\subsection*{1.1.2 统计的应用领域}
统计方法是适用于所有学科领域的通用数据分析方法, 只要有数据的地方就会用到统计方法。随着人们对定量研究的日益重视, 统计方法应用到自然科学和社会科学的众多领域, 统计学已发展成为由若干分支学科组成的学科体系。可以说, 几乎所有的研究领域都要用到统计方法, 比如政府部门、学术研究领域、日常生活、公司或企业的生产经营管理都要用到统计。下面将给出统计在工商管理中的一些应用。

\section*{1. 企业发展战略}
发展战略是一个企业的长远发展方向。制定发展战略一方面需要及时了解和把握整个宏观经济的状况及发展变化趋势, 了解市场的变化, 另一方面还要对企业进行合理的市场定位, 把握企业自身的优势和劣势。所有这些都离不开统计, 需要统计提供可靠的数据,利用统计方法对数据进行科学的分析和预测, 等等。

\section*{2. 产品质量管理}
质量是企业的生命, 是企业持续发展的基础。质量管理中离不开统计的应用。在一些知名的跨国公司, 六西格玛准则成为一种重要的管理理念。质量控制成为统计学在生产领域的一项重要应用。各种统计质量控制图广泛用于监测生产过程。

\section*{3. 市场研究}
企业要在激烈的市场竞争中取得优势, 首先必须了解市场, 要了解市场, 则需要做广泛的市场调查, 取得所需的信息, 并对这些信息进行科学的分析, 以便作为生产和营销的依据, 这些都需要统计的支持。

\section*{4. 财务分析}
上市公司的财务数据是股民投资的重要参考依据。一些投资咨询公司主要是根据上市公司提供的财务和统计数据进行分析, 为股民提供投资参考。企业自身的投资也离不开对财务数据的分析, 其中要用到大量的统计方法。

\section*{5. 经济预测}
企业要对未来的市场状况进行预测, 经济学家也常常对宏观经济或某一方面进行预测。在进行预测时要使用各种统计信息和统计方法。比如, 企业要对产品的市场潜力作出预测, 以便及时调整生产计划, 这就需要利用市场调查取得数据, 并对数据进行统计分析。经济学家在预测通货膨胀时, 要利用有关生产价格指数、失业率、生产能力利用率等统计数据, 通过统计模型进行预测。

\section*{6. 人力资源管理}
企业利用统计方法对员工的年龄、性别、受教育程度、工资等进行分析, 作为企业制定工资计划和奖惩制度的依据。

当然, 统计并不是仅仅为了管理, 它是为自然科学、社会科学的多个领域发展起来的, 为多个学科提供了一种通用的数据分析方法。从某种意义上说, 统计仅仅是一种数据分析的方法。与数学一样, 统计是一种工具, 是一种数据分析的工具。目前, 统计已被应用到生产、生活和科学研究的所有领域。

利用统计方法可以简化繁杂的数据, 比如, 用图表展示数据, 建立数据模型。有人认为统计的全部目的就是让人看懂数据, 其实这仅仅是统计的一个方面, 统计更重要的功能是对数据进行分析, 它提供了一套分析数据的方法和工具。不同的人对数据分析的理解会大不一样, 曲解数据分析是常见的现象。在有些人的心目中, 数据分析就是寻找支持: 他们的心目中可能有了某种 “结论” 性的东西, 或者说他们希望看到一种符合他们需要的某种结论, 而后去找些统计数据来支持他们的结论。这恰恰歪曲了数据分析的本质, 数据分\\
析的真正目的是从数据中找出规律, 从数据中寻找启发, 而不是寻找支持。真正的数据分析事先是没有结论的, 通过对数据的分析才能得出结论。统计不是万能的, 它不能解决你所面临的所有问题。统计可以帮助你分析数据, 并从分析中得出某种结论, 但对统计结论的进一步解释, 则需要你的专业知识。比如, 吸烟会使患肺癌的概率增大, 这是一个统计结论, 但要解释吸烟为什么能引起肺癌, 这就不是统计学家所能做到的了, 需要有更多的医学知识才行。

\section*{1.2}
\section*{统计数据的类型}
统计数据是对现象进行测量的结果。比如, 对经济活动总量的测量可以得到国内生产总值 (GDP); 对股票价格变动水平的测量可以得到股票价格指数; 对人口性别的测量可以得到男或女这样的数据。下面从不同角度说明统计数据的分类。

\subsection*{1.2. 1 分类数据和数值数据}
按照所采用的计量尺度的不同 (1), 可以将统计数据分为分类数据和数值数据两大类。

分类数据 (categorical data) 是只能归于某一类别的非数字型数据, 它是对事物进行分类的结果, 数据表现为类别, 是用文字来表述的。分类数据根据取值是否有序可分为无序分类数据和有序分类数据。无序分类数据的各类别间是不可以排序的。比如, “上市公司所属的行业” 这一变量取值为 “制造业” “金融业” “旅游业” 等, 这些取值之间不存在顺序关系。再比如, “商品的产地” 这一变量的取值为 “甲” “乙” “丙” “丁”, 这些取值之间也不存在顺序关系。有序分类数据也称顺序数据, 其各类别间可以排序。比如, “对商品的评价” 这一变量的取值为 “很好” “好” “一般” “差” “很差”, 这 5 个值之间是有序的。

数值数据 (metric data) 是按数字尺度测量的观察值, 其结果表现为具体的数值。现实中所处理的大多数是数值数据。数值数据根据其取值的不同, 可以分为离散数据 (discrete data)和连续数据(continuous data)。离散数据的取值是有限的, 通常可以列举。连续数据是在一个或多个区间中取任何值的数据, 它的数值是连续不断的, 通常不能一一列举,如 “年龄” “温度” “零件尺寸的误差” 等都是连续数据。

无序分类数据和有序分类数据说明的是事物的品质特征,通常是用文字来表述,其结果均表现为类别, 因而也可统称为定性数据或品质数据(qualitative data); 数值数据说明的是现象的数值特征, 是用数字来表现的, 因此也可称为定量数据或数量数据(quantitative data)。
\footnotetext{(1) 数据的测量尺度有四种: 第一, 分类尺度 (nominal scale)。按照事物的某种属性对其进行的平行的分类, 数据表现为类别。第二, 顺序尺度 (ordinal scale)。对事物类别顺序的测度, 数据表现为有序的类别。第三, 间隔尺度 (interval scale)。对事物类别或次序之间间距的测度, 没有绝对零点, 数据表现为数字。第四, 比率尺度(ratio scale)。对事物类别或次序之间间距的测度,有绝对零点,数据表现为数字。
}

\subsection*{1.2.2 观测数据和实验数据}
按照统计数据的收集方法, 可以将其分为观测数据和实验数据。观测数据 (observational data) 是通过调查或观测收集到的数据, 这类数据是在没有对事物人为控制的条件下得到的, 有关社会经济现象的统计数据几乎都是观测数据。实验数据 (experimental data) 则是在实验中控制实验对象而收集到的数据。比如, 对一种新药疗效的实验数据, 对一种新的农作物品种的实验数据。自然科学领域的大多数数据为实验数据。

\subsection*{1.2.3 截面数据和时间序列数据}
按照被描述的现象与时间的关系, 可以将统计数据分为截面数据和时间序列数据。截面数据 (cross-sectional data) 是在相同或近似相同的时间点上收集的数据, 这类数据通常是在不同的空间获得的, 用于描述现象在某一时刻的变化情况。比如, 2021 年我国各地区的地区生产总值就是截面数据。时间序列数据(time series data)是在不同时间收集到的数据, 这类数据是按时间顺序收集到的, 用于描述现象随时间变化的情况。比如 20102020 年我国的国内生产总值就是时间序列数据。

图 1-1 列出了统计数据的大致分类。

\begin{center}
\end{center}

图 1-1 统计数据的分类

区分数据的类型是十分重要的, 因为对不同类型的数据需要采用不同的统计方法来处理和分析。比如, 对分类数据, 我们通常计算出各类别的频数或频率进行描述性分析, 对\\
列联表进行 $\chi^{2}$ 检验等; 对数值数据, 可以用更多的统计方法进行分析, 如计算各种统计量, 进行参数估计和检验等。

\section*{1.3 统计中的几个基本概念}
统计中的概念很多, 其中有几个概念是经常要用到的, 有必要单独介绍。这些概念包括总体和样本、参数和统计量以及变量等。

\subsection*{1.3. 1 总体和样本}
\section*{1. 总体}
总体 (population) 是包含所研究的全部个体(或数据)的集合, 它通常由所研究的一些个体组成, 如由多个企业构成的集合,多个居民户构成的集合,多个人构成的集合等。组成总体的每个元素称为个体, 在由多个企业构成的总体中, 每个企业就是一个个体; 由多个居民户构成的总体中, 每个居民户就是一个个体; 由多个人构成的总体中, 每个人就是一个个体。

总体范围的确定有时比较容易。比如, 要检验一批灯泡的使用寿命, 这批灯泡构成的集合就是总体, 每个灯泡就是一个个体, 总体的范围很清楚。但在有些场合总体范围的确定则比较困难, 比如, 对于新推出的一种饮料, 要想知道消费者是否喜欢, 首先必须弄清哪些人是消费的对象, 也就是要确定构成该饮料的消费者这一总体, 但事实上, 我们很难确定哪些消费者购买该饮料, 总体范围的确定十分复杂。当总体的范围难以确定时, 可根据研究的目的来定义总体。

总体根据其所包含的单位数目是否可数可以分为有限总体和无限总体。有限总体是指总体的范围能够明确确定, 而且元素是有限可数的。比如, 由若干个企业构成的总体就是有限总体, 一批待检验的灯泡也是有限总体。无限总体是指总体所包括的元素是无限的、不可数的。例如, 在科学实验中, 每个实验数据可以看作总体的一个元素, 而实验则可以无限地进行下去, 因此由实验数据构成的总体就是一个无限总体。

把总体分为有限总体和无限总体主要是为了判别在抽样中每次抽取是否独立。对于无限总体, 每次抽取一个单位, 并不影响下一次的抽样结果, 因此每次抽取可以看作是独立的。对于有限总体, 抽取一个单位后, 总体元素就会减少一个, 前一次的抽样结果往往会影响第二次的抽样结果, 因此每次抽取是不独立的。这些因素会影响到抽样推断的结果。

\section*{2. 样本}
样本 (sample) 是从总体中抽取的一部分元素的集合, 构成样本的元素的数目称为样本量 (sample size)。抽样的目的是根据样本提供的信息推断总体的特征。比如, 从一批灯泡中随机抽取 100 个, 这 100 个灯泡就构成了一个样本, 然后根据这 100 个灯泡的平均使\\
用寿命去推断这批灯泡的平均使用寿命。

\subsection*{1.3.2 参数和统计量}
\section*{1. 参数}
参数 (parameter) 是用来描述总体特征的概括性数字度量, 它是研究者想要了解的总体的某种特征值。对于一个总体, 研究者所关心的参数通常有总体平均数、总体标准差、总体比例等。在统计中, 总体参数通常用希腊字母表示。比如, 总体平均数用 $\mu(\mathrm{mu})$ 表示, 总体标准差用 $\sigma$ (sigma) 表示, 总体比例用 $\pi ( \mathrm{pi}$ ) 表示, 等等。

由于总体数据通常是不知道的, 所以参数通常也是一个未知的常数。比如, 我们不知道某一地区所有人口的平均年龄, 不知道一个城市所有家庭的收人的差异, 不知道一批产品的合格率, 等等。正因为如此, 才需要进行抽样, 根据样本计算出某种统计量, 然后估计总体参数。

\section*{2. 统计量}
统计量 (statistic) 是用来描述样本特征的概括性数字度量。它是根据样本数据计算出来的一个量, 由于抽样是随机的, 因此统计量是样本的函数。对于一个总体, 研究者所关心的统计量主要有样本平均数、样本标准差、样本比例等。样本统计量通常用英文字母来表示。比如, 样本平均数用 $\bar{x}$ (读作 $\mathrm{x}$-bar)表示, 样本标准差用 $s$ 表示, 样本比例用 $p$ 表示, 等等。

由于样本是已经抽出来的, 所以统计量总是知道的。抽样的目的就是要根据样本统计量去估计总体参数。比如, 用样本平均数 $(\bar{x})$ 去估计总体平均数 $(\mu)$, 用样本标准差 ( $s$ ) 去估计总体标准差 $(\sigma)$, 用样本比例 $(p)$ 去估计总体比例 $(\pi)$, 等等。

图 1-2 展示了总体和样本、参数和统计量的关系。

\begin{center}
\end{center}

图 1-2 总体和样本、参数和统计量\\
除了样本平均数、样本比例、样本标准差这类统计量外, 还有一些是为统计分析的需要而构造出来的统计量, 比如用于统计检验的 $z$ 统计量、 $t$ 统计量、 $\chi^{2}$ 统计量、 $F$ 统计量等, 它们的含义将在后面相关的章节中介绍。

\subsection*{1.3.3 变量}
变量 (variable) 是说明现象某种特征的概念, 其特点是从一次观察到下一次观察结果会呈现出差别或变化。如 “商品销售额” “受教育程度” “产品的质量等级” 等都是变量。变量的具体取值称为变量值, 也就是数据。比如商品销售额可以是 20 万元、 30 万元、 50 万元等, 这些数字就是变量值。

统计数据就是统计变量的取值。因此变量的分类与数据的分类一致, 基本的类型有分类变量 (categorical variable) 和数值变量 (metric variable)。

分类变量根据取值是否有序可分为无序分类变量和有序分类变量两种。无序分类变量取值的各类别间是不可以排序的。比如, “上市公司所属的行业” 这一变量取值为 “制造业” “金融业” “旅游业” 等, 这些取值之间不存在顺序关系。有序分类变量也称顺序变量, 其取值的各类别间可以排序。比如, “对商品的评价” 这一变量的取值为 “很好” “好” “一般” “差” “很差”,这 5 个值之间是有序的。

数值变量是取值为数字的变量, 也称为定量变量 (quantitative variable)。例如, “企业销售额” “某只股票的收盘价” “生活费支出” “投郑一枚色子出现的点数” 等变量的取值可以用数字来表示, 都属于数值变量。数值变量根据其取值的不同, 可以分为离散变量 (discrete variable) 和连续变量 (continuous variable)。离散变量是只能取有限个值的变量; 连续变量是可以在一个或多个区间中取任何值的变量, 它的取值是连续不断的。

\section*{思考与练习}
\section*{思考题}
1.1 什么是统计学? 解释描述统计和推断统计的含义。

1.2 统计数据可分为哪几种类型? 解释各类型数据的含义。

1.3 举例说明总体、样本、参数、统计量、变量这几个概念。

1.4 请举出应用统计的几个领域。

\section*{练习题}
1.1 指出下面变量的类型。

(1) 年龄。

(2) 性别。

(3) 汽车产量。

(4) 员工对企业某项改革措施的态度 (赞成、中立、反对)。

(5)购买商品时的支付方式(现金、信用卡、支票)。

1.2 某研究部门准备通过抽取 2000 个家庭来推断该城市所有职工家庭的年人均收入。请描述该项研究的总体和样本, 指出该项研究的参数和统计量。

1.3 一家研究机构从 IT 从业者中随机抽取 1000 人作为样本进行调查, 其中 $60 \%$ 的人回答他们的月收入在 15000 元以上, $50 \%$ 的人回答他们的消费支付方式是用信用卡。回答以下问题:

(1) 这一研究的总体是什么?

(2) 月收入是分类变量还是数值变量?

(3) 消费支付方式是分类变量还是数值变量?

(4) 这一研究涉及截面数据还是时间序列数据?

1.4 一项调查表明, 消费者每月在网上购物的平均花费是 2000 元, 他们选择在网上购物的主要原因是“价格便宜”。回答以下问题:

(1) 这一研究的总体是什么?

(2)“消费者在网上购物的原因”是分类变量还是数值变量?

(3) 研究者所关心的参数是什么?

(4)“消费者每月在网上购物的平均花费是 2000 元” 是参数还是统计量?

(5) 研究者使用的是描述统计方法还是推断统计方法?\\

在美国 1936 年的总统选举中, 两位竞争者分别为民主党的罗斯福和共和党的兰登。一般民意测验认为罗斯福将获胜, 例如盖洛普公司基于对 5 万选民的抽样调查, 预测罗斯福的得票率为 $56 \%$ 。但是美国著名杂志《文学文摘》 (Literary Digest) 宣布,根据其对 240 万人的调查,兰登将获得 $57 \%$ 的选票。最后的投票结果是,罗斯福赢得 2770 万张选票, 而兰登只得到 1600 万张选票, 罗斯福以绝对优势胜出。

值得思考的问题是, 为什么《文学文摘》调查的样本量如此之大, 结果却那样离谱? 细分析起来, 其预测失败的根本原因在于调查方案存在严重失误, 违背了统计学规律, 主要反映在以下两个方面:

\begin{enumerate}
  \item 样本抽选有偏。兰登的支持者主要是富裕阶层、大资产阶级, 而罗斯福的支持者主要是一般工薪阶层、中下层平民。《文学文摘》调查的对象集中在富人圈, 因为《文学文摘》是通过电话簿和俱乐部进行调查的, 而在 1936 年, 美国约有 1100 万户家!庭拥有电话, 大多是富裕家庭, 支持兰登。而俱乐部成员 (如高尔夫球俱乐部等) 则是更富裕的阶层, 他们也支持兰登。美国当时有 900 多万失业人口, 按《文学文摘》的调查方案, 这些失业人口难以被纳入样本中, 而这些人中的绝大多数是支持罗斯福的。

  \item 没有考虑缺失数据的影响。《文学文摘》在进行调查时发放了 1000 万份问卷,但只收回了近 240 万份。例如, 《文学文摘》当年对 $1 / 3$ 的芝加哥选民进行调查, 却只有 $20 \%$ 的比较富裕的阶层给予回答, 而那些忙于生计的一般家庭大多拒绝回应。实际投票中, 在芝加哥市罗斯福以压倒性多数票胜过兰登。这说明, 当回答者和无回答者有显著差异时, 忽略缺失数据进行推断一定会出错。

\end{enumerate}

《文学文摘》的这次调查被称为美国历史上最失败的一次调查, 作为数据收集失败的案例, 多次被写入各类调查图书。《文学文摘》最终也因此破产倒闭。

资料来源: 韦博成. 漫话信息时代的统计学. 北京: 中国统计出版社, 2011 .\\
人们购买住房是喜欢大户型还是喜欢小户型? 对父母的孝敬程度与子女的性别有关系吗? 人们在购买保险的时候, 是选择国内的保险公司, 还是选择国外的保险公司? 这些都是我们感兴趣却又不知道答案的问题。为了回答这些问题, 需要搜集相关的数据进行分析。这就是说, 当研究的问题确定之后, 我们就要考虑进行研究所需要的数据, 这里包括: 我们从哪里获得数据? 如果需要调查, 有那么多的潜在被调查者, 我们应当向谁进行调查? 选中被调查者以后, 我们怎样实施调查? 有些研究问题可能需要通过实验的方法获得数据, 那么怎样使用实验方法获得数据呢? 我们所得到的这些数据都很准确吗? 如果不准确, 误差是怎么产生的? 应当怎样控制误差以便获得较高质量的数据? 这些问题都是统计研究活动所必须考虑的。本章将对上述问题加以讨论。

\section*{2.1 数据的来源}
所有统计数据追踪其初始来源, 都是来自调查或实验。但是, 从使用者的角度看, 统计数据主要来自两个渠道: 一个是数据的间接来源, 即数据是由别人通过调查或实验的方式搜集的, 使用者只是找到它们并加以使用, 对此我们称为数据的间接来源。另一个是通过自己的调查或实验活动直接获得一手数据, 对此我们称为数据的直接来源。本节将对获取数据的这两个渠道分别加以介绍。

\subsection*{2.1.1 数据的间接来源}
如果与研究内容有关的原信息已经存在, 我们只是对这些原信息重新加工、整理, 使之成为我们进行统计分析可以使用的数据, 则把它们称为间接来源的数据。从搜集的范围看, 这些数据可以取自系统外部, 也可以取自系统内部。数据取自系统外部的主要渠道有: 统计部门和各级政府部门公布的有关资料, 如定期发布的统计公报, 定期出版的各类统计年鉴; 各类经济信息中心、信息咨询机构、专业调查机构、行业协会和联合会提供的市场信息与行业发展的数据情报; 各类专业期刊、报纸、图书所提供的文献资料; 各种会议, 如博览会、展销会、交易会及专业性、学术性研讨会上交流的有关资料; 从互联网或图书馆查阅到的相关资料等。取自系统内部的资料, 如果就经济活动而言, 则主要包括业务资料, 如与业务经营活动有关的各种单据、记录; 经营活动过程中的各种统计报表; 各种财务、会计核算和分析资料等。

相对而言, 这种二手资料的搜集比较容易, 采集数据的成本低, 并且能很快获得。二手资料的作用也非常广泛, 除了分析所要研究的问题, 这些资料还可以提供研究问题的背景, 帮助研究者更好地定义问题, 检验和回答某些假设和疑问, 寻找研究问题的思路和途径。因此, 搜集二手资料是研究者首先考虑并采用的。分析也应该从对二手资料的分析开始。

但是, 二手资料有很大的局限性, 研究者在使用二手资料时要保持谨慎的态度。因为二手资料并不是为特定的研究问题而产生的, 在回答所研究的问题方面可能是有欠缺的,\\
如资料的相关性不够, 口径可能不一致, 数据也许不准确, 也许过时了, 等等。因此, 在使用二手资料前, 对二手资料进行评估是必要的。

对二手资料进行评估可以考虑如下一些内容:

(1)资料是谁搜集的? 这主要是考察数据搜集者的实力和社会信誉度。例如, 对于全国性的宏观数据, 与某个专业性的调查机构相比, 政府有关部门公布的数据可信度更高。

(2) 为什么目的而搜集? 为了某个集团的利益而搜集的数据是值得怀疑的。

(3)数据是怎样搜集的? 搜集数据可以有多种方法, 采用不同方法所采集到的数据,其解释力和说服力都是不同的。如果不了解搜集数据所用的方法, 则很难对数据的质量作出客观的评价。数据的质量来源于数据的产生过程。

(4)什么时候搜集的?过时的数据,其说服力自然受到质疑。

使用二手数据, 要注意数据的定义、含义、计算口径和计算方法, 避免错用、误用、滥用。在引用二手数据时, 应注明数据的来源, 以尊重他人的劳动成果。

\section*{2. 1.2 数据的直接来源}
虽然二手数据具有搜集方便、采集快、采集成本低等优点, 但对一个特定的研究问题而言, 二手资料的主要缺陷是针对性不够, 所以仅仅靠二手资料还不能回答研究所提出的问题, 这时就要通过调查和实验的方法直接获得一手资料。我们把通过调查方法获得的数据称为调查数据, 把通过实验方法得到的数据称为实验数据。

调查通常是针对社会现象的。例如, 经济学家通过搜集经济现象的数据来分析经济形势、某种经济现象的发展趋势、经济现象之间的相互联系和影响。社会学家通过搜集有关人的数据来了解人类行为。管理学家通过搜集生产、经营活动的有关数据来分析生产过程的协调性和效率。调查数据通常取自有限总体, 即总体所包含的个体单位是有限的。如果调查针对总体中的所有个体单位进行, 就把这种调查称为普查。普查数据具有信息全面、完整的特点, 对普查数据的全面分析和深人挖掘是统计分析的重要内容。但是, 当总体较大时, 进行普查将是一项很大的工程。由于普查涉及的范围广, 接受调查的单位多, 所以耗时、费力, 调查的成本也非常高, 因此不可能经常进行。事实上, 统计学家所面临的经常是样本的数据, 如何从总体中抽取出一个有效的样本, 就成为统计学家需要考虑的一个问题。对于调查方法将在 2.2 节中专门讨论。

实验大多是针对自然现象的。例如, 化学家通过实验了解不同元素结合后产生的变化, 农学家通过实验了解水分、温度对农作物产量的影响, 医学家通过实验验证新药的疗效。实验作为搜集数据的一种科学的方法也被广泛运用到社会科学中。心理学、教育学的研究中大量使用实验方法获取所需要的数据, 社会学、经济学、管理学中也有许多使用实验方法获得研究数据的案例。关于实验方法, 我们将在 2.3 节中专门讨论。

\section*{2.2 调查方法}
\subsection*{2.2. 1 概率抽样和非概率抽样}
在数据采集阶段, 统计学家面临的一个关键问题是如何抽选出一个好的样本。好的样本都是相对而言的, “相对” 包括两方面的含义:一方面是针对研究的问题而言的。不同的研究问题对样本的要求会有所差别, 对某个研究问题, 这可能是一个不错的样本, 对另一个研究问题, 这个样本可能就是糟糕的。例如, 如果研究顾客的满意度, 样本就应当来自该产品的用户, 而如果要了解消费者对该产品的购买意愿, 样本就应当取自所有潜在的购买者。所以, 进行什么样的抽样设计首先取决于研究目的。另一方面是针对调查费用与估计精度的关系而言的。进行数据搜集总要投人一定的调查费用, 调查也希望获得更多高质量数据。但两者往往是有矛盾的, 一个好的样本应具有最好的性能价格比, 即在相同调查费用的条件下, 获得数据的估计精度最高, 或在相同估计精度的条件下, 调查成本最低。在研究中, 我们对估计结果的精度要求可以有差别, 有些问题很重要, 我们希望估计的精度高一些, 有些数据相对而言不太重要, 放松估计精度而节省大量调查费用也是一个不错的选择。正如对航天器中精密仪器主轴加工精度的要求和制作一根香肠时的精度要求不能相提并论一样, 对投资股票收益率的估计和对电视节目收视率的估计的精度要求也可以有所不同, 因为它们意味着不同的后果。

使用抽样采集数据的具体方式有许多种, 可以将这些不同的方式分为两类: 概率抽样和非概率抽样。

\section*{1. 概率抽样}
概率抽样 (probability sampling) 也称随机抽样, 是指遵循随机原则进行的抽样, 总体中每个单位都有一定的机会被选人样本。它具有下面几个特点:

首先, 抽样时按一定的概率以随机原则抽取样本。所谓随机原则就是在抽取样本时排除主观上有意识地抽取调查单位, 使每个单位都有一定的机会被抽中。需要注意的是, 随机不等于随便, 随机有严格的科学含义, 可以用概率来描述, 而随便则带有人为主观的因素。例如, 要在一栋楼内抽取 10 位居民作为样本, 若采用随机原则, 就需要事先将居住在该楼的居民按某种顺序编上号, 通过一定的随机化程序, 如使用随机数字表, 抽取出样本, 这样可以保证居住在该楼的每位居民都有一定的机会被选中。而如果调查人员站在楼前, 将最先走到楼外的 10 位居民选人样本, 这就是随便而不是随机, 这种方法不能使居住在该楼的所有居民都有一定的机会被选中, 已经在楼外的人不可能被选中, 在调查时段不外出的人也没有机会被选中。随机与随便的本质区别就在于, 是否按照给定的人样概率, 通过一定的随机化程序抽取样本单元。

其次, 每个单位被抽中的概率是已知的, 或是可以计算出来。\\
最后, 当用样本对总体目标量进行估计时, 要考虑到每个样本单位被抽中的概率。这就是说, 估计量不仅与样本单位的观测值 (也称为观察值) 有关, 也与其人样概率有关。

需要提及的是, 概率抽样与等概率抽样是两个不同的概念。当我们谈到概率抽样时,是指总体中的每个单位都有一定的非零概率被抽中, 单位之间被抽中的概率可以相等, 也可以不等。若是前者, 称为等概率抽样; 若是后者, 称为不等概率抽样。

调查实践中经常采用的概率抽样方式有以下几种。

(1) 简单随机抽样。

进行概率抽样需要抽样框, 抽样框 (sampling frame) 通常包括所有总体单位的信息,如企业名录(抽选企业)、学生名册(抽选学生)或住户门牌号码(抽选住户)等。抽样框的作用不仅在于提供备选单位的名单以供抽选, 它还是计算各个单位人样概率的依据。简单随机抽样 (simple random sampling) 就是从包括总体 $N$ 个单位的抽样框中随机地、一个个地抽取 $n$ 个单位作为样本, 每个单位的人样概率是相等的。抽样的随机性是通过抽样的随机化程序体现的, 实施随机化程序可以使用随机数字表, 也可以使用能产生符合要求的随机数序列的计算机程序。

方法一:根据总体单位个数 $N$ 的位数决定在随机数字表中随机抽取几列, 如 $N=$ 678 , 要抽取 $n=5$ 的样本, 这时 $N$ 为 3 位数, 则在随机数字表中随机抽取 3 列, 顺序往下, 选出头 5 个 $001 \sim 678$ 之间互不相同的数, 如果这 3 列随机数字不够, 可另选其他 3 列继续, 直到抽满 $n$ 个单位为止。

方法二: 有时方法一的执行效率可能不高, 通常是在首位数比较小的时候。假设 $N=$ 327 , 首位数是 3 , 比较小。如果按方法一, 在随机数字表中 $001 \sim 327$ 的范围内抽选, 有许多数就会大于 327 , 例如在随机数字表中抽到 486 , 在 $001 \sim 327$ 范围之外, 只好遗弃,比较可惜。这时可采用余数人样的方法, 即 $486 \div 327$, 商为 1 , 余数为 159 , 则第 159 个单位被抽中。如果在随机数字表中抽到 $999,999 \div 327$ 的商为 3 , 余数为 18 , 则第 18 个单位被抽中,依此类推。

在使用随机数字表时, 为克服可能的个人习惯, 增加随机性, 使用随机数字表的页号及起始点应该由随机数产生, 如随意翻开一页, 闭上眼睛, 将火柴随意扔到页面上, 将火柴头所指的数字作为页号, 同样的方法可以用于产生起始行号和起始列号。

简单随机抽样是一种最基本的抽样方法, 是其他抽样方法的基础。这种方法的突出特点是简单、直观, 在抽样框完整时, 可以直接从中抽取样本, 由于抽选的概率相同, 用样本统计量对目标量进行估计及计算估计量误差都比较方便。但简单随机抽样在实际应用中也有一些局限性: 首先, 它要求将包含所有总体单位的名单作为抽样框, 当 $N$ 很大时,构造这样的抽样框并不容易; 其次, 采用这种方法抽出的单位很分散, 给实施调查增加了困难; 最后, 这种方法没有利用其他辅助信息以提高估计的效率。所以, 在规模较大的调查中, 很少直接采用简单随机抽样, 一般是把这种方法和其他抽样方法结合起来使用。

(2) 分层抽样。

分层抽样(stratified sampling)是将抽样单位按某种特征或某种规则划分为不同的\\
层, 然后从不同的层中独立、随机地抽取样本。将各层的样本结合起来, 对总体的目标量进行估计。分层抽样有许多优点, 例如, 这种抽样方法保证了样本中包含有各种特征的抽样单位, 样本的结构与总体的结构比较相近, 可以提高估计的精度; 分层抽样在一定条件下为组织实施调查提供了方便 (当层是按行业或行政区划进行划分时); 分层抽样既可以对总体参数进行估计, 也可以对各层的目标量进行估计等。这些优点使分层抽样在实践中得到了广泛的应用。

(3) 整群抽样。

将总体中若干个单位合并为组, 这样的组称为群。抽样时直接抽取群, 然后对中选群中的所有单位全部实施调查, 这样的抽样方法称为整群抽样 (cluster sampling)。

与简单随机抽样相比, 整群抽样的特点在于: 首先, 抽取样本时只需要群的抽样框,而不必要求抽样框包括所有单位, 这就大大简化了编制抽样框的工作量。其次, 群通常由那些地理位置邻近或隶属于同一系统的单位构成, 调查的地点相对集中, 从而节省了调查费用, 方便了调查的实施。整群抽样的主要缺陷是估计的精度较差, 因为同一群内的单位或多或少有些相似, 在样本量相同的条件下, 整群抽样的抽样误差通常比较大。一般说来, 要得到与简单随机抽样相同的精度, 采用整群抽样需要增加基本调查单位。

(4) 系统抽样。

将总体中的所有单位 (抽样单位) 按一定顺序排列, 在规定的范围内随机抽取一个单位作为初始单位, 然后按事先制定好的规则确定其他样本单位, 这种抽样方法称为系统抽样 (systematic sampling)。典型的系统抽样是先从数字 $1 \sim k$ 中随机抽取一个数字 $r$ 作为初始单位, 以后依次取 $r+k, r+2 k, \cdots$ 。可以把系统抽样看成是将总体内的单位按顺序分成 $k$ 群, 用相同的概率抽取出一群的方法。

系统抽样的主要优点是操作简便, 如果有辅助信息, 对总体内的单位进行有组织的排列, 可以有效地提高估计的精度。系统抽样的缺点是对估计量方差的估计比较困难。系统抽样方法在调查实践中有广泛的应用。

(5) 多阶段抽样。

多阶段抽样采用类似整群抽样的方法, 首先抽取群, 但并不是调查群内的所有单位,而是再进一步抽样, 从选中的群中抽取出若干个单位进行调查。因为取得这些接受调查的单位需要两个步骤, 所以将这种抽样方式称为二阶段抽样。这里, 群是初级抽样单位, 第二阶段抽取的是最终抽样单位。将这种方法推广, 使抽样的阶段数增多, 就称为多阶段抽样 (multi-stage sampling)。例如, 第一阶段抽取初级单位, 第二阶段抽取二级单位, 第三阶段抽取接受调查的最终单位就是三阶段抽样, 同样的方法还可以定义四阶段抽样。不过, 即便是大规模的抽样调查, 抽取样本的阶段也应当尽可能少。因为每增加一个抽样阶段就会增添一份估计误差, 用样本对总体进行估计也就更加复杂。

多阶段抽样具有整群抽样的优点, 它保证了样本相对集中, 从而节约了调查费用; 不需要包含所有低阶段抽样单位的抽样框; 由于实行再抽样, 使调查单位在更广的范围内展开。在较大规模的抽样调查中, 多阶段抽样是经常采用的方法。\\
以上介绍了几种常见的概率抽样方式。概率抽样最主要的优点是, 可以依据调查结果计算估计量误差, 从而得到对总体目标量进行推断的可靠程度。而且, 也可以按照要求的精确度, 计算必要的样本单位数目。所有这些都为统计估计结果的评估提供了有力的依据, 所以, 统计分析的样本主要是概率样本, 即样本是采用概率抽样方式得到的。

\section*{2. 非概率抽样}
非概率抽样 (non-probability sampling) 是相对于概率抽样而言的, 指抽取样本时不是依据随机原则, 而是根据研究目的对数据的要求, 采用某种方式从总体中抽出部分单位对其实施调查。非概率抽样的方式有许多, 可以归为以下几种类型。

(1) 方便抽样。

调查过程中调查员依据方便的原则, 自行确定作为样本的单位。例如, 调查员在街头、公园、商店等公共场所进行拦截式的调查; 厂家在出售产品的柜台前对路过的顾客进行调查等。方便抽样的最大特点是容易实施, 调查成本低, 但这种抽样方式也有明显的缺点。例如, 样本单位的确定带有随意性, 因此, 方便样本无法代表有明确定义的总体, 将方便样本的调查结果推广到总体是没有任何意义的。因此, 如果研究的目的是对总体有关的参数进行推断, 使用方便样本是不合适的。但在科学研究中, 使用方便样本可以产生一些想法以及形成对研究内容的初步认识, 或建立假设。

(2) 判断抽样。

判断抽样是另一种比较方便的抽样方式, 是指研究人员根据经验、判断和对研究对象的了解, 有目的地选择一些单位作为样本, 实施时根据不同的目的有重点抽样、典型抽样、代表抽样等方式。重点抽样是从调查对象的全部单位中选择少数重点单位, 对其实施调查。这些重点单位的数量虽然不多, 但在总体中地位重要。例如, 要了解全国钢铁企业的生产状况, 可以选择产量较大的几个钢铁企业, 如宝钢、鞍钢、首钢等, 对这些重点单位进行调查, 就可以了解钢铁产量的大致情况及产量变化的基本走势。典型抽样是从总体中选择若干个典型的单位进行深人的调研, 目的是通过典型单位来描述或揭示所研究问题的本质和规律, 因此, 选择的典型单位应该具有研究问题的本质或特征。例如, 研究青少年犯罪问题, 可以选择一些典型的犯人, 对其做深人细致的调查, 掌握大量一手资料, 进而分析青少年犯罪的一般规律。代表抽样是通过分析选择具有代表性的单位作为样本, 在某种程度上, 也具有典型抽样的含义。例如, 某奶粉生产企业欲了解消费者对奶粉成分的需求,可以调查一些年轻的母亲, 因为她们购买奶粉的数量较大, 对奶粉的成分有更高的要求, 通过她们可以了解消费者购买奶粉时的意向。判断抽样是主观的, 样本选择的好坏取决于调研者的判断、经验、专业程度和创造性。这种方式的抽样成本比较低, 也容易操作, 但由于样本是人为确定的, 没有依据随机的原则, 调查结果不能用于对总体有关参数进行估计。

(3) 自愿样本。

自愿样本指被调查者自愿参加, 成为样本中的一分子, 向调查人员提供有关信息。例如, 参与报刊和互联网上刊登的调查问卷活动, 向某类节目拨打热线电话等, 都属于自愿\\
样本。自愿样本与抽样的随机性无关, 样本的组成往往集中于某类特定的人群, 尤其集中于对该调查活动感兴趣的人群, 因此, 这种样本是有偏的。我们不能依据样本的信息对总体的状况进行估计, 但自愿样本可以给研究人员提供许多有价值的信息, 可以反映某类群体的一般看法。

(4) 滚雪球抽样。

滚雪球抽样往往用于对稀少群体的调查。在滚雪球抽样中, 首先选择一组调查单位,对其实施调查之后, 再请他们提供另外一些属于研究总体的调查对象, 调查人员根据所提供的线索, 继续进行调查。这个过程持续下去, 就会形成滚雪球效应。例如, 欲对冬泳爱好者进行某项调查, 调查人员首先找到若干名冬泳爱好者, 然后通过他们找到更多的冬泳爱好者。滚雪球抽样也属于非概率抽样, 因为与随机抽取的被调查者相比, 被推荐的被调查者在许多方面与推荐他们的那些人更为相似。滚雪球抽样的主要优点是容易找到属于特定群体的被调查者, 调查的成本也比较低。它适合对特定群体进行资料的搜集和研究。

(5) 配额抽样。

配额抽样类似于概率抽样中的分层抽样, 在市场调查中有广泛的应用。它是首先将总体中的所有单位按一定的标志 (变量) 分为若干类, 然后在每类中采用方便抽样或判断抽样的方式选取样本单位。这种抽样方式操作比较简单, 而且可以保证总体中不同类别的单位都能包括在所抽的样本中, 使得样本的结构和总体的结构类似。但因为在抽取具体样本单位时并不是依据随机原则, 所以它属于非概率抽样。

在配额抽样中, 可以按单一变量控制, 也可以按交叉变量控制。表 2-1 是单一变量控制的例子。在一个城市中采用配额抽样抽出一个 $n=500$ 的样本。控制变量有年龄和性别, 配额是按单个变量分别分配的, 如各个年龄段的配额和性别的配额。这种配额抽样操作比较简便, 但有可能出现偏斜, 如年龄低的均为女性, 年龄高的均为男性。表 2-2 是交叉变量控制的例子。

表 2-1 单一变量控制配额分配表

\begin{center}
\begin{tabular}{cc}
\hline
年龄 & 人数 \\
\hline
$20 \sim 30$ 岁 & 150 \\
$30 \sim 40$ 岁 & 150 \\
$40 \sim 50$ 岁 & 100 \\
50 岁及以上 & 100 \\
\hline
合计 & 500 \\
\hline
性别 & 人数 \\
\hline
男 & 250 \\
\hline
女 & 250 \\
\hline
合计 & 500 \\
\hline
\end{tabular}
\end{center}

\begin{center}
\begin{tabular}{cccc}
\hline
年龄 & 男 & 女 & 合计 \\
\hline
$20 \sim 30$ 岁 & 70 & 80 & 150 \\
$30 \sim 40$ 岁 & 75 & 75 & 150 \\
$40 \sim 50$ 岁 & 55 & 45 & 100 \\
50 岁及以上 & 50 & 50 & 100 \\
\hline
合计 & 250 & 250 & 500 \\
\hline
\end{tabular}
\end{center}

交叉变量配额控制可以保证样本的分布更为均匀, 但现场调查中为了保证配额的实现, 尤其是在调查接近结束时所选的样本单位要同时满足特定的配额, 操作的难度可能要大一些。

\section*{3. 概率抽样与非概率抽样的比较}
概率抽样与非概率抽样是性质不同的两种抽样类型, 在调查中采用何种抽样类型取决于多种因素, 包括研究问题的性质、使用数据要说明的问题、调查对象的特征、调查费用、时间等。

由于非概率抽样不是依据随机原则抽选样本, 样本统计量的分布是不确切的, 因而无法使用样本的结果对总体相应的参数进行推断。如果调查的目的是用样本的调查结果对总体相应的参数进行估计, 并计算估计的误差, 得到总体参数的置信区间, 这时就不适合采用非概率抽样。非概率抽样的特点是操作简便、时效快、成本低, 而且对于抽样中的统计专业技术要求不是很高。非概率抽样适合探索性的研究, 调查的结果用于发现问题, 为更深人的数量分析做准备。非概率抽样也适合市场调查中的概念测试, 如产品包装测试、广告测试等。

概率抽样是依据随机原则抽选样本, 这时样本统计量的理论分布是存在的, 因此可以根据调查的结果对总体的有关参数进行估计, 计算估计误差, 得到总体参数的置信区间,并且可以在进行抽样设计时对估计的精度提出要求, 计算为满足特定精度要求所需要的样本量。所以, 如果调查的目的在于掌握研究对象总体的数量特征, 得到总体参数的置信区间, 就应当使用概率抽样的方法。当然, 概率抽样的技术含量更高, 无论是抽选样本还是对调查数据进行分析, 都要求有较高的统计学专业知识, 调查的成本也比非概率抽样高。

有时在一个研究项目中, 也可以把概率抽样和非概率抽样相结合, 发挥各自的特点,满足研究中的不同需求。

鉴于概率抽样对统计学专业知识的要求, 在本书后面对统计方法的讨论中, 若没有特殊说明, 均假定数据取自概率样本。

同样需要说明, 由于概率抽样有抽取样本的不同方式 (参见前面对概率抽样的讨论),而本书并不是论述抽样技术的专门图书, 所以在本书后面的讨论中均假定样本是采用简单随机抽样的方式抽选出的, 这有助于我们集中把握推断统计的基本原理。对其他抽样方式\\
感兴趣的读者, 请参阅抽样技术专门的图书。

\subsection*{2.2.2 搜集数据的基本方法}
样本单位确定之后, 对这些单位实施调查, 即从样本单位那里得到所需要的数据, 可以采用不同的方法。搜集数据的基本方法有以下几种。

\section*{1. 自填式}
自填式是指在没有调查员协助的情况下由被调查者自己填写, 完成调查问卷。把问卷递送给被调查者的方法有很多, 如调查员分发、邮寄或通过网络方式, 或把问卷刊登在报刊上, 等等。由于被调查者在填答问卷时调查员一般不在现场, 对于问卷的疑问无人解答, 所以这种方法要求问卷结构严谨, 有清楚的说明, 让被调查者一看就知道如何完成问卷。与其他调查方式相比, 自填式问卷应有制作详细、形象友好的说明, 必要时可在问卷上提供调查人员的联系电话, 以便被调查者遇到疑问时与调查人员联络。

自填式方法通常要求被调查者有一定的文化素养, 可以读懂问卷, 能正确理解调查问卷中的问题并进行回答。与其他搜集数据的方式相比, 调查组织者对自填式方法的管理相对容易, 只要把问卷准确地送到被调查者手中即可。自填式的调查成本也是最低的, 增大样本量对调查费用的影响很小, 所以可以进行大范围的调查。这种方式也有利于被调查者, 他们可以选择方便的时间填答问卷, 可以参考有关记录而不必依靠记忆回答。由于填写问卷时调查员不在场, 因而自填式方法可以在一定程度上减少被调查者回答敏感问题的压力。

自填式方法的缺点也是明显的。首先, 问卷的回收率比较低, 因为被调查者往往不够重视, 在完成问卷方面没有压力, 所以可能弃而不答。同时, 由于不重视, 被调查者也容易把问卷丢失和遗忘。所以采用自填式方法时, 通常需要做很多跟踪回访工作以取得较高的回收率。其次, 自填式方法不适合结构复杂的问卷, 因为许多被调查者不会去认真阅读填写问卷的指南, 如果问卷中出现跳答、转答这样的问题, 被调查者往往出现回答错误,而如果问卷中不使用跳答、转答这样的技术手段, 研究人员可能就无法搜集到最合适的所需信息。因此, 自填式方法对调查的内容会有所限制。此外, 自填式方法的调查周期通常都比较长, 调查人员需要对问卷的递送和回收方法进行仔细的研究和选择。最后, 对于在数据搜集过程中出现的问题, 一般难以及时采取调改措施。

\begin{enumerate}
  \setcounter{enumi}{1}
  \item 面访式
\end{enumerate}

面访式是指现场调查中调查员与被调查者面对面, 调查员提问、被调查者回答这种调查方式。面访式的主要优点是, 由于是面对面的交流, 调查人员可以激发被调查者的参与意识, 对不愿意参与的被访者进行说服, 由此提高调查的回答率。调查员可以在现场解释问卷, 回答被调查者的问题, 同时, 可以对被调查者的回答进行鉴别和澄清, 提高调查数据的质量, 并且可以对识字率低的群体实施调查。由于调查问卷是由经过培训的调查员所\\
控制的, 所以在问卷设计中可以采用更多的技术手段, 使得调查问题的组合更为科学、合理。在面访调查中, 还可以借助其他调查工具(图片、照片、卡片、实物等)以丰富调查内容。面访式的数据搜集方法还有一个优点, 即它能对数据搜集所花费的时间进行调节,如果数据搜集进展太慢, 需要加快速度, 就可以雇用更多的调查员, 而这在使用自填式方法时是不可能的。

面访式方法的缺点主要有: 首先, 调查的成本比较高, 因为要有调查员的培训费用、调查员的工资、送给被调查者的小礼品和调查员的交通费用等, 而且调查费用与样本量关系十分密切, 所以, 在大样本调查中, 研究人员面临着巨大的成本压力。其次, 面访这种搜集数据的方式在调查过程的质量控制方面有一定难度, 调查的数据质量与调查员的工作态度、责任心有直接关系, 当大量调查员参与调查时, 如何保证高质量的现场操作是一个很重要的问题。最后, 对于敏感问题, 除非对调查员进行角色篮选, 对调查员的访谈技巧进行专门的技术培训, 否则, 在面对面的条件下, 被调查者通常不会像在自填式方法下那样放松。

\section*{3. 电话式}
电话式是指调查人员通过打电话的方式向被调查者实施调查。电话调查的最大特点是速度快, 能够在很短的时间内完成调查。电话调查特别适合样本单位十分分散的情况, 由于不需要支付调查员的交通费, 数据搜集的成本大大下降。电话调查对调查员也是安全的, 他们不必在晚上走访偏僻的居民区, 而在面访调查中, 这些都是不可避免的。在电话调查中, 对访问过程的控制也比较容易, 因为调查员的工作地点都在一起, 调查中遇到的问题可以得到及时处理和解决, 调查督导对访问实施监听也很容易。目前, 这方面的技术正在朝计算机辅助电话调查 (computer assisted telephone interview, CATI) 方向发展。 CATI 系统把计算机与电话访问连接起来, 调查的问卷被输人计算机, 调查员在计算机屏幕前操作, 随机样本的抽选由计算机完成, 由计算机进行自动拨号, 调查员将调查结果 (用鼠标点击选项) 输人计算机, 设计的程序可以对录人的结果进行逻辑审核, 从而保证数据的合理性。可以在调查过程中得到即时的调查结果统计, 从而发现样本结构、样本分布等有关问题, 并及时采取相应措施, 使得样本的组成更为合理。对于无人接听, 或对方因为忙而无法接受调查等特殊情况, CATI 系统可以自动记录下来, 并在适当的时候向调查人员做出提示, 对这些样本单位进行重新调查。目前在发达国家, 使用 CATI 系统已经成为数据搜集的最主要方法。我国电话拥有率增长很快, 使用电话调查的方式搜集数据有广阔的发展空间。

电话调查也有一定的局限性。首先, 因为电话调查的工具是电话, 如果被调查者没有电话, 调查将无法实施, 所以在电话使用率不高的地区, 电话调查这种方式就会受到限制。其次, 使用电话进行访问的时间不能太长, 人们不愿意通过电话进行冗长的交谈, 被访者对调查的内容不感兴趣时更是如此。再次, 电话调查所使用的问卷要简单, 如果问卷答案的选项过长、过多, 被调查者听了后面忘了前面, 不仅延缓调查进度, 被调查者还很\\
容易挂断电话。最后, 与面访式相比, 电话调查由于不是面对面的交流, 在被访者不愿意接受调查时要说服他们就更为困难。

此外, 搜集数据的方法还有观察式, 即调查人员通过直接观测的方法获取信息, 如利用安置在超市中的录像设备观察顾客挑选商品时的表情, 在十字路口通过计数的方法估算车流量等。

\section*{4. 数据搜集方法的选择}
搜集数据的不同方法各有特点, 在选择数据搜集方法时, 需要考虑以下几个问题。

(1) 抽样框中的有关信息。

抽样框中的有关信息是影响方法选择的一个因素。如果抽样框中没有通信地址, 就不能将自填式问卷寄给被调查者; 如果没有计算机随机数字拨号系统, 又没有电话号码的抽样框, 电话调查的样本就难以产生, 电话访问就无法使用。

(2) 目标总体的特征。

目标总体的特征也会影响数据搜集方法。目标总体的特征表现在多个方面。例如, 如果总体的识字率很低, 对问卷的理解有困难, 就不宜使用自填式方法。样本的地理分布也很重要, 如果样本单位分布很广, 地域跨度大, 进行面访调查的交通费用就会很高, 调查过程的管理和质量监控实施起来也不容易。

(3) 调查问题的内容。

调查问题的内容也会影响数据搜集方法。面访调查比较适合复杂的问题, 因为调查员可以在现场对模糊的问题进行解释和澄清, 并判断被访者对问题是否真正理解, 调查问卷的设计也可以采用更多技术, 如跳答、转答等, 使搜集的数据满足研究的要求。如果调查涉及敏感问题, 那么使用匿名的数据搜集方法如自填式或电话式可能更合适。

(4) 有形辅助物的使用。

有形辅助物的使用对调查常常是有帮助或是必要的, 例如在调查期间展示产品、产品的广告等, 在一些市场调查中, 有时还需要被调查者试用产品, 然后接受调查。在这些情况下, 面访是最合适的方法。采用邮寄问卷的自填式调查方法也有一些效果, 因为可以随问卷同时邮寄有关调查内容的图片。但电话调查中对有形辅助物的使用就受到限制。

(5) 实施调查的资源。

实施调查的资源会对搜集数据的方法产生重大影响。这些资源包括经费预算、人员、调查设备和调查所需时间。面访调查的费用是最高的, 需要支付调查员的劳务费、调查交通费、被访者的礼品费等, 还要找到能够满足调查需要的一定数量的调查员。如果使用计算机辅助电话调查, 就需要有计算机设备和 CATI 操作系统。

(6)管理与控制。

有些数据搜集方法比另一些方法更容易管理。例如, 在电话调查中, 调查员通常集中在调查中心一起工作, 因此, 管理和控制相对简单。而面访调查中调查员分散、独立地工\\
作, 对他们的管理与控制有一定难度。

(7) 质量要求。

质量要求也是确定数据搜集方法的一个重要因素。如果调查员是经过考核选拔出来的, 有较好的素质和责任心, 并经过专门的培训, 这时面访调查就能够有效地减少被访者的回答误差。例如, 对于调查中所使用的概念, 调查员能够给出清晰无误的解释; 有经验的调查员还可以对被访者回答的真实性作出判断, 并使用调查询问的相关技术进行澄清,以保证高质量的数据。回答率也是影响数据质量的一个重要方面。由于面访具有面对面交流的有利条件, 所以一般而言, 面访式的回答率最高, 而自填式的回答率最低。但面访式的调查成本也是最高的,而自填式的调查成本最低。

三种搜集数据方法的特点如表 $2-3$ 所示。

表 2-3 搜集数据不同方法的特点

\begin{center}
\begin{tabular}{llll}
\hline
\multicolumn{1}{c}{项目} & \multicolumn{1}{c}{自填式} & \multicolumn{1}{c}{面访式} & \multicolumn{1}{c}{电话式} \\
\hline
调查时间 & 慢 & 中等 & 快 \\
调查费用 & 低 & 高 & 低 \\
问卷难度 & 要求容易 & 可以复杂 & 要求容易 \\
有形辅助物的使用 & 中等利用 & 充分利用 & 无法利用 \\
调查过程控制 & 简单 & 复杂 & 简单 \\
调查员作用的发挥 & 无法发挥 & 充分发挥 & 一般发挥 \\
回答率 & 最低 & 较高 & 一般 \\
\hline
\end{tabular}
\end{center}

由此可知, 没有哪一种方法在所有方面都是最好的, 因此, 在数据搜集方法的选择中要根据调查所需信息的性质、调查对象的特点、对数据质量和回答率的要求, 以及预算费用和时间要求等多方面因素综合而定。也许没有一种方法是完全适用的, 这时就要考虑研究人员对数据需求的最主要方面。需要说明的是, 各种方法并不是相互排斥的; 相反, 在许多方面恰恰是相互补充的, 因此, 在一项调研活动中将各种方法结合起来使用也许是不错的选择。例如, 对被选中的调查单位首先采用邮寄问卷方式, 让受访者自填, 对没有返回问卷的受访者, 再进行电话追访或面访。

\section*{23 实验方法}
搜集数据的另一类方法是通过实验, 在实验中控制一个或多个变量, 在有控制的条件下得到观测结果。所以, 实验数据是指在实验中控制实验对象而搜集到的变量的数据。例如, 对在一起饲养的一群牲畜, 分别喂给不同的饲料, 以检验不同饲料对牲畜增重的影响。实验是检验变量间因果关系的一种方法。在实验中, 研究人员要控制某一情形的所有相关方面, 操纵少数感兴趣的变量, 然后观察实验的结果。

\subsection*{2.3. 1 实验组和对照组}
实验不仅是搜集数据的一种方式, 而且是一种研究方法。实验法的基本逻辑是: 有意识地改变某个变量的情况 (不妨设为 $\mathrm{A}$ 项), 然后看另一个变量变化的情况 (不妨设为 $\mathrm{B}$项)。如果 B 项随着 $\mathrm{A}$ 项的变化而变化, 就说明 $\mathrm{A}$ 项对 $\mathrm{B}$ 项有影响。为此, 需要将研究对象分为两组, 一个为实验组, 一个为对照组。实验组 (experiment group) 是指随机抽选的实验对象的子集。在这个子集中, 每个单位接受某种特别的处理。而在对照组 (control group) 中, 每个单位不接受实验组成员所接受的某种特别的处理。

早在 17 世纪初, 英国海军就试图运用实验法找到坏血病的起因。当时, 在海上长期航行的水手面临坏血病的威胁, 皮肤上有青灰斑点, 牙龈大量出血, 英国海军部怀疑这是由于缺乏柑橘类水果所导致的。当这个想法被提出时, 恰好有四艘海军军舰正要离开英国本土做长期航行, 为调查是不是因为缺乏柑橘类水果而导致这种疾病, 海军部安排其中一艘军舰上的水手每天喝柑橘汁, 而其他三艘军舰上的水手则没有相橘汁供应。航行还未结束, 没有喝柑橘汁的水手开始成批地生病, 以至于不得不把每天喝柑橘汁的水手分配到这三艘军舰上以帮助这些军舰进港。

在这项实验中, 喝柑橘汁的水手构成了实验组, 没有喝柑橘汁的水手构成了对照组。需要对照组的原因是, 若没有对照组, 就无法判定 A 项是否对 B 项产生影响。设想, 如果四艘军舰上的水手都喝相橘汁, 那么, 没有得坏血病的原因是什么就无法验证。一个好的实验设计都有一个实验组和一个或多个对照组。

但英国海军的实验还是有欠缺的, 主要表现在两点: 首先, 实验组和对照组所处的外部环境应该相同, 在这个原则下, 每艘船上都应该有喝柑橘汁和不喝柑橘汁的实验者, 这样就排除了船的因素的影响。其次, 实验者在哪个组应该随机产生, 否则, 喜欢喝柑橘汁的人跑到了实验组, 而喜欢喝酒的人在对照组, 在研究开始之前两组的人员身体状况就存在差异, 这样就无法说明问题。如果实验对象是随机安排的, 那么健康和不健康的水手在每一组中的数目差不多, 身体状况对导致坏血病的影响就被抵消了, 实验数据才有更高的可信度。

一个好的实验, 对照组和实验组的产生不仅应该是随机的, 而且应该是匹配的。所谓匹配, 是指对实验单位的背景材料进行分析比较, 将情况类似的每对单位分别随机地分配到实验组和对照组。例如, 在实验新药或新的疗法时, 将接受实验的患者按照年龄、性别、病情等变量匹配后分到实验组和对照组。这样, 不同组的患者有大致相同的背景。同时, 分组的结果不让患者知道, 最好主持评价的医生也不知道, 这称为双盲法。双盲法也是在实验设计中应采用的。

\subsection*{2.3.2 实验中的若干问题}
实验法逻辑严密, 可以较好地证明假设, 分析事物间的因果关系, 但在实验过程中也会遇到一些问题。

\section*{1. 人的意愿}
根据前面的讨论, 我们知道, 在划分实验组和对照组时, 应该采用随机原则, 但在实施过程中会遇到挑战。如果研究的对象是人, 这种挑战就更为明显。人都有自己的生活方式和处世原则, 都有自己的爱好和兴趣, 他们未必会按照研究者的要求和布置行事。他们不会让自己的行为处于一定的控制条件下。

\section*{2. 心理问题}
在实验研究中, 人们对被研究非常敏感, 这使得他们更加关注自我, 从而走向另一个极端。记录这种影响的例子之一是 1924-1932 年间, 对西方电气公司的工人生产率的系列调查。在一次调查中, 一组社会学家和公司人事部门的成员想要研究车间照明度对工人劳动生产率的影响。研究者提高照明度, 发现产量增加。令人奇怪的是, 当他们降低照明度, 产量也增加。看来无论做什么, 工人的产量都会增加。后来发现, 产量增加的原因不在于照明度, 而是工人意识到有人在注意他们的行为, 从而表现出一种容易被社会认可和接受的行为, 尽管这种行为并不是他们愿意作出的。

\section*{3. 道德问题}
道德问题使得对人和动物做的实验复杂化了。当某种实验涉及道德问题时, 人们会处于进退两难的䢟尬境地。例如, 有一种理论认为, 人口密度大会导致犯罪率的上升。研究人员通过动物实验, 观察作为实验对象的小白鼠的行为变化。随着被关在一起的小白鼠的密度不断加大, 小白鼠变得越来越烦躁, 最后相互攻击、自相残杀。显然, 对人做这种实验是不道德的, 那么对小白鼠做这种实验就道德吗? 又比如, 在做药物实验时, 如何看待实验组和对照组的结果? 例如, 发明了一种有望治疗艾滋病的新药, 实验组的患者服用这种药, 而对照组的患者不能服用这种药。如果新药是有效的, 对照组的人得不到新药就会面临死亡的威胁。然而, 如果发现这种药有副作用, 从而导致服用该药的人在两年以后有更高的死亡率, 那么, 没有服用这种药的对照组患者则可能避免这种风险。这中间确实存在道德的困境。

\subsection*{2.3.3 实验中的统计}
统计在实验过程中发挥着重要的作用。这些作用主要表现在: 确定进行实验所需要的单位的个数, 以保证实验可以达到统计显著的结果; 将统计的思想融人实验设计, 使实验设计符合统计分析的标准;提供尽可能最有效地同时研究几个变量影响的方法。

确定进行实验所需要的单位的个数, 以便得到关于实验精度预期的结果, 这需要统计学的专业知识。一般来说, 实验数据越多越好。但进行大规模的实验, 搜集数据的成本将非常高, 所需要的时间也更长。统计分析能够为在精度与费用的平衡中作出决断提供参考信息。\\
进行实验设计, 也离不开统计学知识。实验设计本身就是一个统计问题。实验设计是探索如何根据研究问题的需要, 科学地安排实验, 使我们能用尽可能少的实验获得尽可能多的信息。实验设计的有关问题将在后续章节中介绍。

在对实验数据进行分析时, 根据研究的需要, 统计可以提供最恰当的分析方法。一个好的实验应该在两个方面都有效。一方面是内部的有效性, 内部的有效性意味着实验测量的准确性。实验的目的是要考察自变量和因变量之间的因果关系, 而如果实验观察结果受到其他无关变量的影响, 就很难推断自变量与因变量之间的因果关系。另一个方面是外部的有效性, 外部的有效性决定是否可以将实验中发现的因果关系加以推广, 即能否将结果推广到实验环境以外的情况? 如果可以, 可以推广到什么样的总体、什么样的环境、什么样的自变量和因变量? 与实验情况完全相同的环境在社会现实中是很难复制的, 那么, 实验结果是否还有效? 对这些问题给出分析和解释, 需要利用统计方法。例如, 多元回归分析可以将各个变量的影响区分开, 在满足一定条件下定量地比较各个自变量对因变量产生的影响。协方差分析可以通过调整每组内因变量的平均值, 达到将无关变量的影响剔除的目的。此外,多元统计分析的方法在实验数据的分析中也发挥着重要的作用。

\subsection*{2.3.4 实验法案例}
通过实验得到的数据称为实验数据, 实验数据可以作为研究者判断假设的依据。下面的两个案例或许可以使读者对实验数据的作用有更多的体会。

\section*{案例 2.1}
\section*{现场实验帮助 A 公司胜诉}
美国的 A 公司生产著名的运动包, 该公司发现 B 公司(一个大型的商业集团)引进一条生产线, 生产的运动包与 A 公司生产的运动包外形几乎完全一样, 消费者很难区分。 $\mathrm{A}$ 公司指控 B 公司, 说 B 公司误导消费者, 让消费者觉得自己买的是 A 公司的产品, 而实际买的却是 $\mathrm{B}$ 公司的产品。为了证实这一点, 由第三方进行了一次现场实验。实验中选择了两组妇女, 给第一组妇女看的是 A 公司生产的包, 包面上的所有标签都去掉, 所有的标识、说明都印在包的里层。给第二组妇女看的是 B 公司生产的包, 包上的商标明显可见, 所有的标签和悬挂物都按出售时的样子保留。这样做的目的是希望通过这种实验了解妇女购买包时的选择标准。例如, 她们能否区分出包的不同来源或品牌, 她们依据什么进行识别或辨认,如果靠某些东西来辨认,那么这样做的理由是什么。

每组样本都是 200 人, 实验分别在芝加哥、洛杉矶和纽约的大商场进行。调查采用拦截式面访, 被调查者是配额样本, 即按妇女不同的年龄比例分配样本单位。

实验结果表明, 大多数消费者无法区分两种包的不同来源, 她们购买包时主要是看包的款式, 而 A 公司生产的是名牌商品, 这种包的款式是人们所熟悉的。这个结果支持了 $\mathrm{A}$公司的立场。调查数据帮助 A 公司在法庭上胜诉, B公司同意停止销售自己生产的包。

\section*{案例 2.2}
\section*{科普节目效果实验}
为了增加儿童 (4 7 岁) 对天文学基本知识的了解, 培养家长和儿童对天文学和观察天象的积极态度, 提高他们对天文学的鉴赏能力, 有关部门制作了一系列天文学科普节目, 在天文馆播出。为了解系列节目的效果, 需要调查在天文馆的观看经历对儿童产生了什么影响。这种影响可以分为两个层面: 一个是短期的影响, 这可以通过受访者对观看节目的感受得到反映; 另一个是长期的影响, 即看完节目后采取了哪些相关行动。这项节目效果实验调查的设计是这样的: 在儿童观看节目前和观看节目后的几分钟内对他们进行短暂的调查。然后, 在观看节目一个月以后进行另一项跟踪调查。调查的样本量为儿童 500 名,家长 500 名。

一开始的调查是在天文馆现场进行的。在该节目播出期间,每个被抽中的儿童在观看节目前接受访员 (年轻的大学生) 大约 5 分钟的谈话调查。访员所问的问题与被访者年龄相适应, 知识问题与天文学相关, 同时还要询问观察天象的经历(如果有观察天象的经历)。观看节目前询问的问题有看电视节目的习惯, 如是否看过《我们的宇宙》(一个在电视上播出过的天文学科普节目), 有哪些家庭学习资源 (如电视、望远镜、电脑等), 以及是否参观过天文馆或者类似的地方。观看后的询问应该由同一个访员来进行。询问时间不超过 10 分钟, 问题包括对节目中有关信息的回忆、对观察天象的兴趣和态度。在询问结束后,送给孩子一个和天文学相关的小礼物,以感谢他们接受访问。

对于带儿童参观的家长或者看护人,让他们在观看节目之前填写一份简短的问卷,在观看节目后马上填写另一份简短的问卷。填写两份问卷的时间分别不超过 5 分钟和 10 分钟。观看前的问卷询问被访者是否带孩子参与过与天文学有关的活动、孩子参加活动的特点, 以及孩子对天文学的兴趣和了解程度。观看前的访问内容还包括以前观察天象的经历 (如果有这样的经历)、学习天文学的资源 (望远镜、电脑和杂志等)。观看后的访问将了解看护人对带儿童观看有关节目内容的评论、将了解的知识运用到今后活动中的想法, 以及他们个人对天文学的兴趣和态度。

在被调查者接受现场调查以后约一个月, 对被访者进行电话跟踪调查。调查员与在天文馆进行现场调查的是同一个人, 调查时间为 10 分钟。跟踪调查是为了了解被访者在观看节目后是否进行过与天文学有关的活动 (买书、资料或者望远镜, 收看有关天文学的电视节目, 参观天文馆或者类似的场馆, 与孩子共同讨论有关天文学的问题以及观察天象),对天文学和科学的态度以及 (经过一段时间后) 对天文馆科普节目的评价。对家长结束访问后, 要询问家长是否可以问孩子几个问题。如果得到同意, 访员要让孩子回忆一些那天观看天文学节目的问题, 参加天文学方面的一些活动或者观看的有关电视节目, 以及与家长和其他小朋友讨论的问题。对儿童的电话访问时间不超过 10 分钟。

在这个案例中, 我们没有详细讨论如何抽选天文馆, 以及在中选的天文馆现场如何抽选接受调查的孩子和家长, 抽选的原则是随机的。

\section*{2.4 数据的误差}
数据的误差是指通过调查搜集到的数据与研究对象真实结果之间的差异。数据的误差有两类:抽样误差和非抽样误差。

\subsection*{2.4. 1 抽样误差}
抽样误差 (sampling error) 是由抽样的随机性引起的样本结果与总体真值之间的差异。在概率抽样中, 我们依据随机原则抽取样本, 可能抽中由这样一些单位组成的样本,也可能抽中由另外一些单位组成的样本。根据不同的样本, 可以得到不同的观测结果。例如, 为检验一批产品的非优质品率, 随机抽出一个样本, 样本由若干个产品组成, 通过检测得到非优质品率为 $30 \%$ 。如果我们再抽取一个产品数量相同的样本, 检测的结果不太可能是 $30 \%$, 有可能是 $29 \%$, 也有可能是 $31 \%$ 。不同样本会得到不同的结果。但是我们知道, 总体真实的结果只能有一个, 尽管这个真实的结果我们并不知道。不过可以推测, 虽然不同的样本会带来不同的答案, 但这些不同的答案应该在总体真值附近。如果不断地增大样本量, 不同的答案会向总体真值逼近。事实也正是如此, 如果这批产品的数量非常大, 不得不采用抽样的办法检验其质量, 又假设样本由随机抽取出的 1000 个零件组成,经过多次抽样, 得到多个不同样本的检测结果, 就会发现这些结果的分布是有规律的。例如, 如果总体真正的非优质品率是 $30 \%$, 那么, 大部分的样本结果(如反复抽样中 $95 \%$的样本结果)会落在 $27.2 \% \sim 32.8 \%$ 之间。以总体真值 $30 \%$ 为中心,有 $95 \%$ 的样本(100 个样本中大约有 95 个样本)结果在 $\pm 2.8 \%$ 的误差范围内波动, 也就是 $30 \%-2.8 \%=$ $27.2 \%, 30 \%+2.8 \%=32.8 \%$ 。这个 $\pm 2.8 \%$ 的误差是由抽样的随机性带来的, 我们把这种误差称为抽样误差。抽样误差的示意图如图 2-1 所示。

\begin{center}
\end{center}

图 2-1 总体百分比和抽样误差示意图

由此看出, 抽样误差并不是针对某个具体样本的检测结果与总体真实结果的差异而言的, 抽样误差描述的是所有样本可能的结果与总体真值之间的平均差异。例如, 在图 2-1 中, 我们说全部样本中 $95 \%$ 的样本结果与真值之间的差异上下不超过 $2.8 \%$ 的范围。读到这里, 读者可能会问: “既然总体真值都不知道, 怎么可能知道有 $95 \%$ 的样本结果与真值的差异是 $2.8 \%$ 呢? 确实, 总体真值我们是不知道, 否则也就不用调查了。但是, 通过样\\
本可以计算出这个误差。本书第 6 章、第 7 章将介绍这方面的内容。

抽样误差的大小与多方面因素有关。最主要的是样本量的大小, 样本量越大, 抽样误差就越小。当样本量大到与总体单位相同时, 也就是抽样调查变成普查, 这时抽样误差便减小到零, 因为这时已经不存在样本选择的随机性问题, 每个单位都需要接受调查。抽样误差的大小还与总体的变异性有关。总体的变异性越大, 即各单位之间的差异越大, 抽样误差也就越大, 因为有可能抽中特别大或特别小的样本单位, 从而使样本结果偏大或偏小; 反之, 总体的变异性越小, 各单位之间越相似, 抽样误差也就越小。如果所有的单位完全一样, 调查一个就可以精确无误地推断总体, 抽样误差也就不存在了。现实中, 这种情况也是不存在的, 否则, 对这样的总体也就不用进行专门的抽样调查了。

\subsection*{2.4.2 非抽样误差}
非抽样误差 (non-sampling error) 是相对抽样误差而言的, 是指除抽样误差之外的,由其他原因引起的样本观测结果与总体真值之间的差异。抽样误差是一种随机性误差, 只存在于概率抽样中; 非抽样误差则不同, 无论是概率抽样、非概率抽样, 还是在全面调查中, 都有可能产生非抽样误差。非抽样误差有以下几种类型。

\section*{1. 抽样框误差}
在概率抽样中需要根据抽样框抽取样本。抽样框是有关总体全部单位的名录, 在地域抽样中, 抽样框可以是地图。一个好的抽样框应该是, 抽样框中的单位和研究总体中的单位有一一对应的关系。例如, 如果在某个学校抽取一个学生样本, 则抽样框是该学校所有学生的名单, 这时, 名单中的每一个名字都对应着一个学生。该校所有学生的名字都在抽样框中有所反映, 抽样框中的所有名字又确实是该校目前注册的所有学生, 这时, 就存在一一对应的关系。但如果学生的名单是去年的, 新人学学生的名字没有在名单上, 而名单上的学生有些已经毕业, 不再是该校的学生, 这时, 抽样框中的单位与研究总体的单位就不存在一一对应的关系, 使用这样的抽样框抽取样本就会出现一些错误。例如, 由于新人学学生的名字不在抽样框中, 他们不可能被选人样本, 他们那部分的信息就无法知道; 而已毕业学生的名字仍然在名单中, 他们已经不属于研究总体, 但由于他们名字的存在, 抽样过程中的单位人样概率发生变化, 结果导致推论的错误。这些统计推论的错误是抽样框的不完善造成的, 我们把这种误差称为抽样框误差。

构造一个好的抽样框是抽样设计中的一项重要内容。在调查对象确定后, 通常可以选取不同的资料构造抽样框。例如, 上面对学生情况的调查, 抽样框可以是学生名单, 也可以是学生宿舍的号码 (先抽取宿舍, 再从选中的宿舍中抽取学生)。在这种情况下, 设计人员的任务是选择与调查内容最贴切的抽样框。

\section*{2. 回答误差}
回答误差是指被调查者在接受调查时给出的回答与真实情况不符。导致回答误差的原\\
因有多种,主要有理解误差、记忆误差和有意识误差。

(1) 理解误差。

不同的被调查者对调查问题的理解不同, 每个人都按自己的理解回答, 大家的标准不一致, 由此造成理解误差。

例如, 有些表示频率的词, 如 “经常” “频繁” “偶尔” 等, 在调查中经常使用。实际上不同的人对这些词的理解是有差别的。设想在一项关于电视收视率的调查中询问被调查者这样一个问题:

您经常看电视节目吗?

\begin{enumerate}
  \item 从来不看;

  \item 偶尔看;

  \item 有时看;

  \item 经常看;

  \item 天天看。

\end{enumerate}

被调查者对这五项选择的理解可能是不同的。例如, 某人一周看两次电视, 他认为属于 “偶尔看”, 而另一个人同样一周看两次电视, 却可能选择 “有时看” 或 “经常看”。这说明, 问卷中的措辞对减少调查中的非抽样误差起着相当重要的作用。

对这个问题的调查, 比较好的措辞可以是:

您经常看电视节目吗?

\begin{enumerate}
  \item 从来不看;

  \item 平均每周少于 1 次;

  \item 平均每周 1 2 次;

  \item 平均每周 $3 \sim 5$ 次;

  \item 平均每周 6 7 次。

\end{enumerate}

这样, 被调查者对问题的理解就唯一了, 就有可能减少理解误差。

有时, 问卷中问题的排序也会对调查结果产生影响。这方面一个经典的例子取自 1980 年的一项实验调查。调查的对象是美国居民,调查的问题有下面两个:

A. 你是否认为美国应该让那些社会主义国家 (如苏联) 的记者到美国来, 并把他们看到的新闻发回去。

B. 你是否认为像苏联那样的社会主义国家应该让美国新闻记者人境, 并把他们看到的新闻发回美国。

实验结果是, 如果按 A, B 的顺序排列, 对问题 A 赞同的比例有 $54.7 \%$, 对问题 B 赞同的比例有 $63.7 \%$ 。但是,如果按 B, A 的顺序排列,对问题 A 赞同的比例有 $74.6 \%$,对问题 B 赞同的比例有 $81.9 \%$ 。不同的排序得到不同的结果。从心理学的角度分析, 人们在回答问题时总是有意无意地保持一致。美国人可能比较愿意让美国的记者到社会主义国家去,并把新闻发回美国。当把问题 B 放在前面时,就形成一个比较宽松的框架,既然苏联应该让美国记者人境,美国也应该让苏联记者进人美国,所以赞同的比例都比较高。\\
这些对问题的理解都与被调查者的心理活动有关。心理学知识对于设计一份好的调查问卷是有帮助的。

(2) 记忆误差。

有时, 调查的问题是关于一段时期内的现象或事实, 需要被调查者回忆。需要回忆的时间间隔越久,回忆的数据就可能越不准确。所以,缩短调查所涉及的时间间隔可以减少记忆误差。但是, 有些事件是按一定周期发生的。例如, 研究农作物产量与施肥量的关系。产量通常以年度为周期, 而肥料的用量与收获年度有关。在这种情况下, 以年度为调查期更适宜。

(3) 有意识误差。

当调查的问题比较敏感, 被调查者不愿意回答, 迫于各种原因又必须回答时, 就可能会提供一个不真实的数字。产生有意识误差的动因大致有两种: 一种是调查问题涉及个人隐私, 被调查者不愿意告知, 所以造假; 另一种是受利益驱动, 进行数字造假。有意识误差比记忆误差的危害要大。因为记忆误差具有随机性, 有些人可能说高了, 有些人可能说低了, 高低相抵, 调查结果还是具有趋中的倾向; 有意识误差则不同, 它往往偏向一个方向, 是一种系统性偏差。例如, 调查纳税情况时, 被调查者往往高报, 以表现自己没有漏税行为。而对高收人者调查收人时, 被调查者则往往低报, 以避免被视为富人。

减少回答中的有意识误差需要多方面的努力。调查人员要做好被调查者的思想工作,让他们打消顾虑; 调查人员要遵守职业道德, 为被调查者保密; 调查中尽量避免敏感问题。对于政府统计中的调查, 要加强法制化管理, 让 “数字造假” 没有市场。

\section*{3. 无回答误差}
无回答误差是指被调查者拒绝接受调查, 调查人员得到的是一份空白的答卷。无回答也包括那些调查进行时被访者不在家的情况。电话调查中, 拨通后没有人接; 邮寄问卷调查中, 地址写错, 被调查者搬家, 或被调查者虽然收到问卷却把问卷遗忘或丢失等, 这些都可以视为调查中的无回答。

无回答会对调查结果产生什么影响? 如果我们询问被访者的收人, 他拒绝回答他的收人是高还是低。如果他回答了, 调查结果将会发生怎样的变化? 在一项调查中, 如果无回答所占比例很小, 对最后结果的影响不大。但是, 如果无回答占到样本很大的比例, 调查结果的说服力将大打折扣。令人不安的是, 现在调查中的无回答率正在呈上升趋势。更令人不安的是, 一些调查人员还没有深刻认识到无回答对数据质量的危害, 对于存在大量无回答的调查结果不采取任何补救措施, 拿来就用。须知, 这是在用样本推算总体, 失之毫厘必将谬以千里。

无回答误差有时是随机的, 有时是系统性的。假设无回答的产生与调查的内容无关,例如, 邮寄的问卷丢失, 或调查时被访者正在生病, 无法接受调查。在随机状态下, 被访者如果回答, 其结果可能高于平均值, 也可能低于平均值, 高低相互抵消, 不会产生有偏估计。但当无回答的产生与调查内容有关时, 就可能产生系统性误差。例如, 调查收人时拒绝\\
回答者通常是收人比较高的人群, 仅仅用收人低的回答结果进行推算, 偏差就不可避免。

如果无回答误差是随机的, 可以通过增加样本量的方式解决。例如, 调查设计要求完成 1000 个样本单位, 结果回答了 800 个, 无回答率为 $20 \%$, 这时可以再随机抽 250 个单位, 并对其进行调查, 如果无回答率仍旧为 $20 \%$, 就可以得到 200 个单位的回答。在有些情况下, 虽然无回答的产生与调查内容无关, 但如果无回答集中于某类群体, 结果仍然是危险的。例如, 采用电话调查方式了解居民对公共交通的看法。如果电话没人接就用另一个号码代替, 这意味着, 经常不在家或回家晚的人有较小的被调查机会, 而可能恰恰这群人与公共交通有更多的接触。频繁地更换号码也违背了每个人都应该有固定的被调查机会这样的抽样原则。所以, 当遇到被访者不在家这类情况时, 调查人员不要轻易放弃, 应该进行多次回访。大量事实证明, 调查数据的质量与调查员的责任心和耐心密切相关。

无回答的系统性误差令人头疼。解决的途径主要有两个方面: 一方面是预防, 即在调查前做好各方面的准备工作, 尽量把无回答降到最低限度。另一方面, 当无回答出现后,分析无回答产生的原因, 采取一些补救措施。例如, 在无回答单位中再抽取一个样本, 实施更有力的调查, 并以此作为无回答层的代表, 和回答层的数据结合起来对总体进行估计。目前, 对无回答问题的研究仍在进行。

\section*{4. 调查员误差}
调查员误差是指由于调查员的原因而产生的调查误差。例如, 调查员粗心, 在记录调查结果时出现错误。调查员误差还可能来自调查中的诱导, 而调查员本人或许并没有意识到。例如, 在调查过程中调查员有意无意地流露出对调查选项的看法或倾向, 调查员的表情变化、语气变化、语速变化都可能对被调查者产生某种影响。

\section*{5. 测量误差}
如果调查与测量工具有关, 则很有可能产生测量误差。例如, 对小学生的视力状况进行抽样调查, 而视力的测定与现场的灯光、测试距离都有密切关系。调查在不同地点进行, 如果各测试点的灯光、测试距离有差异, 就会给调查结果带来测量误差。调查有时采用观察、记数的方式进行。例如, 在调查商场客流量时, 调查员站在商场门口查点进出的顾客, 在调查汽车流量时, 查点过往的车辆。谁敢保证这种查点一个不错呢?

\subsection*{2.4.3 误差的控制}
上面对调查中的误差问题进行了比较详细的讨论。如何有效地控制各种误差, 提高数据的质量, 这是研究人员和现场调查人员面临的挑战。

抽样误差是由抽样的随机性带来的, 只要采用概率抽样, 抽样误差就不可避免。令人欣慰的是, 抽样误差是可以计算的。在对特定问题的研究中, 研究人员对抽样误差有一个可以容忍的限度。例如, 用抽检的方法检验产品的质量, 对总体合格品率估计的误差不超过 $\pm 1 \%$, 这个 $\pm 1 \%$ 就是允许的抽样误差。允许的抽样误差是多大, 取决于对数据精度的\\
要求。一旦这个误差确定下来, 就可以采用相应的措施进行控制。进行控制的一个主要方法是改变样本量。统计方法已经给出了计算样本量的公式 (见第 7 章)。要求的抽样误差越小, 所需要的样本量就越大。

非抽样误差与抽取样本的随机性无关, 因而在概率抽样和非概率抽样中都会存在(但抽样框误差仅在概率抽样中存在) 。有很多原因会造成非抽样误差, 因此控制起来比较困难。全面讨论非抽样误差的控制问题已经超出本书的范围, 有兴趣的读者可以参考本书后面所列的参考文献, 这里只是简要做些介绍。

如果采用概率抽样, 就需要抽样框, 抽样框误差就可能出现。在有些情况下, 抽样人员对这个问题不够重视, 使用了不太好的抽样框。其实, 对同一个调查问题, 有时可以构造不同的抽样框。例如, 对学校教师进行抽样调查, 以了解他们对建设一流大学的看法,抽样框可以是教师的名单, 可以是教师住所的门牌号码, 可以是教师家的电话号码, 甚至可以是教师上课的教室编号。不同的抽样框, 其质量可能会有所差别, 通过认真分析可以选择出比较好的抽样框。此外, 构造抽样框还需要广泛地搜集有关信息, 对抽样框进行改进, 例如, 把两个抽样框结合起来, 以弥补抽样框覆盖不全的缺陷。

一份好的调查问卷可以有效地减少调查误差。问卷中题目的类型、提问的方式、使用的词汇、问题的组合等, 都可能会对被调查者产生哪怕是十分微小的影响, 而大量微小影响的累加是不可忽视的。做好问卷设计是减少非抽样误差的一个方面。

非抽样误差控制的重要方面是调查过程的质量控制。这包括: 调查员的挑选, 调查员的培训, 督导员的调查专业水平, 对调查过程进行控制的具体措施, 对调查结果进行的检验、评估, 对现场调查人员进行奖惩的制度, 等等。目前在规范的专业性市场, 调查咨询公司都有一些进行质量控制的规章制度和经验。

\section*{思考与练习}
2.1 什么是二手资料? 使用二手资料需要注意些什么?

2.2 比较概率抽样和非概率抽样的特点。举例说明什么情况下适合采用概率抽样, 什么情况下适合采用非概率抽样。

2.3 调查中搜集数据的方法主要有自填式、面访式、电话式。除此之外, 还有哪些搜集数据的方法?

2.4 自填式、面访式、电话式调查各有什么利㢣?

2.5 请举出 (或设计) 几个实验数据的例子。

2.6 你认为应当如何控制调查中的回答误差?

2.7 怎样减少无回答? 请通过一个例子说明你所考虑到的减少无回答的具体措施。

\section*{第 3 章 \\
 数据的图表展示}
\begin{center}
\end{center}

述不同原因引起的寿命减少的天数。

\begin{center}
\begin{tabular}{|lc||lc|}
\hline
\multicolumn{1}{c|}{原因} & 寿命减少的天数 & \multicolumn{1}{c|}{原因} & 寿命减少的天数 \\
\hline
未结婚 (男性) & 3500 & 危险工作, 事故 & 300 \\
惯用左手 & 3285 & 交通事故 & 200 \\
吸烟 (男性) & 2250 & 饮酒 & 130 \\
未结婚 (女性) & 1600 & 滥用药物 & 90 \\
超重 30\% & 1300 & 一般工作, 事故 & 74 \\
超重 $20 \%$ & 900 & 自然放射性 & 8 \\
吸烟 (女性) & 800 & 喝咖啡 & 6 \\
抽雪茄 & 300 & 医疗 X射线 & 6 \\
\hline
\end{tabular}
\end{center}

比较上述三种方式, 你认为哪种方式最好?\\
使用图表描述数据是应用统计的基本技能之一。本章首先介绍数据的预处理方法, 然后介绍频数分布表的生成及图表展示方法。

\section*{3.1 数据的预处理}
数据的预处理是在数据分析之前所做的必要处理, 内容包括数据的审核、笠选、排序等。

\section*{3. 1.1 数据审核}
数据审核就是检查数据中是否有错误。对于通过调查取得的原始数据(raw data), 主要从完整性和准确性两个方面去审核。完整性审核主要是检查应调查的个体是否有遗漏,所有的调查项目是否填写齐全等。准确性审核主要是检查数据是否有错误, 是否存在异常值等。对于异常值要仔细鉴别: 如果异常值属于记录时的错误, 在分析之前应予以纠正;如果异常值是一个正确的值, 则应予以保留。

对于通过其他渠道取得的二手数据, 应着重审核数据的适用性和时效性。二手数据可以来自多种渠道, 有些可能是为特定目的通过专门调查而取得的, 或者是已经按特定目的的需要做了加工整理。对于使用者来说, 首先应弄清楚数据的来源、数据的口径以及有关的背景材料, 以便确定这些数据是否符合分析研究的需要, 不能盲目地生搬硬套。此外,还要对数据的时效性进行审核, 对于时效性较强的问题, 如果所取得的数据过于滞后, 就可能失去研究的意义。

\section*{3. 1.2 数据筛选}
数据䇤选 (data filter) 是根据需要找出符合特定条件的某类数据。比如, 找出销售额在 1000 万元以上的企业; 找出考试成绩在 90 分以上的学生; 等等。数据笠选可借助计算机自动完成。下面通过一个简单的例子说明用 Excel 进行数据篮选的过程。

\section*{例 3. 1}
表 3-1 是 8 名学生 4门课程的考试成绩数据 (单位: 分)。试找出统计学成绩等于 75 分的学生, 英语成绩最高的前三名学生, 四门棵程成绩都高于 70 分的学生。

表 3-1 8 名学生的考试成绩数据

\begin{center}
\begin{tabular}{ccccc}
\hline
姓名 & 统计学成绩 & 数学成绩 & 英语成绩 & 经济学成绩 \\
\hline
张松 & 69 & 68 & 84 & 86 \\
王翔 & 91 & 75 & 95 & 94 \\
田雨 & 54 & 88 & 67 & 78 \\
李华 & 81 & 60 & 86 & 64 \\
\hline
\end{tabular}
\end{center}

\begin{center}
\begin{tabular}{ccccc}
\hline
姓名 & 统计学成绩 & 数学成绩 & 英语成绩 & 经济学成绩 \\
\hline
赵颖 & 75 & 96 & 81 & 83 \\
宋媛 & 83 & 72 & 66 & 71 \\
袁方 & 75 & 58 & 76 & 90 \\
陈风 & 87 & 76 & 92 & 77 \\
\hline
\end{tabular}
\end{center}

解 用 Excel 进行数据第选的具体步骤如下。

太用 Excel 进行数据笠选的操作步骤

第 1 步: 将光标放在数据区域的任意单元格。然后点击【数据】 $\rightarrow$ 【第选】。这时会在每个变量名中出现下拉箭头。

第 2 步: 点击要笠选的变量的下拉箭头即可对该变量进行第选。比如, 要䇤选出统计学成绩为 75 分的学生, 点击统计学成绩变量的下拉箭头, 在【数字第选】下选择【等于】, 并在其后面的方框内选择 (或写入) 75, 得到的结果如图 3-1 所示。

\begin{center}
\begin{tabular}{|c|c|c|c|c|c|}
\hline
4 & A & B & C & D & E \\
\hline
1 & 姓领 & 统计学成细 & 数学成绩- & 英语成绩- & 经济学成约 \\
\hline
6 & 赵颖 & 75 & 96 & 81 & 83 \\
\hline
8 & 袁方 & 75 & 58 & 76 & 90 \\
\hline
10 &  &  &  &  &  \\
\hline
\end{tabular}
\end{center}

图 3-1 篣选出的统计学成绩等于 75 分的学生

要篮选出英语成绩最高的前三名学生, 点击英语成绩变量的下拉箭头, 在【数字管选】下选择

【前 10 项】, 并在对话框中的【最大】后输入数据 3, 结果如图 3-2 所示。

\begin{center}
\begin{tabular}{|c|c|c|c|c|c|}
\hline
$\angle$ & A & B & C & $\mathrm{D}$ & E \\
\hline
1 & 姓全。 & 统计学成约 & 数学成绩- & 英语成绩 $\mathbf{T}$ & 经济学成约 \\
\hline
3 & 王翔 & 91 & 75 & 95 & 94 \\
\hline
5 & 李华 & 81 & 60 & 86 & 64 \\
\hline
9 & 陈风 & 87 & 76 & 92 & 77 \\
\hline
10 &  &  &  &  &  \\
\hline
\end{tabular}
\end{center}

图 3-2 第选出的英语成绩最高的前三名学生

如果要篮选出四门课程成绩都高于 70 分的学生, 由于设定的条件比较多, 需要使用【高级篣选】命令。使用高级笕选时, 必须建立条件区域。这时需要在数据清单上面至少留出三行作为条件区域。然后选择【数据】 $\rightarrow$ 【高级】。在【列表区域】输入要篮选的数据区域; 在【条件区域】输入条件区域。出现的界面如图 3-3 所示。点击【确定】 后出现的结果如图 3-4 所示。

\begin{center}
\end{center}

图 3-3 多条件的高级第选过程

\begin{center}
\begin{tabular}{|c|c|c|c|c|c|}
\hline
4 & A & B & C & D & E \\
\hline
1 & 姓名 & 统计学成绩 & 数学成绩 & 英语成绩 & 经济学成绩 \\
\hline
2 &  & $>70$ & $>70$ & $>70$ & $>70$ \\
\hline
\multicolumn{6}{|l|}{3} \\
\hline
4 & 姓名 & 统计学成绩 & 数学成绩 & 英语成绩 & 经济学成绩 \\
\hline
6 & 王翔 & 91 & 75 & 95 & 94 \\
\hline
9 & 赵颖 & 75 & 96 & 81 & 83 \\
\hline
12 & 陈风 & 87 & 76 & 92 & 77 \\
\hline
13 &  &  &  &  &  \\
\hline
\end{tabular}
\end{center}

图 3-4 笠选出四门课程成绩都高于 70 分的学生

\section*{3. 1.3 数据排序}
数据排序是指按一定顺序将数据排列, 以便研究者通过浏览数据发现一些明显的特征或趋势, 找到解决问题的线索。除此之外, 排序还有助于对数据进行检查纠错, 以及为重新归类或分组等提供方便。在某些场合, 排序本身就是分析的目的之一。例如, 了解究竟谁是中国汽车生产的三巨头, 对于汽车生产厂商而言, 不论它是合作伙伴还是竞争者, 都是很有用的信息。美国的《财富》杂志每年都要在世界范围内排出 500 强企业, 通过这一信息, 企业不仅可以了解自己所处的位置, 清楚自身的差距, 还可以从一个侧面了解到竞争对手的状况, 有效制定企业的发展规划和战略目标。

对于分类数据, 如果是字母型数据, 排序则有升序、降序之分, 但习惯上升序用得更多, 因为升序与字母的自然排列相同; 如果是汉字型数据, 排序方式则很多, 比如按汉字的首个拼音字母排列, 这与字母型数据的排序完全一样, 也可按姓氏笔画排序, 其中也有笔画多少的升序、降序之分。交替运用不同方式排序, 在汉字型数据的检查纠错过程中十\\
分有用。

对于数值数据, 排序只有两种, 即递增和递减。设一组数据为 $x_{1}, x_{2}, \cdots, x_{n}$, 递增排序后可表示为: $x_{(1)}<x_{(2)}<\cdots<x_{(n)}$; 递减排序后可表示为: $x_{(1)}>x_{(2)}>\cdots>x_{(n)}$ 。排序后的数据也称为顺序统计量 (order statistics)。无论是分类数据还是数值数据, 排序均可借助 Excel 很容易地完成。

\section*{3.2 分类数据的整理与展示}
数据经过预处理后, 可根据需要进一步分类或分组。在对数据进行整理时, 首先要弄清所面对的是什么类型的数据, 因为不同类型的数据所采取的处理方式和所适用的处理方法是不同的。对分类数据主要是做分类整理, 对数值数据则主要是做分组整理。

\subsection*{3.2. 1 分类数据的整理}
分类数据本身就是对事物的一种分类, 因此, 在整理时首先列出所分的类别, 然后计算出每一类别的频数、频率或比例、比率等, 即可形成一张频数分布表, 最后根据需要选择适当的图形进行展示, 以便对数据及其特征有一个初步的了解。

频数 (frequency) 是落在某一特定类别或组中的数据个数。把各个类别及落在其中的相应频数全部列出, 并用表格形式表现出来, 称为频数分布 (frequency distribution)。针对一个分类变量生成的频数分布表称为简单频数表; 根据两个分类变量生成的频数分布表称为列联表 (contingency table) 或交叉表(cross table)。下面通过一个例子说明如何使用 Excel 来制作分类数据的频数分布表。

\section*{例 3.2}
为研究不同类型软饮料的市场销售情况, 一家市场调查公司对随机抽取的一家超市进行调查。表 3-2 是调查员随机观察的 50 名顾客购买的饮料类型及顾客性别的记录。生成频数分布表, 观察饮料类型和顾客性别的分布状况, 并进行描述性分析。

表 3-2 顾客性别及购买的饮料类型

\begin{center}
\begin{tabular}{cl||cl||cl}
\hline
顾客性别 & \multicolumn{1}{|c||}{饮料类型} & 顾客性别 & \multicolumn{1}{c||}{饮料类型} & 顾客性别 & \multicolumn{1}{c}{饮料类型} \\
\hline
女 & 碳酸饮料 & 女 & 碳酸饮料 & 女 & 其他 \\
男 & 绿茶 & 男 & 绿茶 & 女 & 碳酸饮料 \\
男 & 矿泉水 & 男 & 其他 & 女 & 其他 \\
女 & 矿泉水 & 女 & 碳酸饮料 & 女 & 果汁 \\
\hline
\end{tabular}
\end{center}

续表

\begin{center}
\begin{tabular}{cl||cl||cl}
\hline
顾客性别 & \multicolumn{1}{|c||}{饮料类型} & 顾客性别 & 饮料类型 & 顾客性别 & 饮料类型 \\
\hline
男 & 碳酸饮料 & 男 & 绿茶 & 男 & 绿茶 \\
男 & 矿泉水 & 男 & 绿茶 & 女 & 果汁 \\
女 & 碳酸饮料 & 女 & 碳酸饮料 & 女 & 碳酸饮料 \\
女 & 绿茶 & 男 & 碳酸饮料 & 女 & 果汁 \\
男 & 果汁 & 女 & 绿茶 & 男 & 矿泉水 \\
男 & 碳酸饮料 & 男 & 矿泉水 & 女 & 碳酸饮料 \\
女 & 矿泉水 & 女 & 绿茶 & 女 & 绿茶 \\
女 & 其他 & 女 & 碳酸饮料 & 女 & 其他 \\
男 & 碳酸饮料 & 女 & 矿泉水 & 女 & 果汁 \\
男 & 绿茶 & 其他 & 男 & 绿茶 &  \\
男 & 碳酸饮料 & 男 & 碳酸饮料 & 女 & 其他 \\
女 & 其他 & 女 & 果汁 & 女 & 矿泉水 \\
男 & 矿泉水 & 男 & 矿泉水 &  &  \\
\hline
\end{tabular}
\end{center}

\begin{itemize}
  \item 解 用 Excel 生成分类数据频数分布表有几种途径, 其中最简单的是使用数据透视表进行计数和汇总。具体过程如下:
\end{itemize}

太用【数据透视表】命令制作分类数据的频数分布表

第 1 步:选择【插入】 $\rightarrow$ 【数据透视表】。

第 2 步: 在【表/区域】框内选定数据区域(在操作前将光标放在任意数据单元格内, 系统会自动选定数据区域)。选择放置数据透视表的位置。系统默认是新工作表,如果要将透视表放在现有工作表中, 选择【现有工作表】, 并在【位置】框内点击工作表的任意单元格 (不要覆盖数据)。点击【确定】。

第 3 步: 用鼠标右键单击数据透视表, 选择【数据透视表选项】, 在弹出的对话框中点击【显示】,并选中【经典数据透视表布局】, 然后点击【确定】。

第 4 步: 将数据透视的一个字段拖至 “行”的位置, 将另一个字段拖至 “列”的位置 (行列可以互换), 再将要计数的变量拖至 “值字段” 的位置, 即可生成需要的频数分布表。

按上述步骤生成的饮料类型的简单频数分布表如表 3-3 所示。生成的顾客性别与饮料类型的列联表如表 3-4 所示。

\begin{center}
\begin{tabular}{lc}
\hline
 & 饮料类型 \\
果汁 & 6 \\
矿泉水 & 10 \\
绿茶 & 11 \\
其他 & 8 \\
碳酸饮料 & 15 \\
\hline
总计 & 50 \\
\hline
\end{tabular}
\end{center}

表 3-4 不同饮料类型和顾客性别的频数分布表

\begin{center}
\begin{tabular}{|c|c|c|c|}
\hline
\multirow{2}{*}{饮料类型} & \multicolumn{2}{|c|}{顾客性别} & \multirow{2}{*}{总计} \\
\hline
 & 男 & 女 &  \\
\hline
果汁 & 1 & 5 & 6 \\
\hline
矿泉水 & 6 & 4 & 10 \\
\hline
绿茶 & 7 & 4 & 11 \\
\hline
其他 & 2 & 6 & 8 \\
\hline
碳酸饮料 & 6 & 9 & 15 \\
\hline
总计 & 22 & 28 & 50 \\
\hline
\end{tabular}
\end{center}

使用 SPSS 生成频数分布表时, 可以得到相应的描述统计量。使用 SPSS 生成频数分布表(包括交叉表)的操作步骤如下:

太生成一个分类变量的频数分布表

第 1 步:选择【分析】 $\rightarrow$ 【描述统计-频率】。

第 2 步: 将要生成频数分布表的变量选入【变量】, 点击【确定】。

(注: 若需要绘制图形, 点击【图表】, 可以绘制条形图、饼图等。)

太生成两个分类变量的列联表 (交叉频数分布表)

第 1 步: 选择【分析】 $\rightarrow$ 【描述统计-交又表】。

第 2 步: 将一个分类变量选入【行】, 将另一个分类变量选入【列】(行和列可以互换)。点击【确定】。

(注: 若需要对列联表进行描述性分析, 点击【单元格】。在【百分比】下选中需要计算的百分比, 如【行】【列】【总计】等; 若需要绘制图形, 点击【显示簇状条形图】。)

以例 3.2 为例, SPSS 生成的饮料类型简单频数表如表 3-5 所示; 顾客性别和饮料类型的列联表如表 3-6 所示。\\
表 3-5 饮料类型的频数分布

\begin{center}
\begin{tabular}{lcccc}
\hline
\multicolumn{1}{c}{饮料类型} & 频率 & 百分比 & 有效百分比 & 累积百分比 \\
\hline
果汁 & 6 & 12.0 & 12.0 & 12.0 \\
矿泉水 & 10 & 20.0 & 20.0 & 32.0 \\
绿茶 & 11 & 22.0 & 22.0 & 54.0 \\
其他 & 8 & 16.0 & 16.0 & 70.0 \\
碳酸饮料 & 15 & 30.0 & 30.0 & 100.0 \\
\hline
总计 & 50 & 100.0 & 100.0 &  \\
\hline
\end{tabular}
\end{center}

表 3-6 饮料类型和顾客性别的列联表 (交叉表)

\begin{center}
\begin{tabular}{lccccccc}
\hline
 & \multicolumn{6}{c}{} & \multicolumn{2}{c}{饮料类型} & \multirow{2}{*}{总计} \\
\cline { 3 - 6 }
 &  & 果汁 & 矿泉水 & 绿茶 & 其他 & 碳酸饮料 &  &  \\
\hline
\multirow{2}{*}{顾客性别} & 男 & 1 & 6 & 7 & 2 & 6 & 22 &  \\
 & 女 & 5 & 4 & 4 & 6 & 9 & 28 &  &  \\
\hline
总计 &  & 6 & 10 & 11 & 8 & 15 & 50 &  &  \\
\hline
\end{tabular}
\end{center}

表 3-5 和表 3-6 中的类别是按拼音字母升序排列的(也可以按降序排列)。表 3-5 除了给出频率外, 还给出了相应的百分比、有效百分比、累积百分比等。由于不存在缺失值, 表中的百分比和缺失百分比完全相同。

表 3-6 给出了按行和按列的总计数。对于列联表, 也可以分别计算出行百分比、列百分比、总计百分比等进行分析, 结果如表 3-7 所示。

表 3-7 顾客性别和饮料类型的列联表及其分析

\begin{center}
\begin{tabular}{|c|c|c|c|c|c|c|c|c|}
\hline
 &  &  & \multicolumn{5}{|c|}{饮料类型} & \multirow{2}{*}{总计} \\
\hline
 &  &  & 果汁 & 矿泉水 & 绿茶 & 其他 & 碳酸饮料 &  \\
\hline
\multirow{8}{*}{顾客性别} & \multirow{4}{*}{男} & 计数 & 1 & 6 & 7 & 2 & 6 & 22 \\
\hline
 &  & 占顾客性别的百分比 & $4.5 \%$ & $27.3 \%$ & $31.8 \%$ & $9.1 \%$ & $27.3 \%$ & $100.0 \%$ \\
\hline
 &  & 占饮料类型的百分比 & $16.7 \%$ & $60.0 \%$ & $63.6 \%$ & $25.0 \%$ & $40.0 \%$ & $44.0 \%$ \\
\hline
 &  & 占总计的百分比 & $2.0 \%$ & $12.0 \%$ & $14.0 \%$ & $4.0 \%$ & $12.0 \%$ & $44.0 \%$ \\
\hline
 & \multirow{4}{*}{女} & 计数 & 5 & 4 & 4 & 6 & 9 & 28 \\
\hline
 &  & 占顾客性别的百分比 & $17.9 \%$ & $14.3 \%$ & $14.3 \%$ & $21.4 \%$ & $32.1 \%$ & $100.0 \%$ \\
\hline
 &  & 占饮料类型的百分比 & $83.3 \%$ & $40.0 \%$ & $36.4 \%$ & $75.0 \%$ & $60.0 \%$ & $56.0 \%$ \\
\hline
 &  & 占总计的百分比 & $10.0 \%$ & $8.0 \%$ & $8.0 \%$ & $12.0 \%$ & $18.0 \%$ & $56.0 \%$ \\
\hline
\multirow{4}{*}{总计} &  & 计数 & 6 & 10 & 11 & 8 & 15 & 50 \\
\hline
 &  & 占顾客性别的百分比 & $12.0 \%$ & $20.0 \%$ & $22.0 \%$ & $16.0 \%$ & $30.0 \%$ & $100.0 \%$ \\
\hline
 &  & 占饮料类型的百分比 & $100.0 \%$ & $100.0 \%$ & $100.0 \%$ & $100.0 \%$ & $100.0 \%$ & $100.0 \%$ \\
\hline
 &  & 占总计的百分比 & $12.0 \%$ & $20.0 \%$ & $22.0 \%$ & $16.0 \%$ & $30.0 \%$ & $100.0 \%$ \\
\hline
\end{tabular}
\end{center}

\section*{3. 2.2 分类数据的图示}
上面介绍了如何建立频数分布表来反映分类数据的频数分布。如果用图形来显示频数\\
分布, 就会更形象和直观。一张好的统计图表, 往往胜过冗长的文字表述。绘制什么样的图形取决于数据的类型。对于分类数据或按不同类别记录的其他数据, 绘制的图形主要有条形图、帕累托图、饼图、环形图等。

\section*{1. 条形图和帕累托图}
条形图 (bar chart) 是用条形的高度或长短来表示数据多少的图形。条形图可以横置或纵置, 纵置时也可以称为柱形图 (column chart)。此外, 对于一个分类变量, 可以绘制简单条形图; 对于两个分类变量, 则可以绘制簇状条形图或堆积条形图。以例 3.2 为例,使用 Excel 绘制的饮料类型和顾客性别的简单条形图如图 3-5 所示; 绘制的饮料类型和顾客性别的簇状条形图和堆积条形图如图 3-6 所示。

帕累托图 (Pareto chart) 是以意大利经济学家帕累托 (V. Pareto) 的名字命名的。该图是按各类别出现的频数多少排序后绘制的条形图。通过对条形的排序, 容易看出哪类数据出现的多, 哪类数据出现的少。根据例 3.2 中的饮料类型绘制的帕累托图如图 3-7 所示。图中左侧的纵轴给出了频数, 右侧的纵轴给出了频数累积百分比。

\begin{center}
\end{center}

饮料类型

(a) 饮料类型

\begin{center}
\end{center}

顾客性别

(b) 顾客性别

图 3-5 饮料类型和顾客性别的条形图

\begin{center}
\end{center}

男轅女 (a) 簇状条形图

\begin{center}
\end{center}

■男 世女

(b) 堆积条形图

图 3-6 饮料类型和顾客性别的簇状条形图和堆积条形图

\begin{center}
\end{center}

图 3-7 不同饮料类型的帕累托图

\section*{2. 饼图和环形图}
饼图 (pie chart) 是用圆形及圆内扇形的度数来表示数值大小的图形, 它主要用于表示一个样本 (或总体) 中各组成部分的数据占全部数据的比例, 对于研究结构性问题十分有用。例如, 根据表 3-3 中不同饮料类型的频数分布数据, 绘制的饼图如图 3-8 所示。

\begin{center}
\end{center}

图 3-8 不同饮料类型构成的饼图

如果想比较男女顾客购买的饮料类型的构成状况, 可以绘制环形图。它用一个环表示一个类别的构成, 多个类别构成的多个环嵌套在一起, 主要用于展示两个或多个分类变量的构成。比如, 在男女分类的基础上又增加了饮料类型的分类。根据表 3-6 中的数据,使用 Excel 绘制的环形图如图 3-9 所示。其中, 里面的环表示男性, 外面的环表示女性。

\begin{center}
\end{center}

口果汁 ■矿泉水 口绿茶 口其他 碳酸饮料

图 3-9 按性别分类绘制的不同饮料类型构成的环形图

\section*{3.3}
数值数据的整理与展示上节介绍的分类数据的整理与图示方法也适用于数值数据。但数值数据还有一些特定的整理和图示方法, 它们并不适用于分类数据和顺序数据。

\subsection*{3.3. 1 数据分组}
数据分组是根据统计研究的需要, 将原始数据按照某种标准分成不同的组别, 分组后的数据称为分组数据(grouped data)。数据分组的主要目的是观察数据的分布特征。数据经分组后再计算出各组中数据出现的频数, 就形成了一张频数分布表。数据分组是将全部变量值依次划分为若干个区间, 并将一个区间的变量值作为一组。在数据分组中, 一个组的最小值称为下限值, 一个组的最大值称为上限值。

下面结合具体的例子说明分组的方法和频数分布表的生成过程。

\section*{例 3.3}
表 3-8 是某电脑公司连续 4 个月每天的销售量数据(单位:台)。试对数据进行分组。

表 3-8 某电脑公司连续 4 个月的销售量

\begin{center}
\begin{tabular}{llllllllll}
\hline
234 & 159 & 187 & 155 & 172 & 183 & 182 & 177 & 163 & 158 \\
143 & 198 & 141 & 167 & 194 & 225 & 177 & 189 & 196 & 203 \\
187 & 160 & 214 & 168 & 173 & 178 & 184 & 209 & 176 & 188 \\
\hline
\end{tabular}
\end{center}

\begin{center}
\begin{tabular}{llllllllll}
续表 & \multicolumn{10}{l}{} \\
\hline
161 & 152 & 149 & 211 & 196 & 234 & 185 & 189 & 196 & 206 &  \\
150 & 161 & 178 & 168 & 174 & 153 & 186 & 190 & 160 & 171 &  \\
228 & 162 & 223 & 170 & 165 & 179 & 186 & 175 & 197 & 208 &  \\
153 & 163 & 218 & 180 & 175 & 144 & 178 & 191 & 197 & 192 &  \\
166 & 196 & 179 & 171 & 233 & 179 & 187 & 173 & 174 & 210 &  \\
154 & 164 & 215 & 233 & 175 & 188 & 237 & 194 & 198 & 168 &  \\
174 & 226 & 180 & 172 & 190 & 172 & 187 & 189 & 200 & 211 &  \\
156 & 165 & 175 & 210 & 207 & 181 & 205 & 195 & 201 & 172 &  \\
203 & 165 & 196 & 172 & 176 & 182 & 188 & 195 & 202 & 213 &  \\
\hline
\end{tabular}
\end{center}

$\cdot$ $\cdot$解 分组和制作频数分布表的具体步骤如下:

第 1 步: 确定组数。数据分多少组合适呢? 一般与数据自身的特点及数据的多少有关。由于分组的目的之一是观察数据分布的特征, 因此组数的多少应适中。组数太少, 数据的分布就会过于集中, 组数太多, 数据的分布就会过于分散, 这些都不便于观察数据分布的特征和规律。组数的确定应以能够显示数据的分布特征和规律为目的。一般情况下,数据所分的组数不应少于 5 组且不多于 15 组, 即 $5 \leqslant K \leqslant 15$ 。实际应用时, 可根据数据的多少和特点及分析的要求来确定组数。本例中由于数据较多, 可分为 10 组。

第 2 步: 确定各组的组距。组距 (class width) 是一个组的上限与下限的差。组距可根据全部数据的最大值和最小值及所分的组数来确定, 即组距 $=$ (最大值一最小值) $\div$ 组数。例如, 对于本例数据, 最大值为 237 , 最小值为 141 , 则组距 $=(237-141) \div 10=$ 9.6。为便于计算, 组距宜取 5 或 10 的倍数, 而且第一组的下限应低于最小变量值, 最后一组的上限应高于最大变量值, 因此组距可取 10 。

第 3 步: 根据分组制作频数分布表。用 Excel 制作频数分布表的过程如下:

\section*{太用【直方图】命令制作数值数据的频数分布表}
使用 Excel【数据分析】工具中的【直方图】命令可生成数值数据的频数分布表。但需要注意的是, Excel 在制作频数分布表时, 每一组的频数包括一个组的上限值, 即 $a<x \leqslant b$ 。而实际分组时, 通常规定 “上组限不在内”, 即当相邻两组的上下限重叠时, 恰好等于某一组上限的变量值不算在本组内, 而计算在下一组内, 即 $x$ 满足 $a \leqslant x<b$ 。因此, 需要输入一列比上限值小的数作为【接收区域】。就例 3.3 而言, 分别输入 $149,159,169,179,189,199,209,219,229,239$ 作为【接收区域】,然后按下列步骤操作。

第 1 步:选择【数据】 $\rightarrow$ 【数据分析】 $\rightarrow$ 【直方图】, 单击【确定】。

第 2 步: 在【输入区域】方框内输入原始数据所在的区域; 在【接收区域】方框内输入上限值所在的区域; 在【输出区域】方框内输入结果输出的位置; 选择【图表输出】。单击【确定】。\\
将 Excel 生成的频数分布表进行适当的整理, 比如, 将 “接受” 修改为 “分组”, 将各组别依次修改为 $140 \sim 150,150 \sim 160, \cdots, 230 \sim 240$, 将 “其他” 修改为 “合计”, 将 “频率” 修改为 “天数”, 并计算出频率, 同时求出合计数, 结果如表 3-9 所示。

表 3-9 某电脑公司销售量的频数分布表

\begin{center}
\begin{tabular}{ccc}
\hline
按销售量分组 (台) & 频数 (天) & 频率 (\%) \\
\hline
$140 \sim 150$ & 4 & 3.33 \\
$150 \sim 160$ & 9 & 7.50 \\
$160 \sim 170$ & 16 & 13.33 \\
$170 \sim 180$ & 27 & 22.50 \\
$180 \sim 190$ & 20 & 16.67 \\
$190 \sim 200$ & 17 & 14.17 \\
$200 \sim 210$ & 10 & 8.33 \\
$210 \sim 220$ & 8 & 6.67 \\
$220 \sim 230$ & 4 & 3.33 \\
$230 \sim 240$ & 5 & 4.17 \\
合计 & 120 & 100 \\
\hline
\end{tabular}
\end{center}

在组距分组中, 如果全部数据中的最大值和最小值与其他数据相差悬殊, 为避免出现空白组(即没有变量值的组)或个别极端值被漏掉, 第一组和最后一组可以采取 “ $\times \times$ 以下” 及 “ $\times \times$ 以上” 这样的开口组。开口组通常以相邻组的组距作为其组距。例如, 在上面的 120 个数据中, 假定将最小值改为 102 , 最大值改为 265 , 采用上面的分组就会出现空白组, 这时可采用开口组, 如表 3-10 所示。

表 3-10 某电脑公司销售量的频数分布表 (开口组)

\begin{center}
\begin{tabular}{ccc}
\hline
按销售量分组 (台) & 频数 (天) & 频率 (\%) \\
\hline
150 以下 & 4 & 3.33 \\
$150 \sim 160$ & 9 & 7.50 \\
$160 \sim 170$ & 16 & 13.33 \\
$170 \sim 180$ & 27 & 22.50 \\
$180 \sim 190$ & 20 & 16.67 \\
$190 \sim 200$ & 17 & 14.17 \\
$200 \sim 210$ & 10 & 8.33 \\
$210 \sim 220$ & 8 & 6.67 \\
$220 \sim 230$ & 4 & 3.33 \\
230 及以上 & 5 & 4.17 \\
合计 & 120 & 100 \\
\hline
\end{tabular}
\end{center}

数据分组后掩盖了各组内的数据分布状况, 为反映各组数据的一般水平, 通常用组中值作为该组数据的一个代表值。组中值 (class midpoint) 是每一组中下限值与上限值中间\\
的值, 即组中值 $=($ 下限值 + 上限值) $/ 2$ 。

为了统计分析的需要, 有时需要观察某一数值以下或某一数值以上的频数或频率之和,这时可以计算出累积频数或累积频率。

\subsection*{3.3.2 数值数据的图示}
如果数据经过了分组, 则上面介绍的条形图、帕累托图、饼图、环形图等都适用于显示数值数据。此外, 数值数据还有以下一些图示方法, 这些方法并不适用于分类数据。

\section*{1. 直方图}
直方图 (histogram) 是用于展示数值数据分布的一种图形, 它是用矩形的宽度和高度 (即面积) 来表示频数分布的。绘制该图时, 在平面直角坐标系中, 用横轴表示数据分组,纵轴表示频数或频率, 这样, 一个组与相应的频数就形成了一个矩形, 将多个矩形并列在一起就是直方图。

以例 3.3 的数据, 用 Excel 绘制直方图时, 先将光标放在任意数据单元格, 然后点击【插人】 $\rightarrow$ 【插人统计图表】, 选择【直方图】, 即可绘制出直方图。根据需要再对直方图作必要的修改, 比如, 要添加每一组的频数标签, 点击任意一个条, 然后点击鼠标右键,并点击【添加数据标签】即可。结果如图 3-10 所示。

\begin{center}
\end{center}

图 3-10 销售量分布的直方图 (默认)

图 3-10 是使用 Excel 默认绘制的直方图, 其中, 第一组的 [141, 156] 表示该组含有数值 141 和 156 , 即包含下限值和上限值; (156, 171] 表示该组不包含下限值 156 , 但包含上限值 171, 其余依此类推。图 3-10 显示, 销售量的分布主要集中在 171 186 之间, 以此为中心两侧依次减少, 基本上呈对称分布, 但右边的尾部比左边的尾部稍长一\\
些, 表示销售额的分布有一定程度的右偏。

Excel 在绘制本例数据的直方图时, 默认将数据分成 7 组, 绘制出 7 个箱子(条)。根据需要, 可以对直方图进行修改。比如, 如果要将数据分成组距为 20 的组, 再绘制直方图, 可以双击分组标签, 在右侧弹出的【设置坐标轴格式】下点击【箱宽度】, 在后面写人箱宽度 (组距) 的值 20 即可。也可以点击【箱数】, 在后面写人要分的组数, 比如, 分成 10 组, 或分成组距为 5 的组, 等等。以例 3.3 的数据为例, 将数据分成组距为 10 的组, 绘制的直方图如图 3-11 所示。

\begin{center}
\end{center}

图 3-11 销售量分布的直方图 (组距为 10)

如果数据集中有远离其他值的极小值或极大值, 按相同的组距或箱数绘制的直方图就会出现空白组。假定例 3.3 的数据中最小值为 80 , 最大值为 260 , 这时再按组距为 10 或箱数为 10 绘制直方图, 就会出现空白组。为避免出现空白组, 可以在【设置坐标轴格式】下点击【溢出箱】(有极大值时) 或【下溢箱】(有极小值时), 并写人要截断的组的下限值和上限值即可。比如, 在【溢出箱】后写人 240 , 在【下溢箱】后写人 140 , 绘制的直方图如图 3-12 所示。

注意: 直方图与条形图不同。首先, 条形图中的每一矩形表示一个类别, 其宽度没有意义, 而直方图的宽度则表示各组的组距。其次, 由于分组数据具有连续性, 直方图的各矩形通常是连续排列, 而条形图则是分开排列。最后, 条形图主要用于展示分类数据, 而直方图则主要用于展示类别化的数值数据。

\section*{2. 箱形图}
箱形图也称箱线图, 它不仅可用于反映一组数据分布的特征, 比如, 分布是否对称等,

\begin{center}
\end{center}

图 3-12 销售量分布的直方图 (组距为 10 , 下溢箱为 150 , 溢出箱为 230 )

还可以对多组数据的分布特征进行比较, 这也是箱形图的主要用途。绘制箱形图的步骤大致如下:

首先, 找出一组数据的中位数 (median) 和两个四分位数(1) (quartiles), 并画出箱子。中位数是一组数据排序后处在 $50 \%$ 位置上的数值。四分位数是一组数据排序后处在 $25 \%$ 位置和 $75 \%$ 位置上的两个分位数值, 分别用 $Q_{25 \%}$ 和 $Q_{75 \%}$ 表示。 $Q_{75 \%}-Q_{25 \%}$ 称为四分位差或四分位距(quartile deviation), 用 $I Q R$ 表示。用两个四分位数画出箱子(四分位差的范围), 并画出中位数在箱子里面的位置。

其次,计算出内围栏和相邻值,并画出须线。内围栏(inter fence)是与 $Q_{25 \%}$ 和 $Q_{75 \%}$的距离等于 1.5 倍四分位差的两个点, 其中 $Q_{25 \%}-1.5 \times I Q R$ 称为下内围栏, $Q_{75 \%}+1.5 \times$ $I Q R$ 称为上内围栏。上下内围栏一般不在箱形图中显示, 只是作为确定离群点的界限 (2)。然后找出上下内围栏之间的最大值和最小值 (即非离群点的最大值和最小值), 称为相邻值 (adjacent value), 其中 $Q_{25 \%}-1.5 \times I Q R$ 范围内的最小值称为下相邻值, $Q_{75 \%}+1.5 \times$ $I Q R$ 范围内的最大值称为上相邻值。用直线将上下相邻值分别与箱子连接, 称为须线 (whiskers)。

最后, 找出离群点, 并在图中单独标出。离群点 (outlier) 是大于上内围栏或小于下内围栏的数值, 也称外部点 (outside value), 在图中用 “ $\bigcirc$ ”单独标出。

(1) 这些统计量将在第 4 章详细介绍。

(2) 也可以设定 3 倍的四分位差作为围栏, 称为外围栏 (outer fence), 其中 $Q_{25} \%-3 \times I Q R$ 称为下外围栏, $Q_{75 \%}+$ $3 \times I Q R$ 称为上外围栏。外围栏也不在箱线图中显示。在外围栏之外的数据也称为极值 (extreme)。Excel 默认根据内围栏确定相邻值。\\
箱形图的一般形式如图 3-13 所示。

\begin{center}
\end{center}

图 3-13 箱形图的示意图

通过箱形图的形状可以看出数据分布的特征。图 3-14 显示了几种不同的箱形图与其所对应的分布形状。

\begin{center}
\end{center}

(a) 对称分布

\begin{center}
\end{center}

(c) 右偏分布

\begin{center}
\end{center}

(b) 左偏分布

\begin{center}
\end{center}

(d) U形分布

图 3-14 不同分布对应的箱形图

下面通过一个例子说明用 Excel 绘制箱形图的方法。

\section*{例 3.4}
从某大学的 5 个学院中各随机抽取 30 名学生, 得到英语考试分数的数据如表 3-11 所示。绘制箱形图分析不同学院学生英语考试分数的分布特征。

表 3-11 5 个学院各 30 名学生的英语考试分数

\begin{center}
\begin{tabular}{ccccc}
\hline
经济学院 & 法学院 & 商学院 & 理学院 & 统计学院 \\
\hline
74 & 83 & 90 & 70 & 78 \\
\hline
\end{tabular}
\end{center}

\begin{center}
\begin{tabular}{|c|c|c|c|c|}
\hline
经济学院 & 法学院 & 商学院 & 理学院 & 统计学院 \\
\hline
77 & 81 & 95 & 73 & 74 \\
\hline
78 & 71 & 95 & 80 & 86 \\
\hline
84 & 68 & 91 & 75 & 66 \\
\hline
85 & 77 & 60 & 60 & 80 \\
\hline
72 & 71 & 87 & 75 & 91 \\
\hline
83 & 62 & 84 & 79 & 78 \\
\hline
92 & 68 & 81 & 69 & 79 \\
\hline
55 & 75 & 70 & 76 & 85 \\
\hline
72 & 78 & 92 & 75 & 76 \\
\hline
76 & 82 & 82 & 64 & 73 \\
\hline
76 & 78 & 83 & 77 & 91 \\
\hline
84 & 75 & 90 & 76 & 87 \\
\hline
78 & 76 & 96 & 78 & 78 \\
\hline
74 & 74 & 87 & 76 & 86 \\
\hline
85 & 70 & 88 & 65 & 71 \\
\hline
79 & 74 & 86 & 81 & 68 \\
\hline
81 & 79 & 91 & 79 & 74 \\
\hline
80 & 71 & 85 & 74 & 90 \\
\hline
58 & 73 & 86 & 66 & 76 \\
\hline
81 & 79 & 95 & 74 & 80 \\
\hline
80 & 82 & 82 & 73 & 77 \\
\hline
87 & 67 & 92 & 80 & 86 \\
\hline
81 & 76 & 78 & 76 & 75 \\
\hline
74 & 77 & 85 & 67 & 79 \\
\hline
85 & 76 & 93 & 83 & 72 \\
\hline
69 & 75 & 86 & 72 & 89 \\
\hline
77 & 66 & 78 & 73 & 84 \\
\hline
81 & 76 & 92 & 70 & 69 \\
\hline
79 & 69 & 83 & 90 & 84 \\
\hline
\end{tabular}
\end{center}

\begin{itemize}
  \item 解 用 Excel 绘制箱形图时, 先将光标放在任意数据单元格, 然后点击【插入】 $\rightarrow$ 【插入统计图表】, 选择【箱形图】, 即可绘制出箱形图。根据需要再对图形进行必要的修改,比如, 选择不同的箱形图式样、更改坐标轴刻度、添加坐标轴标题、添加箱形图的频数标签等, 如图 3-15 所示。
\end{itemize}

图 3-15 中, 在统计学院的箱形图中添加了数据标签, 其中显示了中位数 (78.5)、 $25 \%$ 位置上的分位数(74)、75\% 位置上的分位数(86)、上相邻值 (91)、下相邻值 (66), 在 “父” 位置上显示的是平均数 (79.4)。

图 3-15 显示, 英语分数的整体水平(中位数或平均数)最高的是商学院, 其次是经济学院和统计学院 (二者差异不大), 较低的是法学院和理学院 (二者差异不大)。从分布

\begin{center}
\end{center}

图 3-15 5 个学院各 30 名学生英语考试分数的箱形图

形状看, 除统计学院外, 其他 4 个学院的平均数都低于中位数, 表示英语分数的分布呈现一定的左偏分布, 其中, 经济学院的箱形图中出现了 2 个离群点, 商学院出现了 1 个离群点 (通过添加数据标签可观察其结果), 统计学院的分数则大致对称。

如果数值数据是在不同时间取得的, 即为时间序列数据, 则可以绘制线图。线图 (line plot)主要用于反映现象随时间变化的特征。

\section*{3. 散点图}
散点图 (scatter diagram) 是展示两个数值变量之间关系的图形, 它是横轴代表变量 $x$, 纵轴代表变量 $y$, 每组数据 $\left(x_{i}, y_{i}\right)$ 在坐标系中用一个点表示, $n$ 组数据在坐标系中形成的 $n$ 个点称为散点, 由坐标及散点形成的二维数据图就是散点图。

\section*{例 3.5}
随机抽取 40 个学生, 得到身高和体重的数据如表 3-12 所示。绘制散点图并分析身高和体重之间的关系。

表 3-1240 个学生身高和体重的数据

\begin{center}
\begin{tabular}{cc||cc}
\hline
身高 $(\mathrm{cm})$ & 体重 $(\mathrm{kg})$ & 身高 $(\mathrm{cm})$ & 体重 $(\mathrm{kg})$ \\
\hline
163.7 & 62.1 & 179.2 & 87.4 \\
171.8 & 68.2 & 177.8 & 74.0 \\
161.6 & 64.8 & 170.7 & 72.0 \\
186.0 & 83.6 & 150.1 & 52.2 \\
173.3 & 67.2 & 176.2 & 69.2 \\
161.8 & 57.9 & 169.4 & 68.5 \\
\hline
\end{tabular}
\end{center}

续表

\begin{center}
\begin{tabular}{cc||cc}
\hline
身高 $(\mathrm{cm})$ & 体重 $(\mathrm{kg})$ & 身高 $(\mathrm{cm})$ & 体重 $(\mathrm{kg})$ \\
\hline
174.9 & 73.7 & 168.4 & 57.7 \\
177.4 & 77.8 & 155.3 & 63.6 \\
175.8 & 72.1 & 165.2 & 64.9 \\
166.9 & 69.9 & 174.2 & 82.2 \\
185.1 & 82.1 & 183.6 & 81.3 \\
173.9 & 68.1 & 169.0 & 63.7 \\
163.8 & 64.7 & 173.9 & 74.2 \\
147.9 & 44.7 & 169.5 & 62.9 \\
181.2 & 84.1 & 156.2 & 50.7 \\
169.6 & 77.6 & 165.9 & 66.2 \\
169.8 & 66.0 & 166.1 & 62.7 \\
179.4 & 70.3 & 169.4 & 67.5 \\
178.2 & 77.4 & 181.0 & 77.2 \\
175.9 & 72.0 & 177.6 & 71.1 \\
\hline
\end{tabular}
\end{center}

・解 根据表 3-12 中的数据绘制的散点图如图 3-16所示。

\begin{center}
\end{center}

图 3-1640 个学生身高和体重的散点图

图 3-16 显示, 身高与体重之间具有明显的线性关系, 即随着身高的增加, 体重也随之增加。

\section*{4. 雷达图}
雷达图 (radar chart) 是显示多个变量的常用图示方法, 也称为蜘蛛图 (spider chart)。设有 $n$ 组样本 $s_{1}, s_{2}, \cdots, s_{n}$, 每个样本测得 $p$ 个变量 $x_{1}, x_{2}, \cdots, x_{p}$ 。要绘制\\
这 $p$ 个变量的雷达图, 具体做法是: 先画一个圆, 然后将圆 $p$ 等分, 得到 $p$ 个点, 令这 $p$个点分别对应 $p$ 个变量, 再将这 $p$ 个点与圆心连线, 得到 $p$ 个辐射状的半径, 这 $p$ 个半径分别作为 $p$ 个变量的坐标轴, 再将同一样本的值在 $p$ 个坐标上的点连线。这样, $n$ 个样本形成的 $n$ 个多边形就是雷达图。

雷达图在显示或对比各变量的数值总和时十分有用。假定各变量的取值具有相同的正负号, 则总的绝对值与图形所围成的区域成正比。此外, 利用雷达图可以研究多个样本之间的相似程度。

\section*{例 3.6}
表 3-13 是 2018 年北京、天津、上海和重庆的人均各项消费支出数据。绘制雷达图比较 4 个地区人均消费支出的特点和相似性。

表 3-132018 年北京、天津、上海和重庆的人均各项消费支出单位: 元

\begin{center}
\begin{tabular}{rrcrrcccr}
\hline
\multirow{2}{*}{地区} & 食品烟酒 & 衣着 & \multicolumn{1}{c}{居住} & \begin{tabular}{c}
生活用品 \\
及服务 \\
\end{tabular} & 交通通信 & \begin{tabular}{c}
教育文化 \\
娱乐 \\
\end{tabular} & 医疗保健 & \begin{tabular}{c}
其他用品 \\
及服务 \\
\end{tabular} \\
\hline
 & 8064.9 & 2175.5 & 14110.3 & 2371.9 & 4767.4 & 3999.4 & 3274.5 & 1078.6 \\
天津 & 8647.5 & 1990.0 & 6406.3 & 1818.4 & 4280.9 & 3186.6 & 2676.9 & 896.3 \\
上海 & 10728.2 & 2036.8 & 14208.5 & 2095.5 & 4881.2 & 5049.4 & 3070.2 & 1281.5 \\
重庆 & 6220.8 & 1454.5 & 3498.8 & 1338.9 & 2545.0 & 2087.8 & 1660.0 & 442.8 \\
\hline
\end{tabular}
\end{center}

解 根据表 3-13 的数据绘制的雷达图如图 3-17 所示。

\begin{center}
\end{center}

图 3-17 2018 年北京、天津、上海和重庆人均各项消费支出的雷达图

图 3-17 显示, 从各项支出金额看, 上海和北京的居住支出最多, 而天津和重庆的食品烟酒支出最多。从雷达图所围成的形状看, 4 个地区人均消费支出的结构十分相似。

\section*{3.4 合理使用图表}
一个精心设计的图表是展示数据的有效工具。上面介绍了用图表来展示统计数据的方法, 借助计算机可以很容易地绘制出漂亮的图表, 但需要注意的是, 初学者往往会在图表的修饰上花费太多的时间和精力, 这样做得不偿失, 也未必合理, 或许还会画蛇添足。

精心设计的图表可以准确表达数据所要传递的信息。设计图表时, 应绘制得尽可能简洁, 以清晰地显示数据、合理地表达统计目的为依据。使用图表的目的是让人更容易看懂和理解数据。合理使用图表要注意以下几点。

首先, 在制作图表时, 应避免一切不必要的修饰。过于花哨的修饰往往会使人注重图表本身, 而掩盖了图表所要表达的信息。

其次, 图形的比例应合理。一般而言, 一张图形大体上为 $4: 3$ 的一个矩形, 过长或过高的图形都有可能歪曲数据, 给人留下错误的印象。

最后, 图表应有编号和标题。编号一般使用阿拉伯数字, 如表 1、表 2 等。图表的标题应明确标出表中数据所属的时间(when)、地点(where)和内容(what), 即遵循通常所说的 $3 \mathrm{~W}$ 准则。表的标题通常放在表的上方; 图的标题可放在图的上方, 也可放在图的下方。

\section*{思考与练习}
\section*{思考题}
3.1 简述数值数据分组的步骤。

3.2 直方图与条形图有何区别?

3.3 饼图和环形图有什么不同?

3.4 使用图表时应注意哪几点?

\section*{练习题}
3.1 为评价家电行业售后服务的质量, 随机抽取了由 100 个家庭构成的一个样本。服务质量的等级分别表示为: A. 好; B. 较好; C. 一般; D. 较差; E. 差。调查结果如下:

\begin{center}
\begin{tabular}{llllllllll}
\hline
B & E & C & C & A & D & C & B & A & E \\
D & A & C & B & C & D & E & C & E & E \\
A & D & B & C & C & A & E & D & C & B \\
B & A & C & D & E & A & B & D & D & C \\
C & B & C & E & D & B & C & C & B & C \\
D & A & C & B & C & D & E & C & E & B \\
B & E & C & C & A & D & C & B & A & E \\
\hline
\end{tabular}
\end{center}

续表

\begin{center}
\begin{tabular}{llllllllll}
\hline
B & A & C & D & E & A & B & D & D & C \\
A & D & B & C & C & A & E & D & C & B \\
C & B & C & E & D & B & C & C & B & C \\
\hline
\end{tabular}
\end{center}

(1) 用 Excel 制作频数分布表。

(2)绘制一幅条形图,反映评价等级的分布。

(3) 绘制评价等级的帕累托图。

3.2 为了确定灯泡的使用寿命, 在一批灯泡中随机抽取 100 个进行测试, 所得结果 (单位: 小时)如下:

\begin{center}
\begin{tabular}{llllllllll}
\hline
700 & 716 & 728 & 719 & 685 & 709 & 691 & 684 & 705 & 718 \\
706 & 715 & 712 & 722 & 691 & 708 & 690 & 692 & 707 & 701 \\
708 & 729 & 694 & 681 & 695 & 685 & 706 & 661 & 735 & 665 \\
668 & 710 & 693 & 697 & 674 & 658 & 698 & 666 & 696 & 698 \\
706 & 692 & 691 & 747 & 699 & 682 & 698 & 700 & 710 & 722 \\
694 & 690 & 736 & 689 & 696 & 651 & 673 & 749 & 708 & 727 \\
688 & 689 & 683 & 685 & 702 & 741 & 698 & 713 & 676 & 702 \\
701 & 671 & 718 & 707 & 683 & 717 & 733 & 712 & 683 & 692 \\
693 & 697 & 664 & 681 & 721 & 720 & 677 & 679 & 695 & 691 \\
713 & 699 & 725 & 726 & 704 & 729 & 703 & 696 & 717 & 688 \\
\hline
\end{tabular}
\end{center}

(1) 以组距为 10 进行分组,生成频数分布表。

(2)绘制直方图,说明数据分布的特点。

3.3 一种袋装食品用生产线自动装填, 每袋重量大约为 50 克, 但由于某些原因, 每袋重量不会恰好是 50 克。随机抽取 100 袋食品, 测得的重量数据(单位: 克)如下:

\begin{center}
\begin{tabular}{llllllllll}
\hline
57 & 46 & 49 & 54 & 55 & 58 & 49 & 61 & 51 & 49 \\
51 & 60 & 52 & 54 & 51 & 55 & 60 & 56 & 47 & 47 \\
53 & 51 & 48 & 53 & 50 & 52 & 40 & 45 & 57 & 53 \\
52 & 51 & 46 & 48 & 47 & 53 & 47 & 53 & 44 & 47 \\
50 & 52 & 53 & 47 & 45 & 48 & 54 & 52 & 48 & 46 \\
49 & 52 & 59 & 53 & 50 & 43 & 53 & 46 & 57 & 49 \\
49 & 44 & 57 & 52 & 42 & 49 & 43 & 47 & 46 & 48 \\
51 & 59 & 45 & 45 & 46 & 52 & 55 & 47 & 49 & 50 \\
54 & 47 & 48 & 44 & 57 & 47 & 53 & 58 & 52 & 48 \\
55 & 53 & 57 & 49 & 56 & 56 & 57 & 53 & 41 & 48 \\
\hline
\end{tabular}
\end{center}

\section*{统计学 (第 8 版)}
(1) 生成频数分布表。

(2) 绘制频数分布的直方图。

(3) 说明数据分布的特征。

3.4 根据下面的数据绘制散点图, 说明两个变量之间的米系。

\begin{center}
\begin{tabular}{lllllll}
\hline
$x$ & 2 & 3 & 4 & 1 & 8 & 7 \\
\hline
$y$ & 25 & 25 & 20 & 30 & 16 & 18 \\
\hline
\end{tabular}
\end{center}

3.5 甲、乙两个班各有 40 名学生, 期末统计学考试成绩的分布如下:

\begin{center}
\begin{tabular}{ccc}
\hline
 &  & 人数 \\
考试成绩 & 甲班 & 乙班 \\
\cline { 3 - 4 }
 & 3 & 6 \\
优 & 6 & 15 \\
良 & 18 & 9 \\
中 & 9 & 8 \\
及格 & 4 & 2 \\
不及格 &  &  \\
\hline
\end{tabular}
\end{center}

(1)根据上面的数据, 画出两个班统计学考试成绩的对比条形图和环形图。

(2) 比较两个班统计学考试成绩分布的特点。

(3)画出雷达图,看看两个班统计学考试成绩的分布是否相似。

3.6 我国几个主要城市 1997 年各月份的平均相对湿度数据(单位: \% ) 如下表所示,绘制箱形图,并分析各城市平均相对湿度的分布特征。

\begin{center}
\begin{tabular}{|c|c|c|c|c|c|c|c|c|c|c|}
\hline
月份 & 北京 & 长春 & 南京 & 郑州 & 武汉 & 广州 & 成都 & 昆明 & 兰州 & 西安 \\
\hline
1 & 49 & 70 & 76 & 57 & 77 & 72 & 79 & 65 & 51 & 67 \\
\hline
2 & 41 & 68 & 71 & 57 & 75 & 80 & 83 & 65 & 41 & 67 \\
\hline
3 & 47 & 50 & 77 & 68 & 81 & 80 & 81 & 58 & 49 & 74 \\
\hline
4 & 50 & 39 & 72 & 67 & 75 & 84 & 79 & 61 & 46 & 70 \\
\hline
5 & 55 & 56 & 68 & 63 & 71 & 83 & 75 & 58 & 41 & 58 \\
\hline
6 & 57 & 54 & 73 & 57 & 74 & 87 & 82 & 72 & 43 & 42 \\
\hline
7 & 69 & 70 & 82 & 74 & 81 & 86 & 84 & 84 & 58 & 62 \\
\hline
8 & 74 & 79 & 82 & 71 & 73 & 84 & 78 & 74 & 57 & 55 \\
\hline
9 & 68 & 66 & 71 & 67 & 71 & 81 & 75 & 77 & 55 & 65 \\
\hline
10 & 47 & 59 & 75 & 53 & 72 & 80 & 78 & 76 & 45 & 65 \\
\hline
11 & 66 & 59 & 82 & 77 & 78 & 72 & 78 & 71 & 53 & 73 \\
\hline
12 & 56 & 57 & 82 & 65 & 82 & 75 & 82 & 71 & 52 & 72 \\
\hline
\end{tabular}
\end{center}

\begin{center}
\end{center}

\section*{鳗鱼的公共繁殖场所}
费希尔在 1952 年的一篇文章中举了一个例子, 说明如何由基本的描述统计量的知识引出一个重要的发现。

20 世纪早期, 哥本哈根卡尔堡实验室的施密特(J. Schmidt)发现不同地区所捕获的同种鱼类的脊椎骨和鳃腺的数量有很大不同, 甚至在同一海湾不同地点捕获的同种鱼类也是这样。然而, 鳗鱼的脊椎骨的数量却变化不大。施密特在从冰岛、亚速尔群岛和尼罗河等几乎分离地区的海域里所捕获的鳗鱼的样本中, 通过计算发现了几乎一样的均值和标准偏差值。

施密特由此推断: 各个不同海域内的鳗鱼是由海洋中某公共场所繁殖的。后来名为 “戴纳”(Dana)的科学考察船在一次远征中发现了这个场所。

资料来源: 劳. 统计与真理: 怎样运用偶然性. 北京: 科学出版社, 2004: 92 .\\
利用图表展示数据, 可以让我们对数据分布的形状和特征有一个大致的了解。但要全面把握数据分布的特征, 还需要找到反映数据分布特征的各个代表值。数据分布的特征可以从三个方面进行测度和描述: 一是分布的集中趋势, 反映各数据向其中心值靠拢或聚集的程度; 二是分布的离散程度, 反映各数据远离其中心值的趋势; 三是分布的形状, 反映数据分布的偏态和峰态。这三个方面分别反映了数据分布特征的不同侧面。本章将重点讨论分布特征值的计算方法、特点及其应用场合。

\section*{4.1 集中趋势的度量}
集中趋势 (central tendency) 是指一组数据向某一中心值靠拢的程度, 它反映了一组数据中心点的位置所在。测度集中趋势的统计量主要有平均数、中位数、四分位数、众数等。

\subsection*{4.1.1 平均数}
平均数 (mean) 也称均值, 它是一组数据相加后除以数据的个数得到的结果。平均数是度量数据集中趋势的常用统计量, 在参数估计和假设检验中经常用到。

设一组样本数据为 $x_{1}, x_{2}, \cdots, x_{n}$, 样本量 (样本数据的个数) 为 $n$, 则样本平均数用 $\bar{x}$ 表示, 计算公式为 ${ }^{(1)}$ :


\begin{equation*}
\bar{x}=\frac{x_{1}+x_{2}+\cdots+x_{n}}{n}=\frac{\sum_{i=1}^{n} x_{i}}{n} \tag{4.1}
\end{equation*}


式 (4.1) 也称为简单平均数 (simple mean)。

\section*{例 4.1}
随机抽取 30 个消费者, 得到他们每月网购金额的数据如表 4-1 所示。计算这 30 个人每月的平均网购金额。

\begin{center}
\begin{tabular}{cccccc}
\multicolumn{5}{c}{表 4-1 $\mathbf{3 0}$ 个人的每月网购金额} & 单位: 元 \\
\hline
429.0 & 671.2 & 622.4 & 678.7 & 393.2 & 331.3 \\
477.0 & 450.0 & 536.0 & 450.0 & 478.2 & 583.8 \\
655.9 & 373.5 & 540.1 & 303.6 & 397.4 & 515.3 \\
507.1 & 431.3 & 511.1 & 570.1 & 427.1 & 386.2 \\
512.9 & 455.1 & 465.4 & 452.7 & 437.5 & 625.4 \\
\hline
\end{tabular}
\end{center}

(1) 如果有总体的全部数据 $x_{1}, x_{2}, \cdots, x_{N}$, 总体平均数用 $\mu$ 表示, 其计算公式为: $\mu=\frac{x_{1}+x_{2}+\cdots+x_{N}}{N}=\frac{\sum_{i=1}^{N} x_{i}}{N}$ 。\\
一・解 根据式 (4.1) 有:

$$
\bar{x}=\frac{429.0+671.2+\cdots+437.5+625.4}{30}=\frac{14668.5}{30}=488.95
$$

如果样本数据被分成 $k$ 组, 各组的组中值 (一个组中的中间值, 是组的下限值与上限值的平均数) 分别用 $m_{1}, m_{2}, \cdots, m_{k}$ 表示, 各组的频数分别用 $f_{1}, f_{2}, \cdots, f_{k}$ 表示,则样本平均数的计算公式为:


\begin{equation*}
\bar{x}=\frac{m_{1} f_{1}+m_{2} f_{2}+\cdots+m_{k} f_{k}}{f_{1}+f_{2}+\cdots+f_{k}}=\frac{\sum_{i=1}^{k} m_{i} f_{i}}{n} \tag{4.2}
\end{equation*}


式 (4.2) 也称为加权平均数 (weighted mean) (1)

\section*{例 4.2}
假定将表 4-1 的数据分成组距为 50 的组, 分组结果如表 4-2 所示, 计算这 30 个消费者每月网购金额的平均数。

表 4-230 个消费者每月网购金额的分组表

\begin{center}
\begin{tabular}{cc}
\hline
分组 (元) & 人数 \\
\hline
$300 \sim 350$ & 2 \\
$350 \sim 400$ & 4 \\
$400 \sim 450$ & 6 \\
$450 \sim 500$ & 5 \\
$500 \sim 550$ & 6 \\
$550 \sim 600$ & 2 \\
$600 \sim 650$ & 2 \\
$650 \sim 700$ & 3 \\
\hline
\end{tabular}
\end{center}

——解 计算过程见表 4-3。

表 4-3 网购金额加权平均数计算表

\begin{center}
\begin{tabular}{cccc}
\hline
分组 (元) & 组中值 $\left(m_{i}\right)$ & 人数 $\left(f_{i}\right)$ & $m_{i} f_{i}$ \\
\hline
$300 \sim 350$ & 325 & 2 & 650 \\
$350 \sim 400$ & 375 & 4 & 1500 \\
$400 \sim 450$ & 425 & 6 & 2550 \\
\hline
\end{tabular}
\end{center}

(1) 如果总体数据被分成 $k$ 组, 各组的组中值分别用 $M_{1}, M_{2}, \cdots, M_{k}$ 表示, 各组数据出现的频数分别用 $f_{1}$, $f_{2}, \cdots, f_{k}$ 表示, 则总体加权平均数的计算公式为: $\mu=\frac{M_{1} f_{1}+M_{2} f_{2}+\cdots+M_{k} f_{k}}{f_{1}+f_{2}+\cdots+f_{k}}=\frac{\sum_{i=1}^{k} M_{i} f_{i}}{N}$ 。\\
续表

\begin{center}
\begin{tabular}{cccc}
\hline
分组 $($ 元) & 组中值 $\left(m_{i}\right)$ & 人数 $\left(f_{i}\right)$ & $m_{i} f_{i}$ \\
\hline
$450 \sim 500$ & 475 & 5 & 2375 \\
$500 \sim 550$ & 525 & 6 & 3150 \\
$550 \sim 600$ & 575 & 2 & 1150 \\
$600 \sim 650$ & 625 & 2 & 1250 \\
$650 \sim 700$ & 675 & 3 & 2025 \\
\hline
合计 & - & 30 & 14650 \\
\hline
\end{tabular}
\end{center}

根据式(4.2) 得:

$$
\bar{x}=\frac{\sum_{i=1}^{n} m_{i} f_{i}}{n}=\frac{14650}{30}=488.33
$$

\section*{4. 1.2 中位数和四分位数}
一组数据从小到大排序后, 找出排在某个位置上的数值, 并用该数值代表数据水平的高低, 则这些位置上的数值就是相应的分位数 (quantile), 包括中位数、四分位数等。

\section*{1. 中位数}
中位数 (median) 是一组数据排序后处在中间位置上的数值, 用 $M_{e}$ 表示。中位数是用一个点将全部数据等分成两部分, 每部分包含 $50 \%$ 的数据, 一部分数据比中位数大, 另一部分比中位数小。中位数是用中间位置上的值代表数据的水平, 其特点是不受极端值的影响, 在研究收人分配时很有用。

计算中位数时, 要先对 $n$ 个数据从小到大进行排序, 然后确定中位数的位置, 最后确定中位数的具体数值。如果位置是整数值, 中位数就是该位置所对应的数值; 如果位置是整数加 0.5 的数值, 中位数就是该位置两侧值的平均值。

设一组数据 $x_{1}, x_{2}, \cdots, x_{n}$ 从小到大排序后为 $x_{(1)}, x_{(2)}, \cdots, x_{(n)}$, 则中位数就是 $(n+1) / 2$ 位置上的值。计算公式为:

\[
M_{e}= \begin{cases}x\left(\frac{n+1}{2}\right) & n \text { 为奇数 }  \tag{4.3}\\ \frac{1}{2}\left\{x_{\left(\frac{n}{2}\right)}+x_{\left(\frac{n}{2}+1\right)}\right\} & n \text { 为偶数 }\end{cases}
\]

\section*{例 4.3}
沿用例 4.1。计算 30 个消费者每月网购金额的中位数。

解 首先, 将 30 个消费者每月网购金额数据排序, 然后确定中位数的位置: $(30+1) \div$\\
$2=15.5$, 中位数是排序后的第 15.5 位置上的数值, 即中位数在第 15 个数值 (465.4) 和第 16 个数值(477.0)中间(0.5)的位置上。因此 $(465.4+477.0) / 2=471.2$ 。

\section*{2. 四分位数}
四分位数 (quartile) 是一组数据排序后处于 $25 \%$ 和 $75 \%$ 位置上的数值。它是用 3 个点将全部数据等分为 4 部分, 其中每部分包含 $25 \%$ 的数据。很显然, 中间的四分位数就是中位数, 因此通常所说的四分位数是指处在 $25 \%$ 位置上和处在 $75 \%$ 位置上的两个数值。

与中位数的计算方法类似, 计算四分位数时, 首先对数据进行排序, 然后确定四分位数所在的位置, 该位置上的数值就是四分位数。与中位数不同的是, 四分位数位置的确定方法有多种 ${ }^{1}$, 每种方法得到的结果可能会有一定差异, 但差异不会很大 (一般相差不会超过一个位次)。由于不同软件的计算方法可能不一样, 因此, 对同一组数据使用不同软件得到的四分位数结果也可能会有所差异, 但不会影响分析的结论。

设 $25 \%$ 位置上的四分位数为 $Q_{25 \%}, 75 \%$ 位置上的四分位数为 $Q_{75} \%$, 用 SPSS 计算四分位数位置的公式为:


\begin{equation*}
Q_{25 \%} \text { 位置 }=\frac{n+1}{4}, \quad Q_{75 \%} \text { 位置 }=\frac{3(n+1)}{4} \tag{4.4}
\end{equation*}


如果位置是整数, 四分位数就是该位置对应的数值; 如果是在整数加 0.5 的位置上,则取该位置两侧数值的平均数; 如果是在整数加 0.25 或 0.75 的位置上, 则四分位数等于该位置前面的数值加上按比例分摊的位置两侧数值的差值。

\section*{例 4.4}
沿用例 4.1。计算 30 个消费者每月网购金额的四分位数。

\begin{itemize}
  \item 解 先对 $n$ 个数据从小到大排序, 然后计算出四分位数的位置:
\end{itemize}

$$
Q_{25 \%} \text { 位置 }=\frac{30+1}{4}=7.75, \quad Q_{75 \%} \text { 位置 }=\frac{3 \times(30+1)}{4}=23.25
$$

$Q_{25 \%}$ 在第 7 个数值 (427.1) 和第 8 个数值 (429.0) 之间 0.75 的位置上:

$$
Q_{25 \%}=427.1+0.75 \times(429.0-427.1)=428.525
$$

$Q_{75 \%}$ 在第 23 个数值(540.1)和第 24 个数值(570.1)之间 0.25 的位置上:

$$
Q_{25 \%}=540.1+0.25 \times(570.1-540.1)=547.6
$$

由于 $Q_{25 \%}$ 和 $Q_{75 \%}$ 之间大约包含了 $50 \%$ 的数据,就上面 30 个消费者每月网购金额而言, 可以说大约有一半人的每月网购金额在 428.525 547.6 元之间。
\footnotetext{(1) Excel 四分位数位置的计算公式为: $Q_{25 \%}$ 位置 $=\frac{n+3}{4}, Q_{75 \%}$ 位置 $=\frac{3 n+1}{4}$ 。
}

\section*{4. 1.3 众数}
除平均数、中位数和四分位数外, 有时候也会使用众数作为数据集中趋势的度量。众数 (mode) 是一组数据中出现频数最多的数值, 用 $M_{o}$ 表示。一般情况下, 只有在数据量较大时众数才有意义。从分布的角度看, 众数是一组数据分布的峰值点所对应的数值。如果数据的分布没有明显的峰值, 众数可能不存在; 如果有两个或多个峰值, 也可以有两个或多个众数。就例 4.1 而言, 出现金额最多的是 450 , 因此 $M_{o}=450$ 。

\section*{4. 1.4 几何平均数}
几何平均数 (geometric mean) 是 $n$ 个变量值乘积的 $n$ 次方根, 用 $G$ 表示。计算公式为 (1):


\begin{equation*}
G=\sqrt[n]{x_{1} \times x_{2} \times \cdots \times x_{n}}=\sqrt[n]{\prod_{i=1}^{n} x_{i}} \tag{4.5}
\end{equation*}


式中, $\Pi$ 为连乘符号。

几何平均数是适用于特殊数据的一种平均数, 它主要用于计算平均比率。当所掌握的变量值本身是比率形式时, 采用几何平均法计算平均比率更为合理。在实际应用中, 几何平均数主要用于计算现象的平均增长率。

\section*{例 4.5}
一位投资者持有一种股票,连续 4 年的收益率分别为 $4.5 \%, 2.1 \%, 25.5 \%, 1.9 \%$ 。计算该投资者在这 4 年内的平均收益率。

$\cdot$解 设平均收益率为 $\bar{G}$ ,有

$$
\begin{aligned}
\bar{G} & =\sqrt[n]{\prod_{i=1}^{n} x_{i}}-1 \\
& =\sqrt[4]{104.5 \% \times 102.1 \% \times 125.5 \% \times 101.9 \%-1} \\
& =8.0787 \%
\end{aligned}
$$

即该投资者的年均收益率为 $8.0787 \%$ 。

假定该投资者最初投人 10000 元, 按各年的几何平均收益率计算, 第 4 年的本利总和应为:

$$
\begin{aligned}
& 10000 \times 104.5 \% \times 102.1 \% \times 125.5 \% \times 101.9 \% \\
& =10000 \times(108.0787 \%)^{4} \\
& =13644.57(\text { 元) }
\end{aligned}
$$

如果按算术平均计算, 平均收益率则为:

$$
\bar{G}=(4.5 \%+2.1 \%+25.5 \%+1.9 \%) \div 4=8.5 \%
$$

(1) 当数据中出现零值或负值时, 不宜计算几何平均数。\\
按算术平均收益率计算, 该投资者第 4 年的本利总和应为:

$$
10000 \times(108.5 \%)^{4}=13858.59 \text { (元) }
$$

二者相差 214.02 元, 而这部分收益投资者并没有获得。这说明, 对于比率数据的平均, 采用几何平均要比算术平均更合理。从下面的分析中可以更清楚地看出这一点。设开始的数值为 $y_{0}$, 逐年增长率为 $G_{1}, G_{2}, \cdots, G_{n}$, 则第 $n$ 年的数值为:


\begin{equation*}
y_{n}=y_{0}\left(1+G_{1}\right)\left(1+G_{2}\right) \cdots\left(1+G_{n}\right)=y_{0} \prod_{i=1}^{n}\left(1+G_{i}\right) \tag{4.6}
\end{equation*}


从 $y_{0}$ 到 $y_{n}$ 有 $n$ 年, 每年的增长率都相同, 这个增长率 $G$ 就是平均增长率 $\bar{G}$, 即

\begin{center}
\end{center}


\begin{align*}
& (1+G) n=\prod_{i=1}^{n}\left(1+G_{i}\right)  \tag{4.7}\\
& \bar{G}=\sqrt[n]{\prod_{i=1}^{n}\left(1+G_{i}\right)}-1 \tag{4.8}
\end{align*}


当所平均的各比率数值差别不大时, 算术平均和几何平均的结果相差不大; 而当各比率数值相差较大时, 二者的差别就很明显。

\section*{4. 1.5 众数、中位数和平均数的比较}
众数、中位数和平均数是集中趋势的三个主要测度值, 它们具有不同的特点和应用场合。

从分布的角度看, 众数始终是一组数据分布的最高峰值, 中位数是处于一组数据中间位置上的值, 而平均数则是全部数据的算术平均。因此, 对于具有单峰分布的大多数数据而言, 众数、中位数和平均数之间具有以下关系: 如果数据的分布是对称的, 众数 $\left(M_{o}\right)$ 、中位数 $\left(M_{e}\right)$ 和平均数 $(\bar{x})$ 必定相等, 即 $M_{o}=M_{e}=\bar{x}$; 如果数据是左偏分布,说明数据存在极小值, 必然拉动平均数向极小值一方靠, 而众数和中位数由于是位置代表值, 不受极值的影响, 因此三者之间的关系表现为: $\bar{x}<M_{e}<M_{o}$; 如果数据是右偏分布,说明数据存在极大值, 必然拉动平均数向极大值一方靠, 因此 $M_{o}<M_{e}<\bar{x}$ 。

掌握众数、中位数和平均数的特点, 有助于在实际应用中选择合理的测度值来描述数据的集中趋势。

众数是一组数据分布的峰值, 不受极端值的影响。其缺点是具有不唯一性, 一组数据可能有一个众数, 也可能有两个或多个众数, 还可能没有众数。众数只有在数据量较多时才有意义, 当数据量较少时, 不宜使用众数。众数适合作为分类数据的集中趋势测度值。

中位数是一组数据中间位置上的值, 不受数据极端值的影响。当一组数据的分布偏斜程度较大时, 使用中位数也许是一个不错的选择。中位数适合作为顺序数据的集中趋势测度值。

平均数是针对数值数据计算的, 而且利用了全部数据信息, 它是应用最广泛的集中趋势测度值。当数据呈对称分布或接近对称分布时, 三个代表值相等或接近相等, 这时应选\\
择平均数作为集中趋势的代表值。平均数的主要缺点是易受数据极端值的影响, 对于偏态分布的数据, 平均数的代表性较差。因此, 当数据为偏态分布, 特别是偏斜程度较大时,可以考虑选择中位数或众数, 这时它们的代表性要比平均数好。

\section*{4.2 离散程度的度量}
描述样本数据离散程度的统计量主要有全距、四分位距、方差、标准差以及测度相对离散程度的离散系数等。

\section*{4. 2.1 全距和四分位距}
\begin{enumerate}
  \item 全距
\end{enumerate}

全距 (range) 是一组数据的最大值与最小值之差, 也称极差, 用 $R$ 表示。计算公式为:


\begin{equation*}
R=\max (x)-\min (x) \tag{4.9}
\end{equation*}


例如, 根据例 4.1 中的数据, 计算 30 个消费者每月网购金额的全距为: $R=678.7-$ 303. $6=375.1$ 。由于全距只是利用了一组数据两端的信息, 容易受极端值的影响, 不能全面反映数据的差异状况。虽然全距在实际中很少单独使用, 但它总是作为分析数据离散程度的一个参考值。

\section*{2. 四分位距}
四分位距(inter-quartile range) 是一组数据 $75 \%$ 位置上的四分位数与 $25 \%$ 位置上的四分位数之差, 也称四分位差, 用 $I Q R$ 表示。计算公式为:


\begin{equation*}
I Q R=Q_{75 \%}-Q_{25 \%} \tag{4.10}
\end{equation*}


四分位距反映了中间 $50 \%$ 数据的离散程度:其数值越小, 说明中间的数据越集中, 数值越大, 说明中间的数据越分散。四分位距不受极值的影响。此外, 由于中位数处于数据的中间位置, 因此, 四分位距的大小在一定程度上也说明了中位数对一组数据的代表程度。例如, 根据例 4.1 计算的 30 个消费者每月网购金额的四分位数, $I Q R=547.6-$ $428.525=119.075$ 。

\section*{4. 2.2 方差和标准差}
如果考虑每个数据 $x_{i}$ 与其平均数 $\bar{x}$ 之间的差异, 以此作为一组数据离散程度的度量,结果就要比全距和四分位距更为全面和准确。这就需要求出每个数据 $x_{i}$ 与其平均数 $\bar{x}$ 离差的平均数。但由于 $\left(x_{i}-\bar{x}\right)$ 之和等于 0 , 需要进行一定的处理。一种方法是将离差取绝对值, 求和后再平均, 这一结果称为平均差 (mean deviation) 或称平均绝对离差 (mean absolute deviation); 另一种方法是将离差平方后再求平均数, 这一结果称为方差 (vari-\\
ance)。方差开方后的结果称为标准差 (standard deviation)。方差(或标准差)是实际中应用最广泛的测度数据离散程度的统计量。

设样本方差为 $s^{2}$, 根据原始数据计算样本方差的公式为:


\begin{equation*}
s^{2}=\frac{\sum_{i=1}^{n}\left(x_{i}-\bar{x}\right)^{2}}{n-1} \tag{4.11}
\end{equation*}


样本标准差的计算公式为:


\begin{equation*}
s=\sqrt{\frac{\sum_{i=1}^{n}\left(x_{i}-\bar{x}\right)^{2}}{n-1}} \tag{4.12}
\end{equation*}


如果原始数据被分成 $k$ 组, 各组的组中值分别为 $m_{1}, m_{2}, \cdots, m_{k}$, 各组的频数分别为 $f_{1}, f_{2}, \cdots, f_{k}$, 则加权样本方差的计算公式为(1):


\begin{equation*}
s^{2}=\frac{\sum_{i=1}^{k}\left(m_{i}-\bar{x}\right)^{2} f_{i}}{n-1} \tag{4.13}
\end{equation*}


加权样本标准差的计算公式为:


\begin{equation*}
s=\sqrt{\frac{\sum_{i=1}^{k}\left(m_{i}-\bar{x}\right)^{2} f_{i}}{n-1}} \tag{4.14}
\end{equation*}


与方差不同的是, 标准差具有量纲, 它与原始数据的计量单位相同, 其实际意义要比方差清楚。因此,在对实际问题进行分析时更多地使用标准差。

\section*{例 4.6}
沿用例 4.1。计算 30 个消费者每月网购金额的方差和标准差。

・解 根据式 (4.11) 得:

$$
\begin{aligned}
s^{2} & =\frac{(429.0-488.95)^{2}+(671.2-488.95)^{2}+\cdots+(625.4-488.95)^{2}}{30-1} \\
& =9529.6812
\end{aligned}
$$

标准差为 $\sqrt{9529.6812}=97.62$ 。

(1) 对于总体的 $N$ 个数据, 总体方差 (population variance) 用 $\sigma^{2}$ 表示, 计算公式为 $\sigma^{2}=\frac{\sum_{i=1}^{N}\left(x_{i}-\mu\right)^{2}}{N}$ 。对于分组数据, 总体加权方差的计算公式为 $\sigma^{2}=\frac{\sum_{i=1}^{k}\left(M_{i}-\mu\right)^{2} f_{i}}{N}$ 。开平方后即得到总体的加权标准差。注意: 总体方差通常是不知道的, 都是用样本方差 $s^{2}$ 来推断的。

\section*{例 4.7}
沿用例 4.2。根据表 4-2 的分组数据, 计算网购金额的标准差。

解 计算过程见表 4-4。

表 4-4 30 个消费者每月网购金额分组数据的加权标准差计算表

\begin{center}
\begin{tabular}{cccrc}
\hline
网购金额分组 (元) & 组中值 $\left(m_{i}\right)$ & 人数 $\left(f_{i}\right)$ & $\left(m_{i}-\bar{x}\right)^{2}$ & $\left(m_{i}-\bar{x}\right)^{2} f_{i}$ \\
\hline
$300 \sim 350$ & 325 & 2 & 26676.69 & 53353.38 \\
$350 \sim 400$ & 375 & 4 & 12843.69 & 51374.76 \\
$400 \sim 450$ & 425 & 6 & 4010.69 & 24064.13 \\
$450 \sim 500$ & 475 & 5 & 177.69 & 888.44 \\
$500 \sim 550$ & 525 & 6 & 1344.69 & 8068.13 \\
$550 \sim 600$ & 575 & 2 & 7511.69 & 15023.38 \\
$600 \sim 650$ & 625 & 2 & 18678.69 & 37357.38 \\
$650 \sim 700$ & 675 & 3 & 34845.69 & 104537.07 \\
\hline
合计 & - & 30.00 & 106089.51 & 294666.67 \\
\hline
\end{tabular}
\end{center}

根据式(4.14) 得:

$$
s=\sqrt{\frac{\sum_{i=1}^{k}\left(m_{i}-\bar{x}\right)^{2} f_{i}}{n-1}}=\sqrt{\frac{294666.67}{30-1}}=100.80
$$

\subsection*{4.2. 3 离散系数}
标准差是反映数据离散程度的绝对值, 其数值的大小受原始数据取值大小的影响, 数据的观测值越大, 标准差的值通常也就越大。此外, 标准差与原始数据的计量单位相同,采用不同计量单位计量的数据, 其标准差的值也就不同。因此, 对于不同组别的数据, 如果原始数据的观测值相差较大或计量单位不同, 就不能用标准差直接比较其离散程度, 这时需要计算离散系数。

离散系数 (coefficient of variation, CV) 也称变异系数, 它是一组数据的标准差与其相应的平均数之比。由于离散系数消除了数据取值大小和计量单位对标准差的影响, 因而可以反映一组数据的相对离散程度。其计算公式为:


\begin{equation*}
C V=\frac{s}{\bar{x}} \tag{4.15}
\end{equation*}


离散系数主要用于比较不同样本数据的离散程度。离散系数大说明数据的相对离散程度也大, 离散系数小说明数据的相对离散程度也小(1)。
\footnotetext{(1) 当平均数接近 0 时, 离散系数的值趋于无穷大, 此时必须慎重解释。
}

\section*{例 4.8}
在奥运会女子 10 米气手枪比赛中, 每个运动员首先进行每组 10 枪共 4 组的预赛, 然后根据预赛总成绩确定进入决赛的 8 名运动员。决赛时 8 名运动员再进行 10 枪射击, 将预赛成绩加上决赛成绩来确定最后的名次。在 2008 年 8 月 10 日举行的第 29 届北京奥运会女子 10 米气手枪决赛中, 进入决赛的 8 名运动员的预赛成绩和最后 10 枪的决赛成绩如表 4-5 所示。评价哪名运动员的发挥更稳定。

表 4-58 名运动员 10 米气手枪决赛的成绩

\begin{center}
\begin{tabular}{|c|c|c|c|c|c|c|c|c|c|c|c|}
\hline
姓名 & 国家 & \begin{tabular}{l}
预赛 \\
成绩 \\
\end{tabular} &  &  & 决赛 & 10 枪 & 成绩 & (环) &  &  &  \\
\hline
纳塔利娅 - 帕杰琳娜 & 俄罗斯 & 391 & $10.0 \quad 8.5$ & 10.0 & 10.2 & 10.6 & 10.5 & 9.8 & 9.7 & 9. 5 & 9. 3 \\
\hline
郭文珺 & 中国 & 390 & 10.010 .5 & 10.4 & 10.4 & 10.1 & 10.3 & 9. 4 & 10.7 & 10.8 & 9.7 \\
\hline
卓格巴德拉赫・蒙赫珠勒 & 蒙古国 & 387 & 9.310 .0 & 8.7 & 8.3 & 9.2 & 9.5 & 8.5 & 10.7 & 9.2 & 9.2 \\
\hline
妮诺・萨卢克瓦泽 & 格鲁吉亚 & 386 & 9.810 .3 & 10.0 & 9.5 & 10.2 & 10.7 & 10.4 & 10.6 & 9. 1 & 10.8 \\
\hline
维多利亚・柴卡 & 白俄罗斯 & 384 & $9.3 \quad 9.4$ & 10.4 & 10.1 & 10.2 & 10.5 & 9.2 & 10.5 & 9. 8 & 8. 6 \\
\hline
莱万多夫斯卡 $\cdot$ 萨贡 & 波兰 & 384 & 8.110 .3 & 9.2 & 9. 9 & 9. 8 & 10.4 & 9.9 & 9.4 & 10.7 & 9. 6 \\
\hline
亚斯娜 - 舍卡里奇 & 塞尔维亚 & 384 & 10.29 .6 & 9.9 & 9.9 & 9. 3 & 9.1 & 9.7 & 10.0 & 9.3 & 9. 9 \\
\hline
米拉・内万苏 & 芬兰 & 384 & $8.7 \quad 9.3$ & 9.2 & 10.3 & 9.8 & 10.0 & 9. 7 & 9.9 & 9. 9 & 9. 7 \\
\hline
\end{tabular}
\end{center}

解 如果各运动员决赛 10 枪的平均成绩差异不大, 可以直接比较标准差的大小, 否则需要计算离散系数。8 名运动员最后 10 枪决赛的平均成绩、标准差和离散系数如表 4-6 所示。

表 4-6 8 名运动员 10 枪决赛的平均成绩、标准差和离散系数

\begin{center}
\begin{tabular}{lcccc}
\hline
\multicolumn{1}{c}{运动员} & 国家 & 平均成绩 & 标准差 & 离散系数 \\
\hline
纳塔利娅・帕杰琳娜 & 俄罗斯 & 9.81 & 0.6154 & 0.0627 \\
郭文珺 & 中国 & 10.23 & 0.4373 & 0.0427 \\
卓格巴德拉赫・蒙赫珠勒 & 蒙古国 & 9.26 & 0.7074 & 0.0764 \\
妮诺・萨卢克瓦泽 & 格鲁吉亚 & 10.14 & 0.5461 & 0.0539 \\
维多利亚・柴卡 & 白俄罗斯 & 9.8 & 0.6498 & 0.0663 \\
莱万多夫斯卡・萨贡 & 波兰 & 9.73 & 0.7334 & 0.0754 \\
亚斯娜・舍卡里奇 & 塞尔维亚 & 9.69 & 0.3573 & 0.0369 \\
米拉・内万苏 & 芬兰 & 9.65 & 0.4625 & 0.0479 \\
\hline
\end{tabular}
\end{center}

从离散系数可以看出, 在最后 10 枪的决赛中, 发挥比较稳定的运动员是塞尔维亚的亚斯娜・舍卡里奇和中国的郭文珺, 发挥不稳定的运动员是蒙古国的卓格巴德拉赫・蒙赫珠勒和波兰的莱万多夫斯卡・萨贡。

\section*{4. 2.4 标准分数}
有了平均数和标准差之后, 可以计算一组数据中每个数值的标准分数 (standard score)。它是某个数据与其平均数的离差除以标准差后的值。设样本数据的标准分数为 $z$,则有:


\begin{equation*}
z_{i}=\frac{x_{i}-\bar{x}}{s} \tag{4.16}
\end{equation*}


标准分数可以测度每个数值在该组数据中的相对位置, 并可以用来判断一组数据是否有离群点。比如, 全班的平均考试分数为 80 分, 标准差为 10 分, 而你的考试分数是 90 分, 距离平均分数有多远? 显然是 1 个标准差的距离。这里的 1 就是你考试成绩的标准分数。标准分数说的是某个数据与平均数相比相差多少个标准差。

将一组数据化为标准化得分的过程称为数据的标准化。式 (4.16) 也就是统计上常用的标准化公式, 在对多个具有不同量纲的变量进行处理时, 常常需要对各变量的数据进行标准化处理, 也就是把一组数据转化成具有平均数为 0 、标准差为 1 的新的数据。实际上,标准分数只是将原始数据进行了线性变换, 它并没有改变某个数值在该组数据中的位置,也没有改变该组数据分布的形状。

\section*{例 4.9}
沿用例 4.1。计算 30 个消费者每月网购金额的标准分数。

\begin{itemize}
  \item 解 根据上面的计算结果, $\bar{x}=488.95, s=97.62$ 。以第 1 个消费者的标准分数为例,由式 (4.16) 得:
\end{itemize}

$$
z=\frac{429.0-488.95}{97.62}=-0.6141
$$

结果表示, 第 1 个消费者的每月网购金额比平均每月网购金额低 0.6141 个标准差。

使用 Excel【STANDARDIZE】函数可计算标准分数,操作步骤如下所示:

\section*{太用【STANDARDIZE】函数计算标准分数的操作步骤}
第 1 步: 将光标放在任意空白单元格。然后点击【公式】, 点击插入函数【fx】。

第 2 步: 在【选择类别】中选择【统计】, 并在【选择函数】中点击【STANDARDIZE】, 单击

\section*{【确定】。}
第 3 步: 在【X】输入要计算标准分数的原始数据(最好是点击原始数据所在的单元格, 以方便复制得到多个数据的标准分数); 在【Mean】框后输入该组数据的平均数; 在【Standard\_ dev】框后输入该组数据的标准差。单击【确定】, 即可得到该数据的标准分数(要得到多个数据的标准分数, 向下复制该单元格即可)。

按上述步骤得到的标准分数如表 4-7 所示。\\
表 4-7 30 个消费者每月网购金额的标准分数

\begin{center}
\begin{tabular}{cr||cr||cr}
\hline
网购金额 & 标准分数 & 网购金额 & 标准分数 & 网购金额 & 标准分数 \\
\hline
429.0 & -0.6141 & 622.4 & 1.3670 & 393.2 & -0.9808 \\
477.0 & -0.1224 & 536.0 & 0.4820 & 478.2 & -0.1101 \\
655.9 & 1.7102 & 540.1 & 0.5240 & 397.4 & -0.9378 \\
507.1 & 0.1859 & 511.1 & 0.2269 & 427.1 & -0.6336 \\
512.9 & 0.2453 & 465.4 & -0.2412 & 437.5 & -0.5270 \\
671.2 & 1.8669 & 678.7 & 1.9438 & 331.3 & -1.6149 \\
450.0 & -0.3990 & 450.0 & -0.3990 & 583.8 & 0.9716 \\
373.5 & -1.1826 & 303.6 & -1.8987 & 515.3 & 0.2699 \\
431.3 & -0.5906 & 570.1 & 0.8313 & 386.2 & -1.0526 \\
455.1 & -0.3468 & 452.7 & -0.3713 & 625.4 & 1.3978 \\
\hline
\end{tabular}
\end{center}

根据标准分数, 可以判断一组数据中是否存在离群点 (outlier)。经验表明: 当一组数据对称分布时, 约有 $68 \%$ 的数据在平均数加减 1 个标准差的范围之内; 约有 $95 \%$ 的数据在平均数加减 2 个标准差的范围之内; 约有 $99 \%$ 的数据在平均数加减 3 个标准差的范围之内。可以想象, 一组数据中低于或高于平均数 3 个标准差之外的数值是很少的, 也就是说, 在平均数加减 3 个标准差的范围内几乎包含了全部数据, 而在 3 个标准差之外的数据在统计上也称为离群点。例如, 由表 4-6 可知, 30 个消费者的每月网购金额都在平均数加减 3 个标准差的范围内 (标准分数的绝对值均小于 3 ), 没有离群点。

经验法则适合对称分布的数据。如果一组数据不是对称分布, 经验法则就不再适用,这时可使用切比雪夫不等式 (Chebyshev's inequality), 它对任何分布形态的数据都适用。切比雪夫不等式提供的是 “下界”, 也就是 “所占比例至少是多少”, 对于任意分布形态的数据, 根据切比雪夫不等式, 至少有 $\left(1-1 / k^{2}\right)$ 的数据落在 $\pm k$ 个标准差之内。其中 $k$ 是大于 1 的任意值, 但不一定是整数。对于 $k=2,3,4$, 该不等式的含义是:

\begin{itemize}
  \item 至少有 $75 \%$ 的数据在平均数 $\pm 2$ 个标准差的范围之内。
  \item 至少有 $89 \%$ 的数据在平均数 $\pm 3$ 个标准差的范围之内。
  \item 至少有 $94 \%$ 的数据在平均数 $\pm 4$ 个标准差的范围之内。
\end{itemize}

\section*{4.3 分布形状的度量}
利用直方图可以看出数据的分布是否对称。对于不对称的分布, 要想知道不对称程度则需要计算相应的描述统计量。偏度系数和峰度系数就是对分布不对称程度和峰值高低的一种度量。

\subsection*{4.3. 1 偏度系数}
偏度 (skewness) 是指数据分布的不对称性, 这一概念由统计学家皮尔逊(K. Pearson)\\
于 1895 年首次提出。测度数据分布不对称性的统计量称为偏度系数 (coefficient of skew ness ), 记为 $S K$ 。根据原始数据计算偏度系数时, 通常采用下面的公式:


\begin{equation*}
S K=\frac{n}{(n-1)(n-2)} \sum\left(\frac{x_{i}-\bar{x}}{s}\right)^{3} \tag{4.17}
\end{equation*}


当数据对称分布时, 偏度系数等于 0 。偏度系数越接近 0 , 偏斜程度就越低, 就越接近对称分布。如果偏度系数明显不等于 0 , 表示分布是非对称的。若偏度系数大于 1 或小于 -1 , 视为严重偏斜分布; 若偏度系数在 $0.5 \sim 1$ 或 $-1 \sim-0.5$ 之间, 视为中等偏斜分布; 偏度系数在 $0 \sim 0.5$ 或 $-0.5 \sim 0$ 之间时,视为轻微偏斜。其中负值表示左偏分布(在分布的左侧有长尾), 正值则表示右偏分布(在分布的右侧有长尾)。

\subsection*{4.3. 2 峰度系数}
峰度 (kurtosis) 是指数据分布峰值的高低, 这一概念由统计学家皮尔逊于 1905 年首次提出。测度一组数据分布峰值高低的统计量称为峰度系数 (coefficient of kurtosis), 记作 $K$ 。根据原始数据计算峰度系数时, 通常采用下面的公式:


\begin{equation*}
K=\frac{n(n+1)}{(n-1)(n-2)(n-3)} \sum\left(\frac{x_{i}-\bar{x}}{s}\right)^{4}-\frac{3(n-1)^{2}}{(n-2)(n-3)} \tag{4.18}
\end{equation*}


峰度通常是与标准正态分布相比较而言的。由于标准正态分布的峰度系数为 0 , 当 $K>0$ 时为尖峰分布, 数据分布的峰值比标准正态分布高, 数据相对集中; 当 $K<0$ 时为扁平分布, 数据分布的峰值比标准正态分布低, 数据相对分散。

\section*{例 4. 10}
沿用例 4.1。计算 30 个消费者每月网购金额的偏度系数和峰度系数。

・解 根据式 (4.17) 得偏度系数为:

$$
\begin{aligned}
S K & =\frac{30}{(30-1)(30-2)} \sum\left(\frac{x_{i}-488.95}{97.62}\right)^{3} \\
& =\frac{30}{(30-1)(30-2)} \times 9.217966 \\
& =0.3406
\end{aligned}
$$

结果表示, 网购金额为轻微的右偏。

根据式 (4.18) 得峰度系数为:

$$
\begin{aligned}
K & =\frac{30 \times(30+1)}{(30-1)(30-2)(30-3)} \sum\left(\frac{x_{i}-488.95}{97.62}\right)^{4}-\frac{3 \times(30-1)^{2}}{(30-2)(30-3)} \\
& =-0.4075
\end{aligned}
$$

上面介绍的各描述统计量除了可以用 Excel 的统计函数计算外, 也可以使用【数据分析】工具一次输人多个统计量的计算结果, 以便进行综合性描述分析。用 Excel 的【数据分析】工具计算描述统计量的操作步骤如下:\\
太用【数据分析】工具计算描述统计量

第 1 步: 将光标放在任意空白单元格。然后点击【数据】 $\rightarrow$ 【数据分析】。在分析工具中选择【描述统计】。单击【确定】。

第 2 步: 在【输入区域】输入原始数据所在的区域; 在【输出选项】中选择结果的输出位置; 选择【汇总统计】(其他选项可根据需要选择)。单击【确定】即可得到结果。

以例 4.1 为例, 使用【数据分析】工具计算的描述统计量结果如表 4-8 所示。

表 4-830 个消费者每月网购金额的描述统计量

\begin{center}
\begin{tabular}{lc}
\hline
平均数 & 488.95 \\
标准误差 & 17.82290774 \\
中位数 & 471.2 \\
众数 & 450 \\
标准差 & 97.62008608 \\
方差 & 9529.681207 \\
峰度系数 & -0.407471639 \\
偏度系数 & 0.340565261 \\
区域 & 375.1 \\
最小值 & 303.6 \\
最大值 & 678.7 \\
求和 & 14668.5 \\
观测数 & 30 \\
\hline
\end{tabular}
\end{center}

表 4-8 中给出了描述数据集中趋势的平均数、中位数和众数, 描述数据离散程度的标准差、方差和全距 (输出结果显示为 “区域”), 描述数据分布形状的峰度系数和偏度系数等。利用该输出结果, 可以对 30 个消费者的每月网购金额作综合性的分析。

\section*{思考与练习}
\section*{思考题}
4.1 一组数据的分布特征可以从哪几个方面进行测度?

4.2 简述众数、中位数和平均数的特点和应用场合。

4.3 标准分数有哪些用途?

4.4 为什么要计算离散系数?

\section*{练习题}
4.1 一家汽车零售店的 10 名销售人员 5 月份销售的汽车数量 (单位: 辆) 排序后如下:

$\begin{array}{llllllllll}2 & 4 & 7 & 10 & 10 & 10 & 12 & 12 & 14 & 15\end{array}$

(1) 计算汽车销售量的众数、中位数和平均数。

(2) 计算销售量的四分位数。

(3)计算销售量的标准差。

(4) 说明汽车销售量分布的特征。

4.2 随机抽取 25 个网络用户, 得到他们的年龄数据(单位: 岁)如下:

\begin{center}
\begin{tabular}{lllll}
\hline
19 & 15 & 29 & 25 & 24 \\
23 & 21 & 38 & 22 & 18 \\
30 & 20 & 19 & 19 & 16 \\
23 & 27 & 22 & 34 & 24 \\
41 & 20 & 31 & 17 & 23 \\
\hline
\end{tabular}
\end{center}

(1)计算众数、中位数。

(2) 计算四分位数。

(3)计算平均数和标准差。

(4) 计算偏度系数和峰度系数。

(5)对网民年龄的分布特征进行综合分析。

4.3 某电商 6 月份各天的销售额数据(单位: 万元)如下:

\begin{center}
\begin{tabular}{llllllllll}
\hline
257 & 276 & 297 & 252 & 238 & 310 & 240 & 236 & 265 & 278 \\
271 & 292 & 261 & 281 & 301 & 274 & 267 & 280 & 291 & 258 \\
272 & 284 & 268 & 303 & 273 & 263 & 322 & 249 & 269 & 295 \\
\hline
\end{tabular}
\end{center}

(1)计算该电商日销售额的平均数和中位数。

(2) 计算四分位数。

(3)计算日销售额的标准差。

4.4 在某地区抽取 120 个企业, 按利润额进行分组, 结果如下:

\begin{center}
\begin{tabular}{cc}
\hline
按利润额分组 (万元) & 企业数 (个) \\
\hline
$200 \sim 300$ & 19 \\
$300 \sim 400$ & 30 \\
$400 \sim 500$ & 42 \\
$500 \sim 600$ & 18 \\
600 及以上 & 11 \\
\hline
合计 & 120 \\
\hline
\end{tabular}
\end{center}

计算 120 个企业利润额的平均数和标准差。

4.5 一条产品生产线平均每天的产量为 3700 件, 标准差为 50 件。如果某一天的产量低于或高于平均产量, 并落在 $\pm 2$ 个标准差的范围之外, 就认为该生产线失去控制。下面是一周各天的产量, 问该生产线哪几天失去了控制?

\begin{center}
\begin{tabular}{cccccccc}
\hline
时间 & 周一 & 周二 & 周三 & 周四 & 周五 & 周六 & 周日 \\
\hline
产量 (件) & 3850 & 3670 & 3690 & 3720 & 3610 & 3590 & 3700 \\
\hline
\end{tabular}
\end{center}

4.6 一种产品需要人工组装, 现有三种可供选择的组装方法。为检验哪种方法更好, 随机抽取 15 个工人, 让他们分别用三种方法组装。下面是 15 个工人分别用三种方法在相同时间内组装的产品数量:

\begin{center}
\begin{tabular}{ccc}
\hline
方法 A & 方法 B & 方法 C \\
\hline
164 & 129 & 125 \\
167 & 130 & 126 \\
168 & 129 & 126 \\
165 & 130 & 127 \\
170 & 131 & 126 \\
165 & 130 & 128 \\
164 & 129 & 126 \\
168 & 127 & 127 \\
164 & 128 & 127 \\
162 & 128 & 125 \\
163 & 127 & 126 \\
166 & 128 & 116 \\
167 & 128 & 126 \\
166 & 125 & 125 \\
\hline
165 & 132 &  \\
\hline
\end{tabular}
\end{center}

(1)你准备采用什么方法来评价组装方法的优劣?

(2)如果让你选择一种方法, 你会作出怎样的选择? 试说明理由。

\begin{center}
\end{center}

如果要求一个打火机的可靠性达到 $90 \%$, 而它是由 10 个零部件组成的, 那么, 每个零部件的可靠性应该达到多少?

如果要求一台笔记本电脑的可靠性达到 $90 \%$, 而它是由 1000 个零部件组成的, 那么, 每个零部件的可靠性应该达到多少?

一架波音 737 客机上有 300 多万个零部件, 如果用可靠性 $99.99 \%$ 的零部件去组装它, 这样的飞机您敢坐吗?

上面的问题可以根据下面将要讲授的概率论中独立事件的乘法原则来计算, 即 $A_{1} A_{2} \cdots A_{n}=90 \%$ 。

问题 1 相当于问 $?^{10}=0.90$, 计算结果为 $0.99^{10}=0.9044$, 即只有当每个零部件的可靠性达到 $99 \%$ 时, 由这些零部件组装的打火机的可靠性才能达到 $90 \%$ 。

问题 2 与问题 1 类似, 即相当于问? $?^{1000}=0.90$, 计算结果为 $0.9999^{1000}=0.9048$,即只有当每个零部件的可靠性达到 $99.99 \%$ 时, 由这 1000 个零部件组装的电脑的可靠性才能达到 $90 \%$ 。

我们可以把 $99.99 \%$ 的可靠性 (或合格率) 理解为万分之一的次品率。目前我国众多的生产制造企业对万分之一的次品率感到非常乐观。那么用具有 $99.99 \%$ 的可靠性的零部件去组装波音 737 飞机, 这架飞机整体质量的可靠性可想而知。\\
在前面的章节中, 我们介绍了搜集、整理和描述统计数据的一些基本方法。通过对统计数据的整理和描述, 我们可以对客观事物的概貌有一个了解。然而, 简单的描述方法只能实现对统计数据粗浅的利用, 它与从统计数据中挖掘出规律性的东西相去甚远。统计数据中隐含着非常丰富的重要信息, 要有效地充分利用统计数据, 需要运用推断统计的方法。推断统计就是在搜集、整理观测样本数据的基础上, 对有关总体作出推断,其特点是根据随机的观测样本数据以及问题的条件和假定, 对未知事物作出的以概率形式表述的推断。推断统计的理论和方法在我国通常被认为是概率论与数理统计的内容。由于概率论与数理统计的内容读者大都已经学过, 本章主要对后续章节将用到的概念进行介绍。

\section*{5.1 随机事件及其概率}
\subsection*{5.1. 1 随机事件的几个基本概念}
随机事件是概率论中的一个基本概念, 为了说明它, 先谈一下什么叫试验和事件。在同一组条件下, 对某事物或现象所进行的观察或实验叫做试验, 把观察或试验的结果叫做事件。

例如, 随意抛郑一颗色子 (一个质地均匀、样式对称的正六面体, 六面分别刻有 1 , $2,3,4,5,6$ 六个数字) 就是一次试验。色子落地, 出现 1 点、 2 点 $\cdots \cdots 6$ 点, 或为奇数点、偶数点或点数大于 3 等都是一个事件, 而且这些事件都是在一次试验中可能出现也可能不出现的。与此不同的还有两种事件, 即在一次试验中, 点数小于 7 这一事件, 在每次抛郑后是一定出现的, 而点数大于 6 这一事件, 在每次抛郑后一定不会出现。这样可以引人下面三个概念, 通过互相比较, 随机事件的概念会更清楚些。

(1)随机事件 (random event)。在同一组条件下, 每次试验可能出现也可能不出现的事件, 也叫偶然事件。

(2)必然事件 (certain event)。在同一组条件下,每次试验一定出现的事件。

(3)不可能事件 (impossible event)。在同一组条件下, 每次试验一定不出现的事件。

在郑色子的活动中, 点数小于 7 这一事件就是一个必然事件; 点数大于 6 这一事件就是一个不可能事件。

概率论研究的是随机事件, 并且把必然事件与不可能事件包含在随机事件内作为两个极端来看待。

随机事件简称为事件, 用大写字母 $A, B, C$ 等表示; 必然事件用 $\Omega$ 表示; 不可能事件用 $\Phi$ 表示。

如果一个事件不能分解成两个或更多个事件, 则这个事件称为基本事件 (elementary event) 或简单事件。例如, 在掷色子观察点数的试验中, 分别观察到点数为 1 点, 点数为 2 点, 点数为 3 点, 点数为 4 点, 点数为 5 点, 点数为 6 点, 这是该试验中的 6 个基本\\
事件。

基本事件具有很重要的性质。在一次试验中,只能观察到一个且仅有一个基本事件。一个试验中所有的基本事件的全体称为样本空间或基本空间, 记为 $\Omega$ 。如在郑硬币试验中, $\Omega=\{$ 正, 反 $\}$; 在掷色子试验中, $\Omega=\{1,2,3,4,5,6\}$ 。

对事件也可以进行运算, 如同对集合进行运算一样。对集合进行运算得出新的集合,而对事件进行运算则得出新的事件。关于事件的运算本书不作讨论。

\section*{5. 1.2 事件的概率}
事件 $A$ 的概率是对事件 $A$ 在试验中出现的可能性大小的一种度量, 记事件 $A$ 出现可能性大小的数值为 $P(A), P(A)$ 称为事件 $A$ 的概率 (probability)。基于对概率的不同解释, 概率的定义有所不同, 主要有古典定义、统计定义和主观概率定义。下面分别予以介绍。

\section*{1. 概率的古典定义}
求一个随机事件发生的可能性的大小, 这起源于奢博, 如郑硬币、掷色子等。这些活动都比较简单, 有两个重要的共同点:

(1)结果有限。基本空间中只包含有限个元素。如郑硬币的试验中, 只能出现 “正面朝上” 和 “反面朝上” 两种结果。

(2) 各个结果出现的可能性被认为是相同的。如郑硬币, 出现正面和反面的机会被认为是相等的。

具有上述两个特点的随机试验所研究的问题称为古典概型。

概率的古典定义是, 如果某一随机试验的结果有限, 而且各个结果出现的可能性相等, 则某一事件 $A$ 发生的概率为该事件所包含的基本事件个数 $m$ 与样本空间中所包含的基本事件个数 $n$ 的比值, 记为


\begin{equation*}
P(A)=\frac{\text { 事件 } A \text { 所包含的基本事件个数 }}{\text { 样本空间所包含的基本事件个数 }}=\frac{m}{n} \tag{5.1}
\end{equation*}


古典概率局限在随机试验只有有限个可能结果的范围内, 这使其应用受到了很大限制。因此, 人们又提出了根据某一事件在重复试验中发生的频率来确定其概率的方法, 即概率的统计定义。

\section*{2. 概率的统计定义}
在相同条件下随机试验 $n$ 次, 某事件 $A$ 出现 $m$ 次 $(m \leqslant n)$, 则比值 $m / n$ 称为事件 $A$发生的频率。随着 $n$ 的增大, 该频率围绕某一常数 $p$ 上下波动, 且波动的幅度逐渐减小,趋于稳定, 这个频率的稳定值即为该事件的概率, 记为


\begin{equation*}
P(A)=\frac{m}{n}=p \tag{5.2}
\end{equation*}


\section*{3. 主观概率定义}
概率的统计定义也有其局限性。在实际应用中要求在相同的条件下进行大量重复试验, 而事实上很多现象并不能进行大量重复试验, 特别是一些社会经济现象无法重复。有些现象即使能重复试验, 也很难保证试验条件完全一样。因而, 人们提出主观概率的概念。

所谓主观概率, 是指对一些无法重复的试验, 只能根据以往的经验, 人为确定这个事件的概率。例如, 某企业想投资一个新的项目, 那么投资成功的可能性有多大呢? 由于是一个新项目, 没有对这种项目投资的经验, 只能在综合分析多方面信息的基础上, 主观给出一个概率。比如投资成功的概率为 0.7 , 则投资失败的概率为 0.3 。主观概率法是工商活动中决策者常用的一种判断方法。

主观概率的定义是, 一个决策者根据本人掌握的信息对某个事件发生可能性作出的判断。这里我们只给出主观概率的概念, 它不是本书要讨论的重点。

\section*{5. 1.3 关于概率计算的几个例子}
\section*{例 5.1}
某电子公司所属企业职工人数如表 5-1 所示。从该公司中随机抽取一人,问:

(1)该职工为男性的概率是多少?

(2) 该职工为手机公司职工的概率是多少?

表 5-1 某电子公司所属企业职工人数

\begin{center}
\begin{tabular}{crrr}
\hline
分公司 & 男职工 & 女职工 & 合计 \\
\hline
电脑公司 & 4400 & 1800 & 6200 \\
手机公司 & 3200 & 1600 & 4800 \\
半导体公司 & 900 & 600 & 1500 \\
\hline
合计 & 8500 & 4000 & 12500 \\
\hline
\end{tabular}
\end{center}

——解(1)用 $A$ 表示 “抽中的职工为男性” 这一事件, $A$ 为全公司男性职工的集合, 基本空间为全公司职工的集合。由于全公司每个职工都可能以同等机会被抽中, 因此可按古典概率定义计算事件 $A$ 发生的概率:

$$
P(A)=\frac{\text { 全公司男职工人数 }}{\text { 全公司职工人数 }}=\frac{8500}{12500}=0.68
$$

(2)用 $B$ 表示 “抽中的职工为手机公司职工” 这一事件, $B$ 为手机公司全体职工的集合,基本空间为全公司职工的集合。事件 $B$ 发生的概率为:

$$
P(B)=\frac{\text { 手机公司职工人数 }}{\text { 全公司职工人数 }}=\frac{4800}{12500}=0.384
$$

\section*{例 5.2}
某工厂为了节约用电, 规定每天的用电量指标为 10000 度, 按照上个月的用电记录, 30 天中有 14 天的用电量超过规定指标, 若第二个月仍没有具体的节电措施, 试问该月第一天用电量超过指标的概率。

一解 由于每天的用电量并不相等, 超过用电指标的概率也不相等, 因此不能采用古典概率的定义。但上个月 30 天的记录可以看做重复进行了 30 次试验, 其中事件 $A$ 表示超过用电指标出现了 14 次。根据概率的统计定义, 事件 $A$ 的概率为:

$$
P(A)=\frac{\text { 超过用电指标天数 }}{\text { 试验的天数 }}=\frac{14}{30}=0.467
$$

\section*{例 5.3}
在例 5.2 中, 若第二个月采取了节电措施, 预计超过用电指标的概率将大大降低, 因此上个月超过用电指标的概率就不适用了。想要预计下个月超过用电指标的概率, 要请该厂管理用电的工程师根据采用节电措施后的情况进行预测。该工程师根据该厂过去的用电情况和采取节电措施后可以节电的程度判断, 用电量超过指标的概率为 $10 \%$, 这就属于主观概率。

\section*{5.2 离散型随机变量及其分布}
\subsection*{5.2. 1 随机变量的概念}
\section*{1. 随机事件的数量化}
在 5.1 节概率的基本概念中, 介绍了什么是随机事件, 某随机事件出现的概率怎样计算, 以及概率运算的基本法则。现在回过头来再看看随机事件, 即在同一组条件下, 每次试验可能出现也可能不出现的事件。有的采用数量标识表示, 如检验一批零件, 可能出现不同尺寸或直径; 抛郑一颗色子, 可能出现的点数为 $1,2,3,4,5,6$ 。显然, 这些随机事件都是用数量标识表示的。为了把随机事件数量化, 以便作数学上的处理, 有必要把不采用数量标识表示化为采用数量标识表示。例如, 把每检验一件产品可能出现合格的指定为 0 , 可能出现不合格的指定为 1 ; 或把每次抛郑一枚硬币可能出现正面的指定为 0 , 可能出现反面的指定为 1 。这样把指定的 $0 、 1$ 与合格、不合格一一对应, 或把 $0 、 1$ 与正面、反面一一对应, 就可以把随机事件完全数量化了。

\section*{2. 随机变量的定义}
随机事件可以用一个数量标识来表示, 根据某随机事件 $A$ 出现的概率定义 $P(A)$, 可以\\
把 $A$ 换成数量标识 $X$, 于是 $X$ 具有确定概率 $P(X)$ 。但随机事件有各种不同的情况, 如抛郑一枚硬币可能出现正面或反面, 正面指定为 0 , 即 $X=x_{1}=0$; 反面指定为 1 , 即 $X=x_{2}=1$ 。根据概率定义, $P\left(X=x_{1}\right)=P(X=0)=P$ (出现正面) $=1 / 2 ; P\left(X=x_{2}\right)=P(X=1)=$ $P$ (出现反面) $=1 / 2$ 。显然 0 与 1 是变量 $X$ 的两个可能值, 由此引出随机变量的定义。

在同一组条件下, 如果每次试验可能出现这样或那样的结果, 并且所有的结果都能列举出来, 即 $X$ 的所有可能值 $x_{1}, x_{2}, \cdots, x_{n}$ 都能列举出来, 而且 $X$ 的可能值 $x_{1}$, $x_{2}, \cdots, x_{n}$ 具有确定概率 $P\left(x_{1}\right), P\left(x_{2}\right), \cdots, P\left(x_{n}\right)$, 其中 $P\left(x_{i}\right)=P\left(X=x_{i}\right)$, 称为概率函数 (probability function), 则 $X$ 称为 $P(X)$ 的随机变量, $P(X)$ 称为随机变量 $X$的概率函数。

显然随机变量是基于随机事件的一个概念。既然各随机事件对应于一定的概率, 那么随机变量也对应于一定的概率, 而且用随机变量来研究所对应的一定的概率更全面、更系统。因此, 可以说随机变量是用随机事件描述随机现象的数量关系的推广。

随机变量在概率论与数理统计研究中的应用非常普遍, 可以这样说, 如果微积分是研究变量的数学, 那么概率论与数理统计是研究随机变量的数学。

\section*{3. 两种类型的随机变量}
按照随机变量的特性, 通常可把随机变量分为两类, 即离散型随机变量 (discrete random variable) 和连续型随机变量 (continuous random variable)。

(1) 离散型随机变量。

如果随机变量 $X$ 的所有取值都可以逐个列举出来, 则称 $X$ 为离散型随机变量。例如,在一批产品中取到次品的个数、单位时间内某交换台收到的呼叫次数等都是离散型随机变量。

(2) 连续型随机变量。

如果随机变量 $X$ 的所有取值无法逐个列举出来, 而是取数轴上某一区间内的任一点,则称 $X$ 为连续型随机变量。例如, 一批电子元件的寿命、实际工作中常遇到的测量误差等都是连续型随机变量。

\section*{5.2 .2 离散型随机变量的概率分布}
\section*{1. 离散型随机变量的概率分布}
设有一离散型随机变量 $X$, 可能取值 $x_{1}, x_{2}, \cdots, x_{n}$, 其相应的概率为 $p_{1}, p_{2}, \cdots$, $p_{n}$, 即 $P\left(X=x_{i}\right)=p_{i}(i=1,2, \cdots, n)$ ,如表 5-2 所示。

表 5-2 离散型随机变量的概率分布

\begin{center}
\begin{tabular}{ccccc}
\hline
$X=x_{i}$ & $x_{1}$ & $x_{2}$ & $\cdots$ & $x_{n}$ \\
\hline
$P\left(X=x_{i}\right)=p_{i}$ & $p_{1}$ & $p_{2}$ & $\cdots$ & $p_{n}$ \\
\hline
\end{tabular}
\end{center}

称该表格形式为离散型随机变量 $X$ 的概率分布 (probability distribution), 其中, $P\left(X=x_{i}\right)=p_{i}$ 是 $X$ 的概率函数。因为 $x_{1}, x_{2}, \cdots, x_{n}$ 构成一完备组, 所以

$$
\sum_{i=1}^{n} p_{i}=1
$$

\section*{例 5.4}
已知一批产品的废品率为 $p=0.05$, 合格品率为 $q=1-p=1-0.05=0.95$ 。指定废品用 1 代表, 合格品用 0 代表, 考察任抽取一件为废品或合格品, 即 1 或 0 这一离散型随机变量的概率分布 (如表 5-3 所示)。

表 5-3 概率分布

\begin{center}
\begin{tabular}{ccc}
\hline
$X=x_{i}$ & 1 & 0 \\
$P\left(X=x_{i}\right)=p_{i}$ & 0.05 & 0.95 \\
\hline
\end{tabular}
\end{center}

$\cdot$解 $\sum_{i=0}^{1} p_{i}=0.95+0.05=1$

\begin{center}
\end{center}

图 5-1 0-1 分布其图示法如图 5-1所示。

从图 5-1 和表 5-3 可以看出, 例 5.4 是一个仅在 0 与 1 离散点上的分布, 这种分布一般称为 $0-1$ 分布。

设离散型随机变量 $X$ 只可能取 0 和 1 两个值, 它的概率分布为:

$$
\begin{aligned}
& P(X=1)=p \\
& P(X=0)=1-p=q \\
& P(x)=p^{x} q^{1-x}, \quad x=0,1
\end{aligned}
$$

或

式中, $p, q>0$ 为常量, $p+q=1$, 则称 $X$ 服从

0-1 分布。

0-1 分布也可采用表 5-4 的形式。

表 5-4 0-1 分布

\begin{center}
\begin{tabular}{ccc}
\hline
$X$ & 1 & 0 \\
$P(X)$ & $p$ & $q$ \\
\hline
\end{tabular}
\end{center}

0-1 分布是经常遇到的一种分布。如新生儿的性别、产品质量是否合格、某种试验是否成功、电力消耗是否超负荷等, 都可以用 0-1 分布的离散型随机变量来描述。

\section*{例 5.5}
抛掷一颗色子, 出现的点数是一个离散型随机变量, 其概率分布如表 5-5 所示。

\begin{center}
\begin{tabular}{ccccccc}
\hline
$X$ & 1 & 2 & 3 & 4 & 5 & 6 \\
\hline
$P(X)$ & $\frac{1}{6}$ & $\frac{1}{6}$ & $\frac{1}{6}$ & $\frac{1}{6}$ & $\frac{1}{6}$ & $\frac{1}{6}$ \\
\hline
\end{tabular}
\end{center}

・解 $\sum_{i=1}^{6} p_{i}=\frac{1}{6}+\frac{1}{6}+\frac{1}{6}+\frac{1}{6}+\frac{1}{6}+\frac{1}{6}=1$

其图示法如图 5-2所示。

\begin{center}
\end{center}

图 5-2 均匀分布

例 5.5 描述的这种概率分布叫离散型随机变量 $X$ 的均匀分布(uniform distribution)。

\section*{2. 离散型随机变量的期望值和方差}
为了能够全面了解随机变量 $X$ 的概率性质, 自然应该知道 $X$ 的概率分布。然而在实际问题中, 一个随机变量的概率分布往往不好确定。其实还有不少问题, 并不需要知道 $X$的所有概率性质, 只需要知道它的某些数字特征就够了。因此, 在对随机变量的研究中,确定某些数字特征是重要的。在这些数字特征中, 最重要的是期望值和方差。

(1)期望值。

在离散型随机变量 $X$ 的一切可能值的完备组中, 各可能值 $x_{i}$ 与其对应概率 $p_{i}$ 的乘积之和称为该随机变量 $X$ 的期望值 (expected value), 记作 $E(X)$ 或 $\mu_{0}$

若 $X$ 取数值 $x_{1}, x_{2}, \cdots, x_{n}$, 其对应的概率为 $p_{1}, p_{2}, \cdots, p_{n}$, 则期望值为:


\begin{equation*}
E(X)=x_{1} p_{1}+x_{2} p_{2}+\cdots+x_{n} p_{n}=\sum_{i=1}^{n} x_{i} p_{i} \tag{5.3}
\end{equation*}


若 $X$ 取无穷个数值 $x_{1}, x_{2}, \cdots, x_{n}, \cdots$, 其对应的概率为 $p_{1}, p_{2}, \cdots, p_{n}, \cdots$, 则期望值为:


\begin{equation*}
E(X)=x_{1} p_{1}+x_{2} p_{2}+\cdots+x_{n} p_{n}+\cdots=\sum_{i=1}^{\infty} x_{i} p_{i} \tag{5.4}
\end{equation*}


期望值 $E(X)$ 也称为随机变量 $X$ 的数学期望。

以郑色子为例, 它的期望值为:

$$
\mu=E(X)=\sum_{i=1}^{6} x_{i} p_{i}
$$

$$
\begin{aligned}
& =1 \times \frac{1}{6}+2 \times \frac{1}{6}+3 \times \frac{1}{6}+4 \times \frac{1}{6}+5 \times \frac{1}{6}+6 \times \frac{1}{6} \\
& =3.5
\end{aligned}
$$

这就是说, 各种可能点数的期望值为 3.5 。

由离散型随机变量 $X$ 的期望值定义可看到, 它与加权算术平均数的算法有点类似,其实它是加权算术平均数的一种推广。一般加权算术平均数是具体数据的平均指标, 而这里所说的期望值是随机变量 $X$ 的期望指标。

(2) 方差与标准差。

前面讨论的数学期望是随机变量的一个重要数字特征, 它表示随机变量本身的平均水平或集中程度。这里将讨论随机变量与其数学期望的离差的平均水平, 用它可以测量随机变量的变异程度或离散程度, 它是随机变量的另一个重要数字特征。

例如有一批灯泡, 已知其平均寿命 $E(X)=1000$ 小时, 但仅由这一指标还不能判定这批灯泡质量的好坏, 事实上, 可能其中绝大部分灯泡的寿命都在 950 1 050 小时左右 (质量稳定); 也可能其中约一半是高质量, 寿命大约有 1300 小时, 另一半却是质量很差的, 寿命大约只有 700 小时(质量不稳定)。因此为了正确评价灯泡质量, 除考察灯泡平均寿命指标, 还需要考察各个灯泡寿命 $X$ 与一批灯泡寿命 $E(X)=1000$ 小时的离差的平均水平。若这个值较小, 则表示质量比较稳定, 质量较好; 反之, 质量较差。同样, 在检查棉纱质量时, 不仅需要知道纤维的平均长度, 还要知道纤维长短的均匀度等。

随机变量的方差是用来反映随机变量取值的离散程度的。随机变量的方差定义为每一个随机变量取值与期望值的离差平方之期望值。设随机变量为 $X$, 其方差常用 $\sigma_{X}^{2}, D(X)$或 $V(X)$ 表示,本书采用 $D(X)$ ,则

$$
\sigma^{2}=D(X)=E[X-E(X)]^{2}
$$

由上式可知, 方差实际上就是随机变量 $X$ 的函数 $[X-E(X)]^{2}$ 的数学期望。于是,若 $X$ 是离散型随机变量, 则


\begin{equation*}
\sigma^{2}=D(X)=\sum_{i=1}^{\infty}\left[x_{i}-E(X)\right]^{2} p_{i} \tag{5.5}
\end{equation*}


式中, $p_{i}=P\left\{X=x_{i}\right\}(i=1,2, \cdots)$ 。

在计算方差时,一个常用的简化公式为:


\begin{equation*}
\sigma^{2}=D(X)=E\left(X^{2}\right)-[E(X)]^{2} \tag{5.6}
\end{equation*}


由定义可知, 若 $X$ 的取值比较集中, 则方差较小; 若 $X$ 的取值比较分散, 则方差较大。如果方差为 0 , 则意味着随机变量的取值集中于期望值 $E(X)$, 即随机变量以概率 1 取值 $E(X)$ 。

此外, 称随机变量方差的平方根为标准差, 记为:


\begin{equation*}
\sigma=\sqrt{D(X)} \tag{5.7}
\end{equation*}


由于标准差与随机变量 $X$ 有相同的度量单位, 所以在实际工作中经常使用。对于郑色子的例子, 随机变量 $X$ 的方差为:

$$
\begin{aligned}
\sigma^{2}= & \sum_{i=1}^{n}\left[x_{i}-E(X)\right]^{2} p\left(x_{i}\right) \\
= & (1-3.5)^{2} \times \frac{1}{6}+(2-3.5)^{2} \times \frac{1}{6}+(3-3.5)^{2} \times \frac{1}{6}+(4-3.5)^{2} \times \frac{1}{6} \\
& +(5-3.5)^{2} \times \frac{1}{6}+(6-3.5)^{2} \times \frac{1}{6} \\
= & 2.9167
\end{aligned}
$$

标准差 $\sigma=1.7078$, 说明每次郑得的点数与平均点数 3.5 平均相距 1.7078 点。

(3) 离散系数。

离散系数可用来比较不同期望值的总体之间的离散趋势。计算公式为:

$$
V=\frac{\sigma}{E(X)}
$$

(4) 均值和方差在财务分析中的应用。

在财务分析中, 风险的概念十分重要。风险的高低有时可以单凭主观感受作出判断,也可以用方差或标准差去测量, 从而得出一个比较客观和科学的结果。我们用以下例子具体说明如何利用方差或标准差去评估某个投资的预期平均回报率及其相应的风险, 从而作出投资的决定。

\section*{例 5.6}
一位投资者有一笔现金可用于投资, 现有两个投资项目可供选择。项目 A 和项目 B 有如下资料可供参考 (见表 5-6 和表 5-7)。试比较哪个投资项目更优。

\section*{表 5-6 项目 A}
\begin{center}
\begin{tabular}{ccc}
\hline
持有期回报率 $x(\%)$ & 可能性 (概率) $p(x)$ & $x \cdot p(x)$ \\
\hline
4 & 0.05 & 0.2 \\
5 & 0.1 & 0.5 \\
6 & 0.15 & 0.9 \\
7 & 0.4 & 2.8 \\
8 & 0.15 & 1.2 \\
9 & 0.1 & 0.9 \\
10 & 0.05 & 0.5 \\
\hline
合计 & 1 & 7 \\
\hline
\end{tabular}
\end{center}

\section*{表 5-7 项目 B}
\begin{center}
\begin{tabular}{ccc}
\hline
持有期回报率 $x(\%)$ & 可能性 (概率) $p(x)$ & $x \cdot p(x)$ \\
\hline
5.5 & 0.25 & 1.375 \\
6.5 & 0.25 & 1.625 \\
7.5 & 0.25 & 1.875 \\
8.5 & 0.25 & 2.125 \\
\hline
合计 & 1 & 7 \\
\hline
\end{tabular}
\end{center}

解 在评估两个项目的回报率时, 由于各有不同的可能性, 所以可以运用预期平均回报率来反映个别项目的盈利能力, 以便作出投资选择。

在项目 A 中, 回报率 $4 \%$ 乘以其概率 0.05 , 得出 $0.2 \%$ 的预期回报率, 这个数字同时表示了回报率的量及其出现的机会。将各个预期回报率加在一起, 便得出项目 $\mathrm{A}$ 的预期平均回报率为 $7 \%$ 。用同一方法算出项目 B的预期平均回报率亦为 $7 \%$ 。由于两者的预期平均回报率相同, 所以有必要评估两个项目回报率的稳定性或风险, 以决定哪个项目较优。而稳定性或风险可用方差或标准差反映出来。计算方差及标准差时, 预期平均回报率就是均值, 计算见表 5-8 和表 5-9。

表 5-8 项目 $\mathrm{A}$ 的计算

\begin{center}
\begin{tabular}{cccccc}
\hline
回报率 $x(\%)$ & 预期平均回报率 $\mu(\%)$ & $x-\mu$ & $(x-\mu)^{2}$ & 概率 $p(x)$ & $(x-\mu)^{2} p(x)$ \\
\hline
4 & 7 & -3 & 9 & 0.05 & 0.45 \\
5 & 7 & -2 & 4 & 0.1 & 0.4 \\
6 & 7 & -1 & 1 & 0.15 & 0.15 \\
7 & 7 & 0 & 0 & 0.4 & 0 \\
8 & 7 & 1 & 1 & 0.15 & 0.15 \\
9 & 7 & 2 & 4 & 0.1 & 0.4 \\
10 & 7 & 3 & 9 & 0.05 & 0.45 \\
\hline
合计 & - & - & - & 1 & 2 \\
\hline
\end{tabular}
\end{center}

表 5-9 项目 B 的计算

\begin{center}
\begin{tabular}{cccccc}
\hline
回报率 $x(\%)$ & 预期平均回报率 $\mu(\%)$ & $x-\mu$ & $(x-\mu)^{2}$ & 概率 $p(x)$ & $(x-\mu)^{2} p(x)$ \\
\hline
5.5 & 7 & -1.5 & 2.25 & 0.25 & 0.5625 \\
6.5 & 7 & -0.5 & 0.25 & 0.25 & 0.0625 \\
7.5 & 7 & 0.5 & 0.25 & 0.25 & 0.0625 \\
8.5 & 7 & 1.5 & 2.25 & 0.25 & 0.5625 \\
\hline
合计 & - & - & - & 1 & 1.25 \\
\hline
\end{tabular}
\end{center}

项目 $\mathrm{A}$ 的方差 $\sigma^{2}=\sum(x-\mu)^{2} p(x)=2$

项目 $\mathrm{A}$ 的标准差 $\sigma=\sqrt{2}=1.414$

项目 $\mathrm{B}$ 的方差 $\sigma^{2}=\sum(x-\mu)^{2} p(x)=1.25$

项目 $\mathrm{B}$ 的标准差 $\sigma=\sqrt{1.25}=1.12$

项目 $\mathrm{A}$ 的标准差为 1.414 , 这个值反映了每一个可能出现的回报率与预期平均回报率的平均差别; 数值越大, 回报率的变化越大, 其稳定性越小, 凤险越大。项目 B 的标准差为 1.12 , 比项目 $\mathrm{A}$ 的低, 因此, 其回报率的稳定性较高, 即风险较低。比较二者, 就风险控制而言,投资项目B更优。\\
如果项目 $\mathrm{A}$ 和项目 B 的预期平均回报率不一样, 则需要用离散系数来考察投资风险的大小,帮助投资者作出选择。

\section*{例 5.7}
如果投资项目 A 的预期回报率为 $7 \%$, 标准差为 $5 \%$; 而投资项目 B 的预期回报率为 $12 \%$ ,标准差为 $7 \%$, 哪个投资项目的风险较大?

\begin{itemize}
  \item 解 如果只从标准差考虑, 似乎投资项目 $\mathrm{B}$ 的风险较大, 但由于 $\mathrm{A}$ 与 $\mathrm{B}$ 的预期回报率 (均值)不同,我们需用离散系数进行比较。
\end{itemize}

项目 $\mathrm{A}$ 的离散系数 $V=\frac{0.05}{0.07}=0.714$

项目 $\mathrm{B}$ 的离散系数 $V=\frac{0.07}{0.12}=0.583$

投资项目 $\mathrm{A}$ 每单位回报率承受 0.714 单位的风险, 而投资项目 B每单位回报率的风险为 0.583 单位, 因此, $\mathrm{A}$ 的风险较大。

随机变量 $X$ 的均值和方差的含义与第 4 章中均值和方差的含义是一致的。均值反映随机变量 $X$ 取值的平均水平, 方差反映随机变量 $X$ 取值的离散程度。

\section*{3. 二项分布和泊松分布}
离散型随机变量有许多重要的概率分布, 下面仅介绍两种最常用的概率分布, 即二项分布和泊松分布。

(1) 二项分布。

实际工作中, 许多试验与郑硬币的试验有共同的性质, 它们只包含两个结果。例如在市场调查中考虑对产品的喜好, 调查一个人相当于郑一次硬币, 因为某人喜欢该产品相当于出现正面, 不喜欢该产品相当于出现反面, 则在调查的样本中回答喜欢的人数是一个随机变量, 这种随机变量所服从的概率分布通常称为二项分布(binomial distribution)。类似的例子有很多, 如社会学家感兴趣的是农民脱贫的比例; 酒的制造商感兴趣的是饮酒者中喜欢他们品牌的比例。这些例子所具有的共同性质可总结如下:

\begin{itemize}
  \item 包含 $n$ 个相同的试验。
  \item 每次试验只有两个可能的结果: “成功” 或 “失败”。这里的 “成功” 和 “失败” 是广义的。如在某商品的调查中, “成功” 表示喜欢这种商品, “失败” 表示不喜欢这种商品。
  \item 出现 “成功” 的概率 $p$ 对每一次试验是相同的, “失败” 的概率 $q$ 也是如此, 且 $p+q=1$ 。
  \item 试验是互相独立的。
\end{itemize}

$\cdot$试验“成功”或“失败”可以计数, 即试验结果对应于一个离散型随机变量。\\
通常称具有上述特征的 $n$ 次重复独立试验为 $n$ 重贝努里试验, 简称贝努里试验(Ber noulli trials)或贝努里试验概型。

以 $X$ 表示 $n$ 次重复独立试验中事件 $A$ (成功)出现的次数, 不难得出:


\begin{equation*}
P\{X=x\}=C_{n}^{x} p^{x} q^{n-x}, \quad x=0,1,2, \cdots, n \tag{5.8}
\end{equation*}


显然

$$
\begin{aligned}
& P\{X=x\} \geqslant 0, \quad x=0,1,2, \cdots, n \\
& \sum_{x=0}^{n} C_{n}^{x} p^{x} q^{n-x}=(p+q)^{n}=1
\end{aligned}
$$

注意到 $C_{n}^{x} p^{x} q^{n-x}$ 正好是二项式 $(p+q)$ 的展开式中的第 $x+1$ 项, 故我们称随机变量 $X$ 服从二项分布, 参数为 $n, p$, 记作 $X \sim B(n, p)$ 。

式 (5.8) 中的 $C_{n}^{x}$ 表示从 $n$ 个元素中抽取 $x$ 个元素的组合, 计算公式为:


\begin{equation*}
C_{n}^{x}=\frac{n !}{x !(n-x) !} \tag{5.9}
\end{equation*}


二项分布的期望值和方差分别为:


\begin{align*}
& E(X)=n p  \tag{5.10}\\
& D(X)=n p q \tag{5.11}
\end{align*}


特别地, 当 $n=1$ 时, 二项分布化为:

$$
P\{X=x\}=p^{x} q^{1-x}, \quad x=0,1
$$

称为 $0-1$ 分布。

\section*{例 5.8}
已知 100 件产品中有 5 件次品, 现从中任取 1 件, 有放回地取 3 次, 求在所取的 3 件中恰有 2 件次品的概率。

\begin{itemize}
  \item 解 因为这是有放回地取 3 次, 所以这 3 次试验的条件是完全相同的, 由此可知它是贝努里试验。
\end{itemize}

根据题意, 每次试验取到次品的概率为:

$$
P=\frac{5}{100}=0.05
$$

设 $X$ 为所取的 3 件产品中的次品数, 则 $X \sim B(3,0.05)$, 于是有

$$
P\{X=2\}=C_{3}^{2}(0.05)^{2}(0.95)=0.007125
$$

如果将例 5.8 中的 “有放回” 改为 “无放回”, 那么各次试验条件就不同了, 故不能用二项分布的概率公式来计算, 而只能用古典概型求解。

$$
P\{X=2\}=\frac{C_{5}^{2} C_{95}^{1}}{C_{100}^{3}}=0.0059
$$

这种情况实际上是概率分布中一种很重要的分布, 称为超几何分布。设有 $N$ 件产品,其中有 $M$ 件次品, 现从中任取 $n$ 件(假定 $n \leqslant N$ ), 则在这 $n$ 件中所含的次品件数 $X$ 是一\\
个随机变量, $X$ 的概率函数为:


\begin{equation*}
P\{X=m\}=\frac{C_{M}^{m} C_{\mathrm{N}-M}^{n-m}}{C_{\mathrm{N}}^{n}} \tag{5.12}
\end{equation*}


式中, $m$ 为任取 $n$ 件中次品的件数。

在实际问题中, 真正在完全相同条件下进行的试验是不多见的。对于抽样问题来说,当原产品的批量相当大时, “无放回” 可以当做 “有放回” 来处理。于是可用式 (5.8) 来近似计算取到的产品中含有 $x$ 件次品的概率。

太用 Excel 计算二项分布概率值的操作步骤

第 1 步: 进入 Excel 界面, 单击某一单元格。

第 2 步: 在菜单栏中选择【公式】 $\rightarrow$ 【插入函数】选项, 弹出插入函数的对话框。从【或选择类别 $(\underline{C}):$ 】窗口中选择 “统计",从【选择函数 $(\underline{\mathbf{N}}):$ 】窗口中选择 "BINOM. DIST", 点击【确定】。

第 3 步: 当【BINOM. DIST】对话框出现时: 在【Number\_s】中输入 2 (成功的次数 X)。在【Trials】中输入 3 (试验的总次数 $n$ )。在【Probability\_s】中输入 0.05 (每次试验成功的概率 $p$ )。在【Cumulative】中输入 0 (或 False), 表示计算成功次数恰好等于指定数值的概率 (输入 1 或 True 表示计算成功次数小于或等于指定数值的累积概率值)。点击【确

定】, 得到结果 0.007125 。

计算超几何分布概率的步骤与上述过程类似, 只不过在第 2 步【选择函数 (N):】窗口中选择 "HYPGEOM. DIST" 函数。在第 3 步中输入相应的值即可。如下:

在【Sample\_s】中输入 2 (成功的次数 X)。在【Number\_sample】中输入 $3($ 样本量 $n$ )。在【Population\_s】中输入 5 (总体中成功的次数 M)。在【Number\_pop】中输入 100 (总体中的个体总数 $N)$ 。点击【确定】即可。

(2) 泊松分布。

泊松分布 (Poisson distribution) 是用来描述在一指定时间范围内或在指定的面积或体积之内某一事件出现的次数的分布。下面是一些典型的服从泊松分布的随机变量的例子。

\begin{enumerate}
  \item 某企业每月发生事故的次数。

  \item 单位时间内到达某一服务柜台(服务站、诊所、超市收银台、电话总机等)需要服务的顾客人数。

\end{enumerate}

3)人寿保险公司每天收到的死亡声明的份数。

\begin{enumerate}
  \setcounter{enumi}{3}
  \item 某种仪器每月出现故障的次数。
\end{enumerate}

泊松分布的公式为:


\begin{equation*}
P(X)=\frac{\lambda^{x} \mathrm{e}^{-\lambda}}{x !}, \quad x=0,1,2, \cdots \tag{5.13}
\end{equation*}


式中, $\lambda$ 为给定的时间间隔内事件的平均数。

泊松分布的期望值和方差分别为:

$$
\begin{aligned}
& E(X)=\lambda \\
& D(X)=\lambda
\end{aligned}
$$

\section*{例 5.9}
假定某企业职工在周一请事假的人数 $X$ 近似服从泊松分布, 设周一请事假的平均数为 2.5 人。要求计算:

(1) $X$ 的均值与标准差。

(2)在给定的某周一正好请事假是 5 人的概率。

解 (1) 根据公式, $X$ 的均值和方差均为 2.5 , 即 $D(X)=2.5$ 。

$$
\text { 标准差 }=\sqrt{D(X)}=\sqrt{2.5}=1.581
$$

(2) $X$ 的概率分布为:

$$
P(X)=\frac{\lambda^{x} \mathrm{e}^{-\lambda}}{x !}
$$

式中, $\lambda=2.5, x=5, \mathrm{e}^{-2.5}=0.082085$, 所以

$$
P(5)=\frac{(2.5)^{5} \mathrm{e}^{-2.5}}{5 !}=0.067
$$

在 $n$ 重贝努里试验中, 当成功的概率很小(即 $p \rightarrow 0$ ), 试验次数很大时, 二项分布近似等于泊松分布, 即

$$
C_{n}^{x} p^{x} q^{n-x} \approx \frac{\lambda^{x} \mathrm{e}^{-\lambda}}{x !}
$$

在实际应用中, 当 $p \leqslant 0.25, n>20, n p \leqslant 5$ 时, 用泊松分布近似二项分布的效果良好。

\section*{例 5.10}
已知某批集成电路的次品率为 $0.15 \%$, 随机抽取 1000 块集成电路, 求次品数分别为 $0,1,2,3$ 的概率。

・解 把 “集成电路的次品数” 看成随机变量 $X$, 显然 $X$ 服从二项分布, $p=0.0015$, $n=1000$ 。直接用二项分布公式求解, 需要计算下列各式:

$$
\begin{aligned}
& P\{X=0\}=C_{1000}^{0}(0.0015)^{0}(0.9985)^{1000} \\
& P\{X=1\}=C_{1000}^{1}(0.0015)^{1}(0.9985)^{999} \\
& P\{X=2\}=C_{1000}^{2}(0.0015)^{2}(0.9985)^{998} \\
& P\{X=3\}=C_{1000}^{3}(0.0015)^{3}(0.9985)^{997}
\end{aligned}
$$

这样直接计算相当麻烦, 可以考虑用泊松分布计算。因为 $p=0.0015<0.25, n p=$ $1.5<5$, 所以可以用泊松分布近似计算。其中, $\lambda=n p=1.5$ 。

$$
\begin{aligned}
& P\{X=0\}=\frac{\lambda^{x}}{x !} \mathrm{e}^{-\lambda}=\frac{1.5^{0}}{0 !} \mathrm{e}^{-1.5}=0.223130 \\
& P\{X=1\}=\frac{1.5^{1}}{1 !} \mathrm{e}^{-1.5}=0.334695 \\
& P\{X=2\}=\frac{1.5^{2}}{2 !} \mathrm{e}^{-1.5}=0.251021 \\
& P\{X=3\}=\frac{1.5^{3}}{3 !} \mathrm{e}^{-1.5}=0.125511
\end{aligned}
$$

表 5-10 给出了用二项分布直接计算与用泊松分布近似计算结果的对比, 可以看到,二者之间的差距非常小。

表 5-10 二项分布与泊松分布结果对照表

\begin{center}
\begin{tabular}{cccc}
\hline
$P\{X=x\}$ & $C_{n}^{x} p^{x} q^{n-x}$ & $\frac{\lambda^{x}}{x !} \mathrm{e}^{-\lambda}$ & 绝对差 \\
\hline
$P\{X=0\}$ & 0.222879 & 0.223130 & 0.000251 \\
$P\{X=1\}$ & 0.334821 & 0.334695 & 0.000126 \\
$P\{X=2\}$ & 0.251241 & 0.251021 & 0.000220 \\
$P\{X=3\}$ & 0.125558 & 0.125511 & 0.000047 \\
\hline
\end{tabular}
\end{center}

在实际应用中, 二项分布和泊松分布的计算都可以直接查表, 而无须使用公式。读者可将例 5.8 和例 5.9 的结果用查表方式给出。

\section*{太用 Excel 计算泊松分布概率值的操作步骤}
第 1 步:进入 Excel 界面,单击某一单元格。

第 2 步: 在菜单栏中选择【公式】 $\rightarrow$ 【插入函数】选项, 弹出插入函数的对话框。从【或选择类别 $(\mathbf{C}):$ 】窗口中选择 "统计"; 从【选择函数 (N):】窗口中选择 "POISSON. DIST",点击【确定】。

第 3 步: 当【POISSON. DIST】对话框出现时:在【X】中输入事件出现的次数(本例为 0 )。在【Mean】中输入泊松分布的均值 $\lambda$ (本例为 1.5 )。在【Cumulative】中输入 0 (或 False),表示计算事件出现次数恰好等于指定数值的概率(输入 1 或 True 表示计算事件出现次数小于或等于指定数值的累积概率值)。在【X】 选项中分别填入 $1,2,3$ 即可计算出相应概率。

\section*{5.3 连续型随机变量的概率分布}
\subsection*{5.3. 1 概率密度与分布函数}
由于连续型随机变量可以取某一区间或整个实数轴上的任意值, 所以我们不能像对离\\
散型随机变量那样列出每一个值及其相应的概率, 而必须采用其他方法, 通常用数学函数和分布函数的形式来描述。当用函数 $f(x)$ 来表示连续型随机变量时, 我们将 $f(x)$ 称为概率密度函数 (probability density function)。

概率密度函数应满足以下两个条件:

(1) $f(x) \geqslant 0$

(2) $\int_{-\infty}^{+\infty} f(x) \mathrm{d} x=1$

需要指出的是, $f(x)$ 并不是一个概率, 即 $f(x) \neq P(X=x), f(x)$ 称为概率密度函数, 而 $P(X=x)$ 在连续分布的条件下为零。在连续分布的情况下, 以曲线下面的面积表示概率,如随机变量 $X$ 在 $a$ 与 $b$ 之间的概率可以写成:


\begin{equation*}
P(a<X<b)=\int_{a}^{b} f(x) \mathrm{d} x \tag{5.16}
\end{equation*}


即图 5-3 中阴影部分的面积。

\begin{center}
\end{center}

图 5-3 概率 $P(a<X<b)$

连续型随机变量的概率也可以用分布函数 $F(x)$ 来表示, 分布函数定义为:


\begin{equation*}
F(x)=P(X \leqslant x)=\int_{-\infty}^{x} f(t) \mathrm{d} t, \quad-\infty<x<+\infty \tag{5.17}
\end{equation*}


这也是建立在密度函数 $f(x)$ 的基础之上的, 因此 $P(a<X<b)$ 也可以写成:


\begin{equation*}
\int_{a}^{b} f(x) \mathrm{d} x=F(b)-F(a) \tag{5.18}
\end{equation*}


显然, 连续型随机变量的概率密度是其分布函数的导数, 即


\begin{equation*}
f(x)=F^{\prime}(x) \tag{5.19}
\end{equation*}


连续型随机变量的期望值与方差分别定义为:


\begin{align*}
& E(X)=\int_{-\infty}^{+\infty} x f(x) \mathrm{d} x=\mu  \tag{5.20}\\
& D(X)=\int_{-\infty}^{+\infty}[x-E(x)]^{2} f(x) \mathrm{d} x=\sigma^{2} \tag{5.21}
\end{align*}


\subsection*{5.3.2 正态分布}
在连续型随机变量中, 最重要的一种随机变量是具有钟形概率分布的随机变量, 如\\
图 5-4 所示。人们称它为正态随机变量, 相应的概率分布称为正态分布(normal distri bution)。

\begin{center}
\end{center}

图 5-4 正态分布的概率密度曲线

在社会经济问题中, 许多随机变量的概率分布都服从正态分布。例如, 某地区同年龄组儿童的发育特征, 如身高、体重、肺活量; 某公司年销售量; 在同一条件下产品的质量以平均质量为中心上下波动; 特别差的和特别好的都是少数, 多数处在中间状态; 人群中的高个子和矮个子都是少数, 而中等身材的人居多; 等等。

正态分布是最重要的一种连续型分布, 有着非常广泛的应用。

\section*{1. 正态分布的定义及图形特点}
如果随机变量 $X$ 的概率密度为:


\begin{equation*}
f(x)=\frac{1}{\sigma \sqrt{2 \pi}} \mathrm{e}^{-\frac{1}{2 \sigma^{2}(x-\mu)^{2}}}, \quad-\infty<x<+\infty \tag{5.22}
\end{equation*}


则称 $X$ 服从正态分布, 记作 $X \sim N\left(\mu, \sigma^{2}\right)$, 其中, $-\infty<\mu<+\infty, \sigma>0, \mu$ 为随机变量 $X$ 的均值, $\sigma$ 为随机变量 $X$ 的标准差, 它们是正态分布的两个参数。

$X \sim N\left(\mu, \sigma^{2}\right)$ 通常读成随机变量 $X$ 服从均值为 $\mu$ 、方差为 $\sigma^{2}$ 的正态分布。

为了画出正态分布的图形, 先对概率密度 $f(x)$ 作几点讨论:

(1) $f(x) \geqslant 0$, 即整个概率密度曲线都在 $x$ 轴的上方。

(2) 曲线 $f(x)$ 相对于 $x=\mu$ 对称, 并在 $x=\mu$ 处达到最大值, $f(\mu)=\frac{1}{\sqrt{2 \pi} \sigma}$ 。

(3)曲线的陡缓程度由 $\sigma$ 决定, $\sigma$ 越大, 曲线越平缓; $\sigma$ 越小, 曲线越陡峭。

(4) 当 $x$ 趋于无穷时, 曲线以 $x$ 轴为其渐近线。

综上所述, 可画出正态分布的概率密度曲线, 如图 5-4 所示, 它是一条对称的钟形曲线。

为了说明参数 $\mu, \sigma$ 对曲线位置、形状的影响, 可参见图 5-5。

由图 5-5 可看出, $\mu$ 决定了图形的中心位置, $\sigma$ 决定了图形中曲线的陡峭程度。当 $\sigma$较大时, 曲线趋于平缓; 当 $\sigma$ 较小时, 曲线趋于陡峭。

\begin{center}
\end{center}

(a)

\begin{center}
\end{center}

(b)

图 5-5 参数 $\mu$ 和 $\sigma$ 对曲线位置、形状的影响

\section*{2. 标准正态分布}
当式 (5.22) 中 $\mu=0, \sigma=1$ 时, 有


\begin{equation*}
f(x)=\frac{1}{\sqrt{2 \pi}} \mathrm{e}^{-\frac{x^{2}}{2}}, \quad-\infty<x<+\infty \tag{5.23}
\end{equation*}


相应的正态分布 $N(0,1)$ 称为标准正态分布(standard normal distribution)。对标准正态分布, 通常用 $\varphi(x)$ 表示概率密度函数, 用 $\Phi(x)$ 表示分布函数, 即


\begin{align*}
& \varphi(x)=\frac{1}{\sqrt{2 \pi}} \mathrm{e}^{-\frac{x^{2}}{2}}  \tag{5.24}\\
& \Phi(x)=\int_{-\infty}^{x} \varphi(t) \mathrm{d} t=\int_{-\infty}^{x} \frac{1}{\sqrt{2 \pi}} \mathrm{e}^{-\frac{t^{2}}{2}} \mathrm{~d} t \tag{5.25}
\end{align*}


标准正态分布的概率密度函数 $\varphi(x)$ 和分布函数 $\Phi(x)$ 的图形如图 5-6 所示。

\begin{center}
\end{center}

(a)

\begin{center}
\end{center}

(b)

图 5-6 标准正态分布的概率密度函数和分布函数

标准正态分布的重要性在于, 任何一个一般的正态分布都可以通过线性变换转化为标准正态分布。

设 $X \sim N\left(\mu, \sigma^{2}\right)$, 则


\begin{equation*}
Z=\frac{X-\mu}{\sigma} \sim N(0,1) \tag{5.26}
\end{equation*}


式 (5.26) 就是将一般正态分布转化为标准正态分布的公式。

\section*{3. 正态分布表}
只要将一般正态分布转化为标准正态分布, 通过查表就可解决正态分布的概率计算问题。

对于负值,可采用以下公式:


\begin{equation*}
\Phi(-x)=1-\Phi(x) \tag{5.27}
\end{equation*}


太用 Excel 绘制标准正态分布概率密度函数曲线的操作步骤

第 1 步: 在工作表中的第一列 “A1: A61” 输入一个等差数列, 初始值为 “ -3 ", 步长为 “ 0.1 ",终值为 “ 3 ", 作为标准正态变量的值。

第 2 步: 选中 $\mathrm{B} 1$ 单元格, 选择【公式】 $\rightarrow$ 【插入函数】选项, 弹出插入函数的对话框。从【或选择类别 $(\underline{\mathbf{C}}):$ 】窗口中选择 “统计”; 从【选择函数 (N):】窗口中选择 "NORM. DIST",点击【确定】,弹出函数参数【NORM. DIST】对话框。点击【确定】。在【X】选项中输入 A1 (计算 A1 单元格所对应的数据的概率值)。在【Mean】中输入 0 (均值为 0 )。在【Standard\_dev】中输入 1 (标准差为 1 )。在【Cumulative】中输入 0 或 False, 表示计算的是概率密度, 输入 1 或 True, 表示计算的是累积概率密度。

第 3 步: 单击 B1 单元格, 鼠标指向单元格右下角填充控点, 按住鼠标左键往下拖至 B61 单元格,这样计算出 A1:A61 区域正态变量值对应的正态概率密度值。

第 4 步: 选中数据区域如图 5-7 所示, 然后在菜单栏中选择【插入】, 在【图表】选项选中心按钮并单击, 弹出散点图的下列选项, 选中第一行的第二个 “带平滑线的散点图”, 即可得到如图 5 - 8 所示的图形。

\begin{center}
\begin{tabular}{|r|rr|}
\multicolumn{1}{l}{A} & \multicolumn{1}{c}{A} & \multicolumn{1}{c}{B} \\
\cline { 2 - 3 }
1 & -3 & 0.004432 \\
\hline
2 & -2.9 & 0.005953 \\
3 & -2.8 & 0.007915 \\
4 & -2.7 & 0.010421 \\
\hline
5 & -2.6 & 0.013583 \\
6 & -2.5 & 0.017528 \\
\hline
7 & -2.4 & 0.022395 \\
\hline
8 & -2.3 & 0.028327 \\
9 & -2.2 & 0.035475 \\
10 & -2.1 & 0.043984 \\
11 & -2 & 0.053991 \\
12 & -1.9 & 0.065616 \\
13 & -1.8 & 0.07895 \\
14 & -1.7 & 0.094049 \\
15 & -1.6 & 0.110921 \\
\hline
\end{tabular}
\end{center}

C

图 5-7 标准正态分布概率密度函数的数据清单 (部分)

第 5 步: 选中图形中需要调整格式的地方, 可以对图形的标题、坐标轴、网格线和线条的颜色等等进行任意调整, 此处以简单的曲线图展示标准正态分布概率曲线, 如图 5-9 所示。

\begin{center}
\end{center}

图 5-8 标准正态分布概率密度函数曲线草图

\begin{center}
\end{center}

图 5-9 标准正态分布概率密度函数曲线

\section*{例 5.11}
设 $X \sim N\left(5,3^{2}\right)$, 用 Excel 计算概率:

(1) $P(X \leqslant 10)$ 。

(2) $P(2<X<10)$ 。

\section*{(解}
太用 Excel 计算正态分布概率值的操作步骤 (以例 5.11 第 (1) 题为例)

第 1 步: 进入 Excel 界面, 鼠标点击某一空白单元格。

第 2 步: 选中 $\mathrm{B} 1$ 单元格, 选择【公式】 $\rightarrow$ 【插入函数】选项, 弹出插入函数的对话框。从【或选择类别 ( $\underline{C}):$ 】窗口中选择 “统计”; 从【选择函数 (N):】窗口中选择 "NORM. DIST", 点击【确定】, 出现如图 5-10 所示的界面。

\begin{center}
\end{center}

图 5-10 函数参数\\
填入适当的参数, 得出计算结果为 0.9522 。这一结果与前面例子的结果稍有不同, 属近似计算误差。

或者直接在单元格中输入 " $=$ NORM. DIST $(10,5,3$, TRUE $) "$, 也能得到相同的结果。

第 (2) 题在单元格中只需输入 " = NORM. DIST (10, 5, 3, TRUE) - NORM. DIST (2, 5, 3 , TRUE) " 即可。

\section*{4. 正态分布在质量管理中的应用}
在全面质量管理中, 我们知道

$$
P(\mu-3 \sigma<X<\mu+3 \sigma)=\Phi(3)-\Phi(-3)=0.9973
$$

这表明当某质量特性 $X \sim N\left(\mu, \sigma^{2}\right)$ 时, 其特性值落在区间 $(\mu-3 \sigma, \mu+3 \sigma)$ 外的概率仅为 $0.27 \%$ 。这是一个小概率事件, 通常在一次试验中是不会发生的, 一旦发生就认为质量发生了异常, 质量检验和过程控制就是利用这一思想。

当今风靡全球的六西格玛质量管理体系也是在正态分布原理基础上建立的。

当上下公差不变时, $6 \sigma$ 的质量水准就意味着产品合格率达到 $99.9999998 \%$, 即

$$
P(\mu-6 \sigma<X<\mu+6 \sigma)=\Phi(6)-\Phi(-6)=0.999999998
$$

其特性值落在区间 $(\mu-6 \sigma, \mu+6 \sigma)$ 外的概率仅为十亿分之二。

由于种种随机因素的影响, 任何流程在实际运行中都会产生偏离目标值或者期望值的情况, 我们通常把这种偏移称为漂移。

我们知道在传统的 “ $3 \sigma$ 原则” 下, 质量标准的合格率为 $99.73 \%$ 。即使在没有任何漂移的情况下, 也意味着 $2700 \mathrm{ppm}$ (1ppm 表示百万分之一) 的不合格率。考虑到漂移时是 $66807 \mathrm{ppm}$ 。

通常考虑到 $1.5 \sigma$ 漂移时, $6 \sigma$ 就不是十亿分之二次品率, 不合格率为百万分之 3.4 。即在某生产流程或服务系统中有 100 万个出现缺陷的机会, 而 $6 \sigma$ 质量水准下出现的缺陷不到 4 个。更为详细的内容见有关六西格玛管理的文献。

\section*{思考与练习}
\section*{思考题}
5.1 频率与概率有什么关系?

5.2 根据自己的经验体会举几个服从泊松分布的随机变量的实例。

5.3 根据自己的经验体会举几个服从正态分布的随机变量的实例。

\section*{练习题}
5.1 写出下列随机试验的样本空间:

(1) 记录某班一次统计学测验的平均分数。

(2)某人在公路上骑自行车, 观察该骑车人在遇到第一个红灯停下来以前遇到绿灯的次数。

(3) 生产产品直到有 10 件正品为止, 记录生产产品的总件数。

5.2 某人花 2 元钱买彩票,他抽中 100 元奖的概率是 $0.1 \%$, 抽中 10 元奖的概率是 $1 \%$, 抽中 1 元奖的概率是 $20 \%$ 。假设各种奖不能同时抽中, 试求:

(1) 此人收益的概率分布。

(2)此人收益的期望值。

5.3 一张考卷上有 5 道题目, 每道题列出 4 个备选答案, 其中有一个答案是正确的。某学生凭猜测能答对至少 4 道题的概率是多少?

5.4 设随机变量 $X$ 服从参数为 $\lambda$ 的泊松分布, 已知 $P\{X=1\}=P\{X=2\}$, 求 $P\{X=4\}$ 。

5.5 设 $X \sim N(3,4)$, 试求:

(1) $P\{|X|>2\}$ 。

(2) $P\{X>3\}$ 。

5.6 一工厂生产的电子管寿命 $X$ (以小时计算)服从期望值 $\mu=160$ 的正态分布, 若要求 $P\{120<$ $X<200\} \geqslant 0.08$, 允许的标准差 $\sigma$ 最大为多少?

5.7 一本书排版后一校时出现错误数 $X$ 服从正态分布 $N(200,400)$, 试求:

(1) 出现错误数不超过 230 的概率。

(2)出现错误数在 $190 \sim 210$ 之间的概率。\\

借助统计学方法研究任何实际问题, 首先要做的工作就是收集数据。收集数据是一项很重要的基础工作, 一般方法是查阅各种统计年鉴和报表, 或者运用某种调查方法获取欲研究问题的有关数据。通过抽样调查获取数据的方式在我国方兴未艾。抽样调查的方法很多, 专业性很强, 现在已有不少有关抽样技术的专著。在利用统计方法研究的问题中, 通常把所要调查研究的事物或现象的全体称为总体, 而把组成总体的每个元素 (成员) 称为个体, 一个总体中所含个体的数量称为总体的容量。例如, 要研究某城市居民的家庭收入状况, 那么这个城市所有家庭的收入状况就是我们研究的总体, 每个家庭的收入状况就是个体。

为了推断总体的某些特征, 需要采用一定的抽样技术从总体中抽取若干个体, 这一抽取过程称为抽样。所抽取的部分个体称为样本, 样本中所含个体的数量称为样本量。例如, 在研究某城市居民家庭收入时, 随机抽取 1000 户进行调查, 这 1000 户就是一个样本, 样本量就是 1000 。以 $1000 户$ 户居民家庭收入的信息作为这个城市居民家庭收入估计的依据, 这样的估计可靠性如何? 依据的基本统计理论是什么? 这正是本章要解决的问题。\\
通过抽样或查年鉴得到的原始数据一般是杂乱无章的, 很难从中直接看出有价值的东西。因此, 一般需要对获取的原始数据加以整理, 把我们感兴趣的信息提取出来, 并用简明醒目的方式加以表述。绘制原始数据的散点图、饼图、直方图等方法是直观表达数据的常见方式。统计学中最主要的提取信息的方式就是对原始数据进行一定的运算, 得出某些代表性的数字, 以反映数据某些方面的特征, 这种数字称为统计量。用统计学语言表述就是: 统计量是样本的函数, 它不依赖于任何未知参数。推断统计学的重要作用就是, 通过从总体中抽取样本构造适当的统计量, 由样本性质去推断关于总体的性质。本章将简要介绍统计量的概念, 并介绍以正态分布为基础导出的几个重要分布。

\section*{6.1 统计量}
\section*{6. 1.1 统计量的概念}
在实际应用中, 当我们从某总体中抽取一个样本 $\left(X_{1}, X_{2}, \cdots, X_{n}\right)$ 后, 并不能直接用它去对总体的有关性质和特征进行推断, 这是因为样本虽然是从总体中获取的代表,含有总体性质的信息, 但仍比较分散。为了使统计推断成为可能, 首先必须把分散在样本中我们关心的信息集中起来, 针对不同的研究目的构造不同的样本函数, 这种函数在统计学中称为统计量。

定义 6.1 设 $X_{1}, X_{2}, \cdots, X_{n}$ 是从总体 $X$ 中抽取的容量为 $n$ 的一个样本, 如果由此样本构造一个函数 $T\left(X_{1}, X_{2}, \cdots, X_{n}\right)$, 不依赖于任何未知参数, 则称函数 $T\left(X_{1}\right.$, $\left.X_{2}, \cdots, X_{n}\right)$ 是一个统计量。

通常, 又称 $T\left(X_{1}, X_{2}, \cdots, X_{n}\right)$ 为样本统计量。当获得样本的一组具体观测值 $x_{1}$, $x_{2}, \cdots, x_{n}$ 时, 代人 $T$, 计算出 $T\left(x_{1}, x_{2}, \cdots, x_{n}\right)$ 的数值, 就得到一个具体的统计量值。

\section*{例 6.1}
设 $X_{1}, X_{2}, \cdots, X_{n}$ 是从某总体 $X$ 中抽取的一个样本, 则

$$
\begin{aligned}
& \bar{X}=\frac{1}{n} \sum_{i=1}^{n} X_{i} \\
& S^{2}=\frac{1}{n-1} \sum_{i=1}^{n}\left(X_{i}-\bar{X}\right)^{2}
\end{aligned}
$$

都是统计量, 而 $\sum_{i=1}^{n}\left[X_{i}-E(X)\right]^{2},\left[X_{i}-E(X)\right] / D(X)$ 都不是统计量, 这是因为其中的 $E(X)$ 和 $D(X)$ 都是依赖于总体分布的未知参数。

统计量是样本的一个函数。由样本构造具体的统计量, 实际上是对样本所含的总体信息按某种要求进行加工处理, 把分散在样本中的信息集中到统计量的取值上。不同的统计推断问题要求构造不同的统计量。统计量在统计学中具有极其重要的地位, 它是统计推断\\
的基础。统计量在统计学中的地位相当于随机变量在概率论中的地位。

\subsection*{6.1.2 常用统计量}
在一般的概率论教材中, 把数学期望及方差等概念用所谓 “矩” 的概念来描述。当 $n$充分大时, 有定理可以保证经验分布函数 $F_{n}(x)$ 很靠近总体分布函数 $F(X)$ 。因此, 经验分布函数 $F_{n}(x)$ 的各阶矩就反映了总体各阶矩的信息。通常把经验分布函数的各阶矩称为样本各阶矩。常用的样本各阶矩及其函数都是实际应用中的具体统计量。

(1) $\bar{X}=\frac{1}{n} \sum_{i=1}^{n} X_{i}$ 是样本均值, 它反映出总体 $X$ 数学期望的信息。样本均值是最常用的统计量。

(2) $S^{2}=\frac{1}{n-1} \sum_{i=1}^{n}\left(X_{i}-\bar{X}\right)^{2}$ 是样本方差, 它反映的是总体 $X$ 方差的信息。样本方差 $S^{2}$ 及样本标准差 $S$ 也是最常用的统计量。

(3) $V=S / \bar{X}$ 是样本变异系数, 它反映出总体变异系数 $C$ 的信息。其中变异系数定义为 $C=\sqrt{D(X)} / E(X)$, 它反映出随机变量在以它的均值为单位时取值的离散程度。此统计量消除了均值不同对不同总体的离散程度的影响, 常用来刻画均值不同时不同总体的离散程度。它在投资项目的风险分析、不同群体或行业的收人差距描述中有着广泛的应用。

(4) $m_{k}=\frac{1}{n} \sum_{i=1}^{n} X_{i}^{k}$, 称 $m_{k}$ 为样本 $k$ 阶矩。它反映出总体 $k$ 阶矩的信息。显然, $m_{1}=$ $\bar{X}$ 就是样本均值。

(5) $v_{k}=\frac{1}{n-1} \sum_{i=1}^{n}\left(X_{i}-\bar{X}\right)^{k}$, 称 $v_{k}$ 为样本 $k$ 阶中心矩。它反映出总体 $k$ 阶中心矩的信息。显然, $u_{2}$ 就是样本方差。

(6) $\alpha_{3}=\sqrt{n-1} \sum_{i=1}^{n}\left(X_{i}-\bar{X}\right)^{3} /\left(\sum_{i=1}^{n}\left(X_{i}-\bar{X}\right)^{2}\right)^{3 / 2}$, 称 $\alpha_{3}$ 为样本偏度。它反映出总体偏度的信息。偏度反映了随机变量密度函数曲线在众数 (密度函数在这一点达到最大值)两边的偏斜性。如果 $X \sim N\left(\mu, \sigma^{2}\right)$, 则偏度 $\alpha_{3}=0$ 。

(7) $\alpha_{4}=(n-1) \sum_{i=1}^{n}\left(X_{i}-\bar{X}\right)^{4} /\left[\sum_{i=1}^{n}\left(X_{i}-\bar{X}\right)^{2}\right]^{2}-3$, 称 $\alpha_{4}$ 为样本峰度。它反映出总体峰度的信息。峰度反映了密度函数曲线在众数附近的 “峰” 的尖峭程度。正态随机变量 $X \sim N\left(\mu, \sigma^{2}\right)$ 的峰度 $\alpha_{4}=0$ 。

偏度和峰度的概念在质量控制和可靠性研究中有着极其广泛的应用。

\section*{6.2 由正态分布导出的几个重要分布}
\subsection*{6.2. 1 抽样分布}
近代统计学的创始人之一, 英国统计学家费希尔曾把抽样分布、参数估计和假设检验\\
看做统计推断的三个中心内容。研究统计量的性质和评价一个统计推断的优良性, 完全取决于其抽样分布的性质。所以, 抽样分布的研究是统计学中的重要内容。

在总体 $X$ 的分布类型已知时, 若对任一自然数 $n$ 都能导出统计量 $T=T\left(X_{1}, X_{2}, \cdots\right.$, $\left.X_{n}\right)$ 的分布的数学表达式, 这种分布称为精确的抽样分布。它对样本量 $n$ 较小的统计推断问题非常有用。精确的抽样分布大多是在正态总体情况下得到的。在正态总体条件下, 主要有 $\chi^{2}$ 分布、 $t$ 分布、 $F$ 分布, 常称为统计三大分布。

\subsection*{6.2.2 $\chi^{2}$ 分布}
$\chi$ 分布 (Chi-square distribution) 是由赫尔默特(Helmert)和皮尔逊分别于 1875 年和 1900 年推导出来的。

定义 6.2 设随机变量 $X_{1}, X_{2}, \cdots, X_{n}$ 相互独立, 且 $X_{i}(i=1,2, \cdots, n)$ 服从标准正态分布 $N(0,1)$, 则它们的平方和 $\sum_{i=1}^{n} X_{i}^{2}$ 服从自由度为 $n$ 的 $\chi^{2}$ 分布。

自由度是统计学中常用的一个概念, 它可以解释为独立变量的个数, 还可解释为二次型的秩。例如, $Y=X^{2}$ 是自由度为 1 的 $\chi^{2}$ 分布, $\operatorname{rank}(Y)=1 ; Z=\sum_{i=1}^{n} X_{i}^{2}$ 是自由度为 $n$的 $\chi^{2}$ 分布, $\operatorname{rank}(Z)=n$ 。

关于 $\chi^{2}$ 分布的密度函数较为复杂, 非统计专业只关心应用的读者不必了解, 加之篇幅有限, 所以本书不介绍 $\chi^{2}$ 分布、 $t$ 分布、 $F$ 分布的密度函数。有兴趣的读者可参见数学形式比较严谨的有关数理统计图书。

下面给出当 $n=1, n=4, n=10, n=20$ 时, $\chi^{2}$ 分布的概率密度函数曲线, 如图 6-1 所示。

\begin{center}
\end{center}

图 6-1 $x^{2}$ 分布的示意图

$\chi^{2}$ 分布的数学期望为:

$$
E\left(\chi^{2}\right)=n
$$

$\chi^{2}$ 分布的方差为:

$$
D\left(\chi^{2}\right)=2 n
$$

$\chi^{2}$ 分布具有可加性, 即若 $\chi_{1}^{2} \sim \chi^{2}\left(n_{1}\right), \chi_{2}^{2} \sim \chi^{2}\left(n_{2}\right)$, 且独立, 则

$$
\chi_{1}^{2}+\chi_{2}^{2} \sim \chi^{2}\left(n_{1}+n_{2}\right)
$$

由图 6-1 还可看出, 当自由度足够大时, $\chi^{2}$ 分布的概率密度曲线趋于对称。当 $n \rightarrow+\infty$ 时, $\chi^{2}$ 分布的极限分布是正态分布。

$\chi^{2}(n)$ 的 $p$ 分位数 $\chi_{p}^{2}(n)$ 可由卡方分布表查得。当自由度 $n$ 很大时, $\sqrt{2 \chi^{2}(n)}$ 近似服从 $N(\sqrt{2 n-1}, 1)$ 。实际上, 当自由度 $n>45$ 时, 有


\begin{equation*}
\chi_{p}^{2}(n) \approx \frac{1}{2}\left(\mu_{p}+\sqrt{2 n-1}\right)^{2} \tag{6.1}
\end{equation*}


式中, $\mu_{p}$ 即 $Z_{p}$, 为正态 $p$ 分位数, 可由正态分布表查得。

\section*{6. $2.3 t$ 分布}
$t$ 分布 ( $t$ distribution) 也称为学生氏分布, 是戈塞特 (W. S. Gosset) 于 1908 年在一篇以 “Student” (学生) 为笔名的论文中首次提出的。

定义 6.3 设随机变量 $X \sim N(0,1), Y \sim \chi^{2}(n)$, 且 $X$ 与 $Y$ 独立, 则


\begin{equation*}
t=\frac{X}{\sqrt{Y / n}} \tag{6.2}
\end{equation*}


其分布称为 $t$ 分布, 记为 $t(n)$, 其中, $n$ 为自由度。

$t$ 分布的密度函数是一偶函数, 其图形如图 6-2 所示。

\begin{center}
\end{center}

图 6-2 $t$ 分布的示意图

当 $n \geqslant 2$ 时, $t$ 分布的数学期望 $E(t)=0$ 。

当 $n \geqslant 3$ 时, $t$ 分布的方差 $D(t)=\frac{n}{n-2}$ 。

由图 6-2 可以看出, $t$ 分布的密度函数曲线与标准正态分布 $N(0,1)$ 的密度函数曲线非常相似, 都是单峰偶函数。只是 $t(n)$ 的密度函数的两侧尾部要比 $N(0,1)$ 的两侧尾部粗一些。 $t(n)$ 的方差比 $N(0,1)$ 的方差大一些。

自由度为 1 的分布称为柯西分布, 随着自由度 $n$ 的增加, $t$ 分布的密度函数越来越接近标准正态分布的密度函数。实际应用中, 一般当 $n \geqslant 30$ 时, $t$ 分布与标准正态分布就非\\
常接近。 $t$ 分布的诞生对于统计学中小样本理论及其应用有着重要的促进作用。特别是当戈塞特最初提出 $t$ 分布时并不被人们重视和接受。后来费希尔在农业实验中也遇到小样本问题, 这才发现 $t$ 分布有实用价值。1923 年, 费希尔对 $t$ 分布给出严格而简单的证明, 1925 年编制出 $t$ 分布表之后, 戈塞特的小样本方法才被统计界广泛认可。

下面看一个与 $t$ 分布有关的抽样分布。

设 $X_{1}, X_{2}, \cdots, X_{n}$ 是来自正态分布 $N\left(\mu, \sigma^{2}\right)$ 的一个样本, $\bar{X}=\frac{1}{n} \sum_{i=1}^{n} X_{i}, S^{2}=\frac{1}{n-1}$ $\sum_{i=1}^{n}\left(X_{i}-\bar{X}\right)^{2}$, 则


\begin{equation*}
\frac{\sqrt{n}(\bar{X}-\mu)}{S} \sim t(n-1) \tag{6.3}
\end{equation*}


称为服从自由度为 $(n-1)$ 的 $t$ 分布。

设 $X$ 和 $Y$ 是两个相互独立的总体, $X \sim N\left(\mu_{1}, \sigma^{2}\right), Y \sim N\left(\mu_{2}, \sigma^{2}\right), X_{1}, X_{2}, \cdots$, $X_{n}$ 是来自 $X$ 的一个样本, $Y_{1}, Y_{2}, \cdots, Y_{m}$ 是来自 $Y$ 的一个样本, 记


\begin{align*}
& \bar{X}=\frac{1}{n} \sum_{i=1}^{n} X_{i} \\
& \bar{Y}=\frac{1}{m} \sum_{i=1}^{m} Y_{i} \\
& S_{x}^{2}=\frac{1}{n-1} \sum_{i=1}^{n}\left(X_{i}-\bar{X}\right)^{2} \\
& S_{y}^{2}=\frac{1}{m-1} \sum_{i=1}^{m}\left(Y_{i}-\bar{Y}\right)^{2} \\
& S_{x y}^{2}=\frac{(n-1) S_{x}^{2}+(m-1) S_{y}^{2}}{n+m-2} \\
& \frac{(\bar{X}-\bar{Y})-\left(\mu_{1}-\mu_{2}\right)}{S_{x y}} \sqrt{\frac{m n}{m+n}} \sim t(n+m-2) \tag{6.4}
\end{align*}


则

\subsection*{6.2.4 $F$ 分布}
$F$ 分布 ( $F$ distribution) 是统计学家费希尔首先提出的。 $F$ 分布有着广泛的应用, 在方差分析、回归方程的显著性检验中有重要的地位。

定义 6.4 设随机变量 $Y$ 与 $Z$ 相互独立, 且 $Y$ 和 $Z$ 分别服从自由度为 $m$ 和 $n$ 的 $\chi^{2}$ 分布,随机变量 $X$ 有如下表达式:


\begin{equation*}
X=\frac{Y / m}{Z / n}=\frac{n Y}{m Z} \tag{6.5}
\end{equation*}


则称 $X$ 服从第一自由度为 $m$, 第二自由度为 $n$ 的 $F$ 分布, 记为 $F(m, n)$, 简记为 $X \sim$ $F(m, n) 。 F$ 分布的密度函数的图形如图 6-3 所示。

设随机变量 $X$ 服从 $F(m, n)$ 分布, 则数学期望和方差分别为:


\begin{equation*}
E(X)=\frac{n}{n-2}, \quad n>2 \tag{6.6}
\end{equation*}


\begin{center}
\end{center}

图 6-3 $F$ 分布的密度函数示意图


\begin{equation*}
D(X)=\frac{2 n^{2}(m+n-2)}{m(n-2)(n-4)}, \quad n>4 \tag{6.7}
\end{equation*}


$F$ 分布的 $p$ 分位数 $F_{p}\left(v_{1}, v_{2}\right)$ 可查 $F$ 分布表获得, 且


\begin{equation*}
F_{p}\left(v_{1}, v_{2}\right)=\frac{1}{F_{1-p}\left(v_{2}, v_{1}\right)} \tag{6.8}
\end{equation*}


由此可知, 在 $F$ 分布中, 两个自由度的位置不可互换。这一性质在查 $F$ 分布表时有重要应用。

$F$ 分布与 $t$ 分布还存在如下关系: 如果随机变量 $X$ 服从 $t(n)$ 分布, 则 $X^{2}$ 服从 $F(1, n)$的 $F$ 分布。这在回归分析的回归系数显著性检验中有用。

\section*{6.3 样本均值的分布与中心极限定理}
设 $X_{1}, X_{2}, \cdots, X_{n}$ 为从某一总体中抽出的随机样本, 由以上讨论可知 $X_{1}, X_{2}, \cdots$, $X_{n}$ 为互相独立且与总体有相同分布的随机变量。要知道 $\bar{X}$ 的分布, 必须知道总体的分布。由于正态分布是最常见的分布之一, 所以我们主要介绍在总体分布为正态分布 $N\left(\mu, \sigma^{2}\right)$时样本均值 $\bar{X}$ 的分布。

当总体分布为正态分布 $N\left(\mu, \sigma^{2}\right)$ 时, 可以得到下面的结果:

$\bar{X}$ 的抽样分布 (sampling distribution) 仍为正态分布, $\bar{X}$ 的数学期望为 $\mu$, 方差为 $\sigma^{2} / n$, 则


\begin{equation*}
\bar{X} \sim N\left(\mu, \frac{\sigma^{2}}{n}\right) \tag{6.9}
\end{equation*}


上面的结果表明, $\bar{X}$ 的期望值与总体均值相同, 而方差则缩小为总体方差的 $1 / n$ 。这说明当用样本均值 $\bar{X}$ 去估计总体均值 $\mu$ 时, 平均来说没有偏差(这一点称为无偏性); 当 $n$ 越来越大时, $\bar{X}$ 的离散程度越来越小, 即用 $\bar{X}$ 估计 $\mu$ 越来越准确。

然而在实际问题中, 总体的分布并不总是正态分布或近似正态分布, 此时 $\bar{X}$ 的分布\\
将取决于总体分布的情况。值得庆幸的是, 当抽样个数 $n$ 比较大时, 人们证明了如下的中心极限定理。该定理告诉我们不管总体的分布是什么, 样本均值 $\bar{X}$ 的分布总是近似正态分布, 只要总体的方差 $\sigma^{2}$ 有限。因为无论是什么总体分布, 设总体均值为 $\mu$, 总体方差为 $\sigma^{2}$, 总有


\begin{align*}
& E(\bar{X})=E\left(\frac{1}{n} \sum_{i=1}^{n} X_{i}\right)=\frac{1}{n} \sum_{i=1}^{n} E\left(X_{i}\right)=\mu  \tag{6.10}\\
& D(\bar{X})=D\left(\frac{1}{n} \sum_{i=1}^{n} X_{i}\right)=\frac{1}{n^{2}} \sum_{i=1}^{n} D\left(X_{i}\right)=\frac{\sigma^{2}}{n} \tag{6.11}
\end{align*}


所以当 $n$ 比较大时, $\bar{X}$ 近似服从 $N\left(\mu, \frac{\sigma^{2}}{n}\right)$, 等价地有 $\frac{\bar{X}-\mu}{\sigma / \sqrt{n}} \sim N(0,1)$ 。

中心极限定理 (central limit theorem) : 设从均值为 $\mu$ 、方差为 $\sigma^{2}$ (有限) 的任意一个总体中抽取样本量为 $n$ 的样本, 当 $n$ 充分大时, 样本均值 $\bar{X}$ 的抽样分布近似服从均值为 $\mu$ 、方差为 $\sigma^{2} / n$ 的正态分布。

图 6-4 描述的是 $\bar{X}$ 的抽样分布趋于正态分布的过程。\\

图 6-4 $\bar{X}$ 的抽样分布趋于正态分布的过程\\
中心极限定理要求 $n$ 必须充分大, 那么多大才叫充分大呢? 这与总体的分布形状有关。总体偏离正态越远, 则要求 $n$ 越大。然而在实际应用中, 总体的分布未知。此时, 我们常要求 $n \geqslant 30$ 。顺便指出, 大样本、小样本之间并不是以样本量大小来区分的。在样本量固定的条件下所进行的统计推断、问题分析, 不管样本量有多大, 都称为小样本问题;而在样本量 $n \rightarrow \infty$ 的条件下进行的统计推断、问题分析则称为大样本问题。一般统计学中的 $n \geqslant 30$ 为大样本, $n<30$ 为小样本只是一种经验说法。

在统计学中, 由于正态分布有着十分重要的地位, 因此常把证明其极限分布为正态分布的定理统称为中心极限定理。最早的中心极限定理是在 18 世纪初由德莫佛所证明的,即二项分布以正态分布为其极限分布的定理。现在叙述的中心极限定理是 20 世纪 20 年代林德伯格和勒维证明的在任意分布的总体中抽取样本, 其样本均值的极限分布为正态分布的定理。

\section*{例 6.2}
设从一个均值 $\mu=10$ 、标准差 $\sigma=0.6$ 的总体中随机选取容量 $n=36$ 的样本。假定该总体不是很偏, 要求:

(1) 计算样本均值 $\bar{X}$ 小于 9.9 的近似概率。

(2) 计算样本均值 $\bar{X}$ 超过 9.9 的近似概率。

(3)计算样本均值 $\bar{X}$ 在总体均值 $\mu=10$ 附近 0.1 范围内的近似概率。

一繲 根据中心极限定理, 不论总体的分布是什么形状, 在假定总体分布不是很偏的情形下, 当从总体中随机选取 $n=36$ 的样本时, 样本均值 $\bar{X}$ 近似服从均值 $\mu_{X}=\mu=10$ 、标准差 $\sigma_{X}=\frac{\sigma}{\sqrt{n}}=\frac{0.6}{\sqrt{36}}=0.1$ 的正态分布, 即

$$
\bar{X} \sim N\left(10,0.1^{2}\right)
$$

(1) $P(\bar{X}<9.9)=P\left(\frac{\bar{X}-10}{0.1}<\frac{9.9-10}{0.1}\right)$

$$
\begin{aligned}
& =P\left(Z<\frac{-0.1}{0.1}\right)=P(Z<-1) \\
& =1-P(Z<1)=1-\Phi(1) \\
& =1-0.8413 \\
& =0.1587
\end{aligned}
$$

(2) $P(\bar{X}>9.9)=1-P(\bar{X} \leqslant 9.9)$

$=1-0.1587$

$=0.8413$

(3) $P(9.9<\bar{X}<10.1)=P\left(\frac{9.9-10}{0.1}<\frac{\bar{X}-10}{0.1}<\frac{10.1-10}{0.1}\right)$

$$
=P\left(Z<\frac{10.1-10}{0.1}\right)-P\left(Z<\frac{9.9-10}{0.1}\right)
$$

$$
\begin{aligned}
& =P(Z<1)-P(Z<-1) \\
& =2 P(Z<1)-1=2 \Phi(1)-1 \\
& =2 \times 0.8413-1 \\
& =0.6826
\end{aligned}
$$

\section*{例 6.3}
某汽车电瓶生产厂声称其生产的电瓶具有均值为 60 个月、标准差为 6 个月的寿命分布。现假设质检部门决定检验该厂的说法是否准确, 为此随机抽取了 50 个该厂生产的电瓶进行寿命试验。

(1)假定厂方说法是正确的,试描述 50 个电瓶的平均寿命的抽样分布。

(2)假定厂方说法是正确的,则 50 个样品组成的样本的平均寿命不超过 57 个月的概率为多少?

(1)尽管我们对电瓶的寿命分布的形状不甚了解, 但根据中心极限定理可以推出 50 个电瓶的平均寿命近似服从正态分布, 其均值 $\mu_{X}=\mu=60$, 方差 $\sigma_{X}^{2}=\frac{\sigma_{X}^{2}}{n}=\frac{6^{2}}{50}=0.72$, $\sigma_{X}=\sqrt{\sigma_{X}^{2}}=\sqrt{0.72}=0.85$, 即

$$
\bar{X} \sim N\left(60,0.85^{2}\right)
$$

(2)如果厂方说法是正确的,则观察到 50 个电瓶的平均寿命不超过 57 个月的概率为:

$$
\begin{aligned}
P(\bar{X} \leqslant 57) & =P\left(\frac{\bar{X}-60}{0.85} \leqslant \frac{57-60}{0.85}\right)=P\left(Z \leqslant \frac{57-60}{0.85}\right) \\
& =P(Z \leqslant-3.529)=1-P(Z \leqslant 3.529) \\
& =1-\Phi(3.529)=1-0.9998=0.0002
\end{aligned}
$$

如果厂方的说法正确, 则 50 个电瓶的平均寿命不超过 57 个月的概率为 0.0002 。这是一个不可能事件。根据小概率事件原理, 观察到 50 个电瓶的平均寿命小于或等于 57 个月的事件是不可能的; 反之, 如果真的观察到 50 个电瓶的平均寿命低于 57 个月, 则有理由怀疑厂方说法的正确性, 即可认为厂方的说法是不正确的。

\section*{是考与练习}
\section*{思考题}
6.1 什么是统计量? 为什么要引进统计量? 统计量中为什么不含任何未知参数?

6.2 判断下列样本函数中哪些是统计量? 哪些不是统计量?

$T_{1}=\left(X_{1}+X_{2}+\cdots+X_{10}\right) / 10$

$T_{2}=\min \left(X_{1}, X_{2}, \cdots, X_{10}\right)$

$$
\begin{aligned}
& T_{3}=X_{10}-\mu \\
& T_{4}=\left(X_{10}-\mu\right) / \sigma
\end{aligned}
$$

6.3 简述 $\chi^{2}$ 分布、 $t$ 分布、 $F$ 分布及正态分布之间的关系。

6.4 什么是抽样分布?

6.5 简述中心极限定理的意义。

\section*{二 练习题}
调节一台装瓶机使其对每个瓶子的灌装量均值为 $\mu$ 典司, 通过观察发现这台装瓶机对每个瓶子的灌装量服从标准差 $\sigma=1.0$ 典司的正态分布。随机抽取由这台机器灌装的 9 个瓶子组成一个样本, 并测定每个瓶子的灌装量。试确定样本均值偏离总体均值不超过 0.3 典司的概率。

\begin{center}
\end{center}

\section*{参数估计}
\begin{center}
\end{center}

\section*{“双十一”期间大学生平均消费支出有多少?}
随着电子商务的发展, 网上购物逐渐成为人们生活中的重要消费方式。年轻的大学生群体无疑是网络购物的重要消费群体。网购给人们带来便捷的同时, 也造成了某些超额消费现象, 进而催生了各类网络信贷平台。大学生便是网络信贷平台的重要使用者。 2020 年 11 月, 中国人民大学财经学院的几个学生以 “双十一” 购物促销为背景, 通过问卷调查获得大学生 “双十一” 期间网络购物的消费及信贷情况的相关数据, 根据调查数据作了多角度的分析。其中的一个结论是: “双十一” 期间女生平均消费金额的 95\% 的置信区间为 $1063.39 \sim 1501.99$ 元,男生平均消费金额的 $95 \%$ 的置信区间为 $496.93 \sim$ 814.15 元。

什么是置信区间? 上面的置信区间是如何得出来的? $95 \%$ 的置信区间意味着什么?本章就将回答这些问题。\\
参数估计是推断统计的重要内容之一。它是在抽样及抽样分布的基础上, 根据样本统计量来推断所关心的总体参数。本章将介绍参数估计的基本方法, 包括一个总体参数的估计和两个总体参数的估计, 最后介绍参数估计中样本量的确定问题。

\section*{7.1 参数估计的基本原理}
\subsection*{7.1. 1 估计量与估计值}
如果能够掌握总体的全部数据, 那么只需要做一些简单的统计描述就可以得到所关心的总体特征, 比如, 总体均值、方差、比例等。但现实情况比较复杂, 有些现象的范围比较广, 不可能对总体中的每个单位都进行测定。或者有些总体的个体数很多, 不可能也没有必要进行一一测定。这就需要从总体中抽取一部分个体进行调查, 利用样本提供的信息来推断总体的特征。

参数估计 (parameter estimation) 就是用样本统计量去估计总体的参数。比如, 用样本均值 $\bar{x}$ 估计总体均值 $\mu$, 用样本比例 $p$ 估计总体比例 $\pi$, 用样本方差 $s^{2}$ 估计总体方差 $\sigma^{2}$, 等等。如果将总体参数笼统地用一个符号 $\theta$ 来表示, 用于估计总体参数的统计量用 $\hat{\theta}$表示, 参数估计也就是如何用 $\hat{\theta}$ 来估计 $\theta$ 。

在参数估计中, 用来估计总体参数的统计量称为估计量 (estimator), 用符号 $\hat{\theta}$ 表示。样本均值、样本比例、样本方差等都可以是一个估计量。根据一个具体的样本计算出来的估计量的数值称为估计值 (estimated value)。比如, 要估计一个班学生考试的平均分数,从中抽取一个随机样本, 全班的平均分数是不知道的, 称为参数, 用 $\theta$ 表示, 根据样本计算的平均分数 $\bar{x}$ 就是一个估计量, 用 $\hat{\theta}$ 表示。假定计算出来的样本平均分数为 80 分, 这个 80 分就是估计量的具体数值, 称为估计值。

\subsection*{7.1.2 点估计与区间估计}
参数估计的方法有点估计和区间估计两种。

\section*{1. 点估计}
点估计 (point estimate) 就是用样本统计量 $\hat{\theta}$ 的某个取值直接作为总体参数 $\theta$ 的估计值。比如, 用样本均值 $\bar{x}$ 直接作为总体均值 $\mu$ 的估计值, 用样本比例 $p$ 直接作为总体比例 $\pi$ 的估计值, 用样本方差 $s^{2}$ 直接作为总体方差 $\sigma^{2}$ 的估计值, 等等。假定要估计一个班学生考试成绩的平均分数, 根据抽出的一个随机样本计算的平均分数为 80 分, 用 80 分作为全班平均考试分数的一个估计值, 这就是点估计。再比如, 若要估计一批产品的合格率,抽样结果合格率为 $96 \%$, 将 $96 \%$ 直接作为这批产品合格率的估计值, 这也是一个点估计。

虽然在重复抽样条件下点估计的均值有望等于总体真值 (比如, $E(\bar{x})=\mu$ ), 但由于样本是随机的, 抽出一个具体的样本得到的估计值很可能不同于总体真值。在用点估计值\\
代表总体参数值的同时, 还必须给出点估计值的可靠性, 也就是说, 必须能说出点估计值与总体参数的真实值接近的程度。但一个点估计值的可靠性是由它的抽样标准误差来衡量的, 这表明一个具体的点估计值无法给出估计的可靠性的度量, 因此就不能完全依赖于一个点估计值,而是围绕点估计值构造总体参数的一个区间,这就是区间估计。

\section*{2. 区间估计}
假定参数是射击靶上 10 环的位置, 进行一次射击, 打在靶心 10 环的位置上的可能性很小, 但打在靶子上的可能性很大, 用打在靶上的这个点画出一个区间, 这个区间包含靰心的可能性就很大,这就是区间估计的基本思想。

区间估计(interval estimate)是在点估计的基础上,给出总体参数估计的一个区间范围, 该区间通常由样本统计量加减估计误差得到。与点估计不同, 进行区间估计时, 根据样本统计量的抽样分布可以对样本统计量与总体参数的接近程度给出一个概率度量。下面将以总体均值的区间估计为例来说明区间估计的基本原理。

由样本均值的抽样分布可知, 在重复抽样或无限总体抽样的情况下, 样本均值的数学期望等于总体均值, 即 $E(\bar{x})=\mu$, 样本均值的标准误差为 $\sigma_{x}=\sigma / \sqrt{n}$, 由此可知样本均值 $\bar{x}$落在总体均值 $\mu$ 的两侧各为 1 个抽样标准差范围内的概率为 0.6827 ; 落在 2 个抽样标准差范围内的概率为 0.9545 ; 落在 3 个抽样标准差范围内的概率为 0.9973 ; 等等。

实际上, 可以求出样本均值 $\bar{x}$ 落在总体均值 $\mu$ 的两侧任何一个抽样标准差范围内的概率。但实际估计时, 情况恰好相反。 $\bar{x}$ 是已知的, 而 $\mu$ 是未知的, 也正是将要估计的。由于 $\bar{x}$ 与 $\mu$ 的距离是对称的, 如果某个样本的平均值落在 $\mu$ 的 2 个标准差范围之内, 反过来, $\mu$ 也就被包含在以 $\bar{x}$ 为中心左右 2 个标准差的范围之内。因此约有 $95 \%$ 的样本均值会落在 $\mu$ 的 2 个标准差的范围之内。也就是说, 约有 $95 \%$ 的样本均值所构造的 2 个标准差的区间会包含 $\mu_{0}$ 通俗地说, 如果抽取 100 个样本来估计总体均值, 由 100 个样本所构造的 100 个区间中,约有 95 个区间包含总体均值,另外 5 个区间则不包含总体均值。图 7-1 给出了区间估计的示意图。

在区间估计中, 由样本统计量所构造的总体参数的估计区间称为置信区间 (confidence interval), 其中, 区间的最小值称为置信下限, 最大值称为置信上限。由于统计学家在某种程度上确信这个区间会包含真正的总体参数, 所以给它取名为置信区间。原因是, 如果抽取了许多不同的样本, 比如说抽取 100 个样本, 根据每个样本构造一个置信区间, 这样, 由 100 个样本构造的总体参数的 100 个置信区间中, 有 $95 \%$ 的区间包含总体参数的真值, 有 $5 \%$ 没包含, 则 $95 \%$ 这个值称为置信水平。一般地, 如果将构造置信区间的步骤重复多次, 置信区间中包含总体参数真值的次数所占的比例称为置信水平 (confidence level), 也称为置信度或置信系数 (confidence coefficient)。

在构造置信区间时, 可以用所希望的任意值作为置信水平。比较常用的置信水平及正态分布曲线下右侧面积为 $\alpha / 2$ 时的 $z$ 值 $\left(z_{\alpha / 2}\right)$ 如表 $7-1$ 所示。\\

图 7-1 区间估计示意图

表 7-1 常用置信水平的 $z_{\alpha / 2}$ 值

\begin{center}
\begin{tabular}{cccc}
\hline
置信水平 & $\alpha$ & $\alpha / 2$ & $z_{\alpha / 2}$ \\
\hline
$90 \%$ & 0.10 & 0.05 & 1.645 \\
$95 \%$ & 0.05 & 0.025 & 1.96 \\
$99 \%$ & 0.01 & 0.005 & 2.58 \\
\hline
\end{tabular}
\end{center}

有关置信区间的概念可用图 7-2 来表示。

\begin{center}
\end{center}

图 7-2 置信区间示意图

从图 7-1 和图 7-2 不难看出, 当样本量给定时, 置信区间的宽度随着置信系数的增大而增大。从直觉上说, 区间比较宽时, 才会使这一区间有更大的可能性包含参数的真值; 当置信水平固定时, 置信区间的宽度随样本量的增大而减小, 换言之, 较大的样本所提供的有关总体的信息要比较小的样本多。

对置信区间的理解,有以下几点需要注意:

(1)如果用某种方法构造的所有区间中有 $95 \%$ 的区间包含总体参数的真值, $5 \%$ 的区间不包含, 那么, 用该方法构造的区间称为置信水平为 $95 \%$ 的置信区间。同样, 其他置信水平的区间也可以用类似的方式进行表述。

(2)总体参数的真值是固定的、未知的, 而用样本构造的区间则是不固定的。若抽取不同的样本, 可以得到不同的区间, 从这个意义上说, 置信区间是一个随机区间, 它会因样本的不同而不同, 而且不是所有的区间都包含总体参数的真值。一个置信区间就像是为捕获未知参数而撒出去的网, 不是所有撒网的地点都能捕获到参数。

(3) 在实际问题中, 进行估计时往往只抽取一个样本, 此时所构造的是与该样本相联系的一定置信水平(比如 $95 \%$ ) 下的置信区间。由于用该样本所构造的区间是一个特定的区间, 而不再是随机区间, 所以无法知道这个样本所产生的区间是否包含总体参数的真值。我们只能希望这个区间是大量包含总体参数真值的区间中的一个, 但它也可能是少数几个不包含参数真值的区间中的一个。比如, 从一个总体中抽取 20 个随机样本, 得到总体均值 $\mu$ 的 20 个估计区间,如图 7-3 所示。图中每个区间中间的点表示 $\mu$ 的点估计值,即样本均值 $\bar{x}$ 。可以看出 20 个区间中只有第 8 个区间没有包含总体均值 $\mu$ 。如果这是 $95 \%$的置信区间,那么只有 $5 \%$ 的区间没有包含 $\mu$ 。

\begin{center}
\end{center}

图 7-3 重复构造出的 $\mu$ 的 20 个置信区间

比如, 用 $95 \%$ 的置信水平得到某班学生考试成绩的置信区间为 $60 \sim 80$ 分, 我们不能说 60 80 分这个区间以 $95 \%$ 的概率包含全班学生平均考试成绩的真值, 或者说全班学生的平均考试成绩以 $95 \%$ 的概率落在 $60 \sim 80$ 分之间, 这类表述是错误的, 因为总体均值 $\mu$是一个常数, 而不是一个随机变量。 $\mu$ 要么落在这个范围内, 要么不落在这个范围内, 这里并不涉及概率。我们只是知道, 在多次抽样中有 $95 \%$ 的样本得到的区间包含全班学生平均考试成绩的真值。它的真正意义是如果做了 100 次抽样, 大概有 95 次找到的区间包含真值,有 5 次找到的区间不包含真值。假定全班考试成绩平均数的真值为 70 分, $60 \sim 80$分这个区间一定包含真值, 如果全班考试成绩平均数的真值为 50 分, 那么区间 $60 \sim 80$ 分就绝对不包含真值,无论做多少次试验。因此,这个概率不是用来描述某个特定的区间包含总体参数真值的可能性, 而是针对随机区间而言的。一个特定的区间 “总是包含” 或 “绝对不包含” 参数的真值, 不存在 “以多大的概率包含总体参数” 的问题。但是, 用概率可以知道在多次抽样得到的区间中大概有多少个区间包含参数的真值。

\section*{7. 1.3 评价估计量的标准}
参数估计是用样本估计量 $\hat{\theta}$ 作为总体参数 $\theta$ 的估计。实际上, 用于估计 $\theta$ 的估计量有\\
很多, 比如, 可以用样本均值作为总体均值的估计量, 也可以用样本中位数作为总体均值的估计量, 等等。那么, 究竟用样本的哪种估计量作为总体参数的估计呢? 自然要用估计效果最好的那种估计量。什么样的估计量才算是一个好的估计量呢? 这就需要有一定的评价标准。统计学家给出了评价估计量的一些标准, 主要有以下几个。

\section*{1. 无偏性}
无偏性(unbiasedness)是指估计量抽样分布的数学期望等于被估计的总体参数。设总体参数为 $\theta$, 所选择的估计量为 $\hat{\theta}$, 如果 $E(\hat{\theta})=\theta$, 则称 $\hat{\theta}$ 为 $\theta$ 的无偏估计量。

图 7-4 给出了点估计量无偏和有偏的情形。

\begin{center}
\end{center}

(a) 无偏估计量

\begin{center}
\end{center}

(b) 有偏估计量

图 7-4 无偏和有偏估计量

根据样本均值的抽样分布可知, $E(\bar{x})=\mu, E(p)=\pi$, 同样可以证明, $E\left(s^{2}\right)=\sigma^{2}$, 因此 $\bar{x}, p, s^{2}$ 分别是总体均值 $\mu$, 总体比例 $\pi$, 总体方差 $\sigma^{2}$ 的无偏估计量。

\section*{2. 有效性}
一个无偏的估计量并不意味着它就非常接近被估计的参数, 它还必须与总体参数的离散程度较小。有效性 (efficiency) 是指用于估计同一总体参数的两个无偏估计量, 有更小标准差的估计量更有效。假定有两个用于估计总体参数的无偏估计量, 分别用 $\hat{\theta}_{1}$ 和 $\hat{\theta}_{2}$ 表示, 它们的抽样分布的方差分别用 $D\left(\hat{\theta}_{1}\right)$ 和 $D\left(\hat{\theta}_{2}\right)$ 表示, 如果 $\hat{\theta}_{1}$ 的方差小于 $\hat{\theta}_{2}$ 的方差,即 $D\left(\hat{\theta}_{1}\right)<D\left(\hat{\theta}_{2}\right)$, 就称 $\hat{\theta}_{1}$ 是比 $\hat{\theta}_{2}$ 更有效的一个估计量。在无偏估计的条件下, 估计量的方差越小, 估计就越有效。

图 7-5 说明了两个无偏估计量 $\hat{\theta}_{1}$ 和 $\hat{\theta}_{2}$ 的抽样分布。可以看到, $\hat{\theta}_{1}$ 的方差比 $\hat{\theta}_{2}$ 的方差小, 因此 $\hat{\theta}_{1}$ 的值比 $\hat{\theta}_{2}$ 的值更接近总体的参数, 即 $\hat{\theta}_{1}$ 比 $\hat{\theta}_{2}$ 更有效, 是一个更好的估计量。

\section*{3. 一致性}
一致性 (consistency) 是指随着样本量的增大, 估计量的值越来越接近被估计总体的参数。换言之, 一个大样本给出的估计量要比一个小样本给出的估计量更接近总体的参数。根据样本均值的抽样分布可知, 样本均值抽样分布的标准差为 $\sigma_{x}=\sigma / \sqrt{n}$ 。由于 $\sigma_{x}$ 与\\
样本量大小有关, 样本量越大, $\sigma_{\bar{x}}$ 的值就越小, 因此可以说, 大样本量给出的估计量更接近总体均值 $\mu$ 。从这个意义上说, 样本均值是总体均值的一个一致估计量。对于一致性,可以用图 7-6 直观地说明它的意义。

\begin{center}
\end{center}

图 7-5 两个无偏估计量的抽样分布

\begin{center}
\end{center}

图 7-6 两个不同样本量的样本统计量的抽样分布

\section*{7.2 一个总体参数的区间估计}
研究一个总体时, 所关心的参数主要有总体均值 $\mu$ 、总体比例 $\pi$ 和总体方差 $\sigma^{2}$ 等。本节将介绍如何用样本统计量来构造一个总体参数的置信区间。

\subsection*{7.2. 1 总体均值的区间估计}
在对总体均值进行区间估计时, 需要考虑总体是否为正态分布, 总体方差是否已知,用于构造估计量的样本是大样本(通常要求 $n \geqslant 30 )$ 还是小样本 $(n<30)$ 等几种情况。

\begin{enumerate}
  \item 正态总体、方差已知, 或非正态总体、大样本
\end{enumerate}

当总体服从正态分布且 $\sigma^{2}$ 已知时, 或者总体不是正态分布但为大样本时, 样本均值 $\bar{x}$的抽样分布均为正态分布, 其数学期望为总体均值 $\mu$, 方差为 $\sigma^{2} / n$ 。而样本均值经过标准化以后的随机变量服从标准正态分布, 即


\begin{equation*}
z=\frac{\bar{x}-\mu}{\sigma / \sqrt{n}} \sim N(0,1) \tag{7.1}
\end{equation*}


根据式 (7.1) 和正态分布的性质可以得出, 总体均值 $\mu$ 在 $1-\alpha$ 置信水平下的置信区间为:


\begin{equation*}
\bar{x} \pm z_{\alpha / 2} \frac{\sigma}{\sqrt{n}} \tag{7.2}
\end{equation*}


式中, $\bar{x}-z_{\alpha / 2} \frac{\sigma}{\sqrt{n}}$ 称为置信下限, $\bar{x}+z_{\alpha / 2} \frac{\sigma}{\sqrt{n}}$ 称为置信上限; $\alpha$ 是事先确定的一个概率值,也称为风险值, 它是总体均值不包含在置信区间内的概率; $1-\alpha$ 称为置信水平; $z_{\alpha / 2}$ 是标准正态分布右侧面积为 $\alpha / 2$ 时的 $z$ 值; $z_{\alpha / 2} \frac{\sigma}{\sqrt{n}}$ 是估计总体均值时的估计误差 (estimate\\
error)。这就是说, 总体均值的置信区间由两个部分组成: 点估计值和描述估计量精度的士值, 这个土值称为估计误差。

如果总体服从正态分布但 $\sigma^{2}$ 未知, 或总体并不服从正态分布, 只要是在大样本条件下, 式 (7.1) 中的总体方差 $\sigma^{2}$ 就可以用样本方差 $s^{2}$ 代替, 这时总体均值 $\mu$ 在 $1-\alpha$ 置信水平下的置信区间可以写为:


\begin{equation*}
\bar{x} \pm z_{\alpha / 2} \frac{s}{\sqrt{n}} \tag{7.3}
\end{equation*}


\section*{例 7.1}
一家食品生产企业以生产袋装食品为主, 每天的产量大约为 8000 袋。按规定每袋的重量应为 100 克。为对产品重量进行监测, 企业质检部门经常要进行抽检, 以分析每袋重量是否符合要求。现从某天生产的一批食品中随机抽取 25 袋, 测得每袋重量如表 7-2 所示。

表 7-2 25 袋食品的重量

单位: 克

\begin{center}
\begin{tabular}{rrrrr}
\hline
112.5 & 101.0 & 103.0 & 102.0 & 100.5 \\
102.6 & 107.5 & 95.0 & 108.8 & 115.6 \\
100.0 & 123.5 & 102.0 & 101.6 & 102.2 \\
116.6 & 95.4 & 97.8 & 108.6 & 105.0 \\
136.8 & 102.8 & 101.5 & 98.4 & 93.3 \\
\hline
\end{tabular}
\end{center}

已知产品重量服从正态分布, 且总体标准差为 10 克。试估计该天产品平均重量的置信区间,置信水平为 $95 \%$ 。

一・解 已知 $\sigma=10, n=25$, 置信水平 $1-\alpha=95 \%$, 查标准正态分布表得 $z_{\alpha / 2}=1.96$ 。

根据样本数据计算的样本均值为:

$$
\bar{x}=\frac{\sum_{i=1}^{n} x_{i}}{n}=\frac{2634}{25}=105.36 \text { (克) }
$$

根据式 (7.2) 得:

$$
\bar{x} \pm z_{\alpha / 2} \frac{\sigma}{\sqrt{n}}=105.36 \pm 1.96 \times \frac{10}{\sqrt{25}}
$$

即 $105.36 \pm 3.92=(101.44,109.28)$, 该批食品平均重量 $95 \%$ 的置信区间为 $101.44 \sim$ 109.28 克。(该天生产的食品的平均重量是否在 101.44 109.28 克之间?请读者自己思考。)

\section*{例 7.2}
一家保险公司收集到由 36 位投保人组成的随机样本, 得到每位投保人的年龄数据如\\
表 7 - 3 所示。

表 7-3 36 位投保人的年龄数据

单位: 岁

\begin{center}
\begin{tabular}{llllll}
\hline
23 & 35 & 39 & 27 & 36 & 44 \\
36 & 42 & 46 & 43 & 31 & 33 \\
42 & 53 & 45 & 54 & 47 & 24 \\
34 & 28 & 39 & 36 & 44 & 40 \\
39 & 49 & 38 & 34 & 48 & 50 \\
34 & 39 & 45 & 48 & 45 & 32 \\
\hline
\end{tabular}
\end{center}

试建立投保人平均年龄的 $90 \%$ 的置信区间。

一解 已知 $n=36,1-\alpha=90 \%, z_{\alpha / 2}=1.645$ 。由于总体方差未知, 但为大样本, 可用样本方差来代替总体方差。

根据样本数据计算的样本均值和标准差如下:

$$
\begin{aligned}
& \bar{x}=\frac{\sum_{i=1}^{n} x_{i}}{n}=39.5(\text { 岁 }) \\
& s=\sqrt{\frac{\sum_{i=1}^{n}\left(x_{i}-\bar{x}\right)^{2}}{n-1}}=7.77
\end{aligned}
$$

根据式 (7.3) 得:

$$
\bar{x} \pm z_{\alpha / 2} \frac{s}{\sqrt{n}}=39.5 \pm 1.645 \times \frac{7.77}{\sqrt{36}}
$$

即 $39.5 \pm 2.13=(37.4,41.6)$, 投保人平均年龄的 $90 \%$ 的置信区间为 $37.4 \sim 41.6$ 岁。

\begin{enumerate}
  \setcounter{enumi}{1}
  \item 正态总体、方差未知、小样本
\end{enumerate}

如果总体服从正态分布, 则无论样本量如何, 样本均值 $\bar{x}$ 的抽样分布都服从正态分布。这时, 只要总体方差 $\sigma^{2}$ 已知, 即使是在小样本的情况下, 也可以按式 (7.1) 建立总体均值的置信区间。但是, 如果总体方差 $\sigma^{2}$ 未知, 而且是在小样本情况下, 则需要用样本方差 $s^{2}$ 代替 $\sigma^{2}$, 这时, 样本均值经过标准化以后的随机变量服从自由度为 $n-1$ 的 $t$ 分布, 即


\begin{equation*}
t=\frac{\bar{x}-\mu}{s / \sqrt{n}} \sim t(n-1) \tag{7.4}
\end{equation*}


因此, 需要采用 $t$ 分布来建立总体均值 $\mu$ 的置信区间。

$t$ 分布是类似正态分布的一种对称分布, 它通常要比正态分布平坦和分散。一个特定的 $t$分布依赖于称为自由度的参数。随着自由度的增大, $t$ 分布逐渐趋于正态分布, 如图 7-7\\
所示。

\begin{center}
\end{center}

图 7-7 不同自由度的 $t$ 分布与标准正态分布的比较

根据 $t$ 分布建立的总体均值 $\mu$ 在 $1-\alpha$ 置信水平下的置信区间为:


\begin{equation*}
\bar{x} \pm t_{\alpha / 2} \frac{s}{\sqrt{n}} \tag{7.5}
\end{equation*}


式中, $t_{\alpha / 2}$ 是自由度为 $n-1$ 时, $t$ 分布中右侧面积为 $\alpha / 2$ 时的 $t$ 值, 该值可通过书后所附的 $t$ 分布表查得, 也可利用 Excel 提供的 T. INV 统计函数计算 $t$ 分布的临界值, 其语法为 T. INV $(\alpha, \mathrm{df})$, 其中 $\alpha$ 为对应于双尾 $t$ 分布的概率, 即如果要求 $t_{\alpha / 2}=t_{0.05 / 2}=t_{0.025}$ 的临界值, 应输人的 $\alpha$ 值为 0.05 。例如, 函数 “=T. $\operatorname{INV}(0.05,20)$ ” 的结果为 2.085963 。

\section*{例 7.3}
已知某种灯泡的寿命服从正态分布, 现从一批灯泡中随机抽取 16 个, 测得其使用寿命(单位: 小时)如下:

\begin{center}
\begin{tabular}{lllllllll}
1510 & 1450 & 1480 & 1460 & 1520 & 1480 & 1490 & 1460 \\
1480 & 1510 & 1530 & 1470 & 1500 & 1520 & 1510 & 1470 \\
\end{tabular}
\end{center}

试建立该批灯泡平均使用寿命的 $95 \%$ 的置信区间。

根据抽样结果计算得:

$$
\begin{aligned}
& \bar{x}=\frac{\sum_{i=1}^{n} x_{i}}{n}=\frac{23840}{16}=1490 \text { (小时) } \\
& s=\sqrt{\frac{\sum_{i=1}^{n}\left(x_{i}-\bar{x}\right)^{2}}{n-1}}=\sqrt{\frac{9200}{16-1}}=24.77
\end{aligned}
$$

根据 $\alpha=0.05$ 查 $t$ 分布表得 $t_{\alpha / 2}(n-1)=t_{0.025}(15)=2.131$, 由式 (7.5) 得平均使用寿命的置信区间为:

$$
\bar{x} \pm t_{\alpha / 2} \frac{s}{\sqrt{n}}=1490 \pm 2.131 \times \frac{24.77}{\sqrt{16}}
$$

即 $1490 \pm 13.2=(1476.8,1503.2)$, 该种灯泡平均使用寿命的 $95 \%$ 的置信区间为\\
$1476.8 \sim 1503.2$ 小时。

下面对总体均值的区间估计做一个总结, 如表 7-4 所示。

表 7-4 不同情况下总体均值的区间估计

\begin{center}
\begin{tabular}{cccc}
\hline
总体分布 & 样本量 & $\sigma$ 已知 & $\sigma$ 未知 \\
\hline
 & 大样本 $(n \geqslant 30)$ & $\bar{x} \pm z_{\alpha / 2} \frac{\sigma}{\sqrt{n}}$ & $\bar{x} \pm z_{\alpha / 2} \frac{s}{\sqrt{n}}$ \\
 & 小样本 $(n<30)$ & $\bar{x} \pm z_{\alpha / 2} \frac{\sigma}{\sqrt{n}}$ & $\bar{x} \pm t_{\alpha / 2} \frac{s}{\sqrt{n}}$ \\
 & 大样本 $(n \geqslant 30)$ & $\bar{x} \pm z_{\alpha / 2} \frac{\sigma}{\sqrt{n}}$ & $\bar{x} \pm z_{\alpha / 2} \frac{s}{\sqrt{n}}$ \\
\hline
\end{tabular}
\end{center}

\subsection*{7.2.2 总体比例的区间估计}
这里只讨论大样本情况下总体比例的估计问题。(1) 由样本比例 $p$ 的抽样分布可知, 当样本量足够大时, 比例 $p$ 的抽样分布可用正态分布近似。 $p$ 的数学期望为 $E(p)=\pi ; p$ 的方差为 $\sigma_{p}^{2}=\frac{\pi(1-\pi)}{n}$ 。样本比例经标准化后的随机变量服从标准正态分布, 即


\begin{equation*}
z=\frac{p-\pi}{\sqrt{\pi(1-\pi) / n}} \sim N(0,1) \tag{7.6}
\end{equation*}


与总体均值的区间估计类似, 在样本比例 $p$ 的基础上加减估计误差 $z_{\alpha / 2} \sigma_{p}$, 即得总体比例 $\pi$ 在 $1-\alpha$ 置信水平下的置信区间为:


\begin{equation*}
p \pm z_{\alpha / 2} \sqrt{\frac{\pi(1-\pi)}{n}} \tag{7.7}
\end{equation*}


用式 (7.7) 计算总体比例 $\pi$ 的置信区间时, $\pi$ 值应该是已知的。但实际情况不然, $\pi$ 值恰好是要估计的, 所以, 需要用样本比例 $p$ 来代替 $\pi$ 。这时, 总体比例的置信区间可表示为:


\begin{equation*}
p \pm z_{\alpha / 2} \sqrt{\frac{p(1-p)}{n}} \tag{7.8}
\end{equation*}


式中, $\alpha$ 是显著性水平; $z_{\alpha / 2}$ 是标准正态分布右侧面积为 $\alpha / 2$ 时的 $z$ 值; $z_{\alpha / 2} \sqrt{\frac{p(1-p)}{n}}$ 是估计总体比例时的估计误差。这就是说, 总体比例的置信区间由两部分组成:点估计值和描述估计量精度的土值, 这个土值称为估计误差。

\section*{例 7.4}
某城市想要估计下岗职工中女性所占的比例, 随机抽取了 100 个下岗职工, 其中 65

(1) 对于总体比例的估计, 确定样本量是否足够大的一般经验规则是: 区间 $p \pm 2 \sqrt{p(1-p) / 2}$ 中不包含 0 或 1 ,或者要求 $n p \geqslant 5$ 和 $n(1-p) \geqslant 5$ 。\\
人为女职工。试以 $95 \%$ 的置信水平估计该城市下岗职工中女性比例的置信区间。

——解 已知 $n=100, z_{\alpha / 2}=1.96$ 。根据抽样结果计算的样本比例为:

$$
p=\frac{65}{100}=65 \%
$$

根据式(7.8) 得:

$$
p \pm z_{\alpha / 2} \sqrt{\frac{p(1-p)}{n}}=65 \% \pm 1.96 \times \sqrt{\frac{65 \% \times(1-65 \%)}{100}}
$$

即 $65 \% \pm 9.35 \%=(55.65 \%, 74.35 \%)$, 该城市下岗职工中女性比例的 $95 \%$ 的置信区间为 $55.65 \% \sim 74.35 \%$ 。

\subsection*{7.2. 3 总体方差的区间估计}
这里只讨论正态总体方差的估计问题。根据样本方差的抽样分布可知, 样本方差服从自由度为 $n-1$ 的 $\chi^{2}$ 分布。因此, 用 $\chi^{2}$ 分布构造总体方差的置信区间。

若给定一个显著性水平 $\alpha$, 用 $\chi^{2}$ 分布构造的总体方差 $\sigma^{2}$ 的置信区间可用图 7-8 表示。

\begin{center}
\end{center}

图 7-8 自由度为 $n-1$ 的 $\chi^{2}$ 分布

由图 7-8 可以看出, 建立总体方差 $\sigma^{2}$ 的置信区间, 也就是要找到一个 $\chi^{2}$ 值, 使其满足


\begin{equation*}
\chi_{1-a / 2}^{2} \leqslant \chi^{2} \leqslant \chi_{a / 2}^{2} \tag{7.9}
\end{equation*}


由于 $\frac{(n-1) s^{2}}{\sigma^{2}} \sim \chi^{2}(n-1)$, 可用它来代替 $\chi^{2}$, 于是有


\begin{equation*}
\chi_{1-\alpha / 2}^{2} \leqslant \frac{(n-1) s^{2}}{\sigma^{2}} \leqslant \chi_{\alpha / 2}^{2} \tag{7.10}
\end{equation*}


根据式 (7.10) 可推导出总体方差 $\sigma^{2}$ 在 $1-\alpha$ 置信水平下的置信区间为:


\begin{equation*}
\frac{(n-1) s^{2}}{\chi_{\alpha / 2}^{2}} \leqslant \sigma^{2} \leqslant \frac{(n-1) s^{2}}{\chi_{1-\alpha / 2}^{2}} \tag{7.11}
\end{equation*}


\section*{例 7.5}
沿用例 7.1 的数据, 以 $95 \%$ 的置信水平建立食品总体重量标准差的置信区间。

$$
s^{2}=\frac{\sum_{i=1}^{n}\left(x_{i}-\bar{x}\right)^{2}}{n-1}=\frac{2237.02}{25-1}=93.21
$$

根据显著性水平 $\alpha=0.05$ 和自由度 $n-1=25-1=24$, 查 $\chi^{2}$ 分布表得:

$$
\begin{aligned}
& \chi_{\alpha / 2}^{2}(n-1)=\chi_{0.025}^{2}(25-1)=39.3641 \\
& \chi_{1-\alpha / 2}^{2}(n-1)=\chi_{0.975}^{2}(25-1)=12.4011
\end{aligned}
$$

所以,总体方差 $\sigma^{2}$ 的置信区间为:

$$
\frac{(25-1) \times 93.21}{39.3641} \leqslant \sigma^{2} \leqslant \frac{(25-1) \times 93.21}{12.4011}
$$

即 $56.83 \leqslant \sigma^{2} \leqslant 180.39$ 。相应地, 总体标准差的置信区间则为 $7.54 \leqslant \sigma \leqslant 13.43$ 。该企业生产的食品总体重量标准差的 $95 \%$ 的置信区间为 $7.54 \sim 13.43$ 克。

\section*{7.3 两个总体参数的区间估计}
对于两个总体, 所关心的参数主要有两个总体的均值之差 $\mu_{1}-\mu_{2}$ 、两个总体的比例之差 $\pi_{1}-\pi_{2}$ 、两个总体的方差比 $\sigma_{1}^{2} / \sigma_{2}^{2}$ 等。

\subsection*{7.3. 1 两个总体均值之差的区间估计}
设两个总体的均值分别为 $\mu_{1}$ 和 $\mu_{2}$, 从两个总体中分别抽取样本量为 $n_{1}$ 和 $n_{2}$ 的两个随机样本, 其样本均值分别为 $\bar{x}_{1}$ 和 $\bar{x}_{2}$ 。两个总体均值之差 $\mu_{1}-\mu_{2}$ 的估计量显然是两个样本的均值之差 $\bar{x}_{1}-\bar{x}_{2}$ 。

\begin{enumerate}
  \item 两个总体均值之差的估计:独立样本
\end{enumerate}

(1) 大样本的估计。

如果两个样本是从两个总体中独立抽取的, 即一个样本中的元素与另一个样本中的元素相互独立, 则称为独立样本 (independent sample)。如果两个总体都为正态分布, 或两个总体不服从正态分布但两个样本都为大样本 $\left(n_{1} \geqslant 30\right.$ 和 $\left.n_{2} \geqslant 30\right)$, 根据抽样分布的知识可知, 两个样本均值之差 $\bar{x}_{1}-\bar{x}_{2}$ 的抽样分布服从期望值为 $\mu_{1}-\mu_{2}$ 、方差为 $\frac{\sigma_{1}^{2}}{n_{1}}+\frac{\sigma_{2}^{2}}{n_{2}}$ 的正态分布, 两个样本均值之差经标准化后服从标准正态分布, 即


\begin{equation*}
z=\frac{\left(\bar{x}_{1}-\bar{x}_{2}\right)-\left(\mu_{1}-\mu_{2}\right)}{\sqrt{\frac{\sigma_{1}^{2}}{n_{1}}+\frac{\sigma_{2}^{2}}{n_{2}}}} \sim N(0,1) \tag{7.12}
\end{equation*}


当两个总体的方差 $\sigma_{1}^{2}$ 和 $\sigma_{2}^{2}$ 都已知时, 两个总体均值之差 $\mu_{1}-\mu_{2}$ 在 $1-\alpha$ 置信水平下的置信区间为:


\begin{equation*}
\left(\bar{x}_{1}-\bar{x}_{2}\right) \pm z_{\alpha / 2} \sqrt{\frac{\sigma_{1}^{2}}{n_{1}}+\frac{\sigma_{2}^{2}}{n_{2}}} \tag{7.13}
\end{equation*}


当两个总体的方差 $\sigma_{1}^{2}$ 和 $\sigma_{2}^{2}$ 未知时, 可用两个样本方差 $s_{1}^{2}$ 和 $s_{2}^{2}$ 来代替, 这时, 两个总体均值之差 $\mu_{1}-\mu_{2}$ 在 $1-\alpha$ 置信水平下的置信区间为:


\begin{equation*}
\left(\bar{x}_{1}-\bar{x}_{2}\right) \pm z_{\alpha / 2} \sqrt{\frac{s_{1}^{2}}{n_{1}}+\frac{s_{2}^{2}}{n_{2}}} \tag{7.14}
\end{equation*}


\section*{例 7.6}
某地区教育管理部门想估计两所中学的学生高考时的英语平均分数之差, 为此在两所中学独立抽取两个随机样本, 有关数据如表 7-5 所示。

表 7-5 两个样本的有关数据

\begin{center}
\begin{tabular}{cc}
\hline
中学 1 & 中学 2 \\
\hline
$n_{1}=46$ & $n_{2}=33$ \\
$\bar{x}_{1}=86$ & $\bar{x}_{2}=78$ \\
$s_{1}=5.8$ & $s_{2}=7.2$ \\
\hline
\end{tabular}
\end{center}

试建立两所中学高考英语平均分数之差的 $95 \%$ 的置信区间。

$\cdot$解 根据式 (7.14) 得:

$$
\left(\bar{x}_{1}-\bar{x}_{2}\right) \pm z_{\alpha / 2} \sqrt{\frac{s_{1}^{2}}{n_{1}}+\frac{s_{2}^{2}}{n_{2}}}=(86-78) \pm 1.96 \times \sqrt{\frac{5.8^{2}}{46}+\frac{7.2^{2}}{33}}
$$

即 $8 \pm 2.97=(5.03,10.97)$, 两所中学高考英语平均分数之差的 $95 \%$ 的置信区间为 $5.03 \sim$ 10.97 分。

(2) 小样本的估计。

在两个样本都为小样本的情况下, 为估计两个总体的均值之差, 需要作出以下假定:

\begin{itemize}
  \item 两个总体都服从正态分布。
  \item 两个随机样本独立地分别抽自两个总体。
\end{itemize}

在上述假定下, 无论样本量的大小, 两个样本均值之差都服从正态分布。当两个总体方差 $\sigma_{1}^{2}$ 和 $\sigma_{2}^{2}$ 已知时, 可用式 (7.13) 建立两个总体均值之差的置信区间。当 $\sigma_{1}^{2}$ 和 $\sigma_{2}^{2}$ 未知时, 有以下两种情况。

\begin{enumerate}
  \item 当两个总体的方差 $\sigma_{1}^{2}$ 和 $\sigma_{2}^{2}$ 未知但相等时, 即 $\sigma_{1}^{2}=\sigma_{2}^{2}$, 需要用两个样本的方差 $s_{1}^{2}$ 和 $s_{2}^{2}$ 来估计, 这时, 需要将两个样本的数据组合在一起, 以给出总体方差的合并估计量 $s_{p}^{2}$,计算公式为:
\end{enumerate}


\begin{equation*}
s_{p}^{2}=\frac{\left(n_{1}-1\right) s_{1}^{2}+\left(n_{2}-1\right) s_{2}^{2}}{n_{1}+n_{2}-2} \tag{7.15}
\end{equation*}


这时, 两个样本均值之差经标准化后服从自由度为 $n_{1}+n_{2}-2$ 的 $t$ 分布, 即


\begin{equation*}
t=\frac{\left(\bar{x}_{1}-\bar{x}_{2}\right)-\left(\mu_{1}-\mu_{2}\right)}{s_{p} \sqrt{\frac{1}{n_{1}}+\frac{1}{n_{2}}}} \sim t\left(n_{1}+n_{2}-2\right) \tag{7.16}
\end{equation*}


因此, 两个总体均值之差 $\mu_{1}-\mu_{2}$ 在 $1-\alpha$ 置信水平下的置信区间为:


\begin{equation*}
\left(\bar{x}_{1}-\bar{x}_{2}\right) \pm t_{\alpha / 2}\left(n_{1}+n_{2}-2\right) \sqrt{s_{p}^{2}\left(\frac{1}{n_{1}}+\frac{1}{n_{2}}\right)} \tag{7.17}
\end{equation*}


\begin{enumerate}
  \setcounter{enumi}{1}
  \item 当两个总体的方差 $\sigma_{1}^{2}$ 和 $\sigma_{2}^{2}$ 未知且不相等时, 即 $\sigma_{1}^{2} \neq \sigma_{2}^{2}$, 两个样本均值之差经标准化后近似服从自由度为 $v$ 的 $t$ 分布,自由度 $v$ 的计算公式为:
\end{enumerate}


\begin{equation*}
v=\frac{\left(\frac{s_{1}^{2}}{n_{1}}+\frac{s_{2}^{2}}{n_{2}}\right)^{2}}{\frac{\left(s_{1}^{2} / n_{1}\right)^{2}}{n_{1}-1}+\frac{\left(s_{2}^{2} / n_{2}\right)^{2}}{n_{2}-1}} \tag{7.18}
\end{equation*}


两个总体均值之差在 $1-\alpha$ 置信水平下的置信区间为:


\begin{equation*}
\left(\bar{x}_{1}-\bar{x}_{2}\right) \pm t_{\alpha / 2}(v) \sqrt{\frac{s_{1}^{2}}{n_{1}}+\frac{s_{2}^{2}}{n_{2}}} \tag{7.19}
\end{equation*}


\section*{例 7.7}
为估计两种方法组装产品所需时间的差异, 分别为两种不同的组装方法随机安排 12 个工人,每个工人组装一件产品所需的时间如表 7-6 所示。

表 7-6 两种方法组装产品所需的时间 (一)

单位: 分钟

\begin{center}
\begin{tabular}{cc||cc}
\hline
方法 1 & 方法 2 & 方法 1 & 方法 2 \\
\hline
28.3 & 27.6 & 36.0 & 31.7 \\
30.1 & 22.2 & 37.2 & 26.0 \\
29.0 & 31.0 & 38.5 & 32.0 \\
37.6 & 33.8 & 34.4 & 31.2 \\
32.1 & 20.0 & 28.0 & 33.4 \\
28.8 & 30.2 & 30.0 & 26.5 \\
\hline
\end{tabular}
\end{center}

假定两种方法组装产品的时间服从正态分布, 且方差相等, 试以 $95 \%$ 的置信水平建立两种方法组装产品所需平均时间之差的置信区间。

解 根据样本数据计算得:

方法 1: $\bar{x}_{1}=32.5, s_{1}^{2}=15.996$

方法 2: $\bar{x}_{2}=28.8, s_{2}^{2}=19.358$

总体方差的合并估计量为:

$$
s_{p}^{2}=\frac{\left(n_{1}-1\right) s_{1}^{2}+\left(n_{2}-1\right) s_{2}^{2}}{n_{1}+n_{2}-2}
$$

$$
=\frac{(12-1) \times 15.996+(12-1) \times 19.358}{12+12-2}=17.677
$$

根据 $\alpha=0.05$, 自由度为 $12+12-2=22$, 查 $t$ 分布表得 $t_{0.05 / 2}(22)=2.0739$ 。两个总体均值之差 $\mu_{1}-\mu_{2}$ 在 $95 \%$ 的置信水平下的置信区间为:

$$
\begin{aligned}
& \left(\bar{x}_{1}-\bar{x}_{2}\right) \pm t_{\alpha / 2}\left(n_{1}+n_{2}-2\right) \sqrt{s_{p}^{2}\left(\frac{1}{n_{1}}+\frac{1}{n_{2}}\right)} \\
& =(32.5-28.8) \pm 2.0739 \times \sqrt{17.677 \times\left(\frac{1}{12}+\frac{1}{12}\right)} \\
& =3.7 \pm 3.56
\end{aligned}
$$

即 $(0.14,7.26)$, 两种方法组装产品所需平均时间之差的 $95 \%$ 的置信区间为 $0.14 \sim 7.26$分钟。

\section*{例 7.8}
沿用例 7.7 的数据。假定第一种方法随机安排 12 个工人, 第二种方法随机安排 8 个工人, 即 $n_{1}=12, n_{2}=8$, 所得的有关数据如表 7-7 所示。假定两个总体的方差不相等,试以 $95 \%$ 的置信水平建立两种方法组装产品所需平均时间之差的置信区间。

表 7-7 两种方法组装产品所需的时间 (二)

单位: 分钟

\begin{center}
\begin{tabular}{cc||cc}
\hline
方法 1 & 方法 2 & 方法 1 & 方法 2 \\
\hline
28.3 & 27.6 & 36.0 & 31.7 \\
30.1 & 22.2 & 37.2 & 26.5 \\
29.0 & 31.0 & 38.5 &  \\
37.6 & 33.8 & 34.4 &  \\
32.1 & 20.0 & 28.0 &  \\
28.8 & 30.2 & 30.0 &  \\
\hline
\end{tabular}
\end{center}

根据表 $7-7$ 的数据计算得:

方法 1: $\bar{x}_{1}=32.5, s_{1}^{2}=15.996$

方法 2: $\bar{x}_{2}=27.875, s_{2}^{2}=23.014$

计算的自由度为:

$$
v=\frac{\left(\frac{15.996}{12}+\frac{23.014}{8}\right)^{2}}{\frac{(15.996 / 12)^{2}}{12-1}+\frac{(23.014 / 8)^{2}}{8-1}}=13.188 \approx 13
$$

根据自由度 $=13$ 查 $t$ 分布表得 $t_{0.05 / 2}(13)=2.1604$ 。两个总体均值之差 $\mu_{1}-\mu_{2}$ 在 $1-\alpha$置信水平下的置信区间为:

$$
\left(\bar{x}_{1}-\bar{x}_{2}\right) \pm t_{a / 2}(v) \sqrt{\frac{s_{1}^{2}}{n_{1}}+\frac{s_{2}^{2}}{n_{2}}}
$$

$$
\begin{aligned}
& =(32.5-27.875) \pm 2.1604 \times \sqrt{\frac{15.996}{12}+\frac{23.014}{8}} \\
& =4.625 \pm 4.433
\end{aligned}
$$

即 (0.192, 9.058), 两种方法组装产品所需平均时间之差的 $95 \%$ 的置信区间为 $0.192 \sim$ 9. 058 分钟。

\section*{2. 两个总体均值之差的估计: 匹配样本}
在例 7.7 中使用的是两个独立的样本, 但使用独立样本来估计两个总体均值之差具有潜在的弊端。比如, 在为每种方法随机指派 12 个工人时, 偶尔可能会将技术比较差的 12 个工人指给方法 1 , 而将技术较好的 12 个工人指给方法 2 , 这种不公平的指派可能会掩盖两种方法组装产品所需时间的真正差异。

为解决这一问题, 可以使用匹配样本 (matched sample), 即一个样本中的数据与另一个样本中的数据相对应。比如, 先指定 12 个工人用第一种方法组装产品, 然后再让这 12 个工人用第二种方法组装产品, 这样得到的两种方法组装产品的数据就是匹配数据。匹配样本可以消除由于样本指定的不公平造成的两种方法组装时间上的差异。

使用匹配样本进行估计时, 在大样本条件下, 两个总体均值之差 $\mu_{d}=\mu_{1}-\mu_{2}$ 在 $1-\alpha$置信水平下的置信区间为:


\begin{equation*}
\bar{d} \pm z_{\alpha / 2} \frac{\sigma_{d}}{\sqrt{n}} \tag{7.20}
\end{equation*}


式中, $d$ 表示两个匹配样本对应数据的差值; $\bar{d}$ 表示各差值的均值; $\sigma_{d}$ 表示各差值的标准差。当总体的 $\sigma_{d}$ 未知时, 可用样本差值的标准差 $s_{d}$ 来代替。

在小样本情况下, 假定两个总体各观测值的配对差服从正态分布。两个总体均值之差 $\mu_{d}=\mu_{1}-\mu_{2}$ 在 $1-\alpha$ 置信水平下的置信区间为:


\begin{equation*}
\bar{d} \pm t_{\alpha / 2}(n-1) \frac{s_{d}}{\sqrt{n}} \tag{7.21}
\end{equation*}


\section*{例 7.9}
由 10 名学生组成一个随机样本, 分别采用 A 和 B 两套试卷进行测试, 结果如表 7-8 所示。

表 7-8 10 名学生两套试卷的得分

\begin{center}
\begin{tabular}{cccc}
\hline
学生编号 & 试卷 $\mathrm{A}$ & 试卷 B & 差值 $\boldsymbol{d}$ \\
\hline
1 & 78 & 71 & 7 \\
2 & 63 & 44 & 19 \\
3 & 72 & 61 & 11 \\
4 & 89 & 84 & 5 \\
\hline
\end{tabular}
\end{center}

续表

\begin{center}
\begin{tabular}{cccc}
\hline
学生编号 & 试卷 A & 试卷 B & 差值 $d$ \\
\hline
5 & 91 & 74 & 17 \\
6 & 49 & 51 & -2 \\
7 & 68 & 55 & 13 \\
8 & 76 & 60 & 16 \\
9 & 85 & 77 & 8 \\
10 & 55 & 39 & 16 \\
\hline
\end{tabular}
\end{center}

假定两套试卷分数之差服从正态分布, 试建立两套试卷平均分数之差 $\mu_{d}=\mu_{1}-\mu_{2}$ 的 $95 \%$ 的置信区间。

・解 根据表 $7-8$ 中的数据计算得:

$$
\begin{aligned}
& \bar{d}=\frac{\sum_{i=1}^{n} d_{i}}{n_{d}}=\frac{110}{10}=11 \\
& s_{d}=\sqrt{\frac{\sum_{i=1}^{n}\left(d_{i}-\bar{d}\right)^{2}}{n_{d}-1}}=6.53
\end{aligned}
$$

根据自由度为 $10-1=9$ 查 $t$ 分布表得 $t_{0.05 / 2}(9)=2.262$ 。 。根据式 (7.21) 得两套试卷平均分数之差 $\mu_{d}=\mu_{1}-\mu_{2}$ 的 $95 \%$ 的置信区间为:

$$
\begin{aligned}
\bar{d} \pm t_{\alpha / 2}(n-1) \frac{s_{d}}{\sqrt{n}} & =11 \pm 2.2622 \times \frac{6.53}{\sqrt{10}} \\
& =11 \pm 4.7
\end{aligned}
$$

即 $(6.3,15.7)$, 两套试卷平均分数之差的 $95 \%$ 的置信区间为 $6.3 \sim 15.7$ 分。

\subsection*{7.3. 2 两个总体比例之差的区间估计}
由样本比例的抽样分布可知, 从两个二项总体中抽出两个独立的样本, 则两个样本比例之差的抽样分布服从正态分布。同样, 两个样本的比例之差经标准化后服从标准正态分布,即


\begin{equation*}
Z=\frac{\left(p_{1}-p_{2}\right)-\left(\pi_{1}-\pi_{2}\right)}{\sqrt{\frac{\pi_{1}\left(1-\pi_{1}\right)}{n_{1}}+\frac{\pi_{2}\left(1-\pi_{2}\right)}{n_{2}}}} \sim N(0,1) \tag{7.22}
\end{equation*}


当两个总体比例 $\pi_{1}$ 和 $\pi_{2}$ 未知时, 可用样本比例 $p_{1}$ 和 $p_{2}$ 来代替, 因此, 根据正态分布建立的两个总体比例之差 $\pi_{1}-\pi_{2}$ 在 $1-\alpha$ 置信水平下的置信区间为:


\begin{equation*}
\left(p_{1}-p_{2}\right) \pm z_{\alpha / 2} \sqrt{\frac{p_{1}\left(1-p_{1}\right)}{n_{1}}+\frac{p_{2}\left(1-p_{2}\right)}{n_{2}}} \tag{7.23}
\end{equation*}


\section*{例 7.10}
在某个电视节目的收视率调查中, 从农村随机调查了 400 人, 有 $32 \%$ 的人收看了该节目; 从城市随机调查了 500 人, 有 $45 \%$ 的人收看了该节目。试以 $95 \%$ 的置信水平估计城市与农村收视率之差的置信区间。

一解 设城市收视率 $p_{1}=45 \%$, 农村收视率 $p_{2}=32 \%$ 。当 $\alpha=0.05$ 时, $z_{\alpha / 2}=1.96$ 。因此,置信区间为:

$$
\begin{aligned}
& \left(p_{1}-p_{2}\right) \pm z_{\alpha / 2} \sqrt{\frac{p_{1}\left(1-p_{1}\right)}{n_{1}}+\frac{p_{2}\left(1-p_{2}\right)}{n_{2}}} \\
& =(45 \%-32 \%) \pm 1.96 \times \sqrt{\frac{45 \% \times(1-45 \%)}{500}+\frac{32 \% \times(1-32 \%)}{400}} \\
& =13 \% \pm 6.32 \%
\end{aligned}
$$

即 $(6.68 \%, 19.32 \%)$, 城市与农村收视率之差的 $95 \%$ 的置信区间为 $6.68 \% \sim 19.32 \%$ 。

\subsection*{7.3.3 两个总体方差比的区间估计}
实际工作中经常会遇到比较两个总体方差的问题。比如, 比较用两种不同方法生产的产品性能的稳定性, 比较不同测量工具的精度, 等等。

由于两个样本方差比的抽样分布服从 $F\left(n_{1}-1, n_{2}-1\right)$ 分布, 因此可用 $F$ 分布来构造两个总体方差比 $\sigma_{1}^{2} / \sigma_{2}^{2}$ 的置信区间。用 $F$ 分布构造的两个总体方差比的置信区间可用图 7 - 9 来表示。

\begin{center}
\end{center}

图 7-9 方差比的置信区间示意图

建立两个总体方差比的置信区间, 也就是要找到一个 $F$ 值, 使其满足


\begin{equation*}
F_{1-\alpha / 2} \leqslant F \leqslant F_{\alpha / 2} \tag{7.24}
\end{equation*}


由于 $\frac{s_{1}^{2}}{s_{2}^{2}} \cdot \frac{\sigma_{2}^{2}}{\sigma_{1}^{2}} \sim F\left(n_{1}-1, n_{2}-1\right)$, 故可用它来代替 $F$, 于是有


\begin{equation*}
F_{1-\alpha / 2} \leqslant \frac{s_{1}^{2}}{s_{2}^{2}} \cdot \frac{\sigma_{2}^{2}}{\sigma_{1}^{2}} \leqslant F_{\alpha / 2} \tag{7.25}
\end{equation*}


根据式 (7.25), 可以推导出两个总体方差比 $\sigma_{1}^{2} / \sigma_{2}^{2}$ 在 $1-\alpha$ 置信水平下的置信区\\
间为:


\begin{equation*}
\frac{s_{1}^{2} / s_{2}^{2}}{F_{\alpha / 2}} \leqslant \frac{\sigma_{1}^{2}}{\sigma_{2}^{2}} \leqslant \frac{s_{1}^{2} / s_{2}^{2}}{F_{1-\alpha / 2}} \tag{7.26}
\end{equation*}


式中, $F_{\alpha / 2}$ 和 $F_{1-\alpha / 2}$ 分别是分子自由度为 $n_{1}-1$ 和分母自由度为 $n_{2}-1$ 的 $F$ 分布的右侧面积为 $\alpha / 2$ 和 $1-\alpha / 2$ 的分位数。由于 $F$ 分布表中只给出面积较小的右分位数, 此时可利用下面的关系求得 $F_{1-\alpha / 2}$ 的分位数值:


\begin{equation*}
F_{1-\alpha / 2}\left(n_{1}, n_{2}\right)=\frac{1}{F_{\alpha / 2}\left(n_{2}, n_{1}\right)} \tag{7.27}
\end{equation*}


式中, $n_{1}$ 表示分子自由度; $n_{2}$ 表示分母自由度。

\section*{例 7.11}
为研究男女学生在生活费支出 (单位: 元) 上的差异, 在某大学随机抽取 25 名男学生和 25 名女学生, 得到下面的结果:

男学生: $\bar{x}_{1}=520, s_{1}^{2}=260$

女学生: $\bar{x}_{2}=480, s_{2}^{2}=280$

试以 $90 \%$ 的置信水平估计男女学生生活费支出方差比的置信区间。

\begin{itemize}
  \item 解 根据自由度 $n_{1}=25-1=24$ 和 $n_{2}=25-1=24$ 查 $F$ 分布表得:
\end{itemize}

$$
F_{\alpha / 2}(24,24)=F_{0.05}(24,24)=1.98
$$

根据式(7.27) 得:

$$
F_{1-\alpha / 2}(24,24)=F_{0.95}(24,24)=\frac{1}{1.98}=0.505
$$

根据式(7.26) 得:

$$
\frac{260 / 280}{1.98} \leqslant \frac{\sigma_{1}^{2}}{\sigma_{2}^{2}} \leqslant \frac{260 / 280}{0.505}
$$

即 $0.47 \leqslant \frac{\sigma_{1}^{2}}{\sigma_{2}^{2}} \leqslant 1.84$, 男女学生生活费支出方差比的 $90 \%$ 的置信区间为 $0.47 \sim 1.84$ 。

\section*{7.4 样本量的确定}
在进行参数估计之前, 首先应该确定一个适当的样本量, 也就是应该抽取一个多大的样本来估计总体参数。在进行估计时, 总是希望提高估计的可靠程度。但在一定的样本量下,要提高估计的可靠程度(置信水平),就应扩大置信区间,而过宽的置信区间在实际估计中往往是没有意义的。比如, 我们说某一天会下雨, 置信区间并不宽, 但可靠性相对较低; 如果说第三季度会下一场雨, 尽管很可靠, 但准确性又太差, 也就是说置信区间太宽了, 这样的估计是没有意义的。想要缩小置信区间, 又不降低置信程度, 就需要增加样本量。但样本量的增加会受到许多限制, 比如会增加调查的费用和工作量。通常, 样本量\\
的确定与可以容忍的置信区间的宽度以及对此区间设置的置信水平有一定关系。因此, 如何确定一个适当的样本量, 是抽样估计中需要考虑的问题。

\subsection*{7.4. 1 估计总体均值时样本量的确定}
前面已经讲到, 总体均值的置信区间是由样本均值 $\bar{x}$ 和估计误差两部分组成的。在重复抽样或无限总体抽样条件下, 估计误差为 $z_{\alpha / 2} \frac{\sigma}{\sqrt{n}}$ 。 $z_{\alpha / 2}$ 的值和样本量 $n$ 共同确定了估计误差的大小。一旦确定了置信水平 $1-\alpha, z_{\alpha / 2}$ 的值就确定了。对于给定的 $z_{\alpha / 2}$ 的值和总体标准差 $\sigma$, 可以确定任一希望的估计误差所需要的样本量。令 $E$ 代表所希望达到的估计误差, 即


\begin{equation*}
E=z_{\alpha / 2} \frac{\sigma}{\sqrt{n}} \tag{7.28}
\end{equation*}


由此可以推导出确定样本量的公式:


\begin{equation*}
n=\frac{\left(z_{\alpha / 2}\right)^{2} \sigma^{2}}{E^{2}} \tag{7.29}
\end{equation*}


式中, $E$ 值是使用者在给定的置信水平下可以接受的估计误差; $z_{\alpha / 2}$ 的值可直接由区间估计中所用到的置信水平确定。如果能求出 $\sigma$ 的具体值, 就可以用上面的公式计算所需的样本量。在实际应用中, 如果 $\sigma$ 的值不知道, 可以用以前相同或类似的样本的标准差来代替; 也可以用试验调查的办法, 选择一个初始样本, 以该样本的标准差作为 $\sigma$ 的估计值。

从式 (7.29) 可以看出, 样本量与置信水平成正比, 在其他条件不变的情况下, 置信水平越大, 所需的样本量也就越大; 样本量与总体方差成正比, 总体的差异越大, 所要求的样本量也越大; 样本量与估计误差的平方成反比, 即可以接受的估计误差的平方越大,所需的样本量就越小。

需要说明的是: 根据式 (7.29) 计算出的样本量不一定是整数, 通常将样本量取成较大的整数, 也就是将小数点后面的数值进位成整数, 如 24.68 取 $25,24.32$ 也取 25 等。这就是样本量的圆整法则。

\section*{例 7.12}
拥有工商管理学士学位的大学毕业生年薪的标准差大约为 2000 元。假定想要估计年薪的 $95 \%$ 的置信区间, 希望估计误差为 400 元, 应抽取多少样本?

\begin{itemize}
  \item 解 已知 $\sigma=2000, E=400, z_{\alpha / 2}=1.96$ 。
\end{itemize}

根据式 (7.29) 得:

$$
n=\frac{\left(z_{\alpha / 2}\right)^{2} \sigma^{2}}{E^{2}}=\frac{1.96^{2} \times 2000^{2}}{400^{2}}=96.04 \approx 97
$$

即应抽取 97 人作为样本。

\subsection*{7.4. 2 估计总体比例时样本量的确定}
与估计总体均值时样本量的确定方法类似, 在重复抽样或无限总体抽样条件下, 估计总体比例置信区间的估计误差为 $z_{\alpha / 2} \sqrt{\frac{\pi(1-\pi)}{n}}, z_{\alpha / 2}$ 的值、总体比例 $\pi$ 和样本量 $n$ 共同确定了估计误差的大小。一旦确定了置信水平 $1-\alpha, z_{\alpha / 2}$ 的值就确定了。由于总体比例的值是固定的, 所以估计误差由样本量来确定, 样本量越大, 估计误差就越小, 估计的精度也就越好。因此, 对于给定的 $z_{\alpha / 2}$ 的值, 可以确定任一希望的估计误差所需要的样本量。令 $E$ 代表所希望达到的估计误差, 即


\begin{equation*}
E=z_{\alpha / 2} \sqrt{\frac{\pi(1-\pi)}{n}} \tag{7.30}
\end{equation*}


由此可以推导出重复抽样或无限总体抽样条件下确定样本量的公式:


\begin{equation*}
n=\frac{\left(z_{\alpha / 2}\right)^{2} \pi(1-\pi)}{E^{2}} \tag{7.31}
\end{equation*}


式 (7.31) 中的估计误差 $E$ 必须是使用者事先确定的, 大多数情况下, 取 $E$ 的值小于 0.10 。 $z_{\alpha / 2}$ 的值可直接由区间估计中所用到的置信水平确定。如果能够求出 $\pi$ 的具体值,就可以用式 (7.31) 计算所需的样本量。在实际应用中, 如果 $\pi$ 的值不知道, 可以用类似的样本比例来代替; 也可以用试验调查的办法, 选择一个初始样本, 以该样本的比例作为 $\pi$ 的估计值。当 $\pi$ 的值无法知道时, 通常取使 $\pi(1-\pi)$ 最大时的 0.5 。

\section*{例 7.13}
根据以往的生产统计, 某种产品的合格率约为 90\%, 现要求估计误差为 5\%, 在 $95 \%$的置信区间下,应抽取多少个产品作为样本?

已知 $\pi=90 \%, E=5 \%, z_{\alpha / 2}=1.96$ 。

根据式 (7.31) 得:

$$
n=\frac{\left(z_{\alpha / 2}\right)^{2} \pi(1-\pi)}{E^{2}}=\frac{1.96^{2} \times 0.9 \times(1-0.9)}{0.05^{2}}=138.3 \approx 139
$$

即应抽取 139 个产品作为样本。

\section*{思考与练习}
思考题

7.1 解释估计量和估计值。

7.2 简述评价估计量好坏的标准。

7.3 怎样理解置信区间?\\
7. 4 解释 $95 \%$ 的置信区间。

$7.5 z_{\alpha / 2} \frac{\sigma}{\sqrt{n}}$ 的含义是什么?\\
7. 6 解释独立样本和匹配样本的含义。

7.7 在对两个总体均值之差的小样本估计中, 对两个总体和样本都有哪些假定?

7.8 简述样本量与置信水平、总体方差、估计误差的关系。

\section*{练习题}
7.1 利用下面的信息, 构建总体均值的置信区间。

(1)总体服从正态分布,已知 $\sigma=500, n=15, \bar{x}=8900$, 置信水平为 $95 \%$ 。

(2)总体不服从正态分布,已知 $\sigma=500, n=35, \bar{x}=8900$, 置信水平为 $95 \%$ 。

(3)总体不服从正态分布, $\sigma$ 未知, $n=35, \bar{x}=8900, s=500$, 置信水平为 $90 \%$ 。

(4)总体不服从正态分布, $\sigma$ 未知, $n=35, \bar{x}=8900, s=500$, 置信水平为 $99 \%$ 。

7.2 某大学为了解学生每天上网的时间, 在全校 7500 名学生中采取重复抽样方法随机抽取 36 人,调查他们每天上网的时间, 得到下面的数据 (单位: 小时):

\begin{center}
\begin{tabular}{llllllllll}
\hline
3.3 & 3.1 & 6.2 & 5.8 & 2.3 & 4.1 & 5.4 & 4.5 & 3.2 \\
4.4 & 2.0 & 5.4 & 2.6 & 6.4 & 1.8 & 3.5 & 5.7 & 2.3 \\
2.1 & 1.9 & 1.2 & 5.1 & 4.3 & 4.2 & 3.6 & 0.8 & 1.5 \\
4.7 & 1.4 & 1.2 & 2.9 & 3.5 & 2.4 & 0.5 & 3.6 & 2.5 \\
\hline
\end{tabular}
\end{center}

求该校大学生平均上网时间的置信区间, 置信水平分别为 $90 \%, 95 \%$ 和 $99 \%$ 。

7.3 某企业生产的袋装食品采用自动打包机包装, 每袋标准重量为 100 克。现从某天生产的一批产品中按重复抽样随机抽取 50 袋进行检查, 得到如下数据:

\begin{center}
\begin{tabular}{cc}
\hline
每袋重量 (克) & 袋数 \\
\hline
$96 \sim 98$ & 2 \\
$98 \sim 100$ & 3 \\
$100 \sim 102$ & 34 \\
$102 \sim 104$ & 7 \\
$104 \sim 106$ & 4 \\
\hline
合计 & 50 \\
\hline
\end{tabular}
\end{center}

已知食品每袋的重量服从正态分布, 要求:

(1) 确定该食品平均重量的 $95 \%$ 的置信区间。

(2)如果规定食品重量低于 100 克属于不合格, 确定该批食品合格率的 $95 \%$ 的置信区间。

7.4 假设总体服从正态分布, 利用下面的数据构建总体均值 $\mu$ 的 $99 \%$ 的置信区间。

\begin{center}
\begin{tabular}{lllll}
\hline
16.4 & 17.1 & 17.0 & 15.6 & 16.2 \\
14.8 & 16.0 & 15.6 & 17.3 & 17.4 \\
15.6 & 15.7 & 17.2 & 16.6 & 16.0 \\
15.3 & 15.4 & 16.0 & 15.8 & 17.2 \\
14.6 & 15.5 & 14.9 & 17.7 & 16.3 \\
\hline
\end{tabular}
\end{center}

7.5 利用下面的样本数据构建总体比例 $\pi$ 的置信区间。

(1) $n=44, p=0.51$, 置信水平为 $99 \%$ 。

(2) $n=300, p=0.82$, 置信水平为 $95 \%$ 。

(3) $n=1150, p=0.48$, 置信水平为 $90 \%$ 。

7.6 在一项家电市场调查中, 随机抽取了 200 个居民户, 调查他们是否拥有某一品牌的电视机, 其中拥有该品牌电视机的家庭占 $23 \%$ 。求总体比例的置信区间, 置信水平分别为 $90 \%$ 和 $95 \%$ 。

7.7 一位银行的管理人员想估计每位顾客在该银行的月平均存款额。他假设所有顾客月存款额的标准差为 1000 元, 要求的估计误差在 200 元以内, 置信水平为 $99 \%$ 。应选取多大的样本?

7.8 某居民小区共有居民 500 户,小区管理者准备采用一种新的供水设施,想了解居民是否赞成。采取重复抽样方法随机抽取了 50 户, 其中有 32 户赞成, 18 户反对。

(1) 求总体中赞成该项改革的户数比例的置信区间 $(\alpha=0.05)$ 。

(2)如果小区管理者预计赞成的比例能达到 $80 \%$, 估计误差不超过 $10 \%$, 应抽取多少户进行调查 $(\alpha=0.05)$ ?

7.9 根据下面的样本结果, 计算总体标准差 $\sigma$ 的 $90 \%$ 的置信区间。

(1) $\bar{x}=21, s=2, n=50$ 。

(2) $\bar{x}=1.3, s=0.02, n=15$ 。

(3) $\bar{x}=167, s=31, n=22$ 。

7.10 顾客到银行办理业务时往往需要等待一段时间, 而等待时间的长短与多种因素有关, 比如,银行业务员办理业务的速度、顾客排队的方式等。为此, 某银行准备采取两种排队方式进行试验, 第一种排队方式是所有顾客都进入一个等待队列; 第二种排队方式是顾客在三个业务窗口处列队三排等待。为比较哪种排队方式使顾客等待的时间更短, 银行各随机抽取 10 名顾客, 他们在办理业务时所等待的时间(单位:分钟)如下:

\begin{center}
\begin{tabular}{lllllllllll}
\hline
方式1 & 6.5 & 6.6 & 6.7 & 6.8 & 7.1 & 7.3 & 7.4 & 7.7 & 7.7 & 7.7 \\
\hline
方式 2 & 4.2 & 5.4 & 5.8 & 6.2 & 6.7 & 7.7 & 7.7 & 8.5 & 9.3 & 10.0 \\
\hline
\end{tabular}
\end{center}

(1)构建第一种排队方式等待时间标准差的 $95 \%$ 的置信区间。

(2)构建第二种排队方式等待时间标准差的 $95 \%$ 的置信区间。

(3)根据(1)和(2)的结果,你认为哪种排队方式更好?

7.11 从两个正态总体中分别抽取两个独立的随机样本, 它们的均值和标准差如下表所示:

\begin{center}
\begin{tabular}{cr}
\hline
来自总体 1 的样本 & 来自总体 2 的样本 \\
\hline
$\bar{x}_{1}=25$ & $\bar{x}_{2}=23$ \\
$s_{1}^{2}=16$ & $s_{2}^{2}=20$ \\
\end{tabular}
\end{center}

\section*{统计学 (第 8 版)}
(1)设 $n_{1}=n_{2}=100$, 求 $\mu_{1}-\mu_{2}$ 的 $95 \%$ 的置信区间。

(2)设 $n_{1}=n_{2}=10, \sigma_{1}^{2}=\sigma_{2}^{2}$, 求 $\mu_{1}-\mu_{2}$ 的 $95 \%$ 的置信区间。

(3)设 $n_{1}=n_{2}=10, \sigma_{1}^{2} \neq \sigma_{2}^{2}$, 求 $\mu_{1}-\mu_{2}$ 的 $95 \%$ 的置信区间。

(4)设 $n_{1}=10, n_{2}=20, \sigma_{1}^{2}=\sigma_{2}^{2}$, 求 $\mu_{1}-\mu_{2}$ 的 $95 \%$ 的置信区间。

(5)设 $n_{1}=10, n_{2}=20, \sigma_{1}^{2} \neq \sigma_{2}^{2}$, 求 $\mu_{1}-\mu_{2}$ 的 $95 \%$ 的置信区间。

7.12 下表是由四对观察值组成的随机样本。

\begin{center}
\begin{tabular}{ccc}
\hline
配对号 & 来自总体 $\mathrm{A}$ 的样本 & 来自总体 $\mathrm{B}$ 的样本 \\
\hline
1 & 2 & 0 \\
2 & 5 & 7 \\
3 & 10 & 6 \\
4 & 8 & 5 \\
\hline
\end{tabular}
\end{center}

(1) 计算 $\mathrm{A}$ 与 $\mathrm{B}$ 各对观察值之差, 再利用得出的差值计算 $\bar{d}$ 和 $s_{d}$ 。

(2)设 $\mu_{1}$ 和 $\mu_{2}$ 分别为总体 $\mathrm{A}$ 和总体 $\mathrm{B}$ 的均值,构建 $\mu_{d}=\mu_{1}-\mu_{2}$ 的 $95 \%$ 的置信区间。

7.13 一家人才测评机构对随机抽取的 10 名小企业的经理人用两种方法进行自信心测试, 得到的测试分数如下:

\begin{center}
\begin{tabular}{ccc}
\hline
人员编号 & 方法 1 & 方法 2 \\
\hline
1 & 78 & 71 \\
2 & 63 & 44 \\
3 & 72 & 61 \\
4 & 89 & 84 \\
5 & 91 & 74 \\
6 & 49 & 51 \\
7 & 68 & 55 \\
8 & 76 & 60 \\
9 & 85 & 77 \\
10 & 55 & 39 \\
\hline
\end{tabular}
\end{center}

构建两种方法平均自信心得分之差 $\mu_{d}=\mu_{1}-\mu_{2}$ 的 $95 \%$ 的置信区间。

7.14 从两个总体中各抽取一个 $n_{1}=n_{2}=250$ 的独立随机样本, 来自总体 1 的样本比例为 $p_{1}=40 \%$,来自总体 2 的样本比例为 $p_{2}=30 \%$ 。

(1)构建 $\pi_{1}-\pi_{2}$ 的 $90 \%$ 的置信区间。

(2)构建 $\pi_{1}-\pi_{2}$ 的 $95 \%$ 的置信区间。\\
7. 15 生产工序的方差是工序质量的一个重要度量。当方差较大时, 需要对工序进行改进以减小方差。下面是两部机器生产的袋装茶重量的数据 (单位: 克):

\begin{center}
\begin{tabular}{|c|c|c|c|c|c|}
\hline
\multicolumn{3}{|c|}{机器 1} & \multicolumn{3}{|c|}{机器 2} \\
\hline
3.45 & 3.22 & 3.90 & 3.22 & 3.28 & 3. 35 \\
\hline
3. 20 & 2.98 & 3.70 & 3.38 & 3. 19 & 3. 30 \\
\hline
3. 22 & 3. 75 & 3. 28 & 3. 30 & 3. 20 & 3. 05 \\
\hline
3.50 & 3.38 & 3.35 & 3. 30 & 3.29 & 3. 33 \\
\hline
2. 95 & 3.45 & 3.20 & 3.34 & 3.35 & 3. 27 \\
\hline
3. 16 & 3.48 & 3. 12 & 3. 28 & 3. 16 & 3. 28 \\
\hline
3. 20 & 3. 18 & 3.25 & 3. 30 & 3. 34 & 3.25 \\
\hline
\end{tabular}
\end{center}

构建两个总体方差比 $\sigma_{1}^{2} / \sigma_{2}^{2}$ 的 $95 \%$ 的置信区间。

7.16 根据以往的生产数据, 某种产品的废品率为 $2 \%$ 。如果置信区间为 $95 \%$, 估计误差不超过 $4 \%$, 应抽取多少样本?

\section*{假设检验}
\includegraphics[max width=\textwidth, center]{2024_03_13_d99164feef91279bea73g-143}\\
参数估计 (parameter estimation) 和假设检验 (hypothesis testing) 是统计推断的两个组成部分, 它们都是利用样本对总体进行某种推断, 但推断的角度不同。参数估计讨论的是用样本统计量估计总体参数的方法, 总体参数 $\mu$ 在估计前是未知的。而在假设检验中, 则是先对 $\mu$ 的值提出一个假设, 然后利用样本信息去检验这个假设是否成立。因此可以说, 本章讨论的内容是如何利用样本信息, 对假设成立与否作出判断的一套程序。

\section*{8.1 假设检验的基本问题}
\section*{8. 1.1 假设问题的提出}
假设检验是推论统计中的一项重要内容, 现实生活中有大量的事例可以归结为假设检验的问题。本章的内容不妨从下面的例子谈起。

\section*{例 8.1}
由统计资料得知, 1989 年某地新生儿的平均体重为 3190 克, 现从 1990 年的新生儿中随机抽取 100 个, 测得其平均体重为 3210 克, 问 1990 年的新生儿与 1989 年相比, 体重有无显著差异?

从调查结果看, 1990 年新生儿的平均体重为 3210 克, 比 1989 年新生儿的平均体重 3190 克增加了 20 克, 但这 20 克的差异可能源于不同的情况。一种情况是, 1990 年新生儿的体重与 1989 年相比没有什么差别, 20 克的差异是抽样的随机性造成的; 另一种情况是, 抽样的随机性不可能造成 20 克这样大的差异, 1990 年新生儿的体重与 1989 年新生儿的体重相比确实有所增加。

上述问题的关键点是: 20 克的差异说明了什么? 这个差异能不能用抽样的随机性来解释? 为了回答这个问题, 我们可以采取假设的方法。假设 1989 年和 1990 年新生儿的体重没有显著差异, 如果用 $\mu_{0}$ 表示 1989 年新生儿的平均体重, $\mu$ 表示 1990 年新生儿的平均体重, 我们的假设可以表示为 $\mu=\mu_{0}$ 或 $\mu-\mu_{0}=0$, 现要利用 1990 年新生儿体重的样本信息检验上述假设是否成立。如果成立, 说明这两年新生儿的体重没有显著差异; 如果不成立, 说明 1990 年新生儿的体重有了明显增加。在这里, 问题是以假设的形式提出的, 问题的解决方案是检验提出的假设是否成立。所以假设检验的实质是检验我们关心的参数一一1990 年的新生儿总体平均体重是否等于某个我们感兴趣的数值。

\section*{8. 1.2 假设的表达式}
统计的语言是用一个等式或不等式表示问题的原假设。在新生儿体重这个例子中, 原假设采用等式的方式, 即

$$
H_{0}: \mu=3190 \text { 克 }
$$

这里 $H_{0}$ 表示原假设 (null hypothesis)。由于原假设 $(H)$ 的下标用 0 表示, 所以有些文献中将此称为 “零假设”。 $\mu$ 是我们要检验的参数, 即 1990 年新生儿总体体重的均值。该表达式提出的命题是, 1990 年的新生儿与 1989 年的新生儿在体重上没有什么差异。显然, 3190 克是 1989 年新生儿总体的均值, 是我们感兴趣的数值。如果用 $\mu_{0}$ 表示感兴趣的数值, 原假设更一般的表达式为:

$H_{0}: \mu=\mu_{0}$\\
或 $H_{0}: \mu-\mu_{0}=0$

尽管原假设陈述的是两个总体的均值相等, 却并不表示它是既定的事实, 仅是假设而已。如果原假设不成立, 就要拒绝原假设, 而需要在另一个假设中作出选择, 这个假设称为备择假设 (alternative hypothesis)。在我们的例子中, 备择假设的表达式为:

$H_{1}: \mu \neq 3190$ 克

$H_{1}$ 表示备择假设, 它意味着 1990 年的新生儿与 1989 年的新生儿在体重上有明显差异。备择假设更一般的表达式为:

$$
H_{1}: \mu \neq \mu_{0}
$$

或

$$
H_{1}: \mu-\mu_{0} \neq 0
$$

原假设与备择假设互斥, 肯定原假设, 意味着放弃备择假设; 否定原假设, 意味着接受备择假设。由于假设检验是围绕着对原假设是否成立而展开的, 所以有些文献也把备择假设称为替换假设,表明当原假设不成立时的替换。

\section*{8. 1.3 两类错误}
对于原假设提出的命题, 我们需要作出判断, 这种判断可以用 “原假设正确” 或 “原假设错误” 来表述。当然, 这是依据样本提供的信息进行判断的, 也就是由部分来推断总体。因而判断有可能正确, 也有可能不正确, 也就是说, 我们面临着犯错误的可能。所犯的错误有两种类型, 第 I 类错误是原假设 $H_{0}$ 为真却被我们拒绝了, 犯这种错误的概率用 $\alpha$ 表示, 所以也称 $\alpha$ 错误 ( $\alpha$ error) 或弃真错误; 第 II 类错误是原假设为伪我们却没有拒绝, 犯这种错误的概率用 $\beta$ 表示, 所以也称 $\boldsymbol{\beta}$ 错误( $\boldsymbol{\beta}$ error)或取伪错误。在前面的例子中, $\alpha$ 错误和 $\beta$ 错误分别意味着什么呢?

$\alpha$ 错误: 原假设 $H_{0}: \mu=3190$ 克是正确的, 但我们作出了错误的判断, 认为 $H_{0}$ : $\mu \neq 3190$ 克, 即在假设检验中拒绝了本来是正确的原假设, 这时犯了弃真错误。

$\beta$ 错误: 原假设 $H_{0}: \mu=3190$ 克是错误的, 但我们认为原假设 $H_{0}: \mu=3190$ 克是成立的,即在假设检验中没有拒绝本来是错误的原假设,这时犯了取伪错误。

由此看出, 当原假设 $H_{0}$ 为真, 我们却将其拒绝, 犯这种错误的概率用 $\alpha$ 表示, 那么, 当 $H_{0}$ 为真, 我们没有拒绝 $H_{0}$, 则表明作出了正确的决策, 其概率自然为 $1-\alpha$;当原假设 $H_{0}$ 为伪, 我们却没有拒绝 $H_{0}$, 犯这种错误的概率用 $\beta$ 表示, 那么, 当 $H_{0}$ 为伪, 我们拒绝 $H_{0}$, 这也是正确的决策, 其概率为 $1-\beta$, 正确决策和犯错误的概率可以归纳为表 $8-1$ 。\\
表 8-1 假设检验中各种可能结果的概率

\begin{center}
\begin{tabular}{ccc}
\hline
项目 & 没有拒绝 $H_{0}$ & 拒绝 $H_{0}$ \\
\hline
$H_{0}$ 为真 & $1-\alpha$ (正确决策) & $\alpha$ (弃真错误) \\
$H_{0}$ 为伪 & $\beta$ (取伪错误) & $1-\beta$ (正确决策) \\
\hline
\end{tabular}
\end{center}

自然, 人们希望犯这两类错误的概率越小越好。但对于一定的样本量 $n$, 不能同时做到犯这两类错误的概率都很小。如果减少 $\alpha$ 错误, 就会增大犯 $\beta$ 错误的机会; 若减少 $\beta$ 错误, 就会增大犯 $\alpha$ 错误的机会。这就像在区间估计中, 要想增大估计的可靠性, 就会使区间变宽而降低精度; 要想提高精度, 就要求估计区间变得很窄, 而这样, 估计的可靠性就会大打折扣。当然, 使 $\alpha$ 和 $\beta$ 同时变小的办法也有, 就是增大样本量。但样本量不可能没有限制, 否则就会使抽样调查失去意义。因此, 在假设检验中, 就有一个对两类错误进行控制的问题。

一般来说, 哪类错误所带来的后果更严重, 危害更大, 在假设检验中就应当把哪类错误作为首要的控制目标。但在假设检验中, 大家都在执行这样一个原则, 即首先控制犯 $\alpha$错误原则。这样做的原因主要有两点: 第一, 大家都遵循一个统一的原则, 讨论问题就比较方便。第二, 更主要的原因在于, 从实用的观点看, 原假设是什么常常是明确的, 而备择假设是什么则常常是模糊的。在前面所举的新生儿体重的例子中, 原假设 $H_{0}: \mu=$ 3190 克的数量标准十分清楚, 而备择假设 $H_{1}: \mu \neq 3190$ 克的数量标准则比较模糊。我们不知道 $\mu>3190$ 克还是 $\mu<3190$ 克, 而且大的程度也不清楚。显然, 对于一个含义清楚的假设和一个含义模糊的假设, 我们更愿意接受前者。正是在这个背景下, 我们就更为关心如果 $H_{0}$ 为真, 而我们却把它拒绝了, 犯这种错误的可能性有多大, 而这正是 $\alpha$ 错误所表现的内容。假设检验中犯两类错误的情况如图 8-1 所示。

\begin{center}
\end{center}

(b)

图 8-1 假设检验中犯两类错误的图示

图 8-1 中 (a) 显示, 如果原假设 $H_{0}: \mu=\mu_{0}$ 为真, 样本的观察结果应当在 $\mu_{0}$ 附近,落人阴影中的概率为 $\alpha$ 。我们是根据样本的观察结果作出判断决策, 如果观察结果落人\\
(a)中的阴影部分, 我们便拒绝原假设, 这时就犯了 $\alpha$ 错误, 尽管犯这个错误的概率比较小, 但这种错误是不可避免的。图 8-1 中 (b) 显示, 如果原假设为伪, 被检验的参数 $\mu>$ $\mu_{0}$, 那么当样本观察结果落人阴影 $\beta$ 中时, 我们还是把 $\mu$ 看成 $\mu_{0}$ 而没有拒绝, 这时便犯了取伪错误, 其概率为 $\beta$ 。由图 8-1 还可以看出, 如果临界点沿水平方向右移, $\alpha$ 将变小而 $\beta$ 将变大; 如果向左移, $\alpha$ 将变大而 $\beta$ 将变小。这也说明了在假设检验中 $\alpha$ 和 $\beta$ 此消彼长的关系。

\subsection*{8.1.4 假设检验的流程}
假设检验的一般流程如下:

首先提出原假设和备择假设。在前面新生儿体重这个例子中, 原假设和备择假设分别为:

$$
H_{0}: \mu=3190 \text { 克; } H_{1}: \mu \neq 3190 \text { 克 }
$$

接下来, 需要确定适当的检验统计量, 并计算其数值。在参数的假设检验中, 如同在参数估计中一样, 要借助样本统计量进行统计推断, 这个统计量称为检验统计量。选择哪个统计量作为检验统计量需要考虑一些因素, 例如, 进行检验的样本量是多还是少, 总体标准差 $\sigma$ 已知还是未知, 等等。这些因素与参数估计中确定统计量所考虑的因素相同。

计算统计量类似于分数转化过程, 如同把一般得分转化为标准得分。在上例中, 假定总体 $\sigma$ 已知, 且样本量大 (经验说法是当 $n \geqslant 30$ 时, 称为大样本), 采用 $z$ 统计量, 计算公式为:


\begin{equation*}
z=\frac{\bar{x}-\mu_{0}}{\sigma / \sqrt{n}} \tag{8.1}
\end{equation*}


由此, 可以将上例中的相应数据转化, 如图 8-2 所示。

\begin{center}
\end{center}

图 8-2 将抽样分布转化为 $z$ 分布的对应关系

在图 8-2 上方, 3190.00 是 1989 年新生儿体重的平均值, 假如根据有关资料知道新生儿体重的标准差为 80 克, 即 $\sigma=80$, 由例 8.1 知样本量 $n=100$, 根据抽样分布原理, 当 $\mu=$ $3190, \sigma=80, n=100, \alpha=0.05$ 时得到的置信区间, 3174.32 为置信区间的下限, 3205.68 为置信区间的上限。如果原假设成立, 那么 $95 \%$ 的样本均值应当落在这个范围内。

在图 8-2下方, 两个临界点 3174.32 和 3205.68 分别转化为 $z$ 值, 即 $z_{\alpha / 2}= \pm 1.96$,而与总体均值 3190 对应的 $z$ 值恰好为零。 $z$ 统计量服从标准正态分布, 如图 8-3 所示。\\
图 8-3 可以帮助我们理解假设检验。进行假设检验利用的是小概率原理, 小概率原理是指发生概率很小的随机事件在一次试验中几乎不可能发生。根据这一原理可以作出是否拒绝原假设的决定。但什么样的概率才算小呢? 著名的英国统计学家费希尔把小概率的标准定为 0.05 , 虽然费希尔并没有对为什么选择 0.05 给出充分的解释, 但人们还是沿用了这个标准, 把

\begin{center}
\end{center}

图 8-3 双侧检验示意图 0.05 或比 0.05 更小的概率看成小概率。

如果原假设成立, 那么在一次试验中 $z$ 统计量落人图 8-3 两侧拒绝域的概率只有 0.05 , 这个概率是很小的。如果这个情况真的出现, 我们有理由认为总体的真值不是 3190 克, 也即拒绝原假设。接受备择假设。

样本均值 $\bar{x}=3210, \mu_{0}=3190, \sigma=80, n=100$, 代人式 (8.1), 得:

$$
z=\frac{\bar{x}-\mu_{0}}{\sigma / \sqrt{n}}=\frac{3210-3190}{80 / \sqrt{100}}=2.5
$$

最后可以进行统计决策。由图 8-3 看出, 计算出的 $z$ 值 2.5 落人拒绝域, 所以拒绝原假设, 认为与 1989 年相比, 1990 年新生儿的体重有显著差异。

由图 8-3 还可以看出, 如果根据样本数据计算出的 $z$ 值小于 -1.96 , 同样落人拒绝域, 也要拒绝原假设, 将左右两边的情况结合起来, 我们得到进行假设检验的决策准则:

若 $|z|<\left|z_{\alpha / 2}\right|$, 不拒绝 $H_{0}$; 若 $|z|>\left|z_{\alpha / 2}\right|$, 拒绝 $H_{0}$ 。

\subsection*{8.1.5 利用 $P$ 值进行决策}
前面进行检验的程序是根据检验统计量落人的区域作出是否拒绝原假设的决策。确定 $\alpha$ 以后, 拒绝域的位置也就相应确定了, 其好处是进行决策的界限清晰, 但缺陷是进行决策面临的风险是笼统的。在上面的例子中, $z=2.5$, 落人拒绝域, 我们拒绝原假设, 并知道犯弃真错误的概率 (面临的风险) 为 0.05 ; 如果 $z=2.0$, 同样落人拒绝域, 我们拒绝原假设面临的风险也是 0.05 。 0.05 是一个通用的风险概率, 这是用域表示的缺陷, 但根据不同的样本结果进行决策, 面临的风险事实上是有差别的, 为了精确地反映决策的风险度, 可以利用 $P$ 值进行决策。

为了解什么是 $P$ 值, 让我们回到前面的例子。在例 8.1 中, 根据随机抽样测得 1990 年的样本均值 $\bar{x}=3210$ 克, 与 1989 年的总体均值 3190 克相差 20 克, 20 克的差异究竟是大还是小? 换句话说, 如果原假设成立, 即 1990 年新生儿体重的总体均值与 1989 年新生儿体重的总体均值相同, 那么随机抽取出 $n=100$ 的样本, 其均值大于 3210 克的概率有多大呢? 我们把这个概率称为 $P$ 值, 也就是当原假设为真时样本观察结果或更极端结果出现的概率。如果 $P$ 值很小, 说明这种情况发生的概率很小, 如果这种情况出现了, 根据小概\\
率原理, 我们就有理由拒绝原假设。 $P$ 值越小, 拒绝原假设的理由就越充分。

$P$ 值是通过计算得到的, $P$ 值的大小取决于三个因素: 一是样本数据与原假设之间的差异, 在新生儿体重的例子里这个差异是 20 克; 二是样本量, 这里 $n=100$; 三是被假设参数的总体分布。在这个例子中计算出的 $P=0.01242$, 这就是说, 如果原假设成立, 样本均值等于和大于 3210 克的概率只有 0.01242 , 这是很小的, 由此我们可以拒绝原假设,得到与前面 $z$ 值检验相同的结论, 如图 8-4 所示。

手工计算 $P$ 值比较复杂,但使用计算机

\begin{center}
\end{center}

图 8-4 新生儿体重的抽样分布计算 $P$ 值就很方便。如何使用计算机计算,后面有专门介绍。

$P$ 值的长处是它反映了观察到的实际数据与原假设之间不一致的概率值, 与传统的拒绝域范围相比, $P$ 是一个具体的值,这样就提供了更多的信息。如果事先确定了显著性水平, 如 $\alpha=0.05$, 则在双侧检验中, $P>0.025(\alpha / 2=0.025)$ 不能拒绝原假设;反之, $P<0.025$ 则拒绝原假设。在单侧检验中, $P>0.05$ 不能拒绝原假设, $P<0.05$ 则拒绝原假设。当然, 也可以直接使用 $P$ 值进行决策, 这时 $P$ 值本身就代表了显著性水平。我们也可以使用 $P$ 值, 按照所需要的显著性水平进行判断和决策, 具体做法就是将 $P$ 值和需要的显著性水平进行比较。

\section*{8. 1.6 单侧检验}
图 8-3 是双侧检验的示意图, 它有两个拒绝域、两个临界值, 每个拒绝域的面积为 $\alpha / 2$, 如果原假设的命题为 $\mu=\mu_{0}$ 的形式, 则属于双侧检验, 如在例 8.1 中, 有

$$
H_{0}: \mu=3190 \text { 克; } H_{1}: \mu \neq 3190 \text { 克 }
$$

就属于这种情况。在双侧检验中, 只要 $\mu>\mu_{0}$ 或 $\mu<\mu_{0}$ 二者之中有一个成立, 就可以拒绝原假设。

在另外一些情况下, 我们关心的假设问题带有方向性。有两种情况:一种是我们所考察的数值越大越好, 如灯泡的使用寿命、轮胎行驶的里程数等; 另一种是数值越小越好,如废品率、生产成本等。根据人们的关注点不同, 单侧检验可以有不同的方向。

\section*{1. 左单侧检验}
\section*{例 8.2}
某批发商欲从厂家购进一批灯泡, 根据合同规定, 灯泡的使用寿命平均不得低于

1000 小时。已知灯泡使用寿命服从正态分布, 标准差为 200 小时。在总体中随机抽取 100 个灯泡, 得知样本均值为 960 小时, 问批发商是否应该购买这批灯泡?

解 这是一个单侧检验问题。显然, 如果灯泡的使用寿命超过了 1000 小时, 批发商是欢迎的, 因为他用既定的价格(灯泡使用寿命为 1000 小时的价格)购进了更高质量的产品。因此, 如果样本均值超过 1000 小时, 他会购进这批灯泡。问题在于样本均值为 960 小时他是否应当购进。因为即便总体均值为 1000 小时, 由于抽样的随机性, 样本均值略小于 1000 小时的情况也会经常出现。在这种场合下, 批发商更为关注可以容忍的下限,即当灯泡寿命低于什么水平时拒绝。于是检验的形式为:

$H_{0}: \mu \geqslant 1000$ 小时; $H_{1}: \mu<1000$ 小时

左单侧检验如图 8-5 所示 $(\alpha=0.05)$ 。也可以把左单侧检验称为下限检验。

\section*{2. 右单侧检验}
与左单侧检验的问题相反, 有时我们希望所考察的数值越小越好。请看下面的

\begin{center}
\end{center}

临界值

图 8-5 左单侧检验示意图例题。

\section*{例 8.3}
某种大量生产的袋装食品按规定重量不得少于 250 克。今从一批该食品中随机抽取 50 袋, 发现有 6 袋重量低于 250 克, 若规定不符合标准的比例达到 $5 \%$, 食品就不得出厂,问该批食品能否出厂?

一解 显然, 不符合标准的比例越小越好。在这个产品质量检验的问题中, 我们比较关心次品率的上限, 即不合标准的比例达到多少就要拒绝。由于采用的是产品质量抽查, 即使总体不合标准的比例没有超过 $5 \%$, 属于合格范围, 但由于抽样的随机性, 样本中不合标准的比例略大于 $5 \%$ 的情况也会经常发生。如果采用右单侧检验, 确定拒绝的上限临界点, 那么检验的形式可以写为:

$H_{0}: \mu \leqslant 5 \% ; H_{1}: \mu>5 \%$

右单侧检验如图 8-6 所示 $(\alpha=0.05)$ 。也可以把右单侧检验称为上限检验。

\begin{center}
\end{center}

图 8-6 右单侧检验示意图

\section*{8.2 一个总体参数的检验}
\subsection*{8.2. 1 检验统计量的确定}
根据假设检验的不同内容和进行检验的不同条件, 需要采用不同的检验统计量, 在一个总体参数的检验中, 用到的检验统计量主要有三个: $z$ 统计量, $t$ 统计量, $\chi^{2}$ 统计量。 $z$统计量和 $t$ 统计量常常用于均值和比例的检验, $\chi^{2}$ 统计量则用于方差的检验。选择什么统计量进行检验需要考虑一些因素, 这些因素主要有样本量 $n$ 的大小以及总体标准差 $\sigma$ 是否已知。

\section*{1. 样本量}
样本量大小是选择检验统计量的一个要素。在样本量大的条件下, 如果总体为正态分布, 则样本统计量服从正态分布; 如果总体为非正态分布, 则样本统计量渐近服从正态分布。在这些情况下, 我们都可以把样本统计量视为正态分布, 这时可以使用 $z$ 统计量 $(z$分布)。 $z$ 统计量的计算公式见式 (8.1), 即在总体标准差 $\sigma$ 已知时, 有

$$
z=\frac{\bar{x}-\mu_{0}}{\sigma / \sqrt{n}}
$$

实践中当总体标准差 $\sigma$ 未知时, 可以用样本标准差 $s$ 代替, 上式可以写为:

$$
z=\frac{\bar{x}-\mu_{0}}{s / \sqrt{n}}
$$

样本量较小时, 情况有些复杂。在假设总体为正态分布的前提下, 要看我们是否掌握总体标准差 $\sigma$ 的信息。

\section*{2. 总体标准差 $\sigma$ 是否已知}
在样本量较小的情况下, 如果总体标准差已知, 则样本统计量服从正态分布, 这时可以采用 $z$ 统计量。如果总体标准差未知, 进行检验所依赖的信息有所减少, 这时只能使用样本标准差, 样本统计量服从 $t$ 分布, 应该采用 $t$ 统计量。与正态分布相比, $t$ 分布更为扁平, 在相同概率条件下, $t$ 分布的临界点向两边更为扩展, 临界点与中心距离更远, 这意味着推断的精度下降, 这是总体标准差 $\sigma$ 未知所要付出的代价。

$t$ 统计量的计算公式为:


\begin{equation*}
t=\frac{\bar{x}-\mu_{0}}{s / \sqrt{n}} \tag{8.2}
\end{equation*}


$t$ 统计量的自由度为 $n-1$ 。

由上述讨论看出, 样本量大小是选择检验统计量时一个很重要的考虑因素, 在大样本情况下一般可以使用 $z$ 统计量。但样本量 $n$ 为多大才算大样本, 不同的人可能给出不同的答案, 同时也与检验的对象有关。仅就分布本身而言, 当 $n$ 较小时, $t$ 分布与 $z$ 分布的差\\
异是明显的, 随着 $n$ 的增大, $t$ 分布向 $z$ 分布逼近, 它们之间的差异逐渐缩小, $t$ 分布以 $z$分布为极限。当样本量 $n>30$ 时, $t$ 分布与 $z$ 分布非常接近, 具备了用 $z$ 分布取代 $t$ 分布的理由。所以可以说, 当 $n<30$ 时, 如果 $\sigma$ 未知, 则必须使用 $t$ 统计量; 在 $n>30$ 的条件下,选择 $t$ 统计量还是 $z$ 统计量可以根据使用者的偏好。

总体均值和比例检验统计量的确定标准可以归结为图 8-7。

\begin{center}
\end{center}

图 8-7 检验统计量的确定

\subsection*{8.2.2 总体均值的检验}
\begin{enumerate}
  \item 样本量大
\end{enumerate}

\section*{例 8.4}
某机床厂加工一种零件, 根据经验知道, 该厂加工零件的椭圆度渐近服从正态分布,其总体均值为 $0.081 \mathrm{~mm}$, 今另换一种新机床进行加工, 取 200 个零件进行检验, 得到椭圆度均值为 $0.076 \mathrm{~mm}$, 样本标准差为 $0.025 \mathrm{~mm}$, 问新机床加工零件的椭圆度总体均值与以前有无显著差别?

\begin{itemize}
  \item 
  \begin{itemize}
    \item 解 在这个例题中, 我们所关心的是新机床加工零件的椭圆度总体均值与老机床加工零件的椭圆度均值 $0.081 \mathrm{~mm}$ 是否有所不同, 于是可以假设:
  \end{itemize}
\end{itemize}

$H_{0}: \mu=0.081 \mathrm{~mm}$ (没有显著差别); $H_{1}: \mu \neq 0.081 \mathrm{~mm}$ (有显著差别)

这是一个双侧检验问题, 所以只要 $\mu>\mu_{0}$ 或 $\mu<\mu_{0}$ 二者之中有一个成立, 就可以拒绝原假设。

由题意可知, $\mu_{0}=0.081 \mathrm{~mm}, s=0.025 \mathrm{~mm}, \bar{x}=0.076 \mathrm{~mm}$ 。因为 $n>30$, 故选用 $z$ 统计量。

$$
z=\frac{\bar{x}-\mu_{0}}{s / \sqrt{n}}=\frac{0.076-0.081}{0.025 / \sqrt{200}}=-2.83
$$

通常把 $\alpha$ 称为显著性水平 (significant level)。显著性水平是一个统计专有名词, 在假设检验中, 它的含义是当原假设正确时却被拒绝的概率或风险, 其实这就是前面假设检验\\
中犯弃真错误的概率, 它是人们根据检验的要求确定的。通常取 $\alpha=0.05$ 或 $\alpha=0.01$, 这表明, 当作出接受原假设的决定时, 其正确的概率为 $95 \%$ 或 $99 \%$ 。此时不妨取 $\alpha=0.05$,查表可以得到临界值:

$$
z_{\alpha / 2}= \pm 1.96
$$

$z$ 的下标 $\alpha / 2$ 表示双侧检验。

因为 $|z|>\left|z_{\alpha / 2}\right|$, 根据决策准则, 拒绝 $H_{0}$, 可以认为新老机床加工零件椭圆度的均值有显著差别。

\section*{太用 Excel 中的统计函数功能计算 $P$ 值的操作步骤}
第 1 步: 进入 Excel 表格界面, 选择【插入】下拉菜单。

第 2 步:选择【函数】,或者直接点击功能栏中的【fx】。

第 3 步: 在函数类别选择中选 “统计", 然后, 在函数名的菜单中选择字符 "NORM. S. DIST",然后点击【确定】。

第 4 步: 输入 $z$ 的绝对值, Cumulative 逻辑值输入 1 , 得到一个结果记为 $\delta$ 。

第 5 步: 双侧检验, 计算 $P=2 *(1-\delta)$, 得到 $P$ 值; 单侧检验, 计算 $P=1-\delta$, 得到 $P$ 值。

\begin{center}
\end{center}

图 8-8 标准正态分布 $z$ 值示意图本例 $z$ 的绝对值为 2.83 , 计算得到的函数值 $\delta$ 为 0.997672537 , 这意味着在标准正态分布条件下, $z$ 值 2. 83 左边的面积为 0.997672537 , 如图 8-8 所示。 $z=2.83$ 右边和 $z=-2.83$ 左边的面积是一样的。在我们的例子中是双侧检验, 故最后的 $P$ 值为:

$$
P=2 \times(1-0.997672537)=0.004655
$$

$P$ 值远远小于 $\alpha$, 故拒绝 $H_{0}$, 得到与前面相同的结论。

\section*{例 8.5}
我们再对例 8.2 中的假设进行检验。

解 根据前面的分析, 采用左单侧检验。

在该例中, 已知 $\mu_{0}=1000, \bar{x}=960, \sigma=200, n=100$, 并假定显著性水平 $\alpha=0.05$ 。由图 8-5 可知拒绝域在左侧, 所以临界值为负, 即 $z_{\alpha}=-1.645 。 z$ 的下标 $\alpha$ 表示单侧检验。

进行检验的过程为:

$$
\begin{aligned}
& H_{0}: \mu \geqslant 1000 \text { 小时; } H_{1}: \mu<1000 \text { 小时 } \\
& z=\frac{\bar{x}-\mu_{0}}{\sigma / \sqrt{n}}=\frac{960-1000}{200 / \sqrt{100}}=-2
\end{aligned}
$$

由于 $|z|>\left|z_{\alpha}\right|$, 即 $z$ 的值位于拒绝域, 所以拒绝 $H_{0}$, 即这批灯泡的使用寿命低\\
于 1000 小时,批发商不应购买。

如果使用 $P$ 值检验, 按照前述方法, 找到 NORMSDIST, 在 $z$ 值框内录入样本统计量 $z$ 的绝对值 2 , 与之相对的函数值为 0.97725 , 由于这是单侧检验, 故 $P$ 值为:

$$
P=1-0.97725=0.02275
$$

在单侧检验中, 用 $P$ 值直接与 $\alpha$ 比较, 由于 $P(0.02275)<\alpha(0.05)$, 故拒绝 $H_{0}$ 。

但如果在此例的假设检验中, 取显著性水平 $\alpha=0.02$, 则有 $P>\alpha$, 这时就不能拒绝 $H_{0}$ 。

这进一步说明, 检验的结论是建立在概率的基础上的。不能拒绝 $H_{0}$ 并不一定保证 $H_{0}$ 为真, 只是在规定的显著性水平上不能拒绝原假设。上面的例子说明能在 0.95 的置信水平上拒绝原假设, 却不能在 0.98 的置信水平上拒绝原假设。

\begin{enumerate}
  \setcounter{enumi}{1}
  \item 样本量小, $\sigma$ 已知, 正态总体
\end{enumerate}

\section*{例 8.6}
某电子元件批量生产的质量标准为平均使用寿命 1200 小时, 标准差为 150 小时。某厂宣称它采用一种新工艺生产的元件质量大大超过规定标准。为了进行验证, 随机抽取 20 件作为样本, 测得平均使用寿命为 1245 小时。能否说该厂的元件质量显著高于规定标准?

\begin{itemize}
  \item —解 首先需要规定检验的方向。在本例中, 某厂称其产品质量大大超过规定标准 1200 小时, 要检验这个宣称是否可信, 因而是单侧检验。从逻辑上看, 如果样本均值低于 1200 小时, 则该厂的宣称会被拒绝, 即使略高于 1200 小时, 也会被拒绝。只有当样本均值大大超过 1200 小时, 以至于用抽样的随机性也难以解释时, 才能认为该厂产品的质量确实超过规定标准。所以用右单侧检验更为适宜。
\end{itemize}

由题意可知, $\mu_{0}=1200, \bar{x}=1245, \sigma=150, n=20$, 并规定 $\alpha=0.05$ 。虽然 $n<30$,但由于 $\sigma$ 已知, 可以使用 $z$ 统计量。进行检验的过程为:

$$
\begin{aligned}
& H_{0}: \mu \leqslant 1200 \text { 小时; } H_{1}: \mu>1200 \text { 小时 } \\
& z=\frac{\bar{x}-\mu_{0}}{\sigma / \sqrt{n}}=\frac{1245-1200}{150 / \sqrt{20}}=1.34
\end{aligned}
$$

因为这是右单侧检验, 由图 8-6 可知拒绝域在右侧, 查表得到临界值 $z_{\alpha}=1.645$ 。

由于 $z=1.34$, 在非拒绝域, 所以不能拒绝 $H_{0}$, 即不能说该厂产品质量显著高于规定标准。

若用 $P$ 值检验, 方法与前面相同, 在 $z$ 值框内输入 1.34 , 得到函数值为 0.909 , 由于是单侧检验, 故 $P$ 值为:

$$
P=1-0.9099=0.0901
$$

由于 $P>\alpha$, 故不能拒绝 $H_{0}$, 新产品与老产品的质量未表现出显著差别。

\begin{enumerate}
  \setcounter{enumi}{2}
  \item 样本量小, $\sigma$ 未知, 正态总体
\end{enumerate}

\section*{例 8.7}
某机器制造出的肥㿝厚度为 $5 \mathrm{~cm}$, 今欲了解机器性能是否良好, 随机抽取 10 块肥㿝作为样本, 测得平均厚度为 $5.3 \mathrm{~cm}$, 标准差为 $0.3 \mathrm{~cm}$, 试以 0.05 的显著性水平检验机器性能良好的假设。

解 如果机器性能良好, 生产出的肥㿝厚度将在 $5 \mathrm{~cm}$ 上下波动, 过薄或过厚都不符合产品质量标准, 所以,根据题意,这是双侧检验问题。

由于总体 $\sigma$ 未知, 且样本量 $n$ 较小, 所以应采用 $t$ 统计量。

已知条件为: $\mu_{0}=5, \bar{x}=5.3, s=0.3, n=10, \alpha=0.05$ 。

$$
\begin{aligned}
& H_{0}: \mu=5 ; H_{1}: \mu \neq 5 \\
& t=\frac{\bar{x}-\mu_{0}}{s / \sqrt{n}}=\frac{5.3-5}{0.3 / \sqrt{10}}=3.16
\end{aligned}
$$

当 $\alpha=0.05$, 自由度 $n-1=9$ 时, 查表得 $t_{\alpha / 2}(9)=2.2622$ 。因为 $t>t_{\alpha / 2}$, 样本统计量落入拒绝域, 故拒绝 $H_{0}$, 接受 $H_{1}$, 说明该机器的性能不好。

\section*{太用 Excel 计算 $t$ 统计量检验的 $P$ 值的操作步骤}
第 1 步: 进入 Excel 表格界面, 选择【插入】下拉菜单。

第 2 步: 选择【函数】, 或者直接点击功能栏中的【fx】。

第 3 步: 在函数类别选择中选 “统计”, 至此, 与 $z$ 统计量检验中的 $P$ 值计算步骤完全相同。随后,在函数名的菜单中选择字符 "T. DIST",然后点击【确定】。

第 4 步: 在弹出的 X栏中, 输入计算出的 $t$ 值。在自由度 (Deg\_freedom) 栏中, 输入相应的自由度数值。在 Tails 栏中, 如果是单侧检验, 则在该栏输入 1 ; 如果是双侧检验, 则在该栏输入 2 。

本例中的 $t$ 值为 3.16 , 自由度为 9 , Excel 计算的 $P$ 值为 0.01155 。

\subsection*{8.2.3 总体比例的检验}
比例值总是介于 $0 \sim 1$ 或 $0 \sim 100 \%$ 之间, 在实际问题中, 常常需要检验总体比例是否为某个假设值 $\pi_{0}$ 。例如, 全部产品中合格品的比例, 一批种子的发芽率, 对某项改革措施赞同者的比例, 等等。民意调查中常常遇到大量的比例检验问题, 因为调查的内容总是与比例有关, 例如, 希望了解总体中有多少人支持这种或那种观点, 多少人知道这件事或那件事。对于一个政治候选人来说, 在两人竞选时, $50 \%$ 是一个相当重要的标志, 得票率超过 50\% 将在选举中获胜, 所以选举前的民意测验可以给候选人提供如何与对手竞争的重要信息。如果某候选人在一次民意调查中的支持率为 $48 \%$, 他最终能否如愿当选? $48 \%$ 与\\
$50 \%$ 有差异, 这个差异是由于抽样的随机性, 还是由于该候选人的支持率确实低于竞争对手? 候选人竞选班子中的统计专家会精心设计假设检验, 通过不断的民意调查所提供的信息分析竞选的走势。

如果一个事件只有两种结果, 称为二项分布。可以证明, 在样本量大的情况下, 若 $n p>5, n q>5$, 则可以把二项分布问题转化为正态分布问题近似地去求解。

这就是说, 在总体比例的检验中, 通常采用 $z$ 统计量。一般而言, 在有关比例问题的调查中往往使用大样本量, 因为小样本量的结果是极不稳定的。例如, 随机抽取 10 个人,如果支持者有 4 人, 支持率为 $40 \%$; 如果支持者有 5 人, 支持率则为 $50 \%$, 样本中一个人的态度差异导致调查结果相差 10 个百分点, 这种不稳定性是我们不愿意看到的。

在比例问题的检验中, $z$ 统计量的计算公式为:


\begin{equation*}
z=\frac{p-\pi_{0}}{\sqrt{\frac{\pi_{0}\left(1-\pi_{0}\right)}{n}}} \tag{8.3}
\end{equation*}


式中, $p$ 为样本比例; $\pi_{0}$ 为总体比例 $\pi$ 的假设值。

\section*{例 8.8}
一项统计结果声称, 某市老年人口(年龄在 65 岁以上)所占的比例为 $14.7 \%$, 该市老年人口研究会为了检验该项统计是否可靠, 随机抽选了 400 名居民, 发现其中有 57 人年龄在 65 岁以上。调查结果是否支持该市老年人口比例为 $14.7 \%$ 的看法 $(\alpha=0.05) ?$

$\cdot$解 $H_{0}: \pi=14.7 \% ; H_{1}: \pi \neq 14.7 \%$

$$
\begin{aligned}
& p=\frac{57}{400}=0.1425=14.25 \% \\
& z=\frac{p-\pi_{0}}{\sqrt{\frac{\pi_{0}\left(1-\pi_{0}\right)}{n}}}=\frac{0.1425-0.147}{\sqrt{\frac{0.147 \times(1-0.147)}{400}}}=-0.254
\end{aligned}
$$

这是一个双侧检验, 当 $\alpha=0.05$ 时, 有

$$
z_{\alpha / 2}= \pm 1.96
$$

由于 $|z|<\left|z_{\alpha / 2}\right|$, 不能拒绝 $H_{0}$, 可以认为调查结果支持了该市老年人口所占比例为 $14.7 \%$ 的看法。

\subsection*{8.2. 4 总体方差的检验}
在假设检验中, 有时不仅需要检验正态总体的均值、比例, 而且需要检验正态总体的方差。例如, 在产品质量检验中, 质量标准是通过不同类型的指标反映的, 有些属于均值类型, 如尺寸、重量、抗拉强度等; 有些属于比例类型, 如产品合格率、废品率等; 有些属于方差类型, 如尺寸的方差、重量的方差、抗拉强度的方差等。这里, 方差反映产品的稳定性。方差大, 说明产品的性能不稳定, 波动大。凡与均值有关的指标, 通常也与方差\\
有关, 方差从另一个方面说明研究现象的状况。在经济生活方面, 对方差关注的例子比比皆是。例如, 居民的平均收人说明了收人达到的一般水平, 是衡量经济发展阶段的一个重要指标, 而收人的方差则反映了收人分配的差异情况, 可以用来评价收人的合理性。在投资方面, 收益率的方差是评价投资风险的重要依据。

对方差进行检验的程序, 与均值检验、

\begin{center}
\end{center}

图 8-9 $s^{2}$ 分布的直方图比例检验是一样的, 它们之间的主要区别是所使用的检验统计量不同。方差检验所使用的是 $\chi^{2}$ 统计量。对一个方差为 $\sigma^{2}$ 的正态总体反复抽样, 计算每一个样本方差 $s^{2}$, 则 $s^{2}$的分布大体如图 8-9 所示。

由于 $s^{2}=\frac{\sum\left(x_{i}-\bar{x}\right)^{2}}{n-1}$, 故

$$
\sum\left(x_{i}-\bar{x}\right)^{2}=(n-1) s^{2}
$$

可以证明, $\sum\left(x_{i}-\bar{x}\right)^{2}$ 除以总体方差 $\sigma^{2}$ 将服从 $\chi^{2}$ 分布, 即


\begin{equation*}
\chi^{2}=\frac{(n-1) s^{2}}{\sigma^{2}} \tag{8.4}
\end{equation*}


由第 6 章讨论知道, $\chi^{2}$ 分布以正态分布为极限, 在正常情况下 $\chi^{2}$ 分布是偏态分布,因此用 $\chi^{2}$ 统计量进行检验通常采用单侧检验, 临界点在 $\chi^{2}$ 分布右侧斜尾方向。

\section*{例 8.9}
某厂商生产出一种新型的饮料装瓶机器, 按设计要求, 该机器装一瓶 $1000 \mathrm{~mL}$ 的饮料,误差上下不超过 $1 \mathrm{~mL}$ 。如果达到设计要求, 表明机器的稳定性非常好。现从该机器装完的产品中随机抽取 25 瓶, 分别进行测定 (用样本观测值分别减 1000mL), 得到如表 8-2 所示的结果。

表 8-2 25 瓶饮料容量测试结果

单位: $\mathrm{mL}$

\begin{center}
\begin{tabular}{rcccc}
 & 表 8-2 & $\mathbf{2 5}$ 瓶饮料容量测试结果 & 单位: $\mathrm{mL}$ &  \\
\hline
0.3 & -0.4 & -0.7 & 1.4 & -0.6 \\
-0.3 & -1.5 & 0.6 & -0.9 & 1.3 \\
-1.3 & 0.7 & 1 & -0.5 & 0 \\
-0.6 & 0.7 & -1.5 & -0.2 & -1.9 \\
-0.5 & 1 & -0.2 & -0.6 & 1.1 \\
\hline
\end{tabular}
\end{center}

试以 $\alpha=0.05$ 的显著性水平检验该机器的性能是否达到了设计要求。

解 采用单侧检验, 如果样本统计量 $\chi^{2} \geqslant \chi_{\alpha}^{2}(n-1)$, 则拒绝原假设; 若 $\chi^{2}<\chi_{\alpha}^{2}(n-1)$,则不能拒绝原假设。

$$
H_{0}: \sigma^{2} \leqslant 1 ; \quad H_{1}: \sigma^{2}>1
$$

由样本数据可以计算出 $s^{2}=0.866$, 故

$$
\begin{aligned}
\chi^{2} & =\frac{(n-1) s^{2}}{\sigma^{2}}=\frac{(25-1) \times 0.866}{1} \\
& =20.8
\end{aligned}
$$

查 $\chi^{2}$ 分布表可知, $\chi_{0.05}^{2}(24)=36.415$, 故不能拒绝原假设 $H_{0}$, 可以认为该机器的性能达到了设计要求, 如图 8-10 所示。

\begin{center}
\end{center}

图 8-10 $\chi^{2}$ 检验示意图

\section*{8.3 两个总体参数的检验}
在许多情况下, 人们需要比较两个总体的参数, 看它们是否有显著的差别。例如, 在相同年龄组中, 高学历和低学历的职工收人是否有明显的差异; 同一种教学方法, 在不同的年级或不同内容的课程中是否会有不同的效果; 等等。对此, 可以利用两个总体参数的检验寻找答案。

\subsection*{8.3. 1 检验统计量的确定}
两个总体参数检验的主要内容有: 两个总体均值之差的检验, 两个总体比例之差的检验, 两个总体方差比的检验。与一个总体参数的检验类似, 两个总体参数的检验也涉及检验统计量的选择问题。选择什么检验统计量取决于被检验参数的抽样分布, 而抽样分布与样本量大小和总体方差 $\sigma^{2}$ 是否已知都有关系。

两个总体均值之差的检验中可能出现的情况有:

(1) 总体方差 $\sigma_{1}^{2}, \sigma_{2}^{2}$ 已知或未知。

在 $\sigma_{1}^{2}, \sigma_{2}^{2}$ 已知的条件下, 由抽样分布理论可知, 样本统计量服从 $z$ 分布; 而在 $\sigma_{1}^{2}, \sigma_{2}^{2}$未知的条件下, 样本统计量服从 $t$ 分布。故当 $\sigma_{1}^{2}, \sigma_{2}^{2}$ 已知时, 可以使用 $z$ 检验; 当 $\sigma_{1}^{2}, \sigma_{2}^{2}$未知时, 可以使用 $t$ 检验。

(2) $n\left(n_{1}, n_{2}\right)$ 较大或较小。

当样本量 $n_{1}, n_{2}$ 都较大时, 如果总体方差 $\sigma_{1}$ 和 $\sigma_{2}$ 未知, 可以用样本方差 $s_{1}^{2}, s_{2}^{2}$ 替代, 这时, 样本统计量近似服从 $z$ 分布, 采用 $z$ 作为检验统计量是可行的。但是, 当 $n_{1}$或 $n_{2}$ 不大时, 如果 $\sigma_{1}^{2}, \sigma_{2}^{2}$ 未知, 就应该采用 $t$ 作为检验统计量。

两个总体比例之差的检验中, 一般采用 $z$ 统计量, 其理由与前一节总体比例的检验中采用 $z$ 统计量的理由相同。

两个总体方差比的检验中, 正如第 6 章所讨论的, 此时样本统计量服从自由度为 $n_{1}-1$ 和 $n_{2}-1$ 的 $F$ 分布, 在这种情况下使用 $F$ 作为检验统计量。

上述讨论可以归纳为图 8-11。

\begin{center}
\end{center}

图 8-11 检验统计量的确定

\subsection*{8.3.2 两个总体均值之差的检验}
\begin{enumerate}
  \item $\sigma_{1}^{2}, \sigma_{2}^{2}$ 已知
\end{enumerate}

当两个总体均服从正态分布或虽然两个总体的分布形式未知, 但抽自两个总体的样本量均较大, 且两个总体的方差 $\sigma_{1}^{2}, \sigma_{2}^{2}$ 已知时, 可以证明, 由两个独立样本算出的 $\bar{x}_{1}-\bar{x}_{2}$的抽样分布服从正态分布,标准差为:

$$
\sigma_{x_{1}-x_{2}}=\sqrt{\frac{\sigma_{1}^{2}}{n_{1}}+\frac{\sigma_{2}^{2}}{n_{2}}}
$$

此时, 检验统计量 $z$ 的计算公式为:


\begin{equation*}
z=\frac{\left(\bar{x}_{1}-\bar{x}_{2}\right)-\left(\mu_{1}-\mu_{2}\right)}{\sqrt{\frac{\sigma_{1}^{2}}{n_{1}}+\frac{\sigma_{2}^{2}}{n_{2}}}} \tag{8.5}
\end{equation*}


式中, $\mu_{1}$ 为总体 1 的均值; $\mu_{2}$ 为总体 2 的均值。

\section*{例 8.10}
有两种方法可用于制造某种以抗拉强度为重要特征的产品。根据以往的资料得知, 第一种方法生产出的产品抗拉强度的标准差为 8 千克, 第二种方法的标准差为 10 千克。从采用两种方法生产的产品中各抽一个随机样本, 样本量分别为 $n_{1}=32, n_{2}=40$, 测得 $\bar{x}_{1}=$ 50 千克, $\bar{x}_{2}=44$ 千克。问采用这两种方法生产出来的产品平均抗拉强度是否有显著差别 $(\alpha=0.05)$ ?

解 由于检验两种方法生产出的产品在抗拉强度上是否存在显著差别并不涉及方向,所以是双侧检验。建立假设并进行检验:

$$
H_{0}: \mu_{1}-\mu_{2}=0 \text { (没有显著差别); } H_{1}: \mu_{1}-\mu_{2} \neq 0 \text { (有显著差别) }
$$

本题中 $\sigma_{1}^{2}, \sigma_{2}^{2}$ 已知, 应选用 $z$ 作为检验统计量, 采用式 (8.5) 有

$$
z=\frac{\left(\bar{x}_{1}-\bar{x}_{2}\right)-\left(\mu_{1}-\mu_{2}\right)}{\sqrt{\frac{\sigma_{1}^{2}}{n_{1}}+\frac{\sigma_{2}^{2}}{n_{2}}}}
$$

已知 $\bar{x}_{1}=50, \bar{x}_{2}=44, \sigma_{1}^{2}=8^{2}, \sigma_{2}^{2}=10^{2}, n_{1}=32, n_{2}=40$, 故

$$
z=\frac{50-44-0}{\sqrt{\frac{64}{32}+\frac{100}{40}}}=2.83
$$

$\alpha=0.05$ 时, $z_{\alpha / 2}=1.96$ 。

因为 $z>z_{\alpha / 2}$, 所以拒绝 $H_{0}$, 即采用两种方法生产出来的产品的平均抗拉强度有显著差别。

如果计算 $P$ 值, 方法与一个正态总体均值检验中计算 $P$ 值的方法相同。经计算, 此题中的 $P$ 值为 0.004654 。在双侧检验中, 由于 $P<\alpha / 2$, 故拒绝 $H_{0}$, 得到与前面相同的结论。

\section*{2. $\sigma_{1}^{2}, \sigma_{2}^{2}$ 未知, 且 $n$ 较小}
在 $\sigma_{1}^{2}, \sigma_{2}^{2}$ 未知且 $n$ 较小的情况下, 进行两个总体均值之差的检验需要使用 $t$ 统计量,这里有两种情况: 一种是虽然两个总体方差未知, 但知道 $\sigma_{1}^{2}=\sigma_{2}^{2}$ 。这个条件成立往往是从已有的大量经验中得到的, 或者事先进行了关于两个方差相等的检验, 并得到肯定的结论。这时, $\sigma_{x_{1}-x_{2}}$ 的估计为:


\begin{equation*}
\hat{\sigma}_{\bar{x}_{1}-\bar{x}_{2}}=s_{p} \sqrt{\frac{1}{n_{1}}+\frac{1}{n_{2}}} \tag{8.6}
\end{equation*}


式中:


\begin{equation*}
s_{p}^{2}=\frac{\left(n_{1}-1\right) s_{1}^{2}+\left(n_{2}-1\right) s_{2}^{2}}{n_{1}+n_{2}-2} \tag{8.7}
\end{equation*}


于是, 检验统计量 $t$ 的计算公式为:


\begin{equation*}
t=\frac{\left(\bar{x}_{1}-\bar{x}_{2}\right)-\left(\mu_{1}-\mu_{2}\right)}{s_{p} \sqrt{\frac{1}{n_{1}}+\frac{1}{n_{2}}}} \tag{8.8}
\end{equation*}


$t$ 的自由度为 $n_{1}+n_{2}-2$ 。

另一种情况是 $\sigma_{1}^{2}, \sigma_{2}^{2}$ 未知, 且没有理由判定 $\sigma_{1}^{2}$ 与 $\sigma_{2}^{2}$ 相等, 故认为 $\sigma_{1}^{2} \neq \sigma_{2}^{2}$ 。当 $\sigma_{1}^{2}, \sigma_{2}^{2}$未知时, 自然是用样本方差 $s_{1}^{2}, s_{2}^{2}$ 分别估计 $\sigma_{1}^{2}, \sigma_{2}^{2}, \sigma_{\bar{x}_{1}-\bar{x}_{2}}$ 的估计为:


\begin{equation*}
\hat{\sigma}_{\bar{x}_{1}-\bar{x}_{2}}=\sqrt{\frac{s_{1}^{2}}{n_{1}}+\frac{s_{2}^{2}}{n_{2}}} \tag{8.9}
\end{equation*}


但此时抽样分布不服从自由度为 $n_{1}+n_{2}-2$ 的 $t$ 分布, 而是近似服从自由度为 $f$ 的 $t$分布, $f$ 的计算公式为:


\begin{equation*}
f=\frac{\left(\frac{s_{1}^{2}}{n_{1}}+\frac{s_{2}^{2}}{n_{2}}\right)^{2}}{\frac{\left(\frac{s_{1}^{2}}{n_{1}}\right)^{2}}{n_{1}-1}+\frac{\left(\frac{s_{2}^{2}}{n_{2}}\right)^{2}}{n_{2}-1}} \tag{8.10}
\end{equation*}


这时, 检验统计量 $t$ 的计算公式为:


\begin{equation*}
t=\frac{\left(\bar{x}_{1}-\bar{x}_{2}\right)-\left(\mu_{1}-\mu_{2}\right)}{\sqrt{\frac{s_{1}^{2}}{n_{1}}+\frac{s_{2}^{2}}{n_{2}}}} \tag{8.11}
\end{equation*}


$t$ 的自由度为 $f$ 。

\section*{例 8.11}
尽管存在争议, 但大多数科学家认为, 食用含有高纤维的谷类食物有助于降低癌症发生的可能性。然而一个科学家提出, 如果在早餐中食用高纤维的谷类食物, 那么平均而言, 与早餐没有食用谷物的人群相比, 食用谷物者在午餐中摄取的热量将会减少 (Toronto Star, 1991)。如果这个观点成立, 谷物食品的生产商又将获得一个很好的机会, 它们会宣传说: “多吃谷物吧, 早上也吃, 这样有助于减肥。” 为了验证这个假设, 随机抽取了 35 人, 询问他们早餐和午餐的通常食谱, 根据他们的食谱将其分为两类, 一类为经常谷类食用者 (总体1), 一类为非经常谷类食用者 (总体 2 )。然后测度每人午餐的热量摄取量。经过一段时间的实验, 得到的结果如表 8-3 所示。

表 8 - 335 人热量摄取量

\begin{center}
\begin{tabular}{llllll}
\hline
 & 568 & 681 & 636 & 607 & 555 \\
总体 1 & 496 & 540 & 539 & 529 & 562 \\
 & 589 & 646 & 596 & 617 & 584 \\
\hline
 & 650 & 569 & 622 & 630 & 596 \\
 & 637 & 628 & 706 & 617 & 624 \\
 & 563 & 580 & 711 & 480 & 688 \\
 & 723 & 651 & 569 & 709 & 632 \\
\hline
\end{tabular}
\end{center}

试以 $\alpha=0.05$ 的显著性水平进行检验。

一解 本例中要检验的命题是:早餐食用较多的谷类食物有助于减少午餐中热量的摄取。总体 1 和总体 2 的热量摄取均值分别用 $\mu_{1}, \mu_{2}$ 表示。由于此命题是一个尚未被证实的命题, 在单侧检验中, 原假设对此类命题应持否定态度,故建立的假设为:

$$
H_{0}: \mu_{1}-\mu_{2} \geqslant 0 ; \quad H_{1}: \mu_{1}-\mu_{2}<0
$$

由于 $n_{1}, n_{2}$ 均较小, 且 $\sigma_{1}^{2}, \sigma_{2}^{2}$ 未知, 也无法断定 $\sigma_{1}^{2}=\sigma_{2}^{2}$ 是否成立, 故属于 $n$ 较小, $\sigma_{1}^{2}, \sigma_{2}^{2}$ 未知, 且 $\sigma_{1}^{2} \neq \sigma_{2}^{2}$ 的情况。据此, 采用 $t$ 分布, 其自由度为 $f$ 。

经过计算, 得到 $\bar{x}_{1}=583, \bar{x}_{2}=629.25, s_{1}^{2}=2431.429, s_{2}^{2}=3675.461, f=32.34$,若取 $f=32$, 由 $t$ 分布表可知 $t_{0.05}(32)=1.694$ 。由式 (8.11), 检验统计量 $t$ 值为:

$$
t=\frac{\left(\bar{x}_{1}-\bar{x}_{2}\right)-\left(\mu_{1}-\mu_{2}\right)}{\sqrt{\frac{s_{1}^{2}}{n_{1}}+\frac{s_{2}^{2}}{n_{2}}}}=\frac{(583-629.25)-0}{\sqrt{\frac{2431.429}{15}+\frac{3675.461}{20}}}=-2.48689
$$

由于 $|t|>\left|t_{\alpha}\right|$, 故拒绝 $H_{0}$ 。

本题中, 计算的 $P$ 值为 0.009 , 由于 $P<\alpha$, 所以得到拒绝 $H_{0}$ 的相同结论。

太用 Excel 进行两个总体均值之差的检验的操作步骤

将表 8-3 中总体 1 的 15 个数据输入工作表中的 $\mathrm{Al}: \mathrm{A} 15$, 总体 2 的 20 个数据输入工作表中的 $\mathrm{B} 1: \mathrm{B} 20$ 。然后按如下步骤操作:

第 1 步: 选择【数据】下拉菜单。

第 2 步: 选择【数据分析】选项(如果没有该选项, 则需要先在【加载宏】里面加载【分析数据库】)。

第 3 步: 在分析工具中选择【t 检验, 双样本异方差假设】(注: 对于其他条件的假设, 操作步骤类似)。

第 4 步: 当出现对话框时:在【变量 $\mathbf{1}$ 的区域】方框内输入数据区域 $\mathrm{A} 1: \mathrm{A} 15$, 在【变量 $\mathbf{2}$ 的区域】方框内输入数据区域 $\mathrm{B} 1: \mathrm{B} 20$, 在【假设平均差】方框内输入 0 。本例中, 建立的原假设为 $H_{0}: \mu_{1}-\mu_{2} \geqslant 0$, 故输入 0 。如果我们欲检验两个总体均值之差是否等于某个具体数值, 如 $H_{0}: \mu_{1}-\mu_{2}=4$, 则在该框内输入 4 。在【 【框内输入 0.05 。在【输出选项】中选择输出区域(本例中选 “新工作表”)。点击【确定】。输出结果如表 8-4 所示。

表 8-4 $t$-检验: 双样本异方差假设检验输出表

\begin{center}
\begin{tabular}{lcc}
\hline
 & 变量 1 & 变量 2 \\
\hline
平均 & 583 & 629.25 \\
方差 & 2431.428571 & 3675.460526 \\
观测值 & 15 & 20 \\
假设平均差 & 0 &  \\
$\mathrm{df}$ & 33 &  \\
$\mathrm{t}$ Stat & -2.486888876 &  \\
$\mathrm{P}(\mathrm{T}<\mathrm{t})$ 单尾 & 0.009059379 &  \\
$\mathrm{t}$ 单尾临界 & 1.692360456 &  \\
$\mathrm{P}(\mathrm{T}<\mathrm{t})$ 双尾 & 0.018118757 &  \\
$\mathrm{t}$ 双尾临界 & 2.03451691 &  \\
\hline
\end{tabular}
\end{center}

在有原始数据的条件下,使用 Excel 进行运算十分方便。

\subsection*{8.3.3 两个总体比例之差的检验}
设两个总体服从二项分布, 这两个总体中具有某种特征的单位数的比例分别为 $\pi_{1}$ 和 $\pi_{2}$, 但 $\pi_{1}$ 和 $\pi_{2}$ 未知, 可以用样本比例 $p_{1}$ 和 $p_{2}$ 代替。有以下两种情况:

\section*{1. 检验两个总体比例相等的假设}
该假设的表达式为:

$$
\begin{aligned}
& H_{0}: \pi_{1}-\pi_{2}=0 \\
& H_{0}: \pi_{1}=\pi_{2}
\end{aligned}
$$

在原假设成立的条件下, 最佳的方差是 $p(1-p)$, 其中 $p$ 是将两个样本合并后得到的比例估计量, 即


\begin{equation*}
p=\frac{x_{1}+x_{2}}{n_{1}+n_{2}}=\frac{p_{1} n_{1}+p_{2} n_{2}}{n_{1}+n_{2}} \tag{8.12}
\end{equation*}


式中, $x_{1}$ 表示样本 $n_{1}$ 中具有某种特征的单位数; $x_{2}$ 表示样本 $n_{2}$ 中具有某种特征的单位数。

在大样本条件下, 统计量 $z$ 的表达式为:


\begin{equation*}
z=\frac{p_{1}-p_{2}}{\sqrt{p(1-p)\left(\frac{1}{n_{1}}+\frac{1}{n_{2}}\right)}} \tag{8.13}
\end{equation*}


\section*{例 8.12}
人们普遍认为麦当劳的主要消费群体是青少年, 但对市场进一步细分却发现有不同的观点。一种观点认为小学生更喜欢麦当劳, 另一种观点认为中学生对麦当劳的喜爱程度不亚于小学生。某市场调查公司对此在某地区进行了一项调查, 随机抽取了 100 名小学生和 100 名中学生, 调查的问题是: 如果有麦当劳和其他中式快餐 (如兰州拉面), 你会首选哪种作为经常性午餐? 调查结果如下: 小学生(样本 1) 100 人中有 76 人把麦当劳作为首选的经常性午餐, 中学生 (样本 2) 100 人中有 69 人作出同样的选择。调查结果支持哪种观点?

解 作为第三方的调查公司, 其角色是中立的, 所以建立的假设为:

$$
H_{0}: \pi_{1}-\pi_{2}=0 ; \quad H_{1}: \pi_{1}-\pi_{2} \neq 0
$$

或

$$
H_{0}: \pi_{1}=\pi_{2} ; H_{1}: \pi_{1} \neq \pi_{2}
$$

由式 (8.12) 可得

$$
p=\frac{x_{1}+x_{2}}{n_{1}+n_{2}}=\frac{76+69}{100+100}=0.725
$$

由式 (8.13) 可得

$$
\begin{aligned}
z & =\frac{p_{1}-p_{2}}{\sqrt{p(1-p)\left(\frac{1}{n_{1}}+\frac{1}{n_{2}}\right)}} \\
& =\frac{0.76-0.69}{\sqrt{0.725 \times(1-0.725) \times\left(\frac{1}{100}+\frac{1}{100}\right)}}=1.11
\end{aligned}
$$

由决策准则可知, $z=1.11$ 落入接受域, 故调查结果支持原假设, 即在该地区小学生和中学生对麦当劳的偏爱程度没有显著差异。

\section*{2. 检验两个总体比例之差不为零的假设}
检验两个总体比例之差不为零的假设即检验 $\pi_{1}-\pi_{2}=d_{0}\left(d_{0} \neq 0\right)$, 在这种情况下, 两个样本比例之差 $p_{1}-p_{2}$ 近似服从以 $\pi_{1}-\pi_{2}$ 为数学期望, $\frac{p_{1}\left(1-p_{1}\right)}{n_{1}}+\frac{p_{2}\left(1-p_{2}\right)}{n_{2}}$ 为方差的正态分布, 因而可以选择 $z$ 作为检验统计量:


\begin{align*}
z & =\frac{\left(p_{1}-p_{2}\right)-\left(\pi_{1}-\pi_{2}\right)}{\sqrt{\frac{p_{1}\left(1-p_{1}\right)}{n_{1}}+\frac{p_{2}\left(1-p_{2}\right)}{n_{2}}}} \\
& =\frac{\left(p_{1}-p_{2}\right)-d_{0}}{\sqrt{\frac{p_{1}\left(1-p_{1}\right)}{n_{1}}+\frac{p_{2}\left(1-p_{2}\right)}{n_{2}}}} \tag{8.14}
\end{align*}


\section*{例 8.13}
有一项研究报告说青少年经常上网聊天, 男生的比例至少超过女生 10 个百分点, 即 $\pi_{1}-\pi_{2} \geqslant 10 \%$ ( $\pi_{1}$ 为男生比例, $\pi_{2}$ 为女生比例)。现对 150 个男生和 150 个女生进行上网聊天的频度调查, 其中经常聊天的男生有 68 人, 经常聊天的女生有 54 人。调查结果是否支持研究报告的结论 $(\alpha=0.05)$ ?

\begin{itemize}
  \item 解 $H_{0}: \pi_{1}-\pi_{2} \geqslant 10 \% ; H_{1}: \pi_{1}-\pi_{2}<10 \%$
\end{itemize}

由题意可知, $n_{1}=n_{2}=150, p_{1}=\frac{68}{150}=0.45, p_{2}=\frac{54}{150}=0.36, d_{0}=10 \%$ 。

由式 (8.14) 可得:

$$
\begin{aligned}
z & =\frac{\left(p_{1}-p_{2}\right)-d_{0}}{\sqrt{\frac{p_{1}\left(1-p_{1}\right)}{n_{1}}+\frac{p_{2}\left(1-p_{2}\right)}{n_{2}}}} \\
& =\frac{0.45-0.36-0.1}{\sqrt{\frac{0.45 \times(1-0.45)}{150}+\frac{0.36 \times(1-0.36)}{150}}}=-0.177
\end{aligned}
$$

这是一个左单侧检验, $z_{\alpha}=-1.645$, 由决策准则可知, $z=-0.177$, 落入非拒绝域,故无法推翻原假设, 调查结果支持研究报告的结论。

\subsection*{8.3.4 两个总体方差比的检验}
如果要检验两个总体方差是否相等, 可以通过考察两个方差之比是否等于 1 来进行。实践中会遇到关注两个总体方差是否相等的问题, 如比较两个生产过程的稳定性, 比较两种投资方案的风险等。前面讨论两个总体均值之差的检验时, 假定两个总体方差相等或不\\
相等。事实上, 在许多情况下总体方差是否相等事先往往并不知道, 因此在进行两个总体均值之差的检验之前, 可以先进行两个总体方差是否相等的检验, 由此获得所需要的信息。

为了比较两个未知的总体方差 $\sigma_{1}^{2}$ 和 $\sigma_{2}^{2}$, 我们用两个样本方差的比来判断, 如果 $s_{1}^{2} / s_{2}^{2}$接近 1 , 说明两个总体方差 $\sigma_{1}^{2}$ 和 $\sigma_{2}^{2}$ 很接近; 如果比值远离 1 , 说明 $\sigma_{1}^{2}$ 与 $\sigma_{2}^{2}$ 之间有较大差异。由第 6 章的内容可知, 在两个正态总体条件下, 两个方差之比服从 $F$ 分布, 即


\begin{equation*}
F=\frac{s_{1}^{2} / \sigma_{1}^{2}}{s_{2}^{2} / \sigma_{2}^{2}} \tag{8.15}
\end{equation*}


在原假设 $\sigma_{1}^{2}=\sigma_{2}^{2}$ 下, 检验统计量 $F=\frac{s_{1}^{2}}{s_{2}^{2}}$, 此时 $F$ 统计量的两个自由度分别为: 分子自由度 $n_{1}-1$, 分母自由度 $n_{2}-1$ 。

在单侧检验中, 一般把较大的 $s^{2}$ 放在分子 $s_{1}^{2}$ 的位置, 此时 $F>1$, 拒绝域在 $F$ 分布的右侧, 原假设和备择假设分别为:

$$
H_{0}: \sigma_{1}^{2} \leqslant \sigma_{2}^{2} ; \quad H_{1}: \sigma_{1}^{2}>\sigma_{2}^{2}
$$

临界点为 $F_{\alpha}\left(n_{1}-1, n_{2}-1\right)$ 。这样处理含义明确, 易于理解, 且查表方便。在双侧检验中, 拒绝域在 $F$ 分布的两侧, 两个临界点的位置分别为:

$$
\text { . } F_{\alpha / 2}\left(n_{1}-1, n_{2}-1\right), F_{1-\alpha / 2}\left(n_{1}-1, n_{2}-1\right)
$$

如图 8-12 所示。

\begin{center}
\end{center}

图 8-12 $F$ 分布双侧检验示意图

通常, $F$ 分布表仅给出 $F_{\alpha / 2}$ 的位置,可以用它来推算 $F_{1-\alpha / 2}$ 的位置, 推算公式为:


\begin{equation*}
F_{1-\alpha / 2}=\frac{1}{F_{\alpha / 2}\left(n_{2}-1, n_{1}-1\right)} \tag{8.16}
\end{equation*}


注意在式 (8.16) 中, 等号右边分母 $F_{\alpha / 2}$ 的自由度要调换一下。

\section*{例 8.14}
在例 8.11 中, 得到两个样本的方差分别为 $s_{1}^{2}=2431.429, s_{2}^{2}=3675.461$ (见表 8-4),现以 $\alpha=0.05$ 的显著性水平检验两个总体的方差是否相等。

解 由于是检验 $\sigma_{1}^{2}$ 和 $\sigma_{2}^{2}$ 是否相等, 故采用双侧检验:

$$
H_{0}: \sigma_{1}^{2}=\sigma_{2}^{2} ; H_{1}: \sigma_{1}^{2} \neq \sigma_{2}^{2}
$$

或 $H_{0}: \sigma_{1}^{2} / \sigma_{2}^{2}=1 ; H_{1}: \sigma_{1}^{2} / \sigma_{2}^{2} \neq 1$

$$
F=\frac{s_{1}^{2}}{s_{2}^{2}}=\frac{2431.429}{3675.461}=0.662
$$

对于 $F_{\alpha / 2}\left(n_{1}-1, n_{2}-1\right)$, 查表得

$$
F_{0.025}(14,19)=2.62
$$

(注: 由于自由度 $n_{1}-1=14$ 数值表中没有, 故取 15 。)

$$
F_{0.025}(19,14)=2.84
$$

(由于同样的原因用 20 代替 19 。)

由式 (8.16), 有

$$
F_{1-\alpha / 2}=\frac{1}{F_{\alpha / 2}\left(n_{2}-1, n_{1}-1\right)}=\frac{1}{2.84}=0.352
$$

本例中, 两个临界点分别为 $F_{1-\alpha / 2}=0.352, F_{a / 2}=2.62$, 样本统计量 $F$ 值为 0.662 ,故不能拒绝 $H_{0}$, 可以认为这两个总体的方差没有显著差异。

\subsection*{8.3.5 检验中的匹配样本}
在前面对两个总体参数进行显著性检验的讨论中, 我们都假定样本是独立的。然而在可能的情况下, 采用存在相依关系的匹配样本分析, 可以进一步提高效率。对此用下例说明。

\section*{例 8.15}
一个以减肥为主要目标的健美俱乐部声称, 参加它的训练班至少可以使肥胖者平均体重减轻 8.5 千克以上。为了验证该声称是否可信, 调查人员随机抽取了 10 名参加者, 得到他们的体重记录如表 8-5 所示。

表 8-5 训练前后的体重记录

单位: 千克

\begin{center}
\begin{tabular}{lcccccccccc}
\hline
训练前 & 94.5 & 101 & 110 & 103.5 & 97 & 88.5 & 96.5 & 101 & 104 & 116.5 \\
\hline
训练后 & 85 & 89.5 & 101.5 & 96 & 86 & 80.5 & 87 & 93.5 & 93 & 102 \\
\hline
\end{tabular}
\end{center}

在 $\alpha=0.05$ 的显著性水平下, 调查结果是否支持该俱乐部的声称?

\begin{itemize}
  \item 解 $H_{0}: \mu_{1}-\mu_{2} \leqslant 8.5$ 平均减重没有超过 8.5 千克
\end{itemize}

$H_{1}: \mu_{1}-\mu_{2}>8.5$ 平均减重超过 8.5 千克

此时, 与训练前后的体重相比, 调查人员更关心它们之间的差值。根据上述资料, 可以构造出一个差值(减重)样本,如表 8-6 所示。

表 8-6 差值样本构造表

单位: 千克

\begin{center}
\begin{tabular}{ccc}
\hline
训练前 & 训练后 & 差值 $x$ \\
\hline
94.5 & 85 & 9.5 \\
101 & 89.5 & 11.5 \\
110 & 101.5 & 8.5 \\
103.5 & 96 & 7.5 \\
97 & 86 & 11 \\
\hline
\end{tabular}
\end{center}

续表

\begin{center}
\begin{tabular}{ccc}
\hline
训练前 & 训练后 & 差值 $x$ \\
\hline
88.5 & 80.5 & 8 \\
96.5 & 87 & 9.5 \\
101 & 93.5 & 7.5 \\
104 & 93 & 11 \\
116.5 & 102 & 14.5 \\
\hline
合计 & - & 98.5 \\
\hline
\end{tabular}
\end{center}

差值样本的均值和标准差分别为:

$$
\begin{aligned}
\bar{x} & =\frac{\sum x}{n}=\frac{98.5}{10}=9.85 \\
s & =\sqrt{\frac{\sum\left(x_{i}-\bar{x}\right)^{2}}{n-1}} \\
& =\sqrt{\frac{(9.5-9.85)^{2}+\cdots+(14.5-9.85)^{2}}{10-1}}=2.199
\end{aligned}
$$

由此得到抽样分布的标准差的估计值为:

$$
\hat{\sigma}_{x}=\frac{s}{\sqrt{n}}=\frac{2.199}{\sqrt{10}}=0.695
$$

因为样本量 $n$ 不大, 所以采用 $t$ 分布, 其自由度为 $n-1=10-1=9$ 。由前可知, 这是右单侧检验, 当 $\alpha=0.05$ 时, $t_{\alpha}(9)=1.833$ 。

若把减重 (差值) 视为一总体, 调查人员关心其均值是否大于 8.5 千克。根据上述资料, 可以计算出拒绝原假设的临界点:

$8.5+t_{\alpha} \hat{\sigma}_{\bar{x}}=8.5+1.833 \times 0.695=9.774$

上述决策如图 8-13 所示。

\begin{center}
\end{center}

图 8-13 右单侧检验示意图 (一)

若样本均值 $\bar{x}<9.774$, 接受原假设 $H_{0}: \mu_{1}-\mu_{2} \leqslant 8.5$; 若 $\bar{x}>9.774$, 接受备择假设 $H_{1}: \mu_{1}-\mu_{2}>8.5$ 。这里, $\bar{x}=9.85>$ 9. 774 , 故拒绝原假设, 可以认为该俱乐部的声称是可信的。

作为对比, 考察相同背景下的两个独立样本。若调查人员随机抽取 10 名参加者训练前的体重记录, 又随机抽取另 10 名参加者训练后的体重记录, 假设同样得到表 8-5 中的数据。计算结果如表 8-7 所示。\\
表 8-7 根据表 8-5 计算的结果

\begin{center}
\begin{tabular}{cccc}
\hline
样本 & 样本量 & 均值 & 方差 \\
\hline
训练前 & $n_{1}=10$ & $\bar{x}_{1}=101.25$ & $s_{1}^{2}=63.40$ \\
训练后 & $n_{1}=10$ & $\bar{x}_{2}=91.40$ & $s_{2}^{2}=50.49$ \\
\hline
\end{tabular}
\end{center}

由式 (8.7), 有

$$
\begin{aligned}
s_{p}^{2} & =\frac{\left(n_{1}-1\right) s_{1}^{2}+\left(n_{2}-1\right) s_{2}^{2}}{n_{1}+n_{2}-2} \\
& =\frac{(10-1) \times 63.40+(10-1) \times 50.49}{10+10-2} \\
& =56.945 \\
s_{p} & =\sqrt{56.945}=7.546 \\
\hat{\sigma}_{x_{1}-\bar{x}_{2}} & =s_{p} \sqrt{\frac{1}{n_{1}}+\frac{1}{n_{2}}}=7.546 \times \sqrt{\frac{1}{10}+\frac{1}{10}}=3.375
\end{aligned}
$$

此时, $t$ 分布的自由度为 $n_{1}+n_{2}-2=10+10-2=18, \alpha=0.05$, 故

$$
t_{\alpha}(18)=1.7341
$$

可以计算出拒绝域的临界点为:

$$
8.5+t_{\alpha} \hat{\sigma}_{x_{1}-x_{2}}=8.5+1.7341 \times 3.375=14.352
$$

因为 $\bar{x}_{1}-\bar{x}_{2}=101.25-91.40=9.85<14.352$, 所以不能拒绝 $H_{0}$, 调查结果否认该俱乐部的声称。上述决策如图 8-14 所示。

为什么由表 $8-5$ 中相同的数据会得出不同的结论呢? 通过对比可以看出, 在匹配样本的检验中, 抽样分布的标准差 $\hat{\sigma}_{x}=$ 0.695 , 而在独立样本的检验中, 抽样分布的标准差 $\hat{\sigma}_{x_{1}-x_{2}}=3.375$ 。与较小的标准差相比, 9.85 显著大于 8.5 ; 而与较大的标准差相比, 9.85 大于 8.5 的程度则不显著。由于匹配样本实质上起到了控制观测变量影响因素的作用, 因而可以得到更为精确的推断

\begin{center}
\end{center}

图 8-14 右单侧检验示意图 (二)结果。

需要注意的是, 在什么情况下可以把两个样本看成匹配样本? 可以考虑下面两个例子。一个例子是研究人员检验新稻种和旧稻种是否有显著差异。如果从一个地区抽取新稻种的样本, 从另一个地区抽取旧稻种的样本, 则两个样本是独立的; 但如果将一块地一分为二, 一边用新稻种, 一边用旧稻种, 这时稻种生长所依赖的土壤、水分、气候等自然条件均相同, 这样的样本就是匹配样本。另一个例子是欲检验打字员在使用两种不同型号的打字机时打字速度是否有显著差异。假设让一批打字员使用某种型号的打字机, 让另一批\\
打字员使用另一种型号的打字机, 这时的样本是独立的; 但如果让同一批打字员分别使用不同型号的打字机, 这时的样本就是匹配样本。在两个总体参数的检验问题中, 根据可能的情况采用匹配样本的设计, 可以有效地提高检验的效率。

\section*{8.4 检验问题的进一步说明}
\subsection*{8.4. 1 关于检验结果的解释}
在前面各种类型的检验中, 我们采用是否拒绝原假设 $H_{0}$ 的方式达到检验目的。事实上, 原假设是对总体分布的某个未知特征的一种猜测, 我们并不知道这个猜测是否正确。但是在选择 $\alpha$ 作为显著性的标准时, 却是在 $H_{0}$ 为真的前提下进行的。意思是, 正常情况下事件结果应该与原假设 $H_{0}$ 相差不大,如果发生了与 $H_{0}$ 不一致的概率小于显著性水平 $\alpha$的事件, 则拒绝 $H_{0}$, 否则, 不拒绝 $H_{0}$ 。这种反证法的特点, 保证了犯第 I 类错误的概率不超过 $\alpha$, 即错误地拒绝 $H_{0}$ 的概率不超过 $\alpha$, 但无法提供有关犯第 II 类错误的信息,即不知道错误地接受 $H_{0}$ 的概率。因此, 对于显著性水平 $\alpha$ 的检验准则而言, 如果出现拒绝 $H_{0}$的结果, 我们可以说 “结论 $H_{1}$ 为真出错的概率不超过 $\alpha$ ”。

从假设检验的原理看, 不拒绝原假设意味着我们所构造的与原假设相矛盾的小概率事件没有发生, 但可能会有许多其他的与原假设矛盾的小概率事件, 我们没有也无法证实所有的这些小概率事件不会发生。因此, 我们把假设检验中出现接受 $H_{0}$ 的结果解释为 “没有发现充足的证据拒绝 $H_{0}$ ”, 或更严格地解释为 “在显著性水平 $\alpha$ 下没有发现充足的证据拒绝 $H_{0}$ ”,而不是 “接受原假设 $H_{0}$ ”,因为我们无法证明原假设是真的。

\subsection*{8.4.2 单侧检验中假设的建立}
在单侧检验中,如何建立假设是一个需要考虑的问题。如果是左侧检验,即

$$
H_{0}: \mu \geqslant \mu_{0} ; H_{1}: \mu<\mu_{0}
$$

当 $\left|\cdot \bar{x}-\mu_{0}\right|<\Delta$ 时, 不拒绝 $H_{0}$ 。

如果是右侧检验, 即

$$
H_{0}: \mu \leqslant \mu_{0} ; H_{1}: \mu>\mu_{0}
$$

当 $\left|\bar{x}-\mu_{0}\right|<\Delta$ 时, 不拒绝 $H_{0}$ 。同一个数据却得出相反的结论, 这种情况可以由下面的例子得到说明。

\section*{例 8.16}
某种灯泡的质量标准是平均使用寿命不得低于 1000 小时。已知灯泡批量产品的使用寿命服从正态分布, 且标准差为 100 小时。商店欲从工厂进货, 随机抽取 81 个灯泡检验,测得 $\bar{x}=990$ 小时, 问商店是否该购进这批灯泡 $(\alpha=0.05) ?$\\
第一种, 认为该厂生产的灯泡不会低于规定的质量标准, 故检验 $\mu \geqslant 1000$ 小时是否成立。

$H_{0}: \mu \geqslant 1000$ 小时; $H_{1}: \mu<1000$ 小时

这是左侧检验, 检验统计量 $z$ 为:

$$
z=\frac{990-1000}{100 / \sqrt{81}}=-0.9
$$

而 $z_{\alpha}=-1.645$, 由于 $|z|<\left|z_{\alpha}\right|$, 所以不能拒绝 $H_{0}$, 即可以认为该厂生产的灯泡达到了规定的质量标准。

第二种, 认为该厂生产的灯泡很可能低于规定的质量标准, 故检验 $\mu \leqslant 1000$ 小时是否成立。

$H_{0}: \mu \leqslant 1000$ 小时; $H_{1}: \mu>1000$ 小时

这是右侧检验, 临界值 $z_{\alpha}=1.645$ 在分布曲线的右侧, 检验统计量 $z=-0.9$, 故同样不能拒绝 $H_{0}$ ,即认为灯泡的质量没有达到规定标准。

于是出现了两种情况下的推断似乎矛盾的现象。其实, 这也反映了统计推断的一种特点, 它不是简单地采用那种 “非此即彼” 的逻辑。为了便于说明, 可以认为检验是在两种不同的背景下进行的。第一种假设的背景是, 从过去的历史记录看, 该灯泡厂有良好的声誉, 商店相信该厂的质量一贯不错, 于是选择 $\mu \geqslant \mu_{0}$ 作为原假设。这样做对灯泡厂是有利的, 因为这使得达到质量标准的产品只以很低的概率 $\alpha$ 被拒收。虽然这会使商店面临接受不合格产品的风险, 但厂家良好的历史记录显示发生这种情况的可能性很小。商店也因为增大货源而获利。

第二种假设的背景是, 以往的记录表明, 厂家的产品质量并不是很好, 这时, 商店就可以坚持以 $\mu \leqslant \mu_{0}$ 为原假设。这样做表明商店要求有强有力的证据才能相信这批产品的质量达到了标准。这就类似于说一个人一向表现不好, 则必须有显著的好的表现, 人们才能相信他确实有进步。这样做就达到了至少把 $100(1-\alpha) \%$ 的不合格产品拒之门外的目的。

由此可见, 同一个问题, 由于对背景的了解不同而采取了不同的态度, 具体是通过选择假设的方向来体现的。这样也就不难理解前面所出现的表面矛盾。当产品质量一贯很好时, 我们认为稍差的样本并不成为整批产品非优的有力证据; 当产品质量一贯不好时, 我们认为测试合格的样本也不成为整批产品为优的有力证据。出发点不同, 并无矛盾可言。

当然, 在实际问题的检验中, 我们不可能对问题的背景都有所了解, 如何提出假设, 特别是单侧假设的方向, 便成为一个问题。遗憾的是, 如何确定假设并没有固定的统一标准, 假设的确定通常与所要检验的问题的性质、检验者所要达到的目的有一定关系, 也与检验员的经验和知识水平有关。不过, 在假设检验中, 一般是把希望证明的命\\
题放在备择假设上, 而把原有的、传统的观点或结论放在原假设上, 这样可以更好地体现假设检验的价值。设想一下, 如果我们完全认可原有的东西, 就没有必要去进行检验了, 正是我们对原有的东西产生怀疑, 才去进行检验, 希望能够用事实推翻原有的观念, 得出新的结论。由于推翻原假设需要检验统计量落人拒绝域, 所以在一次试验中原假设是具有优势的, 由于小概率原理, 备择假设在一次试验中不容易发生, 一旦发生,我们就有充足的理由推翻原假设, 这意味着一个新结论的诞生。但是没有拒绝原假设,并不意味着备择假设就是错的, 只是说还没有足够的证据表明原假设不成立。这与法庭上对被告定罪类似, 总是先假定被告无罪 (原假设), 然后看这些证据是否能判定被告有罪 (备择假设)。如果证据不充分, 我们不能说被告一定清白, 只是说目前的证据还无法认定被告有罪。如果证据充分, 则可以得到被告有罪的定论。因而在假设检验中,对统计结论的正确理解是很重要的。接受备择假设一定意味着原假设错误; 没有拒绝原假设并不能表明备择假设一定是错的。

所谓 “原有的、传统的” 是指原有的理论、原有的看法、原有的状况,或者说是那些历史的、经验的、被大多数人所认可和接受的东西, 在没有充分证据证明其错误时, 总是假定是正确的, 处于原假设被保护的位置。而那些新的、可能的、猜测的东西则处于备择假设的位置。人们感兴趣的是那些新的、可能的、猜测的东西, 希望用事实推翻原假设,实现吐故纳新。可以分析下面几个例子:

\begin{itemize}
  \item 采用新技术后, 会使产品的使用寿命延长到 5000 小时以上。
\end{itemize}

分析: 产品的使用寿命没有超过 5000 小时是原来的情况, 在没有充分事实证明前不应轻易否定, 故

$$
\begin{array}{ll}
H_{0}: \mu \leqslant 5000 \text { 小时 } & \text { 不能轻易否定的命题 } \\
H_{1}: \mu>5000 \text { 小时 } & \text { 需要验证的命题 }
\end{array}
$$

\begin{itemize}
  \item 改进生产工艺后, 会使产品的废品率降到 1\%以下。
\end{itemize}

分析: 以前的产品废品率在 $1 \%$ 以上, 改进生产工艺可以使产品废品率下降正是需要验证的命题, 故

$$
\begin{array}{ll}
H_{0}: \pi \geqslant 1 \% & \text { 不能轻易否定的命题 } \\
H_{1}: \pi<1 \% & \text { 需要验证的命题 }
\end{array}
$$

\begin{itemize}
  \item 一项研究认为, 与不吸烟相比, 吸烟容易导致肺癌。
\end{itemize}

分析: 若令 $\pi_{1}$ 为吸烟人群的肺癌发病率, $\pi_{2}$ 为不吸烟人群的肺癌发病率, 则该命题的表述为 $\pi_{1} \geqslant \pi_{2}$ 。也许有人对此有不同看法, 但要推翻这一命题, 验证 “吸烟与肺癌无关” 的新命题, 则需要证据。因此

$$
\begin{array}{ll}
H_{0}: \pi_{1} \geqslant \pi_{2} & \text { 不能轻易否定的命题 } \\
H_{1}: \pi_{1}<\pi_{2} & \text { 需要验证的命题 }
\end{array}
$$

单侧检验中假设的建立, 本质上取决于检验人员对检验问题的价值判断, 因此, 对同一个问题提出不同方向的假设这种情况也是有的。我们不能武断地说哪种对, 哪种错, 只能在价值判断共同标准的前提下, 讨论在具体的条件下哪种假设更为合理。

\section*{思考与练习}
\section*{思考题}
8.1 假设检验和参数估计有什么相同点和不同点?

8.2 什么是假设检验中的显著性水平? 统计显著是什么意思?\\
8. 3 什么是假设检验中的两类错误?

8.4 两类错误之间存在什么样的数量关系?

8.5 解释假设检验中的 $P$ 值。

8.6 显著性水平与 $P$ 值有何区别?

8.7 假设检验依据的基本原理是什么?

8.8 在单侧检验中, 原假设和备择假设的方向应该如何确定?

\section*{练习题}
8.1 已知某炼铁厂的铁水含碳量服从正态分布 $N\left(4.55,0.108^{2}\right)$, 现在测定了 9 炉铁水, 其平均含碳量为 4.484 。如果估计方差没有变化, 能否认为现在生产的铁水平均含碳量为 $4.55(\alpha=0.05)$ ?

8.2 有一种元件, 要求其使用寿命不得低于 700 小时。现从一批这种元件中随机抽取 36 件, 测得其平均寿命为 680 小时。已知该元件寿命服从正态分布, $\sigma=60$ 小时, 试在显著性水平 0.05 下确定这批元件是否合格。

8.3 某地区小麦的一般生产水平为亩产 250 千克, 其标准差为 30 千克。现用一种化肥进行试验,从 25 个地块抽样, 平均亩产量为 270 千克。这种化肥是否使小麦明显增产 $(\alpha=0.05) ?$

8.4 糖厂用自动打包机打包, 每包的标准重量是 100 千克。每天开工后需要检验一次打包机工作是否正常。某日开工后测得 9 包重量 (单位: 千克) 如下:\\
99.3\\
$98.7 \quad 100.5$\\
101.298 .3\\
$99.7 \quad 99.5 \quad 102.1$\\
100.5

已知每包的重量服从正态分布,试检验该日打包机工作是否正常 $(\alpha=0.05)$ 。

8.5 某种大量生产的袋装食品按规定每袋不得少于 250 克。今从一批该食品中任意抽取 50 袋, 发现有 6 袋低于 250 克。若规定不符合标准的比例超过 $5 \%$ 就不得出厂, 问该批食品能否出厂 $(\alpha=0.05) ?$

8.6 某厂家在广告中声称, 该厂生产的汽车轮胎在正常条件下行驶距离超过目前的平均水平 25000 公里。对一个由 15 个轮胎组成的随机样本做了试验, 得到样本均值和标准差分别为 27000 公里和 5000 公里。假定轮胎寿命服从正态分布, 问该厂家的广告所声称的内容是否真实 $(\alpha=0.05) ?$

8.7 某种电子元件的寿命 $x$ 服从正态分布。现测得 16 只元件的寿命(单位:小时)如下:

\begin{center}
\begin{tabular}{llllllll}
159 & 280 & 101 & 212 & 224 & 379 & 179 & 264 \\
222 & 362 & 168 & 250 & 149 & 260 & 485 & 170 \\
\end{tabular}
\end{center}

是否有理由认为元件的平均寿命显著地大于 225 小时 $(\alpha=0.05) ?$

8.8 随机抽取 9 个单位, 测得结果分别为:\\
$\begin{array}{lllllllll}85 & 59 & 66 & 81 & 35 & 57 & 55 & 63 & 66\end{array}$

以 $\alpha=0.05$ 的显著性水平对下列假设进行检验: $H_{0}: \sigma^{2} \leqslant 100 ; H_{1}: \sigma^{2}>100$ 。

$8.9 \mathrm{~A}, \mathrm{~B}$ 两厂生产同样的材料。已知其抗压强度服从正态分布, 且 $\sigma_{A}^{2}=63^{2}, \sigma_{B}^{2}=57^{2}$ 。从 A厂生\\
产的材料中随机抽取 81 个样品, 测得 $\bar{x}_{A}=1070 \mathrm{~kg} / \mathrm{cm}^{2}$; 从 B 厂生产的材料中随机抽取 64 个样品, 测得 $\bar{x}_{B}=1020 \mathrm{~kg} / \mathrm{cm}^{2}$ 。根据以上调查结果, 能否认为 $\mathrm{A}, \mathrm{B}$ 两厂生产的材料的平均抗压强度相同 $(\alpha=$ $0.05)$ ?

8.10 装配一个部件可以采用不同的方法, 所关心的问题是哪种方法的效率更高。劳动效率可以用平均装配时间来反映。现从不同的装配方法中各抽取 12 件产品, 记录各自的装配时间(单位: 分钟)如下:

甲方法: $\begin{array}{llllllllllll}11 & 34 & 29 & 32 & 35 & 38 & 34 & 30 & 29 & 32 & 31 & 26\end{array}$

\begin{center}
\end{center}

两总体为正态总体, 且方差相同, 问这两种方法的装配时间有无显著差别 $(\alpha=0.05) ?$

8.11 调查了 339 名 50 岁以上的人, 在 205 名吸烟者中有 43 个患慢性气管炎, 在 134 名不吸烟者中有 13 人患慢性气管炎。调查数据能否支持 “吸烟者容易患慢性气管炎” 这种观点 $(\alpha=0.05)$ ?

8.12 为了控制贷款规模, 某商业银行有个内部要求, 平均每项贷款的数额不能超过 60 万元。随着经济的发展, 贷款规模有增大的趋势。银行经理想了解在同样的项目条件下, 贷款的平均规模是否明显地超过 60 万元, 故一个 $n=144$ 的随机样本被抽出, 测得 $\bar{x}=68.1$ 万元, $s=45$ 。在 $\alpha=0.01$ 的显著性水平下采用 $P$ 值进行检验。

8.13 有一种理论认为服用阿司匹林有助于减少心脏病的发生, 为了进行验证, 研究人员把自愿参与实验的 22000 人随机平均分成两组, 一组人员每星期服用三次阿司匹林 (样本 1), 另一组人员在相同的时间服用安慰剂 (样本 2)。持续 3 年之后进行检测, 样本 1 中有 104 人患心脏病, 样本 2 中有 189 人患心脏病。以 $\alpha=0.05$ 的显著性水平检验服用阿司匹林是否可以降低心脏病发病率。

8.14 某工厂制造螺检, 规定螺检口径为 $7.0 \mathrm{~cm}$, 方差为 $0.03 \mathrm{~cm}$ 。今从一批螺检中抽取 80 个测量其口径, 得到平均值为 $6.97 \mathrm{~cm}$, 方差为 $0.0375 \mathrm{~cm}$ 。假定螺检口径服从正态分布, 问这批蝶检是否达到规定的要求 $(\alpha=0.05)$ ?

8.15 有人说在大学中男生的学习成绩比女生的学习成绩好。现从一所学校中随机抽取 25 名男生和 16 名女生, 对他们进行相同题目的测试。测试结果表明, 男生的平均成绩为 82 分, 方差为 56 分, 女生的平均成绩为 78 分, 方差为 49 分。假设显著性水平 $\alpha=0.02$, 从上述数据中能得到什么结论?

\section*{第 9 章}
\begin{center}
\end{center}

泰坦尼克号的死亡记录告诉了我们什么?

1912 年 4 月 15 日, 豪华巨轮泰坦尼克号与冰山相撞后沉没。1985 年, 泰坦尼克号的沉船遗骸被发现。美国探险家洛维特在船舱里看见了一幅画, 102 岁高龄的罗丝声称!她就是画中的少女。罗丝开始叙述她当年的故事: 1912 年 4 月 10 日, 被称为“世界工!业史上的奇迹”的泰坦尼克号从英国的南安普敦出发驶往美国纽约。富家少女罗丝与母!亲及未婚夫卡尔一道上船, 另一边, 不羁的少年画家杰克靠在码头上的一场赌博赢到了船票。罗丝不愿嫁给卡尔, 打算投海自尽, 被杰克抱住。很快, 美丽活泼的罗丝与英俊开朗的杰克相爱了。然而悲剧发生了, 泰坦尼克号与冰山相撞。杰克把生存的机会让给了爱人罗丝, 自己则在海中被冻死。老态龙钟的罗丝把那串价值连城的珠宝沉入海底,让它陪着杰克和这段爱情长眠海底。我们后来看到的电影《泰坦尼克号》就是根据罗丝!的回忆拍摄的。

据记载, 当时船上有 1316 名乘客和 892 名船员, 共 2208 人, 事故发生后幸存 718 人, 约 $2 / 3$ 的人在海难中丧生。2 208 人中, 按性别划分, 男性 1738 人, 女性 470 人;按年龄划分, 成年人 2099 人, 儿童 109 人; 按所在舱位划分, 一等舱 325 人, 二等舱 285 人, 三等舱 706 人, 船员舱 892 人。在幸存的 718 人中, 按性别划分, 男性 374 人,女性 344 人; 按年龄划分, 成年人 661 人, 儿童 57 人; 按所在舱位划分, 一等舱 203 人, 二等舱 118 人, 三等舱 178 人, 船员舱 219 人。

以上都是分类数据。数据是枯燥的, 但讲述的问题却是鲜活的。死亡与性别是否有关? 与年龄是否有关? 与所在舱位是否有关? 如何解释这些关系? 当时人们的价值观念和对待死亡的态度有什么联系? 通过本章的学习, 可以掌握对分类数据进行分析的方法。\\
本章讨论的统计方法主要用于分类数据的分析。我们在上一章中介绍了两个总体比例之差的检验, 如果对更多的总体比例进行比较, 则需要采用本章介绍的方法。对分类数据进行分析的统计方法主要是利用 $\chi^{2}$ 分布, 许多教材将其称为 $\chi^{2}$ 检验。 $\chi^{2}$ 检验的应用主要表现在两个方面: 拟合优度检验和独立性检验。列联表是进行独立性检验的重要工具。

\section*{9.1 分类数据与 $x^{2}$ 统计量}
\subsection*{9.1.1 分类数据}
在第 1 章中曾提出统计数据的类型有分类数据和数值型数据等。分类数据是对事物进行分类的结果, 其特征是, 调查结果虽然用数值表示, 但不同数值描述了调查对象的不同特征。例如, 研究青少年家庭状况与行为之间的关系, 青少年家庭状况是一个分类数据,可以分为 “完整家庭” 和 “离异家庭”, 如果调查结果为 “ 1 ”, 表示被调查者来自完整家庭, 调查结果为 “ 2 ”, 则表示被调查者来自离异家庭。青少年行为也可以分为两类, “犯罪” 和 “未犯罪”, 分别用 “ 1 ” 和 “ 2 ” 表示。对这类问题是在汇总数据的基础上进行分析的, 数据汇总的结果表现为频数。

在前面泰坦尼克号海难的例子中, 当时船上共 2208 人, 其中男性 1738 人, 女性 470 人。海难发生后, 幸存者 718 人, 其中男性 374 人, 女性 344 人。这里, 性别是分类变量, 有两个类别: 男性和女性; 幸存的男性 374 人和女性 344 人都是事件结果, 以频数的方式表现。

由第 1 章的讨论我们还知道, 在一些问题的研究中, 数值型数据可以转化为分类数据。例如, “收人” 是一个数值型数据, 但如果研究收人分配, 则可以按照一定的标准把不同收人的被调查者分为不同的类型, 如 “高收人群” “较高收人群” “中等收人群” 等。各收人群的统计结果就是频数。

由上述内容可知, 分类数据的结果是频数, $\chi^{2}$ 检验是对分类数据的频数进行分析的统计方法。

\subsection*{9.1.2 $\chi^{2}$ 统计量}
在第 6 章中对 $\chi^{2}$ 分布已经有所介绍, 这里结合本章研究的问题, 讨论 $\chi^{2}$ 统计量的应用。 $\chi^{2}$ 可以用于测定两个分类变量之间的相关程度。若用 $f_{o}$ 表示观察值频数 (observed frequency), 用 $f_{e}$ 表示期望值频数 (expected frequency), 则 $\chi^{2}$ 统计量可以写为:


\begin{equation*}
\chi^{2}=\sum \frac{\left(f_{o}-f_{e}\right)^{2}}{f_{e}} \tag{9.1}
\end{equation*}


$\chi^{2}$ 统计量有如下特征: 首先, $\chi^{2} \geqslant 0$, 因为它是对平方结果的汇总; 其次, $\chi^{2}$ 统计量\\
的分布与自由度有关; 最后, $\chi^{2}$ 统计量描述了观察值与期望值的接近程度。两者越接近,即 $f_{o}-f_{e}$ 的绝对值越小, 计算出的 $\chi^{2}$ 值就越小; 反之, $f_{o}-f_{e}$ 的绝对值越大, 计算出的 $\chi^{2}$ 值也越大。 $\chi^{2}$ 检验正是通过对 $\chi^{2}$ 的计算结果与 $\chi^{2}$ 分布中的临界值进行比较, 作出是否拒绝原假设的统计决策。

$\chi^{2}$ 分布与自由度的关系如图 9-1 所示。图 9-1 显示了自由度分别为 1, 5 和 10 时相应的 $\chi^{2}$ 分布。

\begin{center}
\end{center}

图 9-1 自由度分别为 1,5 和 10 时的 $\chi^{2}$ 分布

由图 9-1 看出, 自由度越小, 分布越向左边倾斜, 随着自由度的增大, $\chi^{2}$ 分布的偏斜程度逐渐显露出对称性, 随着自由度继续增大, $\chi^{2}$ 分布将趋于对称的正态分布。

利用 $\chi^{2}$ 统计量, 可以对分类数据进行拟合优度检验和独立性检验。

\section*{9.2 拟合优度检验}
拟合优度检验(goodness of fit test)是用 $\chi^{2}$ 统计量进行统计显著性检验的重要内容之一。它是依据总体分布状况, 计算出分类变量中各类别的期望频数, 与分布的观察频数进行对比, 判断期望频数与观察频数是否有显著差异, 从而达到对分类变量进行分析的目的。例如, 在泰坦尼克号的例子中, 我们关注在这次海难中幸存者的性别是否有显著差异。当时船上共有 2208 人, 其中男性 1738 人, 女性 470 人。海难发生后, 幸存者共 718 人, 其中男性 374 人, 女性 344 人。海难后存活比率为 $718 / 2208=0.325$ 。如果是否活下来与性别没有关系, 那么按照这个比率, 在 1738 位男性中应该存活 $1738 \times 0.325=565$人, 在 470 位女性中应该存活 $470 \times 0.325=153$ 人。 565 和 153 就是期望频数, 而实际存活结果就是观察频数。通过期望频数和观察频数的比较, 能够从统计角度作出存活与性别是否有关的判断。

下面我们把这个例子作为一个假设问题提出。

\section*{例 9.1}
1912 年 4 月 15 日, 豪华巨轮泰坦尼克号与冰山相撞沉没。当时船上共有 2208 人,其中男性 1738 人,女性 470 人。海难发生后,幸存者共 718 人,其中男性 374 人,女性 344 人, 以 $\alpha=0.1$ 的显著性水平检验存活状况与性别是否有关。

解 在本例中需要判断观察频数与期望频数是否一致。

$H_{0}$ :观察频数与期望频数一致

$H_{1}$ :观察频数与期望频数不一致

依据式 (9.1), 可以将 $\chi^{2}$ 值的计算过程列表如下(见表 9-1)。

表 9-1 $\chi^{2}$ 计算表

\begin{center}
\begin{tabular}{|c|c|c|c|c|c|}
\hline
 & $f_{o}$ & $f_{e}$ & \begin{tabular}{l}
步骤一 \\
$f_{o}-f_{e}$ \\
\end{tabular} & \begin{tabular}{c}
步骤二 \\
$\left(f_{o}-f_{e}\right)^{2}$ \\
\end{tabular} & \begin{tabular}{c}
步骤三 \\
$\left(f_{o}-f_{e}\right)^{2} / f_{e}$ \\
\end{tabular} \\
\hline
男性 & 374 & 565 & -191 & 36481 & 64.6 \\
\hline
女性 & 344 & 153 & 191 & 36481 & 238.4 \\
\hline
\end{tabular}
\end{center}

$$
\text { 步骤四 } \quad \chi^{2}=\sum \frac{\left(f_{o}-f_{e}\right)^{2}}{f_{e}}=303
$$

表中第一行数据为男性数据, 观察频数 374 , 期望频数 565 , 第二行为女性数据, 观察频数 344, 期望频数 153。表 9-1 可以反映 $\chi^{2}$ 值的计算过程:

步骤一: 计算 $f_{o}-f_{e}$;

步骤二: 计算 $\left(f_{o}-f_{e}\right)^{2}$;

步骤三: 计算 $\left(f_{o}-f_{e}\right)^{2} / f_{e}$;

步骤四: 计算 $\chi^{2}=\sum \frac{\left(f_{o}-f_{e}\right)^{2}}{f_{e}}=303$ 。

自由度的计算公式为 $d f=R-1, R$ 为分类变量类型的个数。在本例中, 分类变量是性别, 有男、女两个类别, 故 $R=2$, 于是自由度 $d f=2-1=1$, 经查 $\chi^{2}$ 分布表, $\chi_{0.1}^{2}(1)=$ 2. 706 , 括号中的数字表示自由度。因为 $\chi^{2}$ 远大于 $\chi_{0.1}^{2}$, 故拒绝 $H_{0}$, 接受 $H_{1}$, 说明存活状况与性别显著相关。

\section*{太用 Excel 计算 $P$ 值的操作步骤}
第 1 步: 将观察值输入一列, 将期望值输入一列。

第 2 步: 进入 Excel 表格界面, 选择【插入】下拉菜单。

第 3 步: 选择【函数】, 或者直接点击功能栏中的【fx】。

第 4 步: 在函数类别选择中选 “统计", 然后, 在函数名的菜单中选择字符 “CHISQ. TEST", 然后点击【确定】。

第 5 步: 在对话框【Actual\_range】中输入观察数据。在对话框【Expected\_range】中输入期望数据。\\
第 6 步: 点击【确定】得到 $P$ 值为 $7.2923 \times 10^{-68}$, 接近零。由于 $P<\alpha$, 所以拒绝原假设。

同样的方法还可以用于对泰坦尼克号这个例子中的年龄、舱位情况进行检验。结果表明, 儿童存活率高于成人, 一等舱、二等舱存活率高于船员舱。对这些结果的深层次分析, 或许有助于我们认识当海难发生时, 人们对待死亡的态度。

\section*{9.3 列联分析:独立性检验}
拟合优度检验是对一个分类变量的检验, 有时我们会遇到两个分类变量的问题, 看这两个分类变量是否存在联系。例如, 原料有不同的等级, 原料又产自不同的地区。原料等级和原料生产地就是两个分类变量。我们关心这两者是否有关联, 是不是某些地区生产的原料有更好的质量。对两个分类变量的分析, 称为独立性检验, 分析过程可以通过列联表的方式呈现, 故有人把这种分析称为列联分析。

\subsection*{9.3. 1 列联表}
列联表 (contingency table) 是将两个以上的变量进行交叉分类的频数分布表。例如,欲分析原料的质量是否与生产地有关, 将 500 件随机抽取的原料按质量和产地构造列联表如下 (见表 9-2)。

表 9-2 原料抽样的结果

\begin{center}
\begin{tabular}{ccccc}
\hline
地区 & 一级 & 二级 & 三级 & 合计 \\
\hline
甲 & 52 & 64 & 24 & 140 \\
乙 & 60 & 59 & 52 & 171 \\
丙 & 50 & 65 & 74 & 189 \\
\hline
合计 & 162 & 188 & 150 & 500 \\
\hline
\end{tabular}
\end{center}

表中的行 (row) 是生产地变量, 这里划分为三类: 甲、乙、丙三个地区。表中的列 (column)是原料等级变量, 这里也划分为三类: 一级、二级、三级。因此, 表 9-2 是一个 $3 \times 3$ 列联表, 表中的每个数据都反映了生产地和原料等级两个方面的信息。由于列联表中的每个变量都可以有两个或两个以上的类别, 列联表会有多种形式。不妨将横向变量 (行) 的划分类别视为 $\mathrm{R}$, 纵向变量 (列) 的划分类别视为 $\mathrm{C}$, 这样可以把每一个具体的列联表称为 $R \times C$ 列联表,如把表 $9-2$ 称为 $3 \times 3$ 列联表。

\subsection*{9.3.2 独立性检验}
独立性检验就是分析列联表中行变量和列变量是否相互独立, 在表 $9-2$ 中, 也就是检验各个地区和原料质量之间是否存在依赖关系。

\section*{例 9.2}
一种原料来自三个不同的地区,原料质量被分成三个不同等级。从这批原料中随机抽取 500 件进行检验, 结果如表 9-2 所示, 要求检验各个地区和原料等级之间是否存在依赖关系 $(\alpha=0.05)$ 。

一解 $H_{0}$ :地区和原料等级之间是独立的(不存在依赖关系)

$H_{1}$ : 地区和原料等级之间不独立(存在依赖关系)

这里分析的关键是获得期望值。

在第一行, 甲地区的合计为 140 , 用 $140 / 500$ 作为甲地区原料比例的估计值。在第一列, 一级原料的合计为 162 , 用 $162 / 500$ 作为一级原料比例的估计值。如果地区和原料等级之间是独立的,则可以用下面的公式估计第一个单元(甲地区,一级)中的期望比例。

令: $A=$ 样本单位来自甲地区的事件

$B=$ 样本单位属于一级原料的事件

根据独立性的概率乘法公式, 有

$$
\begin{aligned}
P(\text { 第一个单元 }) & =P(A B) \\
& =P(A) P(B) \\
& =\left(\frac{140}{500}\right)\left(\frac{162}{500}\right) \\
& =0.09072
\end{aligned}
$$

0.09072 是第一个单元中的期望比例, 相应的频数期望值为:

$$
0.09072 \times 500=45.36
$$

一般地,可以采用下式计算任何一个单元中频数的期望值:


\begin{equation*}
f_{e}=\frac{R T}{n} \times \frac{C T}{n} \times n=\frac{R T \times C T}{n} \tag{9.2}
\end{equation*}


式中, $f_{e}$ 为给定单元中的频数期望值; $R T$ 为给定单元所在行的合计; $C T$ 为给定单元所在列的合计; $n$ 为观察值的总个数, 即样本量。

由表 9-2 和式 (9.2), 计算结果如表 9-3 所示。

表 9-3 $3 \times 3$ 列联表的期望值及 $\chi^{2}$ 计算结果

\begin{center}
\begin{tabular}{ccccccc}
\hline
行 & 列 & $f_{o}$ & $f_{e}$ & $f_{o}-f_{e}$ & $\left(f_{o}-f_{e}\right)^{2}$ & $\left(f_{o}-f_{e}\right)^{2} / f_{e}$ \\
\hline
1 & 1 & 52 & 45.36 & 6.64 & 44.09 & 0.97 \\
1 & 2 & 64 & 52.64 & 11.36 & 129.05 & 2.45 \\
1 & 3 & 24 & 42.00 & -18 & 324 & 7.71 \\
2 & 1 & 60 & 55.40 & 4.60 & 21.16 & 0.38 \\
2 & 2 & 59 & 64.30 & -5.3 & 28.09 & 0.44 \\
\hline
\end{tabular}
\end{center}

续表

\begin{center}
\begin{tabular}{ccccccc}
\hline
行 & 列 & $f_{o}$ & $f_{e}$ & $f_{o}-f_{e}$ & $\left(f_{o}-f_{e}\right)^{2}$ & $\left(f_{o}-f_{e}\right)^{2} / f_{e}$ \\
\hline
2 & 3 & 52 & 51.30 & 0.7 & 0.49 & 0.01 \\
3 & 1 & 50 & 61.24 & -11.24 & 126.34 & 2.06 \\
3 & 2 & 65 & 71.06 & -6.06 & 36.72 & 0.52 \\
3 & 3 & 74 & 56.70 & 17.30 & 299.29 & $\underline{5.28}$ \\
 &  &  &  &  &  & 19.82 \\
\hline
\end{tabular}
\end{center}

$$
\chi^{2}=\sum \frac{\left(f_{o}-f_{e}\right)^{2}}{f_{e}}=19.82
$$

$\chi^{2}$ 的自由度 $=(R-1)(C-1)=4$

令 $\alpha=0.05$, 查表知: $\chi_{0.05}^{2}(4)=9.488$ 。

由于 $\chi^{2}>\chi_{0.05}^{2}(4)$, 故拒绝 $H_{0}$, 接受 $H_{1}$, 即地区和原料等级之间存在依赖关系, 原料的质量受地区的影响。

$\star$ 用 SPSS 进行列联分析的独立性检验的操作步骤

第 1 步: 在 SPSS 数据编辑窗口中定义变量 “生产地”、“质量” 和 “数量”, “质量" 变量的取值有三个: 1, 2 和 3 , 其中, 取值 "1" 表示一级, “ 2 ” 表示二级, “ 3 ”表示三级。

第 2 步: 选择【数据】下拉菜单, 并选择【加权个案】选项进入主对话框。在主对话框中选择

【加权个案】, 将变量 “数量” 选入【频率变量】。

第 3 步: 选择【分析】下拉菜单, 再选择【描述统计】下拉菜单, 最后选择【交叉表】选项进入主对话框。在主对话框中将 “生产地”选入【行】, “质量”选入【列】。

第 4 步:在【统计量】下选择【卡方】。点击【单元格】。在【计数】下选择【观察值】和【期望值】。在【百分比】下选择【行】、【列】和【总计】。在【残差】下选择【未标准化】和【标准化】。点击【继续】。点击【确定】。

输出结果如表 9-4 和表 9-5 所示。

表 9-4 生产地 * 质量交叉表

\begin{center}
\begin{tabular}{llllll}
\hline
 & \multicolumn{4}{c}{质量} & \multirow{2}{*}{合计} \\
\cline { 3 - 5 }
 &  & 1 & \multicolumn{1}{c}{2} & \multicolumn{1}{c}{3} &  \\
 & 计数 & 50 & 65 & 74 & 189.0 \\
 & 期望计数 & 61.2 & 71.1 & 56.7 & $100.0 \%$ \\
 & 占生产地的百分比 & $26.5 \%$ & $34.4 \%$ & $39.2 \%$ & $37.8 \%$ \\
生产地 丙地区 & 占质量的百分比 & $30.9 \%$ & $34.6 \%$ & $49.3 \%$ & $37.8 \%$ \\
 & 占总计的百分比 & $10.0 \%$ & $13.0 \%$ & $14.8 \%$ &  \\
 & 残差 & -11.2 & -6.1 & 17.3 &  \\
 & 标准化残差 & -1.4 & -0.7 & 2.3 &  \\
\hline
\end{tabular}
\end{center}

\begin{center}
\end{center}

卡方检验结果如表 9-5 所示。

表 9-5 卡方检验结果

\begin{center}
\begin{tabular}{lccc}
\hline
 & 值 & 自由度 & 渐近显著性 (双侧) \\
\hline
皮尔逊卡方 & $19.822^{\mathrm{a}}$ & 4 & 0.001 \\
似然比 & 20.732 & 4 & 0.000 \\
有效个案数 & 500 &  &  \\
\hline
\end{tabular}
\end{center}

a. 0 个单元格 $(0.0 \%)$ 的期望计数小于 5 。最小期望计数为 42.00 。

从表 9-5 可以看出, 卡方值为 19.822, 相伴概率为 0.001 , 故应拒绝原假设, 认为原料的质量受地区的影响。

最后剖析一下 $\chi^{2}$ 检验中自由度计算的原理。如前所述, 自由度是可以自由取值的数据的个数, 采用自由度 $=($ 行数 -1$)($ 列数 -1$)=(R-1)(C-1)$ 公式计算。这样做的原因可以通过以下例子说明。

假如现在我们有一个 $3 \times 4$ 列联表, 如表 9-6 所示。

\begin{center}
\begin{tabular}{cccccc}
\hline
 & $C 1$ & $C 2$ & $C 3$ & $C 4$ & 合计 \\
\hline
$R 1$ & $\sqrt{ }$ & $\sqrt{ }$ & $\sqrt{ }$ & $*$ & $R T_{1}$ \\
$R 2$ & $\sqrt{ }$ & $\sqrt{ }$ & $\sqrt{ }$ & $*$ & $R T_{2}$ \\
$R 3$ & $*$ & $*$ & $*$ & 0 & $R T_{3}$ \\
合计 & $C T_{1}$ & $C T_{2}$ & $C T_{3}$ & $C T_{4}$ &  \\
\hline
\end{tabular}
\end{center}

注: $\sqrt{ }$ 表示可以自由取值的数据。

\begin{itemize}
  \item 和 0 表示不能自由取值的数据。
\end{itemize}

由表 9-6 可知, $R T_{1}, R T_{2}$ 和 $R T_{3}$ 分别表示行的合计, $C T_{1}, C T_{2}, C T_{3}$ 和 $C T_{4}$ 分别表示列的合计。

首先考察列联表中的第一行, 在行合计 $R T_{1}$ 已经确定的情况下, 这一行可以自由取值的数据只有 3 个(这里假定取前 3 个), 用 $\sqrt{ }$ 表示, 最后一个无法自由取值, 用 $*$ 表示。类似地, 在第二行, 在行合计 $R T_{2}$ 已经确定的情况下, 这一行可以自由取值的数据也只有 3 个, 因此第 4 个不能自由取值的数据也用 $*$ 表示。在第三行, 第一个数据 $(R 3, C 1)$不能自由取值, 因为在列合计 $C T_{1}$ 已经确定的情况下, 第一列的前两个数据已经自由取值。同理, 第三行中的第二个和第三个数据也不能自由取值, 因此这一行的前三个数据均用 $*$ 表示。第三行的第四个数据也不能自由取值, 用 0 表示, 因为不论从行或列来看, 它前面的数据均是无法自由取值的 (意味着该值已经确定), 在行、列合计确定的情况下,这个值也就无法自由选取。

表 $9-6$ 是一个 $3 \times 4$ 列联表, 自由度为 6 , 即

自由度 $=(R-1)(C-1)=(3-1) \times(4-1)=6$

\section*{9.4 列联表中的相关测量}
前面讨论了利用 $\chi^{2}$ 分布对两个分类变量之间的相关性进行统计检验。如果变量相互独立, 说明它们之间没有联系; 反之, 则认为它们之间存在联系。接下来的问题是, 如果变量之间存在联系, 它们之间的相关程度有多大? 这一节主要讨论这个问题。

对两个变量之间相关程度的测定, 主要用相关系数表示。正如前面所言, 列联表中的变量通常是分类变量, 它们所表现的是研究对象的不同品质类别。所以, 可以把这种分类数据之间的相关称为品质相关。经常用到的品质相关系数有以下几种。

\section*{9.4. $1 \varphi$ 相关系数}
$\varphi$ 相关系数 ( $\varphi$ correlation coefficient $)$ 是描述 $2 \times 2$ 列联表数据相关程度最常用的一种相关系数。计算公式为:


\begin{equation*}
\varphi=\sqrt{\chi^{2} / n} \tag{9.3}
\end{equation*}


式中, $\chi^{2}$ 是按式 (9.1) 计算出的 $\chi^{2}$ 值; $n$ 为列联表中的总频数, 也即样本量。说 $\varphi$ 系数\\
适合 $2 \times 2$ 列联表, 是因为对于 $2 \times 2$ 列联表中的数据, 计算出的 $\varphi$ 系数可以控制在 $0 \sim 1$这个范围。表 9-7 是一个简化的 $2 \times 2$ 列联表。

表 9-7 中, $a, b, c, d$ 均为条件频数。由上节分析可知, 当变量 $X, Y$ 相互独立,不存在相关关系时, 频数间应有下面的关系:

$$
\frac{a}{a+c}=\frac{b}{b+d}
$$

表 9-72 $2 \times 2$ 列联表

\begin{center}
\begin{tabular}{|c|c|c|c|}
\hline
\multirow{2}{*}{因素 $Y$} & \multicolumn{2}{|c|}{因素 $X$} & \multirow{2}{*}{合计} \\
\hline
 & $x_{1}$ & $x_{2}$ &  \\
\hline
$y_{1}$ & $a$ & $b$ & $a+b$ \\
\hline
$y_{2}$ & $c$ & $d$ & $c+d$ \\
\hline
合计 & $a+c$ & $b+d$ &  \\
\hline
\end{tabular}
\end{center}

化简后有

$$
a d=b c
$$

因此, 差值 $a d-b c$ 的大小可以反映变量之间相关程度的高低。差值越大, 说明两个变量的相关程度越高。 $\varphi$ 系数就是以 $a d-b c$ 的差值为基础, 对两个变量相关程度的测定。

由式 (9.2) 知, 在 $2 \times 2$ 列联表中, 每个单元中频数的期望值为:

$$
\begin{aligned}
& e_{11}=\frac{(a+b)(a+c)}{n} \\
& e_{21}=\frac{(a+c)(c+d)}{n} \\
& e_{12}=\frac{(a+b)(b+d)}{n} \\
& e_{22}=\frac{(b+d)(c+d)}{n}
\end{aligned}
$$

由式 (9.1), 有

$$
\begin{aligned}
\chi^{2} & =\frac{\left(a-e_{11}\right)^{2}}{e_{11}}+\frac{\left(b-e_{12}\right)^{2}}{e_{12}}+\frac{\left(c-e_{21}\right)^{2}}{e_{21}}+\frac{\left(d-e_{22}\right)^{2}}{e_{22}} \\
& =\frac{n(a d-b c)^{2}}{(a+b)(c+d)(a+c)(b+d)}
\end{aligned}
$$

将此结果代人式 (9.3), 得到


\begin{align*}
\varphi & =\sqrt{\frac{\chi^{2}}{n}} \\
& =\frac{a d-b c}{\sqrt{(a+b)(c+d)(a+c)(b+d)}} \tag{9.4}
\end{align*}


当 $a d=b c$ 时, 表明变量 $X, Y$ 之间相互独立, 这时 $\varphi=0$ 。若 $b=0, c=0$, 由式 (9.4) 计算的 $\varphi=1$, 这是 $X$ 与 $Y$ 完全相关的一种情况。同样, 若 $a=0, d=0$, 由\\
式 (9.4) 计算的 $\varphi=-1$, 这也是 $X$ 与 $Y$ 完全相关的一种情况。

由于在列联表中变量的位置可以任意变换, 因此 $\varphi$ 的符号在这里没有什么实际意义, $|\varphi|=1$ 只是表明 $X$ 与 $Y$ 完全相关。由表 9-7 可知, 当 $|\varphi|=1$ 时, 必有某个方向对角线上的值全为零,如表 $9-8$ 和表 $9-9$ 所示。

表 9-8 完全相关时的 $2 \times 2$ 列联表

\begin{center}
\begin{tabular}{ccc}
\hline
 &  & $X$ \\
\cline { 2 - 3 }
 & $x_{1}$ & $x_{2}$ \\
\hline
$y_{1}$ & $a$ & 0 \\
$y_{2}$ & 0 & $d$ \\
\hline
\end{tabular}
\end{center}

表 9-9 完全相关时的另一种 $2 \times 2$ 列联表

\begin{center}
\begin{tabular}{ccc}
\hline
 &  & $X$ \\
\cline { 2 - 3 }
 & $x_{1}$ & $x_{2}$ \\
\hline
$y_{1}$ & 0 & $b$ \\
$y_{2}$ & $c$ & 0 \\
\hline
\end{tabular}
\end{center}

表中的含义也是清楚的。例如, 一个变量表示性别(男, 女), 另一个变量表示态度 (赞成, 反对)。 $|\varphi|=1$ 说明, 或者男性全部赞成, 女性全部反对; 或者男性全部反对,女性全部赞成。现实中这种情况比较罕见, 因此, 实际上 $\varphi$ 系数的取值范围在 $0 \sim 1$ 之间, $\varphi$ 的绝对值越大, 说明变量 $X$ 与 $Y$ 的相关程度越高。

但是, 当列联表 $R \times C$ 中的行数 $R$ 或列数 $C$ 大于 2 时, $\varphi$ 系数将随着 $R$ 或 $C$ 的变大而增大, 且 $\varphi$ 值没有上限。这时用 $\varphi$ 系数测定两个变量的相关程度就不够清晰, 可以采用列联相关系数。

\subsection*{9.4. 2 列联相关系数}
列联相关系数又称列联系数 (coefficient of contingency), 简称 $c$ 系数, 主要用于列联表大于 $2 \times 2$ 的情况。 $c$ 系数的计算公式为:


\begin{equation*}
c=\sqrt{\frac{\chi^{2}}{\chi^{2}+n}} \tag{9.5}
\end{equation*}


当列联表中的两个变量相互独立时, 系数 $c=0$, 并且它不可能大于 1 , 这一点从式 (9.5) 中也可以看出来。 $c$ 系数的特点是, 其可能的最大值依赖于列联表的行数和列数,且随着 $R$ 和 $C$ 的增大而增大。例如, 当两个变量完全相关时, 对于 $2 \times 2$ 表, $c=0.7071$;对于 $3 \times 3$ 表, $c=0.8165$; 而对于 $4 \times 4$ 表, $c=0.87$ 。因此, 根据不同的行和列计算的列联系数不便于比较, 除非两个列联表的行数和列数一致。这是列联系数的局限性。但由于其计算简便, 且对总体的分布没有任何要求, 所以列联系数不失为一种适应性较广的测度值。

\subsection*{9.4.3 $V$ 相关系数}
鉴于 $\varphi$ 系数无上限, $c$ 系数小于 1 的情况, 格莱姆 (Gramer) 提出了 $\boldsymbol{V}$ 相关系数(V\\
correlation coefficient)。 $V$ 相关系数的计算公式为:


\begin{equation*}
V=\sqrt{\frac{\chi^{2}}{n \times \min [(R-1),(C-1)]}} \tag{9.6}
\end{equation*}


它的计算也是以 $\chi^{2}$ 值为基础的。式中的 $\min [(R-1),(C-1)]$ 表示取 $(R-1)$, $(C-1)$ 中较小的一个。当两个变量相互独立时, $V=0$; 当两个变量完全相关时, $V=1$ 。所以 $V$ 的取值在 $0 \sim 1$ 之间。如果列联表中有一维为 2 , 即 $\min [(R-1),(C-1)]=1$, 则 $V$ 值就等于 $\varphi$ 值。

\subsection*{9.4.4 数值分析}
用前面例 9.2 的数据分别计算 $\varphi$ 系数、 $c$ 系数和 $V$ 系数。

在例 9.2 中, 我们对原料等级和地区之间的关系进行了独立性检验, 结果表明, 原料等级和地区之间存在相关关系。我们提出的下一个问题是, 这种相关程度有多高, 能否给出数量化描述?

由前已知, $\chi^{2}=19.82$, 列联表的总频数 $n=500$ 。这是一个 $3 \times 3$ 列联表, $\min [(R-1)$, $(C-1)]=3-1=2$ 。于是

$$
\begin{aligned}
& \varphi=\sqrt{\frac{\chi^{2}}{n}}=\sqrt{\frac{19.82}{500}}=0.199 \\
& c=\sqrt{\frac{\chi^{2}}{\chi^{2}+n}}=\sqrt{\frac{19.82}{19.82+500}}=0.195 \\
& V=\sqrt{\frac{\chi^{2}}{n \times \min [(R-1),(C-1)]}}=\sqrt{\frac{19.82}{500 \times 2}}=0.141
\end{aligned}
$$

对于 $\varphi$ 而言, 当 $R>2, C>2$ 时, $\varphi$ 值有可能突破 1 , 相比之下, $\varphi=0.199$ 不能认为很大。

对于 $c$ 而言,其结果必然低于 $\varphi$ 值,因为 $c$ 值总是小于 1 。本例中是 $3 \times 3$ 列联表, $c$的最大可能值是 0.8165 。相比 0.8165 而言,本例中的 $c=0.195$ 并不大。

对于 $V$ 而言, $V=0.141$ 则更小。

综合起来可以认为, 虽然检验表明原料等级和地区存在一定关系, 但这种关系的密切程度却不太高。这意味着, 除了地区之外, 还有其他因素对原料的质量产生更大的影响。

太用 SPSS 计算列联相关系数的操作步骤

第 1 步: 在 SPSS 数据编辑窗口中定义变量 “生产地”、“质量" 和 “数量", “质量” 变量的取值有三个: 1, 2 和 3 , 其中, 取值 “ 1 ” 表示一级, “ 2 ” 表示二级, “ 3 ” 表示三级。

第 2 步: 选择【数据】下拉菜单, 并选择【加权个案】选项进入主对话框。在主对话框中选择【加权个案】, 将变量 “数量” 选入【频率变量】。

第 3 步: 选择【分析】下拉菜单, 再选择【描述统计】下拉菜单, 最后选择【交叉表】选项进入主对话框。在主对话框中将 “生产地” 选入【行】, “质量”选入【列】。

第 4 步: 点击【统计量】。在【名义】下选择【相依系数】和【Phi 和 Cramer 变量】。点击【继续】。点击【确定】。\\
输出结果如表 9-10 所示。

表 9-10 列联相关系数

对称测量

\begin{center}
\begin{tabular}{cccc}
\hline
 & 值 & 渐近显著性 &  \\
\hline
 & Phi & 0.199 & 0.001 \\
名义到名义 & 格莱姆 $\mathrm{V}$ & 0.141 & 0.001 \\
 & 列联系数 & 0.195 & 0.001 \\
有效个案数 &  & 500 &  \\
\hline
\end{tabular}
\end{center}

$\varphi$ 系数、 $V$ 系数和 $c$ 系数分别为 $0.199,0.141$ 和 0.195 。

上面的例子还说明, 对于同一个数据, 系数 $\varphi, c, V$ 的结果不同。同样, 对于不同的列联表, 行数和列数的差异也会影响系数值。因此, 在对不同列联表变量之间的相关程度进行比较时, 不同列联表中行与行、列与列的个数要相同, 并且采用同一种系数, 这样的系数值才具有可比性。

\section*{9.5 列联分析中应注意的问题}
\subsection*{9.5.1 条件百分表的方向}
一般来说, 在列联表中变量的位置是任意的。也就是说, 既可以把变量 $X$ 放在列的位置, 也可以放在行的位置。如果变量 $X$ 与 $Y$ 存在因果关系, 令 $X$ 为自变量 (原因), $Y$为因变量 (结果), 那么一般的做法是把自变量 $X$ 放在列的位置。条件百分表多按自变量的方向计算, 因为这样便于更好地表现原因对结果的影响。表 9-11 就是一个 $2 \times 2$ 列联表。

表 9-11 职业与价值取向

\begin{center}
\begin{tabular}{ccc}
\hline
\multirow{2}{*}{价值取向 $Y$} & 职业 $X$ &  \\
\cline { 2 - 3 }
 & 制造业 & 服务业 \\
\hline
物质报酬 & 105 & 45 \\
$\%$ & 72 & 56 \\
人情关系 & 40 & 35 \\
$\%$ & 28 & 44 \\
合计 & 145 & 80 \\
$\%$ & 100 & 100 \\
\hline
\end{tabular}
\end{center}

表中数据显示, 总共调查了 225 人, 其中制造业 145 人, 服务业 80 人。在制造业被调查者中, 以物质报酬为价值取向的有 105 人, 占该群体的 $72 \%$; 以人情关系为价值取向的有 40 人, 占该群体的 $28 \%$ 。而在服务业被调查者中, 以物质报酬为价值取向的有 45 人, 占该群体的 $56 \%$; 以人情关系为价值取向的有 35 人,占该群体的 $44 \%$ 。数据表明,与制造业相比, 服务业就业人员更注重人情关系。人们的职业背景不同, 工作的价值观也可能不同。

但是, 有时也有例外。如果因变量在样本内的分布不能代表其在总体内的分布, 例如, 为了满足分析的需要, 抽样时扩大了因变量某项内容的样本量, 这时仍按自变量的方向计算百分表就会歪曲实际情况。例如, 社会学家欲研究家庭状况 (自变量) 对青少年犯罪 (因变量) 的影响。该地区有未犯罪青少年 10000 名, 犯罪青少年 150 名。如果从未犯罪青少年中抽取 $1 \%$, 即 100 名进行研究, 则用相同比例从犯罪青少年中抽取的样本量仅为 1.5 人。显然, 这么少的数量无法满足对比研究的需要。因此, 对犯罪青少年的抽样比要扩大, 譬如扩大到 $1 / 2$, 即抽取 75 人。假定从两个样本调查所获得的数据如表 9-12 所示。

表 9-12 家庭状况与青少年犯罪

\begin{center}
\begin{tabular}{|c|c|c|c|}
\hline
\multirow{2}{*}{青少年行为} & \multicolumn{2}{|c|}{家庭状况} & \multirow{2}{*}{合计} \\
\hline
 & 完整家庭 & 离异家庭 &  \\
\hline
犯罪 & 38 & 37 & 75 \\
\hline
未犯罪 & 92 & 8 & 100 \\
\hline
合计 & 130 & 45 & 175 \\
\hline
\end{tabular}
\end{center}

表 9-12 是调查结果的条件分布。由表 9-12 可以计算其条件百分表, 如表 9-13 所示。

表 9-13 家庭状况与青少年犯罪百分表 (一)

\begin{center}
\begin{tabular}{ccc}
\hline
\multirow{2}{*}{青少年行为} & 家庭状况 &  \\
\cline { 2 - 4 }
 & 完整家庭 & 离异家庭 \\
\hline
犯罪 $(\%)$ & 29 & 82 \\
未犯罪 $(\%)$ & 71 & 18 \\
\hline
合计 $($ 人) & 130 & 45 \\
\hline
\end{tabular}
\end{center}

由表 9-13 得到的结果是, 在完整家庭接受调查的 130 人中, 犯罪青少年所占的比例是 $29 \%$, 这个比例高达近 $1 / 3$, 这是令人吃惊的。其实, 这个比例是被歪曲的, 这是由于抽样时扩大了对犯罪青少年抽取的数量。如果把计算百分表的方向变换一下, 改为按因变量方向计算, 则得到表 9-14。\\
表 9-14 家庭状况与青少年犯罪百分表 (二)

\begin{center}
\begin{tabular}{ccc}
\hline
\multirow{2}{*}{家庭状况} & \multicolumn{3}{c}{青少年行为} \\
\cline { 2 - 3 }
 & 犯罪 $(\%)$ & 未犯罪 (\%) &  \\
\hline
完整家庭 & 51 & 92 &  \\
离异家庭 & 49 & 8 &  \\
\hline
合计 (人) & 75 & 100 &  \\
\hline
\end{tabular}
\end{center}

从表 9-14 看出, 在完整家庭中, 未犯罪青少年的比例占 92\%, 而在离异家庭中, 这个比例仅为 $8 \%$ 。完整家庭的青少年未犯罪率远远高于离异家庭的这个比例。家庭状况对青少年行为的影响得到了比较真实的反映。

\subsection*{9.5.2 $\chi^{2}$ 分布的期望值准则}
前面谈到用 $\chi^{2}$ 分布进行独立性检验, 要求样本量必须足够大, 特别是每个单元中的期望频数 (理论频数) 不能过小, 否则应用 $\chi^{2}$ 检验可能会得出错误的结论。关于小单元的频数通常有两条准则: 一条准则是, 如果只有两个单元, 则每个单元的期望频数必须是 5 或 5 以上, 如表 9-15 所示。

表 9-15 说明表 (一)

\begin{center}
\begin{tabular}{ccc}
\hline
以往病史 & $f_{o}$ & $f_{e}$ \\
\hline
未患过肝炎 & 532 & 531 \\
患过肝炎 & 4 & 5 \\
\hline
\end{tabular}
\end{center}

此时有两个单元, 或分为两个类别: 患过肝炎和未患过肝炎。由于样本量足够大, 每个单元的期望频数 $f_{e} \geqslant 5$, 因此可以使用 $\chi^{2}$ 检验。

另一条准则是, 倘若有两个以上的单元, $20 \%$ 的单元的期望频数 $f$ e 小于 5 , 则不能使用 $\chi^{2}$ 检验。根据这个准则, 表 9-16 中的数据可以计算 $\chi^{2}$, 因为 6 个单元中只有 1 个单元的期望频数小于 5 。而表 $9-17$ 中的数据不能用于 $\chi^{2}$ 检验, 因为 7 个单元中有 3 个单元的期望频数小于 5 。

表 9-16 说明表 (二)

\begin{center}
\begin{tabular}{ccc}
\hline
类别 & $f_{o}$ & $f_{e}$ \\
\hline
A & 28 & 26 \\
B & 49 & 47 \\
C & 18 & 23 \\
D & 6 & 4 \\
E & 92 & 88 \\
F & 20 & 25 \\
\hline
合计 & 213 & 213 \\
\hline
\end{tabular}
\end{center}

表 9-17 说明表 (三)

\begin{center}
\begin{tabular}{ccc}
\hline
类别 & $f_{o}$ & $f_{e}$ \\
A & 30 & 32 \\
B & 110 & 113 \\
C & 86 & 87 \\
D & 23 & 24 \\
E & 5 & 2 \\
F & 5 & 4 \\
G & 4 & 1 \\
\hline
合计 & 263 & 263 \\
\hline
\end{tabular}
\end{center}

可以用表 9-17 中的数据来说明第二条准则。仔细观察会发现, 表 9-17 中的 $f_{o}$ 与 $f_{e}$ 非常接近, 最大的差别是 3 , 应当说期望值与观察值拟合得很好, 它们之间并无显著差别。然而进行 $\alpha=0.05$ 的 $\chi^{2}$ 检验, 则会得到

$$
\begin{aligned}
& \chi^{2}=14.01 \\
& \chi_{0.05}^{2}(6)=12.592
\end{aligned}
$$

因为 $\chi^{2}>\chi_{0.05}^{2}(6)$, 所以拒绝原假设 $H_{0}$, 结论是期望值与观察值之间存在显著差别。看来这个结论并不符合逻辑。如果将这个例子中的某些类别合并, 使得 $f_{e} \geqslant 5$, 麻烦就会解除。例如, 将表 9-17 中的类别 E, F, G 合并, 合并后的 $f_{o}=5+5+4=14, f_{e}=2+$ $4+1=7$, 此时虽然 $f_{o}$ 与 $f_{e}$ 之间的差别扩大到 7 , 但通过计算会发现, 合并以后有

$$
\begin{aligned}
& \chi^{2}=7.26 \\
& \chi_{0.05}^{2}(4)=9.448
\end{aligned}
$$

因为 $\chi^{2}<\chi_{0.05}^{2}(4)$, 所以不能拒绝 $H_{0}$, 结论是期望值与观察值之间不存在显著差别。自然, 这是一个更合乎逻辑的结论。

由此可知, 如果期望频数 $f_{e}$ 过小, $\left(f_{o}-f_{e}\right)^{2} / f_{e}$ 将会不适当地增大, 造成对 $\chi^{2}$ 的高估, 从而导致不适当地拒绝 $H_{0}$ 的结论。处理的方法是将较小的 $f_{e}$ 合并, 这样便可得到合理的结论。

\section*{思考与练习}
\section*{思考题}
9.1 简述列联表的构造与列联表的分布。

9.2 用一张报纸、一份杂志或你周围的例子构造一个列联表, 说明这个调查中两个分类变量的关系, 并提出进行检验的问题。

9.3 说明计算 $\chi^{2}$ 统计量的步骤。

9.4 简述 $\varphi$ 系数、 $c$ 系数、 $V$ 系数各自的特点。\\
9. 5 构造下列维数的列联表, 并给出 $\chi^{2}$ 检验的自由度: 2 行 5 列; 4 行 6 列; 3 行 4 列。

\section*{练习题}
9.1 市场研究人员欲研究不同收入群体对某种特定商品是否有相同的购买习惯, 他们调查了四个不同收入组的消费者共 527 人, 购买习惯分为:经常购买, 不购买, 有时购买。调查结果如下表所示:

\begin{center}
\begin{tabular}{ccccc}
\hline
项目 & 低收人组 & 偏低收人组 & 偏高收人组 & 高收人组 \\
\hline
经常购买 & 25 & 40 & 47 & 46 \\
不购买 & 69 & 51 & 74 & 57 \\
有时购买 & 36 & 26 & 19 & 37 \\
\hline
\end{tabular}
\end{center}

(1) 提出假设。

(2) 计算 $\chi^{2}$ 值。

(3)以 $\alpha=0.1$ 的显著性水平进行检验。

(4) 计算 $\varphi$ 系数、 $c$ 系数和 $V$ 系数。

9.2 从总体中随机抽取 $n=200$ 的样本, 调查后按不同属性归类, 得到如下结果:

$$
n_{1}=28, n_{2}=56, n_{3}=48, n_{4}=36, n_{5}=32
$$

依据经验数据, 各类别在总体中的比例分别为:

$$
\pi_{1}=0.1, \pi_{2}=0.2, \pi_{3}=0.3, \pi_{4}=0.2, \pi_{5}=0.2
$$

以 $\alpha=0.1$ 的显著性水平进行检验, 说明现在的情况与经验数据相比是否发生了变化(用 $P$ 值)。

9.3 某报社关心其读者的阅读习惯是否与其文化程度有关, 随机调查了 254 位读者, 得到如下数据:

\begin{center}
\begin{tabular}{ccccc}
\hline
阅读习惯 & 大学以上 & 大学和大专 & 高中 & 高中以下 \\
\hline
早上看 & 6 & 13 & 14 & 17 \\
中午看 & 12 & 16 & 8 & 8 \\
晚上看 & 38 & 40 & 11 & 6 \\
有空看 & 21 & 22 & 9 & 13 \\
\hline
\end{tabular}
\end{center}

以 0.05 的显著性水平检验读者的阅读习惯是否与其文化程度有关。

9.4 教学改革后学生有了更多的选课自由, 但学院领导在安排课程上面临新的问题。例如, MBA 研究生班的学生选课的变化常常很大, 去年的学生很多人选会计课, 而今年的学生很多人选市场营销课。由于事先无法确定究竟有多少学生选各门课程, 所以无法有效进行教学资源的准备。有人提出学生所选课程与其本科所学专业有关。对此, 学院领导对学生本科所学专业和 MBA 三门课程的选修情况做了统计, 得到如下结果:\\
统计学 (第 8 版)

\begin{center}
\begin{tabular}{cccc}
\hline
 &  & MBA 所选课程 &  \\
\cline { 2 - 4 }
本科专业 & 会计 & 统计 & 市场营销 \\
\hline
专业一 & 31 & 13 & 16 \\
专业二 & 8 & 16 & 7 \\
专业三 & 12 & 10 & 17 \\
其他专业 & 10 & 5 & 7 \\
\hline
\end{tabular}
\end{center}

(1) 以 0.05 的显著性水平检验学生本科所学专业是否影响其读 MBA 期间所选的课程。

(2) 计算 $P$ 值。\\

新药在广泛用于临床之前需要先在动物身上进行试验, 证明它安全有效, 然后在健!康的志愿者中进行一个剂量或一个疗程的耐受试验, 证明人体能够耐受并给出临床上将!来能够使用的安全剂量, 最后在患者身上进行临床试验。

我国的《新药审批办法》规定,新药的临床试验分为 I, II , III, IV 期。I 期临床!试验: 初步的临床药理学及人体安全性评价试验, 观察人体对于新药的耐受程度和药物!代谢, 为制定给药方案提供依据。II 期临床试验: 随机盲法对照临床试验, 对新药有效性及安全性作出初步评价, 推荐临床给药剂量。红期临床试验: 扩大的多中心临床试!验, 应遵循随机对照原则, 进一步评价药物的有效性、安全性。IV 期临床试验: 新药上!市后的监测, 在广泛使用条件下考察药物的疗效和不良反应(注意罕见不良反应)。

试验设计是取得数据的有效方法, 而试验设计数据的分析方法主要是本章将要介绍!的方差分析。\\
方差分析是在 20 世纪 20 年代发展起来的一种统计方法, 它是由英国统计学家费希尔进行试验设计时为解释试验数据而率先引人的。目前, 方差分析方法广泛用于分析心理学、生物学、工程和医药领域的试验数据。从形式上看, 方差分析是比较多个总体的均值是否相等, 但本质上它所研究的是变量之间的关系, 这与第 11 章和第 12 章中将要介绍的回归分析方法有许多相同之处, 但又有本质区别。在研究一个 (或多个) 分类型自变量与一个数值型因变量之间的关系时, 方差分析是其中的主要方法之一。本章将要介绍的内容主要包括单因素方差分析和双因素方差分析等。

\section*{10.1 方差分析引论}
与第 8 章介绍的假设检验方法相比, 方差分析不仅可以提高检验的效率, 同时由于它将所有的样本信息结合在一起, 因此增加了分析的可靠性。例如, 设 4 个总体的均值分别为 $\mu_{1}, \mu_{2}, \mu_{3}, \mu_{4}$, 如果用一般假设检验方法, 如 $t$ 检验, 一次只能研究两个样本, 要检验 4 个总体的均值是否相等, 需要作 6 次检验。检验 $1: H_{0}: \mu_{1}=\mu_{2}$ 。检验 $2: H_{0}: \mu_{1}=$ $\mu_{3}$ 。检验 $3: H_{0}: \mu_{1}=\mu_{4}$ 。检验 $4: H_{0}: \mu_{2}=\mu_{3}$ 。检验 $5: H_{0}: \mu_{2}=\mu_{4}$ 。检验 $6: H_{0}$ : $\mu_{3}=\mu_{4}$ 。

很显然, 作这样的两两比较十分烦琐, 而且, 每次检验两个的做法共需进行 6 次。如果 $\alpha=0.05$, 即每次检验犯第 I 类错误的概率都是 0.05 , 作多次检验会使犯第 I 类错误的概率相应增加, 检验完成时, 犯第 I 类错误的概率会大于 0.05 , 即连续作 6 次检验犯第 I 类错误的概率为 $1-(1-\alpha)^{6}=0.265$, 而置信水平则会降低到 0.735 (即 $0.95^{6}$ )。

一般来说, 随着个体显著性检验次数的增加, 偶然因素导致差别的可能性也会增加 (并非均值真的存在差别)。方差分析方法则是同时考虑所有的样本, 因此排除了错误累积的概率, 从而避免拒绝一个真实的原假设。

\section*{10. 1.1 方差分析及其有关术语}
表面上看, 方差分析是检验多个总体均值是否相等的统计方法, 但本质上它所研究的是分类型自变量对数值型因变量的影响, 例如, 变量之间有没有关系, 关系的强度如何等。方差分析 (analysis of variance, ANOVA) 就是通过检验各总体的均值是否相等来判断分类型自变量对数值型因变量是否有显著影响。

为更好地理解方差分析的含义, 先通过一个例子来说明方差分析的有关概念以及方差分析所要解决的问题。

\section*{例 10.1}
消费者与产品生产者、销售者或服务提供者之间经常发生纠纷。发生纠纷后, 消费者常常会向消费者协会投诉。为了对几个行业的服务质量进行评价, 消费者协会在零售业、\\
旅游业、航空业、家电制造业分别抽取了不同的企业作为样本。其中零售业抽取 7 家, 旅游业抽取 6 家, 航空业抽取 5 家, 家电制造业抽取 5 家。每个行业中所抽取的这些企业,假定它们在服务对象、服务内容、企业规模等方面基本上是相同的。然后统计出最近一年中消费者对这 23 家企业投诉的次数, 结果如表 10-1所示。

表 10-1 四个行业被投诉的次数

\begin{center}
\begin{tabular}{cccc}
\hline
零售业 & 旅游业 & 航空业 & 家电制造业 \\
\hline
57 & 68 & 31 & 44 \\
66 & 39 & 49 & 51 \\
49 & 29 & 21 & 65 \\
40 & 45 & 34 & 77 \\
34 & 56 & 40 & 58 \\
53 & 51 &  &  \\
44 &  &  &  \\
\hline
\end{tabular}
\end{center}

一般而言, 被投诉的次数越多, 说明服务的质量越差。消费者协会想知道这几个行业之间的服务质量是否有显著差异。

\begin{itemize}
  \item —解 要分析四个行业之间的服务质量是否有显著差异, 实际上也就是要判断行业对被投诉次数是否有显著影响, 作出这种判断最终被归结为检验这四个行业被投诉次数的均值是否相等。如果它们的均值相等, 就意味着行业对被投诉次数是没有影响的, 也就是它们之间的服务质量没有显著差异; 如果均值不全相等, 则意味着行业对被投诉次数是有影响的,它们之间的服务质量有显著差异。
\end{itemize}

在方差分析中, 所要检验的对象称为因素或因子 (factor)。因素的不同表现称为水平或处理 (treatment)。在每个因子水平下得到的样本数据称为观测值。例如, 在例 10.1 中, 要分析行业对被投诉次数是否有显著影响。这里的行业是要检验的对象, 称为因素或因子; 零售业、旅游业、航空业、家电制造业是行业这一因素的具体表现, 称为水平或处理; 在每个行业下得到的样本数据(被投诉次数)称为观测值。由于这里只涉及行业一个因素, 因此称为单因素 4 水平的试验。因素的每一个水平可以看作一个总体, 如零售业、旅游业、航空业、家电制造业可以看作 4 个总体, 上面的数据可以看作从这 4 个总体中抽取的样本数据。再如, 要在不同的温度下进行一项试验, 温度就是一个因素, 在 $20^{\circ} \mathrm{C}$, $25^{\circ} \mathrm{C}, 30^{\circ} \mathrm{C}, 35^{\circ} \mathrm{C} 4$ 个温度值下做试验, 每个温度值就是一个水平, 共有 4 个水平, 在每个温度下试验得到的数据就是观测值。

在只有一个因素的方差分析 (称为单因素方差分析) 中, 涉及两个变量: 一个是分类型自变量, 一个是数值型因变量。例如, 在上面的例子中, 要研究行业对被投诉次数是否有影响, 这里的行业就是自变量, 它是一个分类型变量, 零售业、旅游业、航空业、家电\\
制造业就是行业这个自变量的具体取值, 称为行业这个因素的水平或处理。被投诉次数是因变量, 它是一个数值型变量, 不同的被投诉次数就是因变量的取值。方差分析要研究的就是行业对被投诉次数是否有显著影响。

\section*{10. 1.2 方差分析的基本思想和原理}
为了分析分类型自变量对数值型因变量的影响, 需要从对数据误差来源的分析人手。

\section*{1. 图形描述}
怎样判断行业对被投诉次数是否有显著影响? 或者说, 行业与被投诉次数之间是否有显著的关系? 我们画出它们的散点图, 例如, 图 10-1 是 4 个行业被投诉次数的散点图,图中的折线是由被投诉次数的均值连接而成的。

\begin{center}
\end{center}

行业.

图 10-1 不同行业被投诉次数的散点图

从散点图可以看出, 不同行业被投诉的次数有明显差异, 而且, 即使是在同一个行业, 不同企业被投诉的次数也明显不同。从图中可以看出, 家电制造业被投诉的次数较多, 而航空公司被投诉的次数较少。这表明行业与被投诉次数之间有一定的关系。如果行业与被投诉次数之间没有关系, 那么它们被投诉次数的均值应该差不多相同, 在散点图上所呈现的模式应该很接近。

\section*{2. 误差分解}
仅仅在散点图上观察还不能提供充分的证据证明不同行业被投诉次数之间有显著差异, 也许这种差异是由抽样的随机性造成的。因此, 需要有更准确的方法来检验这种差异是否显著, 也就是进行方差分析。之所以叫方差分析, 是因为虽然人们感兴趣的是均值,但在判断均值之间是否有差异时需要借助方差。这个名称也表示: 它是通过对数据误差来源的分析来判断不同总体的均值是否相等, 进而分析自变量对因变量是否有显著影响。因此, 进行方差分析时, 需要考察数据误差的来源。下面结合表 10-1 中的数据说明数据的\\
误差来源及其分解过程。

首先, 注意到在同一行业 (同一个总体) 中, 样本的各观测值是不同的。例如, 在零售业中, 所抽取的 7 家企业之间被投诉次数是不同的。由于企业是随机抽取的, 因此它们之间的差异可以看成是随机因素的影响造成的, 或者说是由抽样的随机性造成的随机误差。这种来自水平内部的数据误差也称为组内误差。例如, 零售业中所抽取的 7 家企业被投诉次数之间的误差就是组内误差, 它反映了一个样本内部数据的离散程度。显然, 组内误差只含有随机误差。

其次, 不同行业 (不同总体) 的观测值也是不同的。不同水平之间的数据误差称为组间误差。这种差异可能是由抽样本身形成的随机误差, 也可能是由行业本身的系统性因素造成的系统误差。因此, 组间误差是随机误差和系统误差的总和。例如, 4 个行业被投诉次数之间的误差就是组间误差, 它反映了不同样本之间数据的离散程度。

在方差分析中, 数据的误差是用平方和来表示的。

反映全部数据误差大小的平方和称为总平方和, 记为 SST。例如, 所抽取的全部 23 家企业被投诉次数之间的误差平方和就是总平方和, 它反映了全部观测值的离散状况。

反映组内误差大小的平方和称为组内平方和, 也称误差平方和或残差平方和, 记为 $S S E$ 。例如, 每个样本内部的数据平方和加在一起就是组内平方和, 它反映了每个样本内各观测值的离散状况。

反映组间误差大小的平方和称为组间平方和,也称

因素平方和, 记为 $S S A$ 。例如, 4 个行业被投诉次数之间的误差平方和就是组间平方和, 它反映了样本均值之间的差异程度。

图 10-2 给出了数据误差的分解过程。有关各误差的计算方法将在 10.2 节介绍。

\begin{center}
\end{center}

图 10-2 误差分解图

\section*{3. 误差分析}
如果不同行业对被投诉次数没有影响, 那么在组间误差中只包含随机误差, 而没有系统误差。这时, 组间误差与组内误差经过平均后的数值(称为均方或方差)就应该很接近, 它们的比值就会接近 1 。反之, 如果不同行业对被投诉次数有影响, 则组间误差中除了包含随机误差, 还会包含系统误差, 这时组间误差平均后的数值就会大于组内误差平均后的数值, 它们之间的比值就会大于 1 。当这个比值大到某种程度时, 就认为因素的不同水平之间存在显著差异, 也就是自变量对因变量有显著影响。因此, 判断行业对被投诉次数是否有显著影响这一问题, 实际上也就是检验被投诉次数的差异主要是由什么原因引起的。如果这种差异主要是系统误差, 就认为不同行业对被投诉次数有显著影响。在方差分析的假定前提下(见下面的介绍), 要检验行业(分类型自变量)对被投诉次数(数值型因变量) 是否有显著影响, 在形式上也就转化为检验 4 个行业被投诉次数的均值是否相等。

\subsection*{10.1. 3 方差分析中的基本假定}
方差分析中有三个基本假定:

(1)每个总体都应服从正态分布。也就是说,对于因素的每一个水平, 其观测值是来自正态分布总体的简单随机样本。例如, 在例 10.1 中, 要求每个行业被投诉次数必须服从正态分布。

(2)各个总体的方差 $\sigma^{2}$ 必须相同。也就是说,各组观察数据是从具有相同方差的正态总体中抽取的。例如, 在例 10.1 中, 要求每个行业被投诉次数的方差都相同。

(3)观测值是独立的。例如,在例 10.1 中,要求每个被抽中的企业被投诉次数都与其他企业被投诉次数相互独立。

在上述假定成立的前提下, 要分析自变量对因变量是否有影响, 在形式上也就转化为检验自变量的各个水平 (总体) 的均值是否相等。例如, 判断行业对被投诉次数是否有显著影响,实际上也就是检验具有相同方差的 4 个正态总体的均值(被投诉次数的均值)是否相等。

尽管不知道 4 个总体的均值, 但可以使用

\begin{center}
\end{center}

图 10-3 $H_{0}$ 为真时 $x$ 的抽样分布样本数据来检验它们是否相等。如果 4 个总体的均值相等, 可以期望 4 个样本的均值也会很接近。事实上, 4 个样本的均值越接近, 推断 4 个总体均值相等的证据也就越充分; 反之,样本均值越不同, 推断总体均值不同的证据就越充分。换句话说, 样本均值变动越小, 越支持 $H_{0}$; 样本均值变动越大, 越支持 $H_{1}$ 。如果原假设 $H_{0}: \mu_{1}=\mu_{2}=\mu_{3}=\mu_{4}$ (4 个行业被投诉次数的均值相同) 为真, 则意味着每个样本都来自均值为 $\mu$ 、方差为 $\sigma^{2}$ 的同一个正态总体。由样本均值的抽样分布可知, 来自正态总体的一个简单随机样本的样本均值 $\bar{x}$ 服从均值为 $\mu$ 、方差为 $\sigma^{2} / n$ 的正态分布, 如图 10-3 所示。

如果 $\mu_{1}, \mu_{2}, \mu_{3}, \mu_{4}$ 完全不同, 则意味着 4 个样本分别来自均值不同的 4 个正态总体, 因此有 4 个不同的抽样分布, 如图 10-4 所示。在这种情况下, 各样本均值就不像 $H_{0}$ 为真时那样接近了。

\begin{center}
\end{center}

图 10-4 4 个行业被投诉次数的均值全不相同时 $x$ 的抽样分布

\section*{10. 1.4 问题的一般提法}
设因素有 $k$ 个水平, 每个水平的均值分别用 $\mu_{1}, \mu_{2}, \cdots, \mu_{k}$ 表示, 要检验 $k$ 个水平 (总体) 的均值是否相等, 需要提出如下假设:\\
$H_{0}: \mu_{1}=\mu_{2}=\cdots=\mu_{k}$\\
自变量对因变量没有显著影响\\
$H_{1}: \mu_{1}, \mu_{2}, \cdots, \mu_{k}$ 不全相等\\
自变量对因变量有显著影响

在例 10.1 中, 设零售业被投诉次数的均值为 $\mu_{1}$, 旅游业被投诉次数的均值为 $\mu_{2}$, 航空业被投诉次数的均值为 $\mu_{3}$, 家电制造业被投诉次数的均值为 $\mu_{4}$ 。为检验行业对被投诉次数是否有影响, 需要提出如下假设:

$$
\begin{array}{ll}
H_{0}: \mu_{1}=\mu_{2}=\mu_{3}=\mu_{4} & \text { 行业对被投诉次数没有显著影响 } \\
H_{1}: \mu_{1}, \mu_{2}, \mu_{3}, \mu_{4} \text { 不全相等 } & \text { 行业对被投诉次数有显著影响 }
\end{array}
$$

\section*{10.2 单因素方差分析}
根据所分析的分类型自变量的多少, 方差分析可分为单因素方差分析和双因素方差分析。当方差分析只涉及一个分类型自变量时, 称为单因素方差分析 (one-way analysis of variance)。

单因素方差分析研究的是一个分类型自变量对一个数值型因变量的影响。例如, 要检验不同行业被投诉次数的均值是否相等, 这里只涉及行业一个因素, 因而属于单因素方差分析。

\subsection*{10.2. 1 数据结构}
进行单因素方差分析时, 需要得到下面的数据结构, 如表 10-2 所示。

表 10-2 单因素方差分析的数据结构

\begin{center}
\begin{tabular}{ccccc}
\hline
观测值 & \multicolumn{2}{c}{因素 $(i)$} &  & $A_{k}$ \\
\cline { 2 - 5 }
$(j)$ & $A_{1}$ & $A_{2}$ & $\cdots$ & $x_{k 1}$ \\
1 & $x_{11}$ & $x_{21}$ & $\cdots$ & $x_{k 2}$ \\
2 & $x_{12}$ & $x_{22}$ & $\cdots$ & $\vdots$ \\
$n$ & $\vdots$ & $\vdots$ & $\vdots$ & $x_{k n}$ \\
\hline
\end{tabular}
\end{center}

为叙述方便, 在单因素方差分析中, 用 $A$ 表示因素, 因素的 $k$ 个水平 (总体) 分别用 $A_{1}, A_{2}, \cdots, A_{k}$ 表示, 每个观测值用 $x_{i j}(i=1,2, \cdots, k ; j=1,2, \cdots, n)$ 表示, 即 $x_{i j}$ 表示第 $i$ 个水平 (总体) 的第 $j$ 个观测值。例如, $x_{21}$ 表示第二个水平的第一个观测值。其中, 从不同水平中所抽取的样本量可以相等, 也可以不相等。

\subsection*{10.2.2 分析步骤}
为检验自变量对因变量是否有显著影响, 首先需要提出 “两个变量在总体中没有关系” 的原假设, 然后构造一个用于检验的统计量来检验这一假设是否成立。具体来说, 方差分析包括提出假设、构造检验的统计量、作出统计决策等步骤。

\section*{1. 提出假设}
在方差分析中, 原假设所描述的是在按照自变量的取值分成的类中, 因变量的均值相等。因此, 检验因素的 $k$ 个水平 (总体) 的均值是否相等, 需要提出如下形式的假设:

$$
\begin{array}{ll}
H_{0}: \mu_{1}=\mu_{2}=\cdots=\mu_{i}=\cdots=\mu_{k} & \text { 自变量对因变量没有显著影响 } \\
H_{1}: \mu_{i}(i=1,2, \cdots, k) \text { 不全相等 } & \text { 自变量对因变量有显著影响 }
\end{array}
$$

式中, $\mu_{i}$ 为第 $i$ 个总体的均值。

如果拒绝原假设 $H_{0}$, 则意味着自变量对因变量有显著影响, 也就是自变量与因变量之间有显著关系 (1); 如果不拒绝原假设 $H_{0}$, 则意味着没有证据表明自变量对因变量有显著影响,也就是说,不能认为自变量与因变量之间有显著关系。

\section*{2. 构造检验的统计量}
为检验 $H_{0}$ 是否成立, 需要确定检验的统计量。如何构造这一统计量呢? 在此结合表 10-2 的数据结构说明其计算过程。

(1)计算各样本的均值。

假定从第 $i$ 个总体中抽取一个容量为 $n_{i}$ 的简单随机样本, 令 $\bar{x}_{i}$ 为第 $i$ 个总体的样本均值, 则有


\begin{equation*}
\bar{x}_{i}=\frac{\sum_{j=1}^{n_{i}} x_{i j}}{n_{i}}, \quad i=1,2, \cdots, k \tag{10.1}
\end{equation*}


式中, $n_{i}$ 为第 $i$ 个总体的样本量; $x_{i j}$ 为第 $i$ 个总体的第 $j$ 个观测值。例如, 根据表 10-1 中的数据, 计算零售业的样本均值为:

$$
\bar{x}_{1}=\frac{\sum_{j=1}^{7} x_{1 j}}{n_{1}}=\frac{57+66+49+40+34+53+44}{7}=49
$$

同样可以得到旅游业、航空业、家电制造业的均值。结果见表 10-3。
\footnotetext{(1) 需要注意的是, 拒绝 $H_{0}$ 时, 只是表明至少有两个总体的均值不相等, 并不意味着所有的均值都不相等。
}
表 10-3 4 个行业被投诉的次数及其均值

\begin{center}
\begin{tabular}{cccc}
\hline
零售业 & 旅游业 & 航空业 & 家电制造业 \\
\hline
57 & 68 & 31 & 44 \\
66 & 39 & 49 & 51 \\
49 & 29 & 21 & 65 \\
40 & 45 & 34 & 77 \\
34 & 56 & 40 & 58 \\
53 & 51 &  &  \\
44 & $\bar{x}_{2}=48$ & $\bar{x}_{3}=35$ & 59 \\
$\bar{x}_{1}=49$ & 6 & 5 &  \\
7 & $\bar{x}=\frac{57+66+\cdots+77+58}{23}=47.86956$ &  &  \\
 &  &  &  \\
\hline
\end{tabular}
\end{center}

(2)计算全部观测值的总均值。

它是全部观测值的总和除以观测值总个数的结果。令总均值为 $\bar{x}$, 则有


\begin{equation*}
\bar{x}=\frac{\sum_{i=1}^{k} \sum_{j=1}^{n_{i}} x_{i j}}{n}=\frac{\sum_{i=1}^{k} n_{i} \bar{x}_{i}}{n} \tag{10.2}
\end{equation*}


式中, $n=n_{1}+n_{2}+\cdots+n_{k}$ 。

根据表 10-1 的数据, 计算的总均值见表 10-3。

(3)计算各误差平方和。

为构造检验的统计量, 在方差分析中, 需要计算三个误差平方和, 它们是总平方和、组间平方和 (因素平方和)、组内平方和(误差平方和或残差平方和)。

\begin{enumerate}
  \item 总平方和 (sum of squares for total), 记为 $S S T$ 。它是全部观测值 $x_{i j}$ 与总均值 $\bar{x}$的误差平方和, 其计算公式为:
\end{enumerate}


\begin{equation*}
\operatorname{SST}=\sum_{i=1}^{k} \sum_{j=1}^{n_{i}}\left(x_{i j}-\bar{x}\right)^{2} \tag{10.3}
\end{equation*}


例如, 在表 $10-3$ 中已经计算出 $\bar{x}=47.869$ 565。计算总平方和为:

$$
\begin{aligned}
S S T & =(57-47.869565)^{2}+\cdots+(58-47.869565)^{2} \\
& =4164.608696
\end{aligned}
$$

它反映了全部 23 个观测值与这 23 个观测值平均数之间的差异。

\begin{enumerate}
  \setcounter{enumi}{1}
  \item 组间平方和 (sum of squares for factor $\mathbf{A}$ ), 记为 $S S A$, 它是各组均值 $\bar{x}_{i}(i=1$, $2, \cdots, k)$ 与总均值 $\bar{x}$ 的误差平方和, 反映各样本均值之间的差异程度, 又称为因素平方\\
和。其计算公式为:
\end{enumerate}


\begin{equation*}
\operatorname{SS} A=\sum_{i=1}^{k} n_{i}\left(\bar{x}_{i}-\bar{x}\right)^{2} \tag{10.4}
\end{equation*}


例如, 根据表 10-3 的有关结果, 计算组间平方和为:

$$
\begin{aligned}
\operatorname{SS} A= & \sum_{i=1}^{4} n_{i}\left(\bar{x}_{i}-\bar{x}\right)^{2} \\
= & 7 \times(49-47.869565)^{2}+6 \times(48-47.869565)^{2} \\
& +5 \times(35-47.869565)^{2}+5 \times(59-47.869565)^{2} \\
= & 1456.608696
\end{aligned}
$$

\begin{enumerate}
  \setcounter{enumi}{2}
  \item 组内平方和 (sum of squares for error), 记为 $S S E$ 。它是每个水平或组的各样本数据与其组均值的误差平方和, 反映每个样本各观测值的离散状况, 又称为误差平方和。该平方和反映了随机误差的大小,其计算公式为:
\end{enumerate}


\begin{equation*}
\operatorname{SSE}=\sum_{i=1}^{k} \sum_{j=1}^{n_{i}}\left(x_{i j}-\bar{x}_{i}\right)^{2} \tag{10.5}
\end{equation*}


在例 10.1 中, 先求出每个行业被投诉次数与其均值的误差平方和, 然后将四个行业的误差平方和加总, 即为 $S S E$ 。例如, 根据表 10-3 的有关结果, 计算误差平方和分别为:

零售业:

$$
\sum_{j=1}^{7}\left(x_{1 j}-\bar{x}_{1}\right)^{2}=(57-49)^{2}+(66-49)^{2}+\cdots+(44-49)^{2}=700
$$

旅游业:

$$
\sum_{j=1}^{6}\left(x_{2 j}-\bar{x}_{2}\right)^{2}=(68-48)^{2}+(39-48)^{2}+\cdots+(51-48)^{2}=924
$$

航空业:

$$
\sum_{j=1}^{5}\left(x_{3 j}-\bar{x}_{3}\right)^{2}=(31-35)^{2}+(49-35)^{2}+\cdots+(40-35)^{2}=434
$$

家电制造业:

$$
\sum_{j=1}^{5}\left(x_{4 j}-\bar{x}_{4}\right)^{2}=(44-59)^{2}+(51-59)^{2}+\cdots+(58-59)^{2}=650
$$

将其加总可以得到:

$$
S S E=700+924+434+650=2708
$$

上述三个平方和之间的关系为:


\begin{equation*}
\sum_{i=1}^{k} \sum_{j=1}^{n_{i}}\left(x_{i j}-\bar{x}\right)^{2}=\sum_{i=1}^{k} n_{i}\left(\bar{x}_{i}-\bar{x}\right)^{2}+\sum_{i=1}^{k} \sum_{j=1}^{n_{i}}\left(x_{i j}-\bar{x}_{i}\right)^{2} \tag{10.6}
\end{equation*}


即

总平方和 $(S S T)=$ 组间平方和 $(S S A)+$ 组内平方和 $(S S E)$

上面的计算结果也可以验证这一点:\\
$4164.608696=1456.608696+2708$

从上述三个误差平方和可以看出, SSA 是对随机误差和系统误差大小的度量, 它反映了自变量 (行业) 对因变量(被投诉次数)的影响, 也称为自变量效应或因子效应; SSE 是对随机误差大小的度量, 它反映了除自变量对因变量的影响之外其他因素对因变量的总影响, 因此 $S S E$ 也称为残差变量, 它所引起的误差称为残差效应; $S S T$ 是对全部数据总误差程度的度量, 它反映了自变量和残差变量的共同影响, 因此等于自变量效应加残差效应。

(4) 计算统计量。

由于各误差平方和的大小与观测值的多少有关, 为了消除观测值多少对误差平方和大小的影响, 需要将其平均, 也就是用各平方和除以它们所对应的自由度, 这一结果称为均方 (mean square), 也称方差。三个平方和所对应的自由度分别为:

$S S T$ 的自由度为 $n-1$, 其中 $n$ 为全部观测值的个数。

$S S A$ 的自由度为 $k-1$, 其中 $k$ 为因素水平 (总体) 的个数。

$S S E$ 的自由度为 $n-k$ 。

由于要比较的是组间均方和组内均方之间的差异, 所以通常只计算 SSA 的均方和 $S S E$ 的均方。 SSA 的均方也称组间均方或组间方差, 记为 $M S A$, 其计算公式为:


\begin{equation*}
M S A=\frac{\text { 组间平方和 }}{\text { 自由度 }}=\frac{S S A}{k-1} \tag{10.7}
\end{equation*}


例如,根据例 10.1 计算的 MSA 为:

$$
M S A=\frac{S S A}{k-1}=\frac{1456.608696}{4-1}=485.536232
$$

$S S E$ 的均方也称为组内均方或组内方差, 记为 $M S E$, 其计算公式为:


\begin{equation*}
M S E=\frac{\text { 组内平方和 }}{\text { 自由度 }}=\frac{S S E}{n-k} \tag{10.8}
\end{equation*}


例如, 根据例 10.1 计算的 $M S E$ 为:

$$
M S E=\frac{S S E}{n-k}=\frac{2708}{23-4}=142.526316
$$

将上述 $M S A$ 和 $M S E$ 进行对比, 即得到所需要的检验统计量 $F$ 。当 $H_{0}$ 为真时, 二者的比值服从分子自由度为 $k-1$ 、分母自由度为 $n-k$ 的 $F$ 分布, 即


\begin{equation*}
F=\frac{M S A}{M S E} \sim F(k-1, n-k) \tag{10.9}
\end{equation*}


例如, 根据例 10.1 计算得

$$
F=\frac{M S A}{M S E}=\frac{485.536232}{142.526316}=3.406643
$$

\section*{3. 作出统计决策}
如果原假设 $H_{0}: \mu_{1}=\mu_{2}=\cdots=\mu_{i}=\cdots=\mu_{k}$ 成立, 则表明没有系统误差, 组间方差 $M S A$ 与组内方差 $M S E$ 的比值差异就不会太大; 如果组间方差显著大于组内方差, 说明各\\
水平 (总体) 之间的差异显然不仅有随机误差, 还有系统误差。用例 10.1 来说, 如果行业对被投诉次数没有影响, 那么 4 个行业被投诉次数的均值之间的差异与每个行业被投诉次数的内部差异相比, 不会相差很大; 反之, 则意味着行业对被投诉次数有影响。可见,判断因素的水平是否对观测值有显著影响, 实际上也就是比较组间方差与组内方差之间差异的大小。那么, 它们之间的差异大到何种程度才表明有系统误差存在呢? 这就需要用检验统计量进行判断。将统计量的值 $F$ 与给定的显著性水平 $\alpha$ 下的临界值 $F_{a}$ 进行比较, 从而作出对原假设 $H_{0}$ 的决策。

根据给定的显著性水平 $\alpha$, 在 $F$ 分布表中查找与分子自由度 $d f_{1}=k-1$ 、分母自由度 $d f_{2}=n-k$ 相应的临界值 $F_{\alpha}(k-1, n-k)$ 。

若 $F>F_{\alpha}$, 则拒绝原假设 $H_{0}: \mu_{1}=\mu_{2}=\cdots=\mu_{k}$, 表明 $\mu_{i}(i=1,2, \cdots, k)$ 之间的差异是显著的; 也就是说, 所检验的因素(行业)对观测值(被投诉次数)有显著影响。

若 $F<F_{a}$, 则不拒绝原假设 $H_{0}$, 没有证据表明 $\mu_{i}(i=1,2, \cdots, k)$ 之间有显著差异; 也就是说, 这时还不能认为所检验的因素(行业)对观测值(被投诉次数)有显著影响。(1)

例如, 根据上面的计算结果, 计算出 $F=3.406$ 643。若取显著性水平 $\alpha=0.05$, 根据分子自由度 $d f_{1}=k-1=4-1=3$ 和分母自由度 $d f_{2}=n-k=23-4=19$, 查 $F$ 分布表得到临界值 $F_{0.05}(3,19)=3.13$ 。由于 $F>F_{\alpha}$, 因此拒绝原假设 $H_{0}: \mu_{1}=\mu_{2}=\mu_{3}=\mu_{4}$, 表明 $\mu_{1}, \mu_{2}, \mu_{3}, \mu_{4}$ 之间有显著差异, 即行业对被投诉次数有显著影响。

\section*{4. 方差分析表}
上面详细介绍了方差分析的计算步骤和过程。为使计算过程更加清晰, 通常将上述过程的内容列在一张表内, 这就是方差分析表 (analysis of variance table)。其一般形式如表 $10-4$ 所示。

表 10-4 方差分析表的一般形式

\begin{center}
\begin{tabular}{|c|c|c|c|c|c|c|c|}
\hline
 & 误差来源 & \begin{tabular}{c}
平方和 \\
$S S$ \\
\end{tabular} & \begin{tabular}{c}
自由度 \\
$d f$ \\
\end{tabular} & \begin{tabular}{c}
均方 \\
$M S$ \\
\end{tabular} & $F$ 值 & $P$ 值 & $F$ 临界值 \\
\hline
组间 & (因素影响) & SSA & $k-1$ & MSA & $M S A / M S E$ &  &  \\
\hline
组内 & (误差) & SSE & $n-k$ & MSE &  &  &  \\
\hline
总和 &  & SST & $n-1$ &  &  &  &  \\
\hline
\end{tabular}
\end{center}

将例 10.1 的计算结果列成方差分析表, 如表 10-5 所示。

(1) 当检验的因素只有两个水平时, 单因素方差分析与两个独立样本均值之差的 $t$ 检验的结果完全相同, 因为当因素的水平 $k=2$ 时, 检验的 $t$ 统计量与 $F$ 统计量的关系为: $F=t^{2}$ 。\\
表 10-54个行业被投诉次数的方差分析表

\begin{center}
\begin{tabular}{ccccccc}
\hline
差异源 & \multicolumn{1}{c}{SS} & $d f$ & $M S$ & $F$ 值 & $P$ 值 & $F$ 临界值 \\
\hline
组间 & 1456.6087 & 3 & 485.5362 & 3.4066 & 0.0388 & 3.1274 \\
组内 & 2708 & 19 & 142.5263 &  &  &  \\
总计 & 4164.6087 & 22 &  &  &  &  \\
\hline
\end{tabular}
\end{center}

\section*{5. 用 Excel 进行方差分析}
从上面介绍的分析过程可以看到, 进行方差分析需要大量的计算工作, 而手工计算是十分烦琐的。幸运的是这些计算工作可以由计算机来完成, 目前的统计软件中都有现成的方差分析程序。只要理解了方差分析的基本原理, 就可以对计算机输出的结果进行合理的解释和分析。在这里, 结合表 10-1 的数据, 给出用 Excel 进行方差分析的步骤和结果。

\section*{$\star$ 用 Excel 进行方差分析的操作步骤}
第 1 步: 点击【数据】, 并点击【数据分析】选项。

第 2 步:在分析工具中选择【单因素方差分析】, 点击【确定】。

第 3 步: 当对话框出现时:在【输入区域】方框内输入数据单元格区域 $\mathrm{A} 3: \mathrm{D} 9$ 。在【 $\boldsymbol{\alpha}$ 】方框内输入 0.05 (可根据需要确定)。在【输出选项】中选择输出区域 (这里选新工作表组)。

结果如图 10-5 所示。

\begin{center}
\end{center}

图 10-5 用 Excel 进行方差分析的步骤 (一)

点击【确定】后, 得到输出结果。\\
输出结果如表 10-6 所示。其中的 “SUMMARY” 部分是有关各样本的一些描述统计量, 它们可以作为方差分析的参考信息。“方差分析” 部分即为方差分析表, 其中, SS 表示平方和; $\mathrm{df}$ 为自由度; $\mathrm{MS}$ 表示均方; $\mathrm{F}$ 为检验的统计量; $\mathrm{P}$-value 为用于检验的 $P$值; $F$ crit 为给定 $\alpha$ 水平下的 $F$ 临界值。

表 10-6 Excel 输出的方差分析结果

SUMMARY

\begin{center}
\begin{tabular}{|c|c|c|c|c|c|c|c|}
\hline
\multicolumn{2}{|l|}{组} & \multicolumn{2}{|l|}{计数} & 求和 & 平均 & \multicolumn{2}{|r|}{方差} \\
\hline
\multicolumn{2}{|c|}{列 1} & \multicolumn{2}{|l|}{7} & 343 & \multicolumn{2}{|l|}{49} & 116. 666667 \\
\hline
\multicolumn{2}{|c|}{列 2} & \multicolumn{2}{|l|}{6} & 288 & \multicolumn{2}{|l|}{48} & 184.8 \\
\hline
\multicolumn{2}{|c|}{列 3} & \multicolumn{2}{|l|}{5} & 175 & \multicolumn{2}{|l|}{35} & 108.5 \\
\hline
\multicolumn{2}{|c|}{列 4} & \multicolumn{2}{|l|}{5} & 295 & 59 & \multicolumn{2}{|c|}{162.5} \\
\hline
\multicolumn{8}{|c|}{方差分析} \\
\hline
差异源 & SS &  & df & MS & F & P-value & $F$ crit \\
\hline
组间 & 1456.608696 &  & 3 & 485.536232 & 3. 406643 & 0.038765 & 3. 127354 \\
\hline
组内 & 2708 &  & 19 & 142.526316 &  &  &  \\
\hline
总计 & 4164.608696 &  & 22 &  &  &  &  \\
\hline
\end{tabular}
\end{center}

从方差分析表可以看到, 由于 $F=3.406643>F_{0.05}(3,19)=3.127354$, 所以拒绝原假设 $H_{0}$, 表明 $\mu_{1}, \mu_{2}, \mu_{3}, \mu_{4}$ 之间的差异是显著的, 即行业对被投诉次数的影响是显著的。

进行决策时, 可以将方差分析表中的 $P$ 值与显著性水平 $\alpha$ 的值进行比较。若 $P<\alpha$,则拒绝 $H_{0}$; 若 $P>\alpha$, 则不拒绝 $H_{0}$ 。在本例中, $P=0.038765<\alpha=0.05$, 所以拒绝 $H_{0}$ 。

\subsection*{10.2.3 关系强度的测量}
例 10.1 的方差分析结果显示, 不同行业被投诉次数的均值之间有显著差异, 这意味着行业 (自变量) 与被投诉次数 (因变量) 之间的关系是显著的。从图 10-1 中也可以看出, 不同行业被投诉次数之间是有明显差异的。

表 10-6 中给出了组间平方和(组间 $S S$ ), 它度量了自变量(行业)对因变量(被投诉次数)的影响效应。实际上, 只要组间平方和(组间 $S S$ ) 不等于零, 就表明两个变量之间有关系 (只是是否显著的问题)。当组间平方和比组内平方和(组内 $S S )$ ,而且大到一定程度时, 就意味着两个变量之间的关系显著, 大得越多, 表明它们之间的关系就越强; 反之, 当组间平方和比组内平方和小时, 就意味着两个变量之间的关系不显著, 小得越多, 表明它们之间的关系就越弱。\\
那么, 怎样度量它们之间的关系强度呢? 可以用组间平方和 (SSA) 占总平方和 (SST) 的比例大小来反映, 将这一比例记为 $R^{2}$, 即


\begin{equation*}
R^{2}=\frac{S S A(\text { 组间 } S S)}{S S T(\text { 总 } S S)} \tag{10.10}
\end{equation*}


其平方根 $R$ 就可以用来测量两个变量之间的关系强度。(1)

例如, 根据表 10-6 中的结果计算得

$$
R^{2}=\frac{S S A(\text { 组间 } S S)}{S S T(\text { 总 } S S)}=\frac{1456.608696}{4164.608696}=0.349759=34.9759 \%
$$

这表明, 行业 (自变量) 对被投诉次数(因变量)的影响效应占总效应的 $34.9759 \%$,而残差效应则占 $65.0241 \%$ 。也就是说,行业对被投诉次数差异解释的比例达到近 $35 \%$,而其他因素 (残差变量) 所解释的比例为 $65 \%$ 以上。尽管 $R^{2}$ 并不高, 但行业对被投诉次数的影响已经达到统计上显著的程度。

$R^{2}$ 的平方根 (类似于第 11 章中介绍的相关系数 $r$ ) 可以用来测量自变量与因变量之间的关系强度。例如, 根据上面的结果可以计算出 $R=0.591404$, 这表明行业与被投诉次数之间有中等以上的关系。

\subsection*{10.2.4 方差分析中的多重比较}
通过例 10.1 的分析得出的结论是: 不同行业被投诉次数的均值不完全相同。但究竟哪些均值之间不相等? 这种差异到底出现在哪些行业之间? 也就是说, $\mu_{1}$ 与 $\mu_{2} 、 \mu_{1}$ 与 $\mu_{3}$ 、 $\mu_{1}$ 与 $\mu_{4} 、 \mu_{2}$ 与 $\mu_{3} 、 \mu_{2}$ 与 $\mu_{4} 、 \mu_{3}$ 与 $\mu_{4}$ 之间究竟是哪两个均值不同? 这就需要作进一步的分析, 所使用的方法就是多重比较方法 (multiple comparison procedures), 它是通过对总体均值之间的配对比较来进一步检验到底哪些均值之间存在差异。

多重比较方法有许多种, 这里介绍由费希尔提出的最小显著差异方法 (least significant difference), 缩写为 $L S D$ 。使用该方法进行检验的具体步骤为:

第 1 步: 提出假设: $H_{0}: \mu_{i}=\mu_{j} ; H_{1}: \mu_{i} \neq \mu_{j}$ 。

第 2 步: 计算检验统计量: $\bar{x}_{i}-\bar{x}_{j}$ 。

第 3 步: 计算 $L S D$, 其公式为:


\begin{equation*}
L S D=t_{\alpha / 2} \sqrt{M S E\left(\frac{1}{n_{i}}+\frac{1}{n_{j}}\right)} \tag{10.11}
\end{equation*}


式中, $t_{a / 2}$ 为 $t$ 分布的临界值, 通过查 $t$ 分布表得到, 其自由度为 $n-k$, 这里的 $k$ 是因素中水平的个数; $M S E$ 为组内方差; $n_{i}$ 和 $n_{j}$ 分别是第 $i$ 个样本和第 $j$ 个样本的样本量。

第 4 步: 根据显著性水平 $\alpha$ 作出决策。如果 $\left|\bar{x}_{i}-\bar{x}_{j}\right|>L S D$, 则拒绝 $H_{0}$; 如果 $\left|\bar{x}_{i}-\bar{x}_{j}\right|<L S D$, 则不拒绝 $H_{0}$ 。

(1) 参见第 11 章, 在那里, 将 $R^{2}$ 定义为判定系数, 其平方根定义为相关系数。

\section*{例 10.2}
根据表 10-6 中的输出结果, 对 4 个行业的均值作多重比较 $(\alpha=0.05)$ 。

・解 第 1 步:提出假设。

检验 $1: H_{0}: \mu_{1}=\mu_{2} ; H_{1}: \mu_{1} \neq \mu_{2}$

检验 2: $H_{0}: \mu_{1}=\mu_{3} ; H_{1}: \mu_{1} \neq \mu_{3}$

检验 $3: H_{0}: \mu_{1}=\mu_{4} ; H_{1}: \mu_{1} \neq \mu_{4}$

检验 $4: H_{0}: \mu_{2}=\mu_{3} ; H_{1}: \mu_{2} \neq \mu_{3}$

检验 $5: H_{0}: \mu_{2}=\mu_{4} ; H_{1}: \mu_{2} \neq \mu_{4}$

检验 $6: H_{0}: \mu_{3}=\mu_{4} ; H_{1}: \mu_{3} \neq \mu_{4}$

第 2 步: 计算检验统计量。

$\left|\bar{x}_{1}-\bar{x}_{2}\right|=|49-48|=1$\\
$\left|\bar{x}_{1}-\bar{x}_{3}\right|=|49-35|=14$\\
$\left|\bar{x}_{1}-\bar{x}_{4}\right|=|49-59|=10$\\
$\left|\bar{x}_{2}-\bar{x}_{3}\right|=|48-35|=13$\\
$\left|\bar{x}_{2}-\bar{x}_{4}\right|=|48-59|=11$\\
$\left|\bar{x}_{3}-\bar{x}_{4}\right|=|35-59|=24$

第 3 步: 计算 $L S D$ 。根据表 $10-6$ 的结果, $M S E=142.526316$ 。由于 4 个行业的样本量不同, 需要分别计算 $L S D$ 。根据自由度 $=n-k=23-4=19$, 查 $t$ 分布表得 $t_{\alpha / 2}=t_{0.025}=$ 2.093。各检验的 LSD 如下:

检验 1: $L S D_{1}=2.093 \times \sqrt{142.526316 \times\left(\frac{1}{7}+\frac{1}{6}\right)}=13.90$

检验 2: $L S D_{2}=2.093 \times \sqrt{142.526316 \times\left(\frac{1}{7}+\frac{1}{5}\right)}=14.63$

检验 3: $L S D_{3}=L S D_{2}=14.63$

检验 $4: L S D_{4}=2.093 \times \sqrt{142.526316 \times\left(\frac{1}{6}+\frac{1}{5}\right)}=15.13$

检验 $5: L S D_{5}=L S D_{4}=15.13$

检验 6: $L S D_{6}=2.093 \times \sqrt{142.526316 \times\left(\frac{1}{5}+\frac{1}{5}\right)}=15.80$

第 4 步: 作出决策。

$\left|\bar{x}_{1}-\bar{x}_{2}\right|=1<13.90$, 不拒绝 $H_{0}$, 不能认为零售业与旅游业被投诉次数之间有显著差异;

$\left|\bar{x}_{1}-\bar{x}_{3}\right|=14<14.63$, 不拒绝 $H_{0}$, 不能认为零售业与航空业被投诉次数之间有显著差异;

$\left|\bar{x}_{1}-\bar{x}_{4}\right|=10<14.63$, 不拒绝 $H_{0}$, 不能认为零售业与家电制造业被投诉次数之间\\
有显著差异;

$\bar{x}_{2}-\bar{x}_{3} \mid=13<15.13$, 不拒绝 $H_{0}$, 不能认为旅游业与航空业被投诉次数之间有显著差异;

$\left|\bar{x}_{2}-\bar{x}_{4}\right|=11<15.13$, 不拒绝 $H_{0}$, 不能认为旅游业与家电制造业被投诉次数之间有显著差异;

$\bar{x}_{3}-\bar{x}_{4} \mid=24>15.80$, 拒绝 $H_{0}$, 认为航空业与家电制造业被投诉次数之间有显著差异。

\section*{10.3 双因素方差分析}
\subsection*{10.3. 1 双因素方差分析及其类型}
单因素方差分析只是考虑一个分类型自变量对数值型因变量的影响。在对实际问题的研究中, 有时需要考虑几个因素对试验结果的影响。例如, 分析影响彩电销售量的因素时, 需要考虑品牌、销售地区、价格、质量等多个因素的影响。当方差分析涉及两个分类型自变量时, 称为双因素方差分析 (two-way analysis of variance)。先看下面的例子。

\section*{例 10.3}
有 4 个品牌的彩电在 5 个地区销售, 为分析彩电的品牌 (品牌因素) 和销售地区 (地区因素) 对销售量的影响, 取得以下每个品牌在各地区的销售量数据(单位:台),如表 10-7 所示。试分析品牌和地区对彩电的销售量是否有显著影响 $(\alpha=0.05)$ 。

表 10-7 4 个品牌的彩电在 5 个地区的销售量数据

\begin{center}
\begin{tabular}{|c|c|c|c|c|c|c|}
\hline
 &  & \multicolumn{5}{|c|}{地区因素} \\
\hline
 &  & 地区 1 & 地区 2 & 地区 3 & 地区 4 & 地区 5 \\
\hline
 & 品牌 1 & 365 & 350 & 343 & 340 & 323 \\
\hline
品牌 & 品牌 2 & 345 & 368 & 363 & 330 & 333 \\
\hline
\multirow[t]{2}{*}{因素} & 品牌 3 & 358 & 323 & 353 & 343 & 308 \\
\hline
 & 品牌 4 & 288 & 280 & 298 & 260 & 298 \\
\hline
\end{tabular}
\end{center}

在上面的例子中, 品牌和地区是两个分类型自变量, 销售量是一个数值型因变量。同时分析品牌和地区对销售量的影响, 分析究竟是一个因素在起作用, 还是两个因素都起作用, 抑或是两个因素都不起作用, 这就是一个双因素方差分析问题。

在双因素方差分析中, 由于有两个影响因素, 例如, 彩电的品牌因素和地区因素, 如果品牌和地区对销售量的影响是相互独立的, 分别判断品牌和地区对销售量的影响, 这时的双因素方差分析称为无交互作用 (non-interaction) 的双因素方差分析, 或无重复双因\\
素 (two-factor without replication) 分析; 如果除了品牌和地区对销售量的单独影响, 两个因素的搭配还会对销售量产生一种新的影响, 例如, 某个地区对某种品牌的彩电有特殊偏好, 这就是两个因素结合后产生的新效应, 这时的双因素方差分析称为有交互作用的双因素方差分析,或可重复双因素(two-factor with replication)分析。

\subsection*{10.3.2 无交互作用的双因素方差分析}
\section*{1. 数据结构}
在无交互作用的双因素方差分析中, 由于有两个因素, 在获取数据时, 需要将一个因素安排在 “行” (row) 的位置, 称为行因素; 另一个因素安排在 “列” (column) 的位置, 称为列因素。设行因素有 $k$ 个水平, 行 1 , 行 $2, \cdots$, 行 $k$; 列因素有 $r$ 个水平, 列 1 ,列 $2, \cdots$, 列 $r$ 。行因素和列因素的每一个水平都可以搭配成一组, 观察它们对试验数据的影响。共抽取 $k r$ 个观察数据, 其数据结构如表 10-8 所示。

表 10-8 双因素方差分析的数据结构

\begin{center}
\begin{tabular}{|c|c|c|c|c|c|c|}
\hline
 &  & \multicolumn{4}{|c|}{列因素(j)} & \multirow{2}{*}{}\begin{tabular}{c}
平均值 \\
$\bar{x}_{i}$. \\
\end{tabular} \\
\hline
 &  & 列 1 & 列 2 & $\cdots$ & 列 $r$ &  \\
\hline
\multirow{4}{*}{}\begin{tabular}{c}
行因素 \\
(i) \\
\end{tabular} & 行 1 & $x_{11}$ & $x_{12}$ & $\cdots$ & $x_{1 r}$ & $\bar{x}_{1}$. \\
\hline
 & 行 2 & $x_{21}$ & $x_{22}$ & $\cdots$ & $x_{2 r}$ & $\bar{x}_{2}$. \\
\hline
 & $\vdots$ & $\vdots$ & $\vdots$ & $\vdots$ & $\vdots$ & : \\
\hline
 & 行 $k$ & $x_{k 1}$ & $x_{k 2}$ & $\cdots$ & $x_{k r}$ & $\bar{x}_{k}$. \\
\hline
\multicolumn{2}{|c|}{平均值 $\bar{x} . j$} & $\bar{x} \cdot 1$ & $\bar{x} \cdot 2$ & $\cdots$ & $\bar{x} \cdot r$ & $\bar{x}$ \\
\hline
\end{tabular}
\end{center}

表 10-8 中, 行因素共有 $k$ 个水平, 列因素共有 $r$ 个水平。每一个观测值 $x_{i j}(i=1$, $2, \cdots, k ; j=1,2, \cdots, r)$ 看作从由行因素的 $k$ 个水平和列因素的 $r$ 个水平所组合成的 $k \times r$ 个总体中抽取的样本量为 1 的独立随机样本。这 $k \times r$ 个总体中的每一个总体都服从正态分布, 且有相同的方差。

表 10-8 中, $\bar{x}_{i}$. 是行因素的第 $i$ 个水平下各观测值的平均值, 其计算公式为:


\begin{equation*}
\bar{x}_{i}=\frac{\sum_{j=1}^{r} x_{i j}}{r}, \quad i=1,2, \cdots, k \tag{10.12}
\end{equation*}


$\bar{x} \cdot j$ 是列因素的第 $j$ 个水平下各观测值的平均值, 其计算公式为:


\begin{equation*}
\bar{x}_{\cdot j}=\frac{\sum_{i=1}^{k} x_{i j}}{k}, \quad j=1,2, \cdots, r \tag{10.13}
\end{equation*}


$\bar{x}$ 是全部 $k r$ 个样本数据的总平均值, 其计算公式为:


\begin{equation*}
\overline{\bar{x}}=\frac{\sum_{i=1}^{k} \sum_{j=1}^{r} x_{i j}}{k r} \tag{10.14}
\end{equation*}


\section*{2. 分析步骤}
与单因素方差分析类似, 双因素方差分析也包括提出假设、构造检验统计量、作出统计决策等步骤。

(1) 提出假设。

为了检验两个因素的影响, 需要对两个因素分别提出如下假设。

对行因素提出的假设为:

$$
\begin{array}{ll}
H_{0}: \mu_{1}=\mu_{2}=\cdots=\mu_{i}=\cdots=\mu_{k} & \text { 行因素(自变量)对因变量没有显著影响 } \\
H_{1}: \mu_{i}(i=1,2, \cdots, k) \text { 不全相等 } & \text { 行因素 (自变量) 对因变量有显著影响 }
\end{array}
$$

式中, $\mu_{i}$ 为行因素的第 $i$ 个水平的均值。

对列因素提出的假设为:

$$
\begin{array}{ll}
H_{0}: \mu_{1}=\mu_{2}=\cdots=\mu_{j}=\cdots=\mu_{r} & \text { 列因素(自变量)对因变量没有显著影响 } \\
H_{1}: \mu_{j}(j=1,2, \cdots, r) \text { 不全相等 } & \text { 列因素(自变量)对因变量有显著影响 }
\end{array}
$$

式中, $\mu_{j}$ 为列因素的第 $j$ 个水平的均值。

(2)构造检验统计量。

为检验 $H_{0}$ 是否成立, 需要分别确定检验行因素和列因素的统计量。与单因素方差分析构造统计量的方法一样, 这里也需要从总平方和的分解人手。总平方和是全部样本观测值 $x_{i j}(i=1,2, \cdots, k ; j=1,2, \cdots, r)$ 与总的样本均值 $\bar{x}$ 的误差平方和, 记为 SST, 即


\begin{align*}
\operatorname{SST} & =\sum_{i=1}^{k} \sum_{j=1}^{r}\left(x_{i j}-\bar{x}\right)^{2} \\
& =\sum_{i=1}^{k} \sum_{j=1}^{r}\left(\bar{x}_{i \cdot}-\bar{x}\right)^{2}+\sum_{i=1}^{k} \sum_{j=1}^{r}\left(\bar{x}_{\cdot j}-\bar{x}\right)^{2}+\sum_{i=1}^{k} \sum_{j=1}^{r}\left(x_{i j}-\bar{x}_{i \cdot}-\bar{x}_{\cdot j}+\bar{x}\right)^{2} \tag{10.15}
\end{align*}


其中, 分解后的等式右边的第一项是行因素所产生的误差平方和, 记为 SSR, 即


\begin{equation*}
\operatorname{SSR}=\sum_{i=1}^{k} \sum_{j=1}^{r}\left(\bar{x}_{i}-\bar{x}\right)^{2} \tag{10.16}
\end{equation*}


第二项是列因素所产生的误差平方和,记为 $S S C ,$ 即


\begin{equation*}
\operatorname{SSC}=\sum_{i=1}^{k} \sum_{j=1}^{r}\left(\bar{x}_{\cdot j}-\bar{x}\right)^{2} \tag{10.17}
\end{equation*}


第三项是除行因素和列因素之外的剩余因素所产生的误差平方和, 称为随机误差平方和, 记为 $S S E$, 即


\begin{equation*}
\operatorname{SSE}=\sum_{i=1}^{k} \sum_{j=1}^{r}\left(x_{i j}-\bar{x}_{i \cdot}-\bar{x}_{\cdot j}+\bar{x}\right)^{2} \tag{10.18}
\end{equation*}


上述各平方和的关系为:


\begin{equation*}
S S T=S S R+S S C+S S E \tag{10.19}
\end{equation*}


在上述误差平方和的基础上计算均方, 也就是将各平方和除以相应的自由度。与各误

\section*{统计学 (第 8 版)}
差平方和相对应的自由度分别是:

总平方和 $S S T$ 的自由度为 $k r-1$;

行因素的误差平方和 $S S R$ 的自由度为 $k-1$;

列因素的误差平方和 $S S C$ 的自由度为 $r-1$;

随机误差平方和 $S S E$ 的自由度为 $(k-1)(r-1)$ 。

为构造检验统计量, 需要计算下列均方:

行因素的均方, 记为 MSR, 即


\begin{equation*}
M S R=\frac{S S R}{k-1} \tag{10.20}
\end{equation*}


列因素的均方, 记为 $M S C$, 即


\begin{equation*}
M S C=\frac{S S C}{r-1} \tag{10.21}
\end{equation*}


随机误差项的均方, 记为 $M S E$, 即


\begin{equation*}
M S E=\frac{S S E}{(k-1)(r-1)} \tag{10.22}
\end{equation*}


为检验行因素对因变量的影响是否显著, 采用下面的统计量:


\begin{equation*}
F_{R}=\frac{M S R}{M S E} \sim F(k-1,(k-1)(r-1)) \tag{10.23}
\end{equation*}


为检验列因素的影响是否显著, 采用下面的统计量:


\begin{equation*}
F_{C}=\frac{M S C}{M S E} \sim F(r-1,(k-1)(r-1)) \tag{10.24}
\end{equation*}


(3)作出统计决策。

计算出检验统计量后, 根据给定的显著性水平 $\alpha$ 和两个自由度, 查 $F$ 分布表得到相应的临界值 $F_{\alpha}$, 然后将 $F_{R}$ 和 $F_{C}$ 与 $F_{\alpha}$ 进行比较。

若 $F_{R}>F_{\alpha}$, 则拒绝原假设 $H_{0}: \mu_{1}=\mu_{2}=\cdots=\mu_{i}=\cdots=\mu_{k}$, 表明 $\mu_{i}(i=1,2, \cdots, k)$之间的差异是显著的; 也就是说, 所检验的行因素对观测值有显著影响。

若 $F_{C}>F_{a}$, 则拒绝原假设 $H_{0}: \mu_{1}=\mu_{2}=\cdots=\mu_{j}=\cdots=\mu_{r}$, 表明 $\mu_{j}(j=1,2, \cdots, r)$之间的差异是显著的; 也就是说, 所检验的列因素对观测值有显著影响。

为使计算过程更加清晰, 通常将上述过程的内容列成方差分析表, 其一般形式如表 10 - 9 所示。

表 10-9 双因素方差分析表

\begin{center}
\begin{tabular}{ccccccc}
\hline
误差来源 & 误差平方和 $S S$ & 自由度 $d f$ & 均方 $M S$ & $F$ 值 & $P$ 值 & $F$ 临界值 \\
\hline
行因素 & $S S R$ & $k-1$ & $M S R$ & $F_{R}$ &  &  \\
列因素 & $S S C$ & $r-1$ & $M S C$ & $F_{C}$ &  &  \\
误差 & $S S E$ & $(k-1)(r-1)$ & $M S E$ &  &  &  \\
总和 & $S S T$ & $k r-1$ &  &  &  &  \\
\hline
\end{tabular}
\end{center}

\section*{例 10.4}
根据例 10.3 中的数据, 分析品牌和地区对销售量是否有显著影响 $(\alpha=0.05)$ 。

解 首先对两个因素分别提出如下假设。

行因素 (品牌):\\
$H_{0}: \mu_{1}=\mu_{2}=\mu_{3}=\mu_{4}$\\
品牌对销售量没有显著影响\\
$H_{1}: \mu_{1}, \mu_{2}, \mu_{3}, \mu_{4}$ 不全相等\\
品牌对销售量有显著影响

列因素(地区:

$$
\begin{array}{ll}
H_{0}: \mu_{1}=\mu_{2}=\mu_{3}=\mu_{4}=\mu_{5} & \text { 地区对销售量没有显著影响 } \\
H_{1}: \mu_{1}, \mu_{2}, \mu_{3}, \mu_{4}, \mu_{5} \text { 不全相等 } & \text { 地区对销售量有显著影响 }
\end{array}
$$

双因素方差分析的计算较复杂, 可直接利用 Excel 给出计算结果, 步骤与上面介绍的类似, 只需要将第 2 步中的【单因素方差分析】改为【无重复双因素分析】即可。表 10-10 就是 Excel 输出的方差分析结果。

表 10-10 Excel 输出的方差分析结果

\begin{center}
\begin{tabular}{|c|c|c|c|c|c|c|}
\hline
SUMMARY & 观测数 & 求和 & 平均 & \multicolumn{2}{|l|}{方差} &  \\
\hline
行 1 & 5 & 1721 & 344.2 & \multicolumn{2}{|l|}{233.7} &  \\
\hline
行 2 & 5 & 1739 & 347.8 & \multicolumn{2}{|l|}{295.7} &  \\
\hline
行 3 & 5 & 1685 & 337 & \multicolumn{2}{|l|}{442.5} &  \\
\hline
行 4 & 5 & 1424 & 284.8 & \multicolumn{2}{|l|}{249.2} &  \\
\hline
列 1 & 4 & 1356 & 339 & \multicolumn{2}{|l|}{1224.666667} &  \\
\hline
列 2 & 4 & 1321 & 330.25 & \multicolumn{2}{|l|}{1464.25} &  \\
\hline
列 3 & 4 & 1357 & 339.25 & \multicolumn{2}{|l|}{822.9166667} &  \\
\hline
列 4 & 4 & 1273 & 318.25 & \multicolumn{2}{|l|}{1538.916667} &  \\
\hline
列 5 & 4 & 1262 & 315.5 & \multicolumn{2}{|c|}{241.6666667} &  \\
\hline
\multicolumn{7}{|l|}{方差分析} \\
\hline
差异源 & SS & df & MS & F & $\mathrm{P}$ - value & $F$ crit \\
\hline
行 & 13004.55 & 3 & 4334.85 & 18. 10777318 & 9. $45615 \mathrm{E}-05$ & 3. 490294821 \\
\hline
列 & 2011.7 & 4 & 502.925 & 2. 100845894 & 0.143664887 & 3. 259166727 \\
\hline
误差 & 2872.7 & 12 & 239.3916667 &  &  &  \\
\hline
总计 & 17888.95 & 19 &  &  &  &  \\
\hline
\end{tabular}
\end{center}

方差分析 : 无重复双因素分析\\
表 10-10 中的 “行” 指行因素, 即品牌因素; “列” 指列因素, 即地区因素。根据方差分析表的计算结果得出以下结论:

由于 $F_{R}=18.107773>F_{\alpha}=3.490294$, 所以拒绝原假设 $H_{0}$, 表明 $\mu_{1}, \mu_{2}, \mu_{3}, \mu_{4}$ 之间的差异是显著的,这说明品牌对销售量有显著影响。

由于 $F_{C}=2.100846<F_{\alpha}=3.259167$, 所以不拒绝原假设 $H_{0}$, 表明 $\mu_{1}, \mu_{2}, \mu_{3}, \mu_{4}$, $\mu_{5}$ 之间的差异不显著, 不能认为地区对销售量有显著影响。

直接用 P-value 进行分析, 结论也是一样。用于检验行因素的 P-value $=9.45615 \mathrm{E}-$ $05<\alpha=0.05$, 所以拒绝原假设 $H_{0}$; 用于检验列因素的 $\mathrm{P}-\mathrm{value}=0.143665>\alpha=0.05$, 所以不拒绝原假设 $H_{0}$ 。

\section*{3. 关系强度的测量}
例 10.4 的方差分析结果显示, 不同品牌的销售量均值之间有显著差异, 这意味着品牌 (行自变量) 与销售量 (因变量) 之间的关系是显著的。而不同地区的销售量的均值之间没有显著差异, 表明地区 (列自变量) 与销售量 (因变量) 之间的关系是不显著的。那么, 两个变量合起来与销售量之间的关系强度究竟如何呢?

表 10-10 中给出了行自变量 (品牌) 的平方和 (行 SS) 、列自变量 (地区) 的平方和 (列 SS) 、误差平方和 (误差 SS)。其中, 行平方和度量了品牌这个自变量对因变量(销售量)的影响效应; 列平方和度量了地区这个自变量对因变量(销售量)的影响效应。这两个平方和加在一起则度量了两个自变量对因变量的联合效应, 联合效应与总平方和的比值定义为 $R^{2}$, 其平方根 $R$ 则反映了这两个自变量合起来与因变量之间的关系强度(1), 即


\begin{equation*}
R^{2}=\frac{\text { 联合效应 }}{\text { 总效应 }}=\frac{S S R+S S C}{S S T} \tag{10.25}
\end{equation*}


例如, 根据表 10-10 的输出结果计算, 得

$$
R^{2}=\frac{S S R+S S C}{S S T}=\frac{13004.55+2011.70}{17888.95}=0.8394=83.94 \%
$$

这表明, 品牌因素和地区因素合起来总共解释了销售量差异的 $83.94 \%$, 其他因素 (残差变量) 只解释了销售量差异的 $16.06 \%$ 。而 $R=0.9162$, 表明品牌和地区两个因素合起来与销售量之间有较强的关系。

当然, 也可以分别考察品牌和地区与销售量之间的关系, 这就需要作每个自变量与销售量的单因素方差分析, 并分别计算每个 $R^{2}$ 。下面分别给出品牌和地区因素与销售量的单因素方差分析结果, 见表 10-11 和表 10-12, 请读者自己进行分析。

通过表 10-11 和表 10-12 可以发现, 单因素方差分析与双因素方差分析得出的结论一致。但双因素方差分析中的误差平方和等于 2872.7 , 比分别进行单因素方差分析时的任何一个平方和(4884.4 和 15 877.25) 都小, 而且 $P$ 值也变得更小了。这是因为在双因
\footnotetext{(1) $R^{2}$ 称为多重判定系数, $R$ 称为多重相关系数, 见第 12 章多元线性回归。
}
素方差分析中, 误差平方和不包括两个自变量中的任何一个, 因而减少了残差效应。而在分别作单因素方差分析时, 将行因素(品牌)作自变量时, 列因素(地区)被包括在残差中; 同样, 将列因素作自变量时, 行因素被包括在残差中。因此, 对于两个自变量而言,进行双因素方差分析要优于分别对两个因素进行单因素方差分析。(1)

表 10-11 品牌与销售量的单因素方差分析结果

\begin{center}
\begin{tabular}{lccllcc}
\hline
差异源 & SS & df & \multicolumn{1}{l}{MS} & F & P-value & F crit \\
\hline
组间 & 13004.55 & 3 & 4334.85 & 14.19981983 & $8.97271 \mathrm{E}-05$ & 3.238866952 \\
组内 & 4884.4 & 16 & 305.275 &  &  &  \\
总计 & 17888.95 & 19 &  &  &  &  \\
\hline
\end{tabular}
\end{center}

表 10-12 地区与销售量的单因素方差分析结果

\begin{center}
\begin{tabular}{lcccccc}
\hline
差异源 & SS & df & MS & F & P-value & F crit \\
\hline
组间 & 2011.7 & 4 & 502.925 & 0.475137382 & 0.753443326 & 3.055568243 \\
组内 & 15877.25 & 15 & 1058.483333 &  &  &  \\
总计 & 17888.95 & 19 &  &  &  &  \\
\hline
\end{tabular}
\end{center}

\subsection*{10.3.3 有交互作用的双因素方差分析}
在上面的分析中, 假定两个因素对因变量的影响是独立的, 但如果两个因素搭配在一起会对因变量产生一种新的效应, 就需要考虑交互作用对因变量的影响, 这就是有交互作用的双因素方差分析。

\section*{例 10.5}
城市道路交通管理部门为研究不同的路段和不同的时段对行车时间的影响, 让一名交通警察分别在两个路段的高峰期与非高峰期亲自驾车进行试验, 通过试验共获得 20 个行车时间 (单位: 分钟) 的数据, 如表 10-13 所示。试分析路段、时段以及路段和时段的交互作用对行车时间的影响 $(\alpha=0.05)$ 。

解 设行变量有 $k$ 个水平, 例如, 表 10-13 中的行变量(时段)有 2 个水平, 即高峰期和非高峰期; 列变量有 $r$ 个水平, 例如, 表 10-13 中的列变量(路段)有 2 个水平, 即路段 1 和路段 2 ; 行变量中每个水平的行数(Excel 中称为每个样本的行数)为 $m$, 例如,表 10-13 中的行变量的每个水平 (即每个样本) 的行数各有 5 行; 观察数据的总数为 $n$,例如, 表 10-13 中共有 $n=20$ 个数据。

与无交互作用的方差分析方法类似,有交互作用的双因素方差分析也需要提出假设、

(1) 类似于用两个自变量的多元回归代替分别作两个一元回归, 见第 12 章多元线性回归。\\
构造检验的统计量、作出统计决策等步骤。提出假设时, 需要对行变量、列变量和交互作用变量分别提出假设, 方法与上述类似, 这里不再赘述。有交互作用的双因素方差分析表的数据结构与表 10-13 类似, 一般形式如表 10-14 所示。

表 10-13 不同时段和不同路段的行车时间

\begin{center}
\begin{tabular}{|c|c|c|c|}
\hline
 &  & \multicolumn{2}{|c|}{路段(列变量)} \\
\hline
 &  & 路段 1 & 路段 2 \\
\hline
\multirow{10}{*}{}\begin{tabular}{c}
时段 \\
(行变量) \\
\end{tabular} & \multirow{5}{*}{高峰期} & 26 & 19 \\
\hline
 &  & 24 & 20 \\
\hline
 &  & 27 & 23 \\
\hline
 &  & 25 & 22 \\
\hline
 &  & 25 & 21 \\
\hline
 & \multirow{5}{*}{非高峰期} & 20 & 18 \\
\hline
 &  & 17 & 17 \\
\hline
 &  & 22 & 13 \\
\hline
 &  & 21 & 16 \\
\hline
 &  & 17 & 12 \\
\hline
\end{tabular}
\end{center}

表 10-14 有交互作用的双因素方差分析表的一般形式

\begin{center}
\begin{tabular}{|c|c|c|c|c|c|c|}
\hline
误差来源 & 平方和 $S S$ & 自由度 $d f$ & 均方 $M S$ & $F$ 值 & $P$ 值 & $F$ 临界值 \\
\hline
行因素 & SSR & $k-1$ & $M S R=\frac{S S R}{k-1}$ & $F_{R}=\frac{M S R}{M S E}$ &  &  \\
\hline
列因素 & SSC & $r-1$ & $M S C=\frac{S S C}{r-1}$ & $F_{C}=\frac{M S C}{M S E}$ &  &  \\
\hline
交互作用 & SSRC & \begin{tabular}{l}
$(k-1)$ \\
$(r-1)$ \\
\end{tabular} & $M S R C=\frac{S S R C}{(k-1)(r-1)}$ & $F_{R C}=\frac{M S R C}{M S E}$ &  &  \\
\hline
误差 & SSE & $k r(m-1)$ & $M S E=\frac{S S E}{k n(m-1)}$ &  &  &  \\
\hline
总和 & SST & $n-1$ &  &  &  &  \\
\hline
\end{tabular}
\end{center}

设: $x_{i j l}$ 为对应于行因素的第 $i$ 个水平和列因素的第 $j$ 个水平的第 $l$ 行的观测值; $\bar{x}_{i}$. 为行因素的第 $i$ 个水平的样本均值; $\bar{x} . j$ 为列因素的第 $j$ 个水平的样本均值; $\bar{x}_{i j}$ 为对应于行因素的第 $i$ 个水平和列因素的第 $j$ 个水平组合的样本均值; $\bar{x}$ 为全部 $n$ 个观测值的总均值。

各平方和的计算公式如下:

总平方和 $(S S T)$ :


\begin{equation*}
\operatorname{SST}=\sum_{i=1}^{k} \sum_{j=1}^{r} \sum_{l=1}^{m}\left(x_{i j l}-\bar{x}\right)^{2} \tag{10.26}
\end{equation*}


行变量平方和 (SSR):


\begin{equation*}
\operatorname{SSR}=m \sum_{i=1}^{k}\left(\bar{x}_{i} \cdot-\bar{x}\right)^{2} \tag{10.27}
\end{equation*}


列变量平方和 ( $S S C$ ) :


\begin{equation*}
\operatorname{SSC}=k m \sum_{j=1}^{r}\left(\bar{x}_{\cdot j}-\overline{\bar{x}}\right)^{2} \tag{10.28}
\end{equation*}


交互作用平方和 ( $S S R C)$ :


\begin{equation*}
\operatorname{SSRC}=m \sum_{i=1}^{k} \sum_{j=1}^{r}\left(\bar{x}_{i j}-\bar{x}_{i \cdot}-\bar{x}_{\cdot j}+\overline{\bar{x}}\right)^{2} \tag{10.29}
\end{equation*}


误差平方和 ( $S S E$ ) :


\begin{equation*}
S S E=S S T-S S R-S S C-S S R C \tag{10.30}
\end{equation*}


下面针对本例中提出的问题, 说明用 Excel 进行有交互作用的双因素方差分析的步骤, 并对结果进行分析。首先, 将数据按图 10-6 的形式输入到 Excel 工作表中, 然后按下列步骤操作。

\section*{太用 Excel 进行有交互作用的双因素方差分析的操作步骤}
第 1 步: 点击【数据】, 并点击【数据分析】选项。

第 2 步: 在分析工具中选择【方差分析:可重复双因素分析】, 然后点击【确定】。

\begin{center}
\begin{tabular}{|c|c|c|c|c|c|c|}
\hline
 & A & B & C. & E & G & $\mathrm{H}$ \\
\hline
1 &  & 路段1 & 路段2 & \multicolumn{3}{|c|}{方差分析: 可重复双因素分析} \\
\hline
2 & 1 & 26 & 19 & \multirow{5}{*}{}\begin{tabular}{l}
输入 \\
输入区域 $(\mathrm{I}):$ \\
毎一样本的行数 (R): \\
$a(\mathrm{~B}):$ \\
\end{tabular} & \multirow[b]{2}{*}{$\$ \$ \$ 1: \$ C \$ 11$ 国} &  \\
\hline
3 & 2 & 24 & 20 &  &  & 磆定 \\
\hline
4 & 3 & 27 & 23 &  & \multirow{3}{*}{}\begin{tabular}{l}
5 \\
0.05 \\
\end{tabular} & 取消 \\
\hline
5 & 4 & 25 & 22 &  &  & 帮助 (W) \\
\hline
6 & 5 & 25 & 21 &  &  &  \\
\hline
7 & 6 & 20 & 18 & \multirow{5}{*}{}\begin{tabular}{l}
输出选项 \\
○输出区域 (0): \\
○ 新工作表组 $(\mathfrak{P}):$ \\
○ 新工作蒲 ( (W) \\
\end{tabular} &  &  \\
\hline
8 & 7 & 17 & 17 &  & $\$ 4 \$ 13$ &  \\
\hline
9 & 8 & 22 & 13 &  &  &  \\
\hline
10 & 9 & 21 & 16 &  &  &  \\
\hline
11 & 10 & 17 & 12 &  &  &  \\
\hline
\end{tabular}
\end{center}

第 3 步: 当对话框出现时:在【输入区域】方框内输入 $\mathrm{Al}: \mathrm{C} 11$ 。在【 $\boldsymbol{\alpha}$ 】方框内输入 0.05 (可根据需要确定)。在【每一样本的行数】方框内输入 5 。在【输出选项】中选择输出区域 (这里选新工作表组)。结果如图 10-6 所示。

图 10-6 用 Excel 进行方差分析的步骤 (二)

点击【确定】后得到输出结果。

由表 10-15 的输出结果可知, 用于检验 “时段” (行因素, 输出表中为 “样本”) 的 $\mathrm{P}$-value $=0.0000<\alpha=0.05$, 拒绝原假设, 表明不同时段的行车时间之间有显著差异, 即时段对行车时间有显著影响; 用于检验 “路段” (列因素) 的 $\mathrm{P}-$ value $=0.0002<\alpha=0.05$,同样拒绝原假设, 表明不同路段的行车时间之间有显著差异, 即路段对行车时间也有显著影响; 交互作用反映的是时段因素和路段因素联合产生的对行车时间的附加效应, 用于检\\
验的 $\mathrm{P}-\mathrm{value}=0.9118>_{\alpha}=0.05$, 因此不拒绝原假设, 没有证据表明时段和路段的交互作用对行车时间有显著影响。

表 10-15 Excel 输出的有交互作用的双因素方差分析结果

方差分析:可重复双因素分析

\begin{center}
\begin{tabular}{cccc}
\hline
SUMMARY & 路段 1 & 路段 2 & 总计 \\
\hline
计数 & 5 & 5 & 10 \\
求和 & 127 & 105 & 232 \\
平均 & 25.4 & 21 & 23.2 \\
方差 & 1.3 & 2.5 & 7.0667 \\
计数 & 5 & 5 & 10 \\
求和 & 97 & 76 & 173 \\
平均 & 19.4 & 15.2 & 17.3 \\
方差 & 5.3 & 6.7 & 10.2333 \\
\hline
\end{tabular}
\end{center}

总计

计数 10

求和 224

$\begin{array}{lll}\text { 平均 } & 22.4 & 18.1\end{array}$

方差 $\quad 12.9333 \quad 13.4333$

\begin{center}
\begin{tabular}{crrrrrr}
\hline
方差分析 &  &  &  &  &  &  \\
\hline
差异源 & \multicolumn{1}{l}{SS} & df & MS & F & P-value & F crit \\
样本 & 174.0500 & 1 & 174.0500 & 44.0633 & 0.0002 & 4.4940 \\
列 & 92.4500 & 1 & 92.4500 & 23.4051 & 0.0002 & 4.4940 \\
交互 & 0.0500 & 1 & 0.0500 & 0.0127 & 0.9118 & 4.4940 \\
内部 & 63.2000 & 16 & 3.9500 &  &  &  \\
总计 & 329.75 & 19 &  &  &  &  \\
\hline
\end{tabular}
\end{center}

10.1 什么是方差分析? 它研究的是什么?

10.2 要检验多个总体均值是否相等时, 为什么不作两两比较, 而用方差分析方法?

10.3 方差分析包括哪些类型? 它们有何区别?

10.4 方差分析中有哪些基本假定?

10.5 简述方差分析的基本思想。

10.6 解释因子和处理的含义。

10.7 解释组内误差和组间误差的含义。

10.8 解释组内方差和组间方差的含义。

10.9 简述方差分析的基本步骤。

10.10 方差分析中多重比较的作用是什么?

10.11 什么是交互作用?

10.12 解释无交互作用和有交互作用的双因素方差分析。

10.13 解释 $R^{2}$ 的含义和作用。

\section*{练习题}
10.1 从 3 个总体中各抽取容量不同的样本数据, 结果如下:

\begin{center}
\begin{tabular}{ccc}
\hline
样本 1 & 样本 2 & 样本 3 \\
\hline
158 & 153 & 169 \\
148 & 142 & 158 \\
161 & 156 & 180 \\
154 & 149 &  \\
169 &  &  \\
\hline
\end{tabular}
\end{center}

检验 3 个总体的均值之间是否有显著差异 $(\alpha=0.01)$ 。

10.2 某家电制造公司准备购进一批 5 号电池, 现有 A, B, C 3 个电池生产企业愿意供货, 为比较它们生产的电池质量, 从每个企业各随机抽取 5 只电池, 经试验得其寿命数据(单位:小时)如下:

\begin{center}
\begin{tabular}{cccc}
\hline
 &  & 电池生产企业 &  \\
\cline { 2 - 4 }
试号 & B & C &  \\
\hline
1 & 50 & 32 & 45 \\
2 & 50 & 28 & 42 \\
3 & 43 & 30 & 38 \\
4 & 40 & 34 & 48 \\
5 & 39 & 26 & 40 \\
\hline
\end{tabular}
\end{center}

试分析 3 个企业生产的电池的平均寿命之间有无显著差异 $(\alpha=0.05)$ 。如果有差异, 用 $L S D$ 方法检验哪些企业之间有差异。

10.3 一家产品制造公司的管理者想比较 A, B, C 3 种培训方式对产品组装时间是否有显著影响,将 20 名新员工随机分配给这 3 种培训方式。培训结束后, 参加培训的员工组装一件产品所花的时间(单位:分钟)如下:

\begin{center}
\begin{tabular}{ccc}
\hline
培训方式 $\mathrm{A}$ & 培训方式 $\mathrm{B}$ & 培训方式 C \\
\hline
8.8 & 8.2 & 8.6 \\
9.3 & 6.7 & 8.5 \\
8.7 & 7.4 & 9.1 \\
9.0 & 8.0 & 8.2 \\
8.6 & 8.2 & 8.3 \\
8.3 & 7.8 & 7.9 \\
9.5 & 8.8 & 9.9 \\
9.4 & 8.4 & 9.4 \\
9.2 & 7.9 &  \\
\hline
\end{tabular}
\end{center}

取显著性水平 $\alpha=0.05$, 确定不同培训方式对产品组装的时间是否有显著影响。

10.4 某企业准备采用 3 种方法组装一种新的产品, 为确定哪种方法每小时组装的产品数量最多,随机抽取了 30 名工人, 并指定每个人使用其中的一种方法。通过对每名工人组装的产品数进行方差分析得到下面的结果:

方差分析表

\begin{center}
\begin{tabular}{c|c|c|c|c|c|c}
\hline
差异源 & SS & df & MS & F & P-value & F crit \\
\hline
组间 &  &  & 210 &  & 0.245946 & 3.354131 \\
组内 & 3836 &  &  & - & - & - \\
总计 &  & 29 & - & - & - & - \\
\hline
\end{tabular}
\end{center}

(1)完成上面的方差分析表。

(2)若显著性水平 $\alpha=0.05$, 检验采用 3 种方法组装的产品数量之间是否有显著差异。

10.5 有 5 种不同品种的种子和 4 种不同的施肥方案, 在 20 块同样面积的土地上, 将 5 种种子和 4 种施肥方案搭配起来进行试验, 取得的收获量数据如下:

\begin{center}
\begin{tabular}{ccccc}
\hline
\multirow{4}{c}{品种} & \multicolumn{4}{c}{施肥方案} \\
\cline { 2 - 5 }
 & 1 & 2 & 3 & 4 \\
\hline
 & 12.0 & 9.5 & 10.4 & 9.7 \\
 & 13.7 & 11.5 & 12.4 & 9.6 \\
4 & 14.3 & 12.3 & 11.4 & 11.1 \\
5 & 14.2 & 14.0 & 12.5 & 12.0 \\
 & 13.0 & 14.0 & 13.1 & 11.4 \\
\hline
\end{tabular}
\end{center}

检验种子的不同品种对收获量的影响是否显著, 不同的施肥方案对收获量的影响是否显著 $(\alpha=$ $0.05)$ 。

10.6 为研究食品的包装方法和销售地区对其销售量是否有影响, 在 3 个不同地区用 3 种不同包装\\
方法进行销售, 获得的销售量数据如下:

\begin{center}
\begin{tabular}{cccc}
\hline
 &  & 包装方法 (B) &  \\
\cline { 2 - 4 }
销售地区 $(\mathrm{A})$ & $\mathrm{B}_{1}$ & $\mathrm{~B}_{2}$ & $\mathrm{~B}_{3}$ \\
\hline
$\mathrm{A}_{1}$ & 45 & 75 & 30 \\
$\mathrm{~A}_{2}$ & 50 & 50 & 40 \\
$\mathrm{~A}_{3}$ & 35 & 65 & 50 \\
\hline
\end{tabular}
\end{center}

检验不同的地区和不同的包装方法对该食品的销售量是否有显著影响 $(\alpha=0.05)$ 。

10.7 一家超市连锁店进行一项研究, 以确定超市所在的位置和竞争者的数量对其销售额是否有显著影响。下面是获得的月销售额数据(单位: 万元):

\begin{center}
\begin{tabular}{ccccc}
\hline
\multirow{2}{*}{超市位置} & \multicolumn{3}{c}{竞争者数量} &  \\
\cline { 2 - 5 }
 & 0 & 1 & 2 & 3 个及以上 \\
\hline
\multirow{3}{*}{位于市内居民小区} & 41 & 38 & 49 & 47 \\
 & 30 & 31 & 51 & 40 \\
 & 45 & 39 & 44 & 39 \\
 & 25 & 29 & 48 & 43 \\
 & 31 & 35 & 50 & 53 \\
\hline
 & 22 & 30 & 29 & 24 \\
 & 18 & 22 & 28 & 27 \\
\hline
\end{tabular}
\end{center}

取显著性水平 $\alpha=0.01$, 检验:

(1) 竞争者的数量对销售额是否有显著影响。

(2)超市的位置对销售额是否有显著影响。

(3)竞争者的数量和超市的位置对销售额是否有交互影响。\\

高尔顿 (Galton) 被誉为现代回归和相关技术的创始人。1875 年, 高尔顿利用琬豆实验来确定尺寸的遗传规律。他挑选了 7 组不同尺寸的琬豆, 并说服他在英国不同地区的朋友每一组种植 10 粒种子, 然后把原始的琬豆种子(父代)与新长的琬豆种子(子代) 进行尺寸比较。

结果出来之后, 他发现并非每一个子代都与父代一样, 尺寸小的琬豆会得到更大的子代, 而尺寸大的琬豆却得到较小的子代。高尔顿把这一现象叫作 “返祖”(趋向于祖先的某种平均类型), 后来又称之为 “向平均回归”。一个总体中在某一时期具有某一极端特征(低于或高于总体均值)的个体(或者是单个个体或者是整个子代)在未来的某一时期将减弱它的极端性, 这一趋势现在被称作“回归效应”。人们发现它的应用很广,而不仅限于从上一代到下一代的琬豆大小问题。

正如高尔顿进一步发现的那样, 平均来说, 非常矮小的父辈倾向于有偏高的子代,而非常高大的父辈倾向于有偏矮的子代。在第一次考试中成绩最差的那些学生在第二次考试中倾向于有更好的成绩(比较接近所有学生的平均成绩),而第一次考试中成绩最好的那些学生在第二次考试中倾向于有较差的成绩 (比较接近所有学生的平均成绩)。同样, 平均来说, 第一年利润最低的公司第二年不会最差, 而第一年利润最高的公司第二年不会最好。\\
第 10 章介绍了分类型自变量与数值型因变量之间关系的分析方法。本章主要介绍数值型自变量和数值型因变量之间关系的分析方法, 这就是相关与回归分析。相关与回归是处理变量之间关系的一种统计方法。从所处理的变量多少来看, 如果研究的是两个变量之间的关系, 则称为简单相关与简单回归分析; 如果研究的是两个以上变量之间的关系, 则称为多元相关与多元回归分析。从变量之间的关系形态来看, 有线性相关与线性回归分析及非线性相关与非线性回归分析。本章主要讨论简单线性相关和简单线性回归的基本原理与方法。

\section*{11.1 变量间关系的度量}
\section*{11. 1.1 变量间的关系}
在生产和经营活动中, 经常要对变量之间的关系进行分析。例如, 在企业生产中, 要对影响生产成本的各种因素进行分析, 以达到控制成本的目的; 在农业生产中, 需要研究农作物产量与施肥量之间的关系, 以便分析施肥量对产量的影响, 进而确定合理的施肥量; 在商业活动中, 需要分析广告费支出与销售量之间的关系, 进而通过广告费支出来预测销售量; 等等。统计分析的目的在于根据统计数据确定变量之间的关系形态及关联的程度, 并探索内在的数量规律。人们在实践中发现, 变量之间的关系可分为两种类型: 函数关系和相关关系。

函数关系是人们比较熟悉的。设有两个变量 $x$ 和 $y$, 变量 $y$ 随变量 $x$ 一起变化, 并完全依赖于 $x$, 当变量 $x$ 取某个数值时, $y$ 依据确定的关系取相应的值, 则称 $y$ 是 $x$ 的函数,记为 $y=f(x)$, 其中 $x$ 称为自变量, $y$ 称为因变量。下面给出几个函数关系的例子。

\section*{例 11.1}
设某种产品的销售额为 $y$, 销售量为 $x$, 销售价格为 $p$, 则 $x$ 与 $y$ 之间的关系可表示为 $y=p x$ 。这就是说, 在销售价格不变的情况下, 对于该商品的某一销售量, 总有一个销售额与之对应,即销售额完全由销售量确定,二者之间为线性函数关系。

\section*{例 11.2}
企业的原材料消耗额(y)与产量 $\left(x_{1}\right)$ 、单位产品消耗 $\left(x_{2}\right)$ 、原材料价格 $\left(x_{3}\right)$ 之间的关系可表示为 $y=x_{1} x_{2} x_{3}$ 。这里的 $y$ 与 $x_{1}, x_{2}, x_{3}$ 之间是一种确定的函数关系, 但不是线性函数关系。

函数关系是一一对应的确定关系。但在实际问题中, 变量之间的关系往往不那么简单。例如, 考察居民储蓄与居民家庭收人这两个变量, 它们之间就不存在完全确定的关系。也就是说, 收人水平相同的家庭, 他们的储蓄额往往不同; 反之, 储蓄额相同的家\\
庭, 他们的收人水平也可能不同。可见, 居民储蓄并不完全由居民家庭收人确定, 因为家庭收人尽管与家庭储蓄有密切的关系, 但它并不是影响储蓄的唯一因素, 还有银行利率、消费水平等其他影响因素。正是由于影响一个变量的因素非常多, 才造成了变量之间关系的不确定性。变量之间存在的不确定的数量关系称为相关关系 (correlation)。

下面是相关关系的几个例子。

\section*{例 11.3}
从遗传学角度看, 子女的身高 $(y)$ 与其父母身高 $(x)$ 有很大关系。一般来说, 父母较高时, 其子女通常也比较高, 父母较矮时, 其子女通常也较矮。但实际情况并不完全是这样, 因为它们之间不是完全确定的关系。显然, 子女的身高并不完全由父母身高一个因素所决定,还受其他许多因素的影响,因此二者之间属于相关关系。

\section*{例 11.4}
考察一个人的收入水平 $(y)$ 和其受教育程度(x)两个变量, 它们之间不存在确定的函数关系。也就是说, 受教育程度相同的人, 他们的收入水平往往不同, 同样, 收入水平相同的人, 受教育程度也可能不同。因为受教育程度尽管与一个人的收入多少有关系, 但它并不是影响收入的唯一因素, 还有其他因素(如职业、工作年限等)的影响。因此, 收入水平与受教育程度之间是一种相关关系。

\section*{例 11.5}
农作物的单位面积产量 $(y)$ 与施肥量 $(x)$ 有密切的关系。在一定条件下, 施肥量越多, 单位面积产量就越高。但产量并不是由施肥量一个因素决定的, 还有其他许多因素的影响, 如降雨量、温度、管理水平等。因此, 农作物的单位面积产量与施肥量之间并不是函数关系,而是一种相关关系。

从上面的几个例子中可看出相关关系的特点: 一个变量的取值不能由另一个变量唯一确定, 当变量 $x$ 取某个值时, 变量 $y$ 的取值可能有几个。对这种关系不确定的变量显然不能用函数关系进行描述, 但也不是无任何规律可循。通过对大量数据的观察与研究, 就会发现许多变量之间存在着一定的客观规律。例如, 平均来说, 父亲较高时, 其子女一般也较高; 收人水平高的家庭, 其家庭储蓄一般也较多。相关与回归分析正是探索与描述这类变量之间关系及其规律的统计方法。

\section*{11. 1.2 相关关系的描述与测度}
相关分析就是对两个变量之间线性关系的描述与度量,它要解决的问题包括:

(1) 变量之间是否存在关系?

(2)如果存在关系,它们之间是什么样的关系?

(3)变量之间的关系强度如何?

(4)样本所反映的变量之间的关系能否代表总体变量之间的关系?

为解决这些问题,在进行相关分析时,对总体主要有以下两个假定:

第一,两个变量之间是线性关系。

第二, 两个变量都是随机变量。

在进行相关分析时,首先需要绘制散点图来判断变量之间的关系形态。如果是线性关系, 则可以利用相关系数来测度两个变量之间的关系强度, 然后对相关系数进行显著性检验, 以判断样本所反映的关系能否代表两个变量总体上的关系。

\section*{1. 散点图}
对于两个变量 $x$ 和 $y$, 通过观察或试验可以得到若干组数据, 记为 $\left(x_{i}, y_{i}\right) \quad(i=1$, $2, \cdots, n)$ 。用坐标的横轴代表变量 $x$, 纵轴代表变量 $y$, 每组数据 $\left(x_{i}, y_{i}\right)$ 在坐标系中用一个点表示, $n$ 组数据在坐标系中形成的 $n$ 个点称为散点, 由坐标及散点形成的二维数据图称为散点图 (scatter diagram)。

散点图是描述变量之间关系的一种直观方法, 从中可以大体上看出变量之间的关系形态及关系强度。图 11-1 就是不同形态的散点图。

\begin{center}
\end{center}

(a) 正线性相关

\begin{center}
\end{center}

(c) 完全正线性相关

\begin{center}
\end{center}

(e) 非线性相关

\begin{center}
\end{center}

(b)负线性相关

\begin{center}
\end{center}

(d) 完全负线性相关

\begin{center}
\end{center}

(f) 不相关\\
从图 11-1 可以看出, 相关关系的表现形态大体上可分为线性相关、非线性相关、完全相关和不相关等几种。就两个变量而言, 如果变量之间的关系近似地表现为一条直线,则称为线性相关,如图 11-1 (a) 和图 11-1 (b) 所示; 如果变量之间的关系近似地表现为一条曲线, 则称为非线性相关或曲线相关, 如图 11-1 (e) 所示; 如果一个变量的取值完全依赖于另一个变量, 各观测点落在一条直线上, 则称为完全相关, 如图 11-1 (c) 和图 11-1 (d) 所示, 这实际上就是函数关系; 如果两个变量的观测点很分散, 无任何规律, 则表示变量之间没有相关关系, 如图 11-1 (f) 所示。

在线性相关中, 若两个变量的变动方向相同, 一个变量的数值增加, 另一个变量的数值也随之增加, 或一个变量的数值减少, 另一个变量的数值也随之减少, 则称为正相关,如图 11-1 (a) 所示; 若两个变量的变动方向相反, 一个变量的数值增加, 另一个变量的数值随之减少, 或一个变量的数值减少, 另一个变量的数值随之增加, 则称为负相关, 如图 11-1 (b) 所示。

\section*{例 11.6}
一家大型商业银行在多个地区设有分行, 其业务主要是进行基础设施建设、国家重点项目建设、固定资产投资等项目的贷款。近年来, 该银行的贷款额平稳增长, 但不良贷款额也有较大比例的提高, 这给银行业务的发展带来较大压力。为弄清楚不良贷款形成的原因,管理者希望利用银行业务的有关数据做些定量分析,以便找出控制不良贷款的办法。表 11-1 就是该银行所属的 25 家分行的有关业务数据。

表 11-1 某商业银行的主要业务数据

\begin{center}
\begin{tabular}{cccccc}
\hline
\begin{tabular}{c}
分行 \\
编号 \\
\end{tabular} & \begin{tabular}{c}
不良贷款 \\
(亿元) \\
\end{tabular} & \begin{tabular}{c}
贷款余额 \\
(亿元) \\
\end{tabular} & \begin{tabular}{c}
累计应收贷款 \\
(亿元) \\
\end{tabular} & \begin{tabular}{c}
贷款项目个数 \\
(个) \\
\end{tabular} & \begin{tabular}{c}
固定资产投资额 \\
(亿元) \\
\end{tabular} \\
\hline
1 & 0.9 & 67.3 & 6.8 & 5 & 51.9 \\
2 & 1.1 & 111.3 & 19.8 & 16 & 90.9 \\
3 & 4.8 & 173.0 & 7.7 & 17 & 73.7 \\
4 & 3.2 & 80.8 & 7.2 & 10 & 14.5 \\
5 & 7.8 & 199.7 & 16.5 & 19 & 63.2 \\
6 & 2.7 & 16.2 & 2.2 & 1 & 2.2 \\
7 & 1.6 & 107.4 & 10.7 & 17 & 20.2 \\
8 & 12.5 & 185.4 & 27.1 & 18 & 43.8 \\
9 & 1.0 & 96.1 & 1.7 & 10 & 55.9 \\
10 & 2.6 & 72.8 & 9.1 & 14 & 64.3 \\
\hline
\end{tabular}
\end{center}

\begin{center}
\begin{tabular}{cccccc}
\hline
\begin{tabular}{c}
续表 \\
分行 \\
编号 \\
\end{tabular} & \begin{tabular}{c}
不良贷款 \\
(亿元) \\
\end{tabular} & \begin{tabular}{c}
贷款余额 \\
(亿元 $)$ \\
\end{tabular} & \begin{tabular}{c}
累计应收贷款 \\
(亿元) \\
\end{tabular} & \begin{tabular}{c}
贷款项目个数 \\
(个) \\
\end{tabular} & \begin{tabular}{c}
固定资产投资额 \\
(亿元) \\
\end{tabular} \\
\hline
11 & 0.3 & 64.2 & 2.1 & 11 & 42.7 \\
12 & 4.0 & 132.2 & 11.2 & 23 & 76.7 \\
13 & 0.8 & 58.6 & 6.0 & 14 & 22.8 \\
14 & 3.5 & 174.6 & 12.7 & 26 & 117.1 \\
15 & 10.2 & 263.5 & 15.6 & 34 & 146.7 \\
16 & 3.0 & 79.3 & 8.9 & 15 & 29.9 \\
17 & 0.2 & 14.8 & 0.6 & 2 & 42.1 \\
18 & 0.4 & 73.5 & 5.9 & 11 & 25.3 \\
19 & 1.0 & 24.7 & 5.0 & 4 & 13.4 \\
20 & 6.8 & 139.4 & 7.2 & 28 & 64.3 \\
21 & 11.6 & 368.2 & 16.8 & 32 & 163.9 \\
22 & 1.6 & 95.7 & 3.8 & 10 & 44.5 \\
23 & 1.2 & 109.6 & 10.3 & 14 & 67.9 \\
24 & 7.2 & 196.2 & 15.8 & 16 & 39.7 \\
25 & 3.2 & 102.2 & 12.0 & 10 & 97.1 \\
\hline
 &  &  &  &  &  \\
\hline
\end{tabular}
\end{center}

管理者想知道:不良贷款是否与贷款余额、累计应收贷款、贷款项目个数、固定资产投资额等因素有关?如果有关,它们之间是一种什么样的关系?关系强度如何?试绘制散点图,并分析不良贷款与贷款余额、累计应收贷款、贷款项目个数、固定资产投资额之间的关系。

解 用 Excel 绘制的散点图如图 11-2 至图 11-5 所示。

\begin{center}
\end{center}

图 11-2 不良贷款与贷款余额的散点图

\begin{center}
\end{center}

图 11-3 不良贷款与累计应收贷款的散点图

\begin{center}
\end{center}

图 11-4 不良贷款与贷款项目个数的散点图

\begin{center}
\end{center}

图 11-5 不良贷款与固定资产投资额的散点图

从各散点图可以看出, 不良贷款与贷款余额、累计应收贷款、贷款项目个数、固定资产投资额之间都具有一定的线性关系。从各散点的分布情况看, 不良贷款与贷款余额的线性关系比较密切,与固定资产投资额之间的关系最不密切。

\section*{2. 相关系数}
通过散点图可以判断两个变量之间有无相关关系, 并对变量间的关系形态作出大致的描述, 但散点图不能准确反映变量之间的关系强度。因此, 为准确度量两个变量之间的关系强度, 需要计算相关系数。

相关系数 (correlation coefficient) 是根据样本数据计算的度量两个变量之间线性关系强度的统计量。若相关系数是根据总体全部数据计算的, 称为总体相关系数, 记为 $\rho$; 若是根据样本数据计算的, 则称为样本相关系数, 记为 $r$ 。样本相关系数的计算公式为:


\begin{equation*}
r=\frac{n \sum x y-\sum x \sum y}{\sqrt{n \sum x^{2}-\left(\sum x\right)^{2}} \cdot \sqrt{n \sum y^{2}-\left(\sum y\right)^{2}}} \tag{11.1}
\end{equation*}


按上述公式计算的相关系数也称为线性相关系数 (linear correlation coefficient), 或称为 Pearson 相关系数 (Pearson's correlation coefficient)。

\section*{$\star$ Excel 中的统计函数}
使用 Excel 中的 CORREL 或 PEARSON 函数可以计算两组数据的相关系数。其语法为 CORREL (Array1, Array2)。Array1 和 Array2 是两个变量的数据区域。

\section*{例 11.7}
根据表 11-1 中的数据, 计算不良贷款、货款余额、累计应收贷款、贷款项目个数、固定资产投资额之间的相关系数。

\begin{itemize}
  \item 解 用 Excel【数据分析】中的【相关系数】工具计算的相关矩阵如表 $11-2$ 所示。\\
表 11-2 不良贷款、贷款余额、累计应收贷款、贷款项目个数、固定资产投资额之间的相关矩阵
\end{itemize}

\begin{center}
\begin{tabular}{lrrrrr}
\hline
 & 不良贷款 & 贷款余额 & 累计应收贷款 & 贷款项目个数 & 固定资产投资额 \\
\hline
不良贷款 & 1 &  &  &  &  \\
贷款余额 & 0.843571 & 1 &  &  &  \\
累计应收贷款 & 0.731505 & 0.678772 & 1 &  &  \\
贷款项目个数 & 0.700281 & 0.848416 & 0.585831 & 1 &  \\
固定资产投资额 & 0.518518 & 0.779702 & 0.472431 & 0.746646 & 1 \\
\hline
\end{tabular}
\end{center}

从相关矩阵可以看出, 在不良贷款与其他几个变量的关系中, 与货款余额的相关系数最大,而与固定资产投资额的相关系数最小。

为解释相关系数各数值的含义, 首先需要对相关系数的性质有所了解。相关系数的性质可总结如下:

(1) $r$ 的取值范围是 $[-1,1]$, 即 $-1 \leqslant r \leqslant 1$ 。若 $0<r \leqslant 1$, 表明 $x$ 与 $y$ 之间存在正线性相关关系; 若 $-1 \leqslant r<0$, 表明 $x$ 与 $y$ 之间存在负线性相关关系; 若 $r=1$, 表明 $x$ 与 $y$ 之间为完全正线性相关关系; 若 $r=-1$, 表明 $x$ 与 $y$ 之间为完全负线性相关关系。可见当 $|r|=1$ 时, $y$ 的取值完全依赖于 $x$, 二者之间为函数关系; 当 $r=0$ 时, $y$ 的取值与 $x$无关,二者之间不存在线性相关关系。

(2) $r$ 具有对称性。 $x$ 与 $y$ 之间的相关系数 $r_{x y}$ 和 $y$ 与 $x$ 之间的相关系数 $r_{x x}$ 相等, 即 $r_{x y}=r_{y x}$ 。

(3) $r$ 的数值大小与 $x$ 和 $y$ 的原点及尺度无关。改变 $x$ 和 $y$ 的数据原点及计量尺度,并不改变 $r$ 的数值大小。

(4) $r$ 仅仅是 $x$ 与 $y$ 之间线性关系的一个度量, 它不能用于描述非线性关系。这意味着, $r=0$ 只表示两个变量之间不存在线性相关关系, 并不说明变量之间没有任何关系, 它们之间可能存在非线性相关关系。变量之间的非线性相关程度较大时, 可能会导致 $r=0$ 。因此, 当 $r=0$ 或 $r$ 很小时, 不能轻易得出两个变量之间不存在相关关系的结论, 而应结合散点图作出合理的解释。

(5) $r$ 虽然是两个变量之间线性关系的一个度量, 却不意味着 $x$ 与 $y$ 一定有因果关系。

了解相关系数的性质有助于对其实际意义作出解释。但根据实际数据计算出的 $r$ 的取值一般在 $-1 \sim 1$ 之间, $|r| \rightarrow 1$ 说明两个变量之间的线性关系强; $|r| \rightarrow 0$ 说明两个变量之间的线性关系弱。对于一个具体的 $r$ 的取值, 根据经验可将相关程度分为以下几种情况: 当 $|r| \geqslant 0.8$ 时, 可视为高度相关; $0.5 \leqslant|r|<0.8$ 时, 可视为中度相关; $0.3 \leqslant$ $|r|<0.5$ 时, 可视为低度相关; 当 $|r|<0.3$ 时, 说明两个变量之间的相关程度极弱,可视为不相关。但这种解释必须建立在对相关系数的显著性进行检验的基础之上。

\subsection*{11.1.3 相关关系的显著性检验}
一般情况下, 总体相关系数 $\rho$ 是未知的, 通常将样本相关系数 $r$ 作为 $\rho$ 的近似估计值。但由于 $r$ 是根据样本数据计算出来的, 因此会受到抽样波动的影响。由于抽取的样本不\\
同, $r$ 的取值也就不同, 因此 $r$ 是一个随机变量。能否根据样本相关系数说明总体的相关程度呢? 这就需要考察样本相关系数的可靠性, 也就是进行显著性检验。

\section*{1. $r$ 的抽样分布}
为了对总体相关系数 $\rho$ 的显著性进行检验, 需要考察 $r$ 的抽样分布。 $r$ 的抽样分布随总体相关系数 $\rho$ 和样本量 $n$ 的大小而变化。当样本数据来自正态总体时, 随着 $n$ 的增大, $r$ 的抽样分布趋于正态分布, 尤其是在总体相关系数 $\rho$ 很小或接近 0 时, 趋于正态分布的趋势非常明显。而当 $\rho$ 远离 0 时, 除非 $n$ 非常大, 否则 $r$ 的抽样分布呈现一定的偏态。因为 $r$ 是在 $\rho$ 的周围分布的, 当 $\rho$ 的数值接近 1 或 -1 时, 比如 $\rho=0.96, r$ 的值可能以 0.96 为中心朝两个方向变化, 又由于 $r$ 的取值范围在 $-1 \sim 1$ 之间, 所以一方的变化以 1 为限,全距是 0.04 , 而另一方的变化以 -1 为限, 全距是 1.96 , 两个方向变化的全距不等, 因此 $r$ 的抽样分布也不可能对称。但当 $\rho$ 等于 0 或接近 0 时, 两个方向变化的全距接近相等,所以 $r$ 的抽样分布也就接近对称。

总之, 当 $\rho$ 为较大的正值时, $r$ 呈现左偏分布; 当 $\rho$ 为较大的负值时 $r$ 呈现右偏分布。只有当 $\rho$ 接近 0 , 而样本量 $n$ 很大时, 才能认为 $r$ 是接近正态分布的随机变量。然而, 以样本 $r$ 来估计总体 $\rho$ 时, 总是假设 $r$ 为正态分布, 但这一假设常常会带来一些严重后果。

\section*{2. $r$ 的显著性检验}
如果对 $r$ 服从正态分布的假设成立, 则可以应用正态分布来检验。但从上面对 $r$ 抽样分布的讨论可知, 对 $r$ 的正态性假设具有很大的风险, 因此通常情况下不采用正态检验, 而采用费希尔提出的 $t$ 检验, 该检验可以用于小样本, 也可以用于大样本。检验的具体步骤如下。

第 1 步: 提出假设。

$$
H_{0}: \rho=0 ; H_{1}: \rho \neq 0
$$

第 2 步: 计算检验的统计量。


\begin{equation*}
t=|r| \sqrt{\frac{n-2}{1-r^{2}}} \sim t(n-2) \tag{11.2}
\end{equation*}


第 3 步: 进行决策。根据给定的显著性水平 $\alpha$ 和自由度 $d f=n-2$ 查 $t$ 分布表, 得出 $t_{\alpha / 2}(n-2)$的临界值。若 $|t|>t_{a / 2}$, 则拒绝原假设 $H_{0}$, 表明总体的两个变量之间存在显著的线性关系。(1)

\section*{例 11.8}
根据表 11-2 计算的相关系数, 检验不良贷款与贷款余额之间的相关系数是否显著
\footnotetext{(1) 需要注意的是, 即使统计检验表明相关系数在统计上是显著的, 也并不一定意味着两个变量之间就存在重要的相关性。因为在大样本情况下, 几乎总是导致相关系数显著。比如, $r=0.1$, 在大样本情况下, 也可能使得 $r$ 通过检验。但实际上,一个变量取值的差异能由另一个变量的取值来解释的比例只有 $10 \%$, 这很难说明两个变量之间有实际意义上的显著关系。对这一问题的进一步理解, 请参见 11.2 节中有关判定系数 $R^{2}$ 的讨论。
}
$(\alpha=0.05) 。$

第 1 步: 提出假设。

$$
H_{0}: \rho=0 ; H_{1}: \rho \neq 0
$$

第 2 步: 计算检验的统计量。

$$
t=|r| \sqrt{\frac{n-2}{1-r^{2}}}=|0.8436| \times \sqrt{\frac{25-2}{1-0.8436^{2}}}=7.5344
$$

第 3 步: 进行决策。根据显著性水平 $\alpha=0.05$ 和自由度 $n-2=25-2=23$ 查 $t$ 分布表得: $t_{\alpha / 2}(n-2)=2.0687$ 。由于 $t=7.5344>t_{\alpha / 2}=2.0687$, 所以拒绝原假设 $H_{0}$, 说明不良贷款与贷款余额之间存在显著的正线性相关关系。

为了对其他相关系数进行检验, 表 11-3 给出了各相关系数检验的统计量, 请读者自行检验并进行分析。

表 11-3 各相关系数检验的统计量

\begin{center}
\begin{tabular}{lllll}
\hline
 & 不良贷款 & 贷款余额 & 累计应收贷款 & 贷款项目个数 \\
\hline
贷款余额 & 7.5335 &  &  &  \\
累计应收贷款 & 5.1452 & 4.4329 &  &  \\
贷款项目个数 & 4.7046 & 7.6868 & 3.4667 &  \\
固定资产投资额 & 2.9082 & 5.9719 & 2.5707 & 5.3828 \\
\hline
\end{tabular}
\end{center}

注: 本表统计量直接利用 Excel 计算得出, 没有计算误差。

\section*{11.2 一元线性回归}
相关分析的目的在于测度变量之间的关系强度, 它所使用的测度工具就是相关系数。而回归分析则侧重于考察变量之间的数量关系, 并通过一定的数学表达式将这种关系描述出来, 进而确定一个或几个变量 (自变量) 的变化对另一个特定变量 (因变量) 的影响程度。具体来说, 回归分析主要解决以下几个方面的问题:

(1) 从一组样本数据出发, 确定变量之间的数学关系式。

(2) 对这些关系式的可信程度进行各种统计检验, 并从影响某一特定变量的诸多变量中找出哪些变量的影响是显著的, 哪些是不显著的。

(3)利用所求的关系式, 根据一个或几个变量的取值来估计或预测另一个特定变量的取值, 并给出这种估计或预测的可靠程度。

\subsection*{11.2.1 一元线性回归模型}
\section*{1. 回归模型}
进行回归分析时, 首先需要确定哪个变量是因变量, 哪个变量是自变量。在回归分析\\
中, 被预测或被解释的变量称为因变量 (dependent variable), 用 $y$ 表示。用来预测或解释因变量的一个或多个变量称为自变量 (independent variable), 用 $x$ 表示。例如, 在分析贷款余额对不良贷款的影响时, 目的是要预测一定的贷款余额条件下不良贷款是多少, 因此, 不良贷款是被预测的变量, 称为因变量, 而用来预测不良贷款的贷款余额就是自变量。

当回归中只涉及一个自变量时, 称为一元回归, 若因变量 $y$ 与自变量 $x$ 之间为线性关系, 则称为一元线性回归。在回归分析中, 假定自变量 $x$ 是可控制的, 而因变量 $y$ 是随机的, 但很多情况下并非如此。本章所讨论的回归方法对于自变量是预先固定的和自变量是随机的情况都适用, 但固定自变量的情况比较容易描述, 因此下面主要讲述固定自变量的回归问题。

对于具有线性关系的两个变量, 可以用一个线性方程来表示它们之间的关系。描述因变量 $y$ 如何依赖于自变量 $x$ 和误差项 $\varepsilon$ 的方程称为回归模型 (regression model)。只涉及一个自变量的一元线性回归模型可表示为:


\begin{equation*}
y=\beta_{0}+\beta_{1} x+\varepsilon \tag{11.3}
\end{equation*}


在一元线性回归模型中, $y$ 是 $x$ 的线性函数 $\beta_{0}+\beta_{1} x$ 加上误差项 $\varepsilon_{0} \beta_{0}+\beta_{1} x$ 反映了由于 $x$ 的变化而引起的 $y$ 的线性变化; $\varepsilon$ 是被称为误差项的随机变量, 反映了除 $x$ 和 $y$ 之间的线性关系之外的随机因素对 $y$ 的影响, 是不能由 $x$ 和 $y$ 之间的线性关系所解释的变异性。式中的 $\beta_{0}$ 和 $\beta_{1}$ 称为模型的参数。

式 (11.3) 称为理论回归模型, 对这一模型, 有以下几个主要假定:

(1) 因变量 $y$ 与自变量 $x$ 之间具有线性关系。

(2)在重复抽样中, 自变量 $x$ 的取值是固定的, 即假定 $x$ 是非随机的。

在上述两个假定下, 对于任何一个给定的 $x$ 值, $y$ 的取值都对应着一个分布, 因此, $E(y)=\beta_{0}+\beta_{1} x$ 代表一条直线。但由于单个数据点是从 $y$ 的分布中抽出来的, 可能不在这条直线上, 因此, 必须包含一个误差项 $\varepsilon$ 来描述模型的数据点。

(3) 误差项 $\varepsilon$ 是一个期望值为 0 的随机变量, 即 $E(\varepsilon)=0$ 。这意味着在式 (11.3) 中,由于 $\beta_{0}$ 和 $\beta_{1}$ 都是常数, 有 $E\left(\beta_{0}\right)=\beta_{0}, E\left(\beta_{1}\right)=\beta_{1}$ 。因此对于一个给定的 $x$ 值, $y$ 的期望值为 $E(y)=\beta_{0}+\beta_{1} x$ 。这实际上等于假定模型的图示为一条直线。

(4) 对于所有的 $x$ 值, $\varepsilon$ 的方差 $\sigma^{2}$ 都相同。这意味着对于一个特定的 $x$ 值, $y$ 的方差都等于 $\sigma^{2}$ 。

(5)误差项 $\varepsilon$ 是一个服从正态分布的随机变量且独立,即 $\varepsilon \sim N\left(0, \sigma^{2}\right)$ 。独立性意味着一个特定的 $x$ 值所对应的 $\varepsilon$ 与其他 $x$ 值所对应的 $\varepsilon$ 不相关。因此, 一个特定的 $x$ 值所对应的 $y$ 值与其他 $x$ 值所对应的 $y$ 值也不相关。这表明, 在 $x$ 取某个确定值的情况下, $y$ 的变化由误差项 $\varepsilon$ 的方差 $\sigma^{2}$ 来决定。当 $\sigma^{2}$ 较小时, $y$ 的观测值将非常靠近直线; 当 $\sigma^{2}$ 较大时, $y$ 的观测值将偏离直线。由于 $\sigma^{2}$ 是常数, 所以 $y$ 的取值不受 $x$ 取值的影响。由于自变量 $x$ 在数据收集前假设是固定的, 因此, 对于任何一个给定的 $x$ 值, $y$ 都服从期望值为 $\beta_{0}+\beta_{1} x$ 、方差为 $\sigma^{2}$ 的正态分布, 且对于不同的 $x$ 具有相同的方差。关于回归模型的假定,\\
如图 11-6 所示。

\begin{center}
\end{center}

图 11-6 对应不同 $x$ 的 $y$ 和 $\varepsilon$ 的分布

从图 11-6 可以看出, $E(y)$ 的值随着 $x$ 的不同而变化, 但无论 $x$ 怎样变化, $\varepsilon$ 和 $y$的概率分布都是正态分布,并且具有相同的方差。

\section*{2. 回归方程}
根据回归模型中的假定, $\varepsilon$ 的期望值等于 0 , 因此 $y$ 的期望值 $E(y)=\beta_{0}+\beta_{1} x$, 也就是说, $y$ 的期望值是 $x$ 的线性函数。描述因变量 $y$ 的期望值如何依赖于自变量 $x$ 的方程称为回归方程 (regression equation)。一元线性回归方程的形式为:


\begin{equation*}
E(y)=\beta_{0}+\beta_{1} x \tag{11.4}
\end{equation*}


一元线性回归方程的图示是一条直线, 因此也称为直线回归方程。其中 $\beta_{0}$ 是回归直线在 $y$ 轴上的截距, 是当 $x=0$ 时 $y$ 的期望值; $\beta_{1}$ 是直线的斜率, 它表示 $x$ 每变动一个单位, $y$ 的平均变动值。

\section*{3. 估计的回归方程}
如果回归方程中的参数 $\beta_{0}$ 和 $\beta_{1}$ 已知, 对于一个给定的 $x$ 值, 利用式 (11.4) 就能计算出 $y$ 的期望值。但总体回归参数 $\beta_{0}$ 和 $\beta_{1}$ 是未知的, 必须利用样本数据去估计它们。用样本统计量 $\hat{\beta}_{0}$ 和 $\hat{\beta}_{1}$ 代替回归方程中的未知参数 $\beta_{0}$ 和 $\beta_{1}$, 这时就得到了估计的回归方程 (estimated regression equation)。它是根据样本数据求出的回归方程的估计。

对于一元线性回归, 估计的回归方程形式为:


\begin{equation*}
\hat{y}=\hat{\beta}_{0}+\hat{\beta}_{1} x \tag{11.5}
\end{equation*}


式中, $\hat{\beta}_{0}$ 是估计的回归直线在 $y$ 轴上的截距; $\hat{\beta}_{1}$ 是直线的斜率, 表示 $x$ 每变动一个单位, $y$ 的平均变动值。

\subsection*{11.2.2 参数的最小二乘估计}
对于第 $i$ 个 $x$ 值, 估计的回归方程可表示为:


\begin{equation*}
\hat{y}_{i}=\hat{\beta}_{0}+\hat{\beta}_{1} x_{i} \tag{11.6}
\end{equation*}


对于 $x$ 和 $y$ 的 $n$ 对观测值, 用于描述其关系的直线有多条, 究竟用哪条直线来代表两个变量之间的关系, 要有一个明确的原则。我们自然会想到距离各观测点最近的一条直线, 用它来代表 $x$与 $y$ 之间的关系与实际数据的误差比其他任何直线都小。卡尔 - 高斯 (Carl Gauss, 1777-1855)提出用最小化图 (见图 11-7) 中垂直方向的离差平方和来估计参数 $\beta_{0}$ 和 $\beta_{1}$, 根据这一方法确定模型参数 $\beta_{0}$ 和 $\beta_{1}$ 的方法称为最小二乘法, 也称最小平方法 (method of least squares), 它是通过使因变量的观测值 $y_{i}$ 与估计值 $\hat{y}_{i}$ 之间的离差平方和达到最小来估计 $\beta_{0}$ 和 $\beta_{1}$ 的方法。

最小二乘法的思想可用图 11-7 表示。

\begin{center}
\end{center}

图 11-7 最小二乘法示意图

用最小二乘法拟合的直线具有一些优良的性质。首先, 根据最小二乘法得到的回归直线能使离差平方和达到最小, 虽然这并不能保证它就是拟合数据的最佳直线(1), 但这毕竟是一条与数据拟合良好的直线应有的性质。其次, 由最小二乘法求得的回归直线可知 $\beta_{0}$ 和 $\beta_{1}$ 的估计量的抽样分布。最后, 在某些条件下, $\beta_{0}$ 和 $\beta_{1}$ 的最小二乘估计量同其他估计量相比,其抽样分布具有较小的标准差。正是基于上述性质,最小二乘法广泛用于回归模型参数的估计。

根据最小二乘法,使


\begin{equation*}
\sum\left(y_{i}-\hat{y}_{i}\right)^{2}=\sum\left(y_{i}-\hat{\beta}_{0}-\hat{\beta}_{1} x_{i}\right)^{2} \tag{11.7}
\end{equation*}


最小。令 $Q=\sum\left(y_{i}-\hat{y}_{i}\right)^{2}$, 在给定样本数据后, $Q$ 是 $\hat{\beta}_{0}$ 和 $\hat{\beta}_{1}$ 的函数, 且最小值总是存在。根据微积分的极值定理, 对 $Q$ 求相应于 $\hat{\beta}_{0}$ 和 $\hat{\beta}_{1}$ 的偏导数, 令其等于 0 , 便可求出 $\hat{\beta}_{0}$ 和 $\hat{\beta}_{1}$, 即

\[
\left\{\begin{array}{l}
\left.\frac{\partial Q}{\partial \beta_{0}}\right|_{\beta_{0}=\hat{\beta}_{0}}=-2 \sum_{i=1}^{n}\left(y_{i}-\hat{\beta}_{0}-\hat{\beta}_{1} x_{i}\right)=0  \tag{11.8}\\
\left.\frac{\partial Q}{\partial \beta_{1}}\right|_{\beta_{1}=\hat{\beta}_{1}}=-2 \sum_{i=1}^{n} x_{i}\left(y_{i}-\hat{\beta}_{0}-\hat{\beta}_{1} x_{i}\right)=0
\end{array}\right.
\]

解上述方程组得:

\[
\left\{\begin{array}{l}
\hat{\beta}_{1}=\frac{n \sum_{i=1}^{n} x_{i} y_{i}-\sum_{i=1}^{n} x_{i} \sum_{i=1}^{n} y_{i}}{n \sum_{i=1}^{n} x_{i}^{2}-\left(\sum_{i=1}^{n} x_{i}\right)^{2}}  \tag{11.9}\\
\hat{\beta}_{0}=\bar{y}-\hat{\beta}_{1} \bar{x}
\end{array}\right.
\]
\footnotetext{(1) 许多其他拟合直线也具有这种性质。
}
由式 (11.9) 可知, 当 $x=\bar{x}$ 时, $\hat{y}=\bar{y}$, 即回归直线 $\hat{y}_{i}=\hat{\beta}_{0}+\hat{\beta}_{1} x_{i}$ 通过点 $(\bar{x}, \bar{y})$,这是回归直线的重要特征之一。

\section*{例 11.9}
根据例 11.6 的数据, 求不良贷款对贷款余额的估计方程。

・・解 根据式 (11.9) 得:

$$
\left\{\begin{array}{l}
\hat{\beta}_{1}=\frac{25 \times 17080.14-3006.7 \times 93.2}{25 \times 516543.37-3006.7^{2}}=0.037895 \\
\hat{\beta}_{0}=3.728-0.037895 \times 120.268=-0.8295
\end{array}\right.
$$

即不良贷款对贷款余额的估计方程为 $\hat{y}=-0.8295+0.037895 x$ 。回归系数 $\hat{\beta}_{1}=0.037895$表示货款余额每增加 1 亿元, 不良贷款平均增加 0.037895 亿元。在回归分析中, 对截距 $\hat{\beta}_{0}$ 常常不能赋予任何真实意义,例如,在不良贷款与贷款余额的回归中, $\hat{\beta}_{0}=-0.8295$,如果要解释, 它是指当货款余额为 0 时, 不良贷款的平均值为 -0.8295 亿元。但是, 当贷款余额为 0 (没有货款)时,自然也就不会有不良货款,而在这里,它的值为一个负数,这就很难解释得通。因此, 在回归分析中, 对截距 $\hat{\beta}_{0}$ 通常不作实际意义上的解释。

将 $x_{i}$ 的各个取值代入上述估计方程, 可以得到不良贷款的各个估计值 $\hat{y}_{i}$ 。由图 11-8 可以看出散点图与回归直线的关系。

\begin{center}
\end{center}

图 11-8 不良贷款对贷款余额的回归直线

回归分析中的计算量较大, 特别是多元回归, 用手工计算几乎是不可能的。因此, 在实际分析中, 回归的计算完全依赖于计算机。除专门的统计软件外, 为大多数人所熟悉的 Excel 软件也有回归分析功能。下面将结合本例, 说明用 Excel 进行回归的具体步骤。

首先, 将不良贷款与货款余额的数据输入 Excel 工作表中的 A2: B26 单元格。然后按下列步骤进行操作。\\
$\star$ 用 Excel 进行回归分析的操作步骤

第 1 步: 点击【数据】, 并点击【数据分析】选项。

第 2 步:在分析工具中选择【回归】, 点击【确定】。

第 3 步: 当对话框出现时: 在【Y 值输入区域】方框内输入数据区域 $A 2: A 26$ 。在【X 值输入区

域】方框内输入数据区域 B2:B26。在【置信度】选项中给出所需的数值(这里我们使用默认值 $95 \%$ )。在【输出选项】中选择输出区域 (这里我们选新工作表组)。在【残差】中选择所需的选项 (这里我们暂时未选)。结果如图 11-9 所示。

\begin{center}
\end{center}

图 11-9 用 Excel 进行回归的步骤

点击【确定】后得到输出的结果。

本例的 Excel 输出结果如表 11-4 所示。

表 11-4 Excel 输出的回归分析结果

SUMMARY OUTPUT

回归统计\\
Multiple R\\
0. 843571\\
$R$ Square\\
0.711613\\
Adjusted R Square\\
0. 699074\\
标准误差

\begin{enumerate}
  \item 979948\\
观测值\\
续表
\end{enumerate}

\begin{center}
\begin{tabular}{|c|c|c|c|c|c|c|}
\hline
 & df & SS & MS & F & Significance F &  \\
\hline
回归 & 1 & 222.48598 & 222.48597956 & 6. 753844 & 1. $18349 \mathrm{E}-07$ &  \\
\hline
残差 & 23 & 90.164421 & 3. 920192 &  &  &  \\
\hline
\multirow[t]{2}{*}{总计} & 24 & 312.6504 &  &  &  &  \\
\hline
 & Coefficients & 标准误差 & t Stat & P-value & Lower $95 \%$ & Upper $95 \%$ \\
\hline
Intercept & -0.829521 & 0.723043 & -1.147263 & 0.263068 & -2.325248 & 0.666206 \\
\hline
X Variable 1 & 0.037895 & 0.005030 & 7.533515 & 0.000000 & 0.027489 & 0.048300 \\
\hline
\end{tabular}
\end{center}

Excel 输出的回归结果包括以下几个部分:

第一部分是 “回归统计”, 这部分给出了回归分析中的一些常用统计量, 包括相关系数 (Multiple R)、判定系数 (R Square)、调整的判定系数 (Adjusted R Square)、标准误差、观测值等。

第二部分是 “方差分析”, 这部分给出的是回归分析的方差分析表, 包括回归和残差的自由度 (df)、总平方和 (SS)、均方 (MS) 、检验统计量 (F)、F 检验的显著性水平 (Significance F)。这部分的主要作用是对回归方程的线性关系进行显著性检验。下面将详细介绍。

第三部分是回归参数估计的有关内容。包括回归方程的截距(Intercept)和斜率(X Variable 1) 的系数 (Coefficients)、标准误差、用于检验回归系数的 $t$ 统计量 (t Stat) 和 $P$ 值(P-value),以及截距和斜率的置信区间(Lower 95\%和 Upper 95\%) 等。

此外, 还有 “残差分析” 部分, 这里暂时未给出其输出结果。对于本章内容所涉及的一些结果, 后面会陆续介绍。

\subsection*{11.2.3 回归直线的拟合优度}
回归直线 $\hat{y}_{i}=\hat{\beta}_{0}+\hat{\beta}_{1} x_{i}$ 在一定程度上描述了变量 $x$ 与 $y$ 之间的数量关系, 根据这一方程, 可依据自变量 $x$ 的取值来佔计或预测因变量 $y$ 的取值。但估计或预测的精度如何将取决于回归直线对观测数据的拟合程度。可以想象, 如果各观测数据的散点都落在一条直线上, 那么这条直线就是对数据的完全拟合, 直线充分代表了各个点, 此时用 $x$ 来估计 $y$ 是没有误差的。各观测点越是紧密围绕直线, 说明直线对观测数据的拟合程度越好, 反之则越差。回归直线与各观测点的接近程度称为回归直线对数据的拟合优度 (goodness of fit)。为说明直线的拟合优度, 需要计算判定系数。

\section*{1. 判定系数}
判定系数是对估计的回归方程拟合优度的度量。为说明它的含义, 需要对因变量 $y$ 取\\
值的变差进行研究。

因变量 $y$ 的取值是不同的, $y$ 取值的这种波动称为变差。变差的产生来自两个方面:一是由自变量 $x$ 的取值不同造成的; 二是除 $x$ 以外的其他因素(如 $x$ 对 $y$ 的非线性影响、

\begin{center}
\end{center}

图 11-10 变差分解图测量误差等) 的影响。对一个具体的观测值来说, 变差的大小可以用实际观测值 $y$ 与其均值 $\bar{y}$ 之差 $y-\bar{y}$ 来表示。而 $n$ 次观测值的总变差可由这些离差的平方和来表示, 称为总平方和,记为 $S S T$, 即


\begin{equation*}
S S T=\sum\left(y_{i}-\bar{y}\right)^{2} \tag{11.10}
\end{equation*}


从图 11-10 可以看出, 每个观测点的离差都可以分解为:


\begin{equation*}
y-\bar{y}=(y-\hat{y})+(\hat{y}-\bar{y}) \tag{11.11}
\end{equation*}


将式 (11.11) 两边平方, 并对所有 $n$ 个点求和, 有


\begin{equation*}
\sum\left(y_{i}-\bar{y}\right)^{2}=\sum\left(y_{i}-\hat{y}_{i}\right)^{2}+\sum\left(\hat{y}_{i}-\bar{y}\right)^{2}+2 \sum\left(y_{i}-\hat{y}_{i}\right)\left(\hat{y}_{i}-\bar{y}\right) \tag{11.12}
\end{equation*}


可以证明, $\sum\left(y_{i}-\hat{y}_{i}\right)\left(\hat{y}_{i}-\bar{y}\right)=0$, 因此


\begin{equation*}
\sum\left(y_{i}-\bar{y}\right)^{2}=\sum\left(y_{i}-\hat{y}_{i}\right)^{2}+\sum\left(\hat{y}_{i}-\bar{y}\right)^{2} \tag{11.13}
\end{equation*}


式 (11.13) 的左边称为总平方和 $S S T$, 它可分解为两部分: $\sum\left(\hat{y}_{i}-\bar{y}\right)^{2}$ 是回归值 $\hat{y}_{i}$与均值 $\bar{y}$ 的离差平方和, 根据估计的回归方程, 估计值 $\hat{y}_{i}=\hat{\beta}_{0}+\hat{\beta}_{1} x_{i}$, 因此可以把 $\hat{y}_{i}-\bar{y}$看作由于自变量 $x$ 的变化引起的 $y$ 的变化, 而其平方和 $\sum\left(\hat{y}_{i}-\bar{y}\right)^{2}$ 则反映了 $y$ 的总变差中由于 $x$ 与 $y$ 之间的线性关系引起的 $y$ 的变化部分, 它是可以由回归直线来解释的 $y_{i}$ 的变差部分, 称为回归平方和, 记为 $S S R ; \sum\left(y_{i}-\hat{y}_{i}\right)^{2}$ 是各实际观测点与回归值的残差 $y_{i}-\hat{y}_{i}$ 平方和, 它是除了 $x$ 对 $y$ 的线性影响之外的其他因素引起的 $y$ 的变化部分, 是不能由回归直线来解释的 $y_{i}$ 的变差部分, 称为残差平方和或误差平方和, 记为 $S S E$ 。三个平方和的关系为:


\begin{equation*}
\text { 总平方和 }(S S T)=\text { 回归平方和 }(S S R)+\text { 残差平方和 }(S S E) \tag{11.14}
\end{equation*}


从图 11-10 可以直观地看出, 回归直线拟合的好坏取决于 $S S R$ 及 $S S E$ 的大小, 或者说取决于回归平方和 $S S R$ 占总平方和 $S S T$ 的比例 ( $S S R / S S T)$ 的大小。各观测点越是靠近直线, $S S R / S S T$ 则越大, 直线拟合得越好。回归平方和占总平方和的比例称为判定系数 (coefficient of determination), 记为 $R^{2}$, 其计算公式为:


\begin{equation*}
R^{2}=\frac{S S R}{S S T}=\frac{\sum\left(\hat{y}_{i}-\bar{y}\right)^{2}}{\sum\left(y_{i}-\bar{y}\right)^{2}}=1-\frac{\sum\left(y_{i}-\hat{y}_{i}\right)^{2}}{\sum\left(y_{i}-\bar{y}\right)^{2}} \tag{11.15}
\end{equation*}


判定系数 $R^{2}$ 测度了回归直线对观测数据的拟合程度。若所有观测点都落在直线上,残差平方和 $S S E=0$, 则 $R^{2}=1$, 拟合是完全的; 如果 $y$ 的变化与 $x$ 无关, $x$ 完全无助于\\
解释 $y$ 的变差, $\hat{y}=\bar{y}$, 则 $R^{2}=0$ 。可见 $R^{2}$ 的取值范围是 $[0,1] 。 R^{2}$ 越接近 1 , 表明回归平方和占总平方和的比例越大, 回归直线与各观测点越接近, 用 $x$ 的变化来解释 $y$ 值变差的部分就越多, 回归直线的拟合程度就越好; 反之, $R^{2}$ 越接近 0 , 回归直线的拟合程度就越差。

在一元线性回归中, 相关系数 $r$ 实际上是判定系数的平方根。根据这一结论, 不仅可以由相关系数直接计算判定系数 $R^{2}$, 也可以进一步理解相关系数的意义。相关系数 $r$ 与回归系数 $\hat{\beta}_{1}$ 的正负号是相同的, 实际上, 相关系数 $r$ 从另一个角度说明了回归直线的拟合优度。 $|r|$ 越接近 1 , 表明回归直线对观测数据的拟合程度越好。但用 $r$ 说明回归直线的拟合优度要慎重, 因为 $r$ 的值总是大于 $R^{2}$ 的值(除非 $r=0$ 或 $|r|=1$ )。比如, 当 $r=0.5$时, 表面上看相关程度似乎接近一半, 但 $R^{2}=0.25$, 实际上这只能解释总变差的 $25 \%$ 。 $r=0.7$ 才能解释近一半的变差, $r<0.3$ 意味着只有很少一部分变差可由回归直线来解释。

\section*{例 11.10}
根据例 11.6 的数据, 计算不良贷款对贷款余额回归的判定系数, 并解释其意义。

一解 根据表 11-4 Excel 输出的回归分析结果可知, 总平方和 $S S T=312.6504$; 回归平方和 $S S R=222.4860$; 残差平方和 $S S E=90.1644$ 。根据式(11.15)得:

$$
R^{2}=\frac{S S R}{S S T}=\frac{222.4860}{312.6504}=0.7116=71.16 \%
$$

实际上,表 11-4 中直接给出了判定系数(R Square)为 0.711613 。

判定系数的实际意义是: 在不良贷款取值的变差中, 有 $71.16 \%$ 可以由不良贷款与贷款余额之间的线性关系来解释, 或者说, 在不良贷款取值的变动中, 有 $71.16 \%$ 是由贷款余额决定的。不良贷款取值的差异有 $2 / 3$ 以上是由贷款余额决定的, 可见二者之间有较强的线性关系。

\section*{2. 估计标准误差}
判定系数可用于度量回归直线的拟合程度, 相关系数也可以起到类似的作用。而残差平方和则可以说明实际观测值 $y_{i}$ 与回归估计值 $\hat{y}_{i}$ 之间的差异程度。估计标准误差 (standard error of estimate) 就是度量各实际观测点在直线周围的散布状况的一个统计量,它是均方残差 (MSE) 的平方根, 用 $s_{e}$ 来表示, 其计算公式为:


\begin{equation*}
s_{e}=\sqrt{\frac{\sum\left(y_{i}-\hat{y}_{i}\right)^{2}}{n-2}}=\sqrt{\frac{S S E}{n-2}}=\sqrt{M S E} \tag{11.16}
\end{equation*}


估计标准误差是对误差项 $\varepsilon$ 的标准差 $\sigma$ 的估计, 它可以看作在排除了 $x$ 对 $y$ 的线性影响后, $y$ 随机波动大小的一个估计量。从估计标准误差的实际意义看, 它反映了用估计的回归方程预测因变量 $y$ 时预测误差的大小。各观测点越靠近直线, $s_{e}$ 越小, 回归直线对各观测点的代表性就越好, 根据估计的回归方程进行预测也就越准确。若各观测点全部落在\\
直线上, 则 $s_{e}=0$, 此时用自变量来预测因变量是没有误差的。可见 $s_{e}$ 从另一个角度说明了回归直线的拟合优度。

从式 (11.16) 容易看出, 回归直线是对 $n$ 个观测点拟合的所有直线中, 估计标准误差最小的一条直线, 因为回归直线是当 $\sum\left(y_{i}-\hat{y}_{i}\right)^{2}$ 最小时确定的。

\section*{例 11.11}
根据例 11.9 的有关结果, 计算不良贷款对贷款余额回归的估计标准误差, 并解释其意义。

・解 根据表 11-4 Excel 输出的回归分析结果可知, $S S E=90.1644$ 。根据式 (11.16) 得:

$$
s_{e}=\sqrt{\frac{\operatorname{SSE}}{n-2}}=\sqrt{\frac{90.1644}{25-2}}=1.9799(\text { 亿元) }
$$

实际上, 表 11-4 中直接给出了该值, 即标准误差=1.979 948。这就是说, 根据货款余额来估计不良贷款时, 平均的估计误差为 1.9799 亿元。

\subsection*{11.2.4 显著性检验}
回归分析的主要目的是根据所建立的估计方程用自变量 $x$ 来估计或预测因变量 $y$ 的取值。建立了估计方程后, 还不能马上进行估计或预测, 因为该估计方程是根据样本数据得出的, 它是否真实地反映了变量 $x$ 和 $y$ 之间的关系, 需要通过检验来证实。

根据样本数据拟合回归方程时, 实际上已经假定变量 $x$ 与 $y$ 之间存在线性关系, 即 $y=\beta_{0}+\beta_{1} x+\varepsilon$, 并假定误差项 $\varepsilon$ 是一个服从正态分布的随机变量, 且对不同的 $x$ 具有相同的方差。但这些假设是否成立, 需要通过检验来证实。

回归分析中的显著性检验主要包括两方面内容: 一是线性关系的检验; 二是回归系数的检验。

\section*{1. 线性关系的检验}
线性关系检验是检验自变量 $x$ 和因变量 $y$ 之间的线性关系是否显著, 或者说, 它们之间能否用一个线性模型 $y=\beta_{0}+\beta_{1} x+\varepsilon$ 来表示。为检验两个变量之间的线性关系是否显著,需要构造用于检验的统计量。该统计量的构造是以回归平方和 (SSR) 和残差平方和 ( $S S E$ ) 为基础的。将 $S S R$ 除以其相应的自由度(SSR 的自由度是自变量的个数 $k$, 一元线性回归中自由度为 1) 的结果称为均方回归, 记为 MSR; 将 $S S E$ 除以其相应的自由度 ( $S S E$ 的自由度为 $n-k-1$, 一元线性回归中自由度为 $n-2$ ) 的结果称为均方残差, 记为 $M S E$ 。如果原假设成立 $\left(H_{0}: \beta_{1}=0\right.$, 两个变量之间的线性关系不显著), 则比值 $M S R /$ $M S E$ 的抽样分布服从分子自由度为 1 、分母自由度为 $n-2$ 的 $F$ 分布, 即

$$
F=\frac{\operatorname{SSR} / 1}{\operatorname{SSE} /(n-2)}
$$


\begin{equation*}
=\frac{M S R}{M S E} \sim F(1, n-2) \tag{11.17}
\end{equation*}


所以当原假设 $H_{0}: \beta_{1}=0$ 成立时, $M S R / M S E$ 的值应接近 1 , 但如果原假设 $H_{0}$ : $\beta_{1}=0$ 不成立, $M S R / M S E$ 的值将变得无穷大。因此, 较大的 $M S R / M S E$ 值将导致拒绝原假设 $H_{0}$, 此时就可以断定变量 $x$ 与 $y$ 之间存在显著的线性关系。线性关系检验的具体步骤如下。

第 1 步: 提出假设。

$$
H_{0}: \beta_{1}=0 \text { 两个变量之间的线性关系不显著 }
$$

第 2 步: 计算检验统计量 $F$ 。

$$
F=\frac{S S R / 1}{S S E /(n-2)}=\frac{M S R}{M S E}
$$

第 3 步: 作出决策。确定显著性水平 $\alpha$, 并根据分子自由度 $d f_{1}=1$ 和分母自由度 $d f_{2}=$ $n-2$ 查 $F$ 分布表, 找到相应的临界值 $F_{\alpha}$ 。若 $F>F_{\alpha}$, 拒绝 $H_{0}$, 表明两个变量之间的线性关系是显著的; 若 $F<F_{\alpha}$, 不拒绝 $H_{0}$, 没有证据表明两个变量之间的线性关系显著。

\section*{例 11.12}
根据例 11.9 的有关结果,检验不良货款与贷款余额之间线性关系的显著性 $(\alpha=0.05)$ 。

-解 第 1 步: 提出假设。

$$
H_{0}: \beta_{1}=0 \text { 两个变量之间的线性关系不显著 }
$$

第 2 步: 计算检验统计量 $F$ 。

$$
F=\frac{\operatorname{SSR} / 1}{\operatorname{SSE} /(n-2)}=\frac{222.48598 / 1}{90.164421 /(25-2)}=\frac{222.48598}{3.920192}=56.75384
$$

第 3 步: 作出决策。根据显著性水平 $\alpha=0.05$, 分子自由度 $d f_{1}=1$ 和分母自由度 $d f_{2}=$ $25-2=23$, 查 $F$ 分布表, 找到相应的临界值 $F_{\alpha}=4.28$ 。由于 $F>F_{\alpha}$, 拒绝 $H_{0}$, 表明不良货款与货款余额之间的线性关系是显著的。

实际上, 在 Excel 输出的回归分析结果中, 方差分析部分给出了线性关系显著性检验的全部结果 (见表 $11-4$ )。方差分析部分除给出检验统计的 $F$ 值外, 还给出了用于检验的显著性 $F$, 即 Significance $\mathrm{F}$, 它就是用于检验的 $P$ 值。将 Significance $\mathrm{F}$ 的值与给定的显著性水平 $\alpha$ 的值进行比较, 如果 Significance $\mathrm{F}<\alpha$, 则拒绝原假设 $H_{0}$, 表明因变量 $y$ 与自变量 $x$ 之间有显著的线性关系; 如果 Significance $\mathrm{F}>\alpha$, 则不拒绝原假设 $H_{0}$, 没有证据表明因变量 $y$ 与自变量 $x$ 之间有显著的线性关系。在表 $11-4$ 的输出结果中, Significance $F=$ $1.18349 \mathrm{E}-07<\alpha=0.05$, 这说明不良贷款与贷款余额之间存在显著的线性关系。所得结论与统计量检验的结果相同。

\section*{2. 回归系数的检验}
回归系数的显著性检验是要检验自变量对因变量的影响是否显著。在一元线性回归模\\
型 $y=\beta_{0}+\beta_{1} x+\varepsilon$ 中, 如果回归系数 $\beta_{1}=0$, 则回归线是一条水平线, 表明因变量 $y$ 的取值不依赖于自变量 $x$, 即两个变量之间没有线性关系。如果回归系数 $\beta_{1} \neq 0$, 也不能得出两个变量之间存在线性关系的结论, 要看这种关系是否具有统计意义上的显著性。回归系数的显著性检验就是检验回归系数 $\beta_{1}$ 是否等于 0 。为检验原假设 $H_{0}: \beta_{1}=0$ 是否成立, 需要构造用于检验的统计量。为此, 需要研究回归系数 $\beta_{1}$ 的抽样分布。

估计的回归方程 $\hat{y}_{i}=\hat{\beta}_{0}+\hat{\beta}_{1} x_{i}$ 是根据样本数据计算的。当抽取不同的样本时, 就会得出不同的估计方程。实际上, $\hat{\beta}_{0}$ 和 $\hat{\beta}_{1}$ 是根据最小二乘法得到的用于估计参数 $\beta_{0}$ 和 $\beta_{1}$ 的统计量, 它们都是随机变量, 都有自己的分布。根据检验的需要, 这里只讨论 $\hat{\beta}_{1}$ 的分布。统计证明, $\hat{\beta}_{1}$ 服从正态分布, 其数学期望为 $E\left(\hat{\beta}_{1}\right)=\beta_{1}$, 标准差为:


\begin{equation*}
\sigma_{\beta_{1}}=\frac{\sigma}{\sqrt{\sum x_{i}^{2}-\frac{1}{n}\left(\sum x_{i}\right)^{2}}} \tag{11.18}
\end{equation*}


式中, $\sigma$ 是误差项 $\varepsilon$ 的标准差。

由于 $\sigma$ 未知, 将 $\sigma$ 的估计量 $s_{e}$ 代人式 (11.18), 得到 $\sigma_{\beta_{1}}$ 的估计量, 即 $\hat{\beta}_{1}$ 的估计的标准差为:


\begin{equation*}
s_{\hat{\beta}_{1}}=\frac{s_{e}}{\sqrt{\sum x_{i}^{2}-\frac{1}{n}\left(\sum x_{i}\right)^{2}}} \tag{11.19}
\end{equation*}


这样就可以构造出用于检验回归系数 $\beta_{1}$ 的统计量 $t$ :


\begin{equation*}
t=\frac{\hat{\beta}_{1}-\beta_{1}}{s_{\hat{\beta}_{1}}} \tag{11.20}
\end{equation*}


该统计量服从自由度为 $n-2$ 的 $t$ 分布。如果原假设成立, 则 $\beta_{1}=0$, 检验的统计量为:


\begin{equation*}
t=\frac{\hat{\beta}_{1}}{s_{\beta_{1}}} \tag{11.21}
\end{equation*}


回归系数的显著性检验的具体步骤如下:

第 1 步: 提出检验。

$$
H_{0}: \beta_{1}=0 ; H_{1}: \beta_{1} \neq 0
$$

第 2 步: 计算检验统计量 $t$ 。

$$
t=\frac{\hat{\beta}_{1}}{s_{\beta_{1}}}
$$

第 3 步: 作出决策。确定显著性水平 $\alpha$, 并根据自由度 $d f=n-2$ 查 $t$ 分布表, 找到相应的临界值 $t_{\alpha / 2}$ 。若 $|t|>t_{\alpha / 2}$, 则拒绝 $H_{0}$, 回归系数等于 0 的可能性小于 $\alpha$, 表明自变量 $x$ 对因变量 $y$ 的影响是显著的, 换言之, 两个变量之间存在显著的线性关系; 若 $|t|<t_{\alpha / 2}$, 则不拒绝 $H_{0}$ ,没有证据表明 $x$ 对 $y$ 的影响显著,或者说,二者之间尚不存在显著的线性关系。

\section*{例 11.13}
根据例 11.9 的有关结果, 检验回归系数的显著性 $(\alpha=0.05)$ 。

・・解 第 1 步:提出假设。

$$
H_{0}: \beta_{1}=0 ; H_{1}: \beta_{1} \neq 0
$$

第 2 步: 计算检验统计量 $t$ 。

$$
t=\frac{\hat{\beta}_{1}}{s_{\hat{\beta}_{1}}}=\frac{0.037895}{0.005030}=7.533797
$$

第 3 步: 作出决策。根据给定显著性水平 $\alpha=0.05$, 自由度 $=n-2=25-2=23$, 查 $t$分布表, 得 $t_{\alpha / 2}=t_{0.025}=2.0687$ 。由于 $t=7.533797>t_{0.025}=2.0687$, 拒绝原假设 $H_{0}$, 这意味着货款余额是影响不良贷款的一个显著因素。

在实际应用中, 可以直接利用 Excel 输出的参数估计表部分进行检验。表中除了给出检验的统计量, 还给出了用于检验的 $P$ 值 (P-value)。检验时可直接将 P-value 与给定的显著性水平 $\alpha$ 进行比较。若 $\mathrm{P}$-value $<\alpha$, 则拒绝假设 $H_{0}$; 若 $\mathrm{P}$-value $>\alpha$, 则不拒绝假设 $H_{0}$ 。在本例中, $\mathrm{P}$-value $=0.000000<\alpha=0.05$, 所以拒绝 $H_{0}$ 。

在一元线性回归中, 自变量只有一个, 上面介绍的 $F$ 检验和 $t$ 检验是等价的; 也就是说, 如果 $H_{0}: \beta_{1}=0$ 被 $t$ 检验拒绝, 它也将被 $F$ 检验拒绝。但在多元回归分析中, 这两种检验的意义是不同的。 $F$ 检验只是用来检验总体回归关系的显著性, 而 $t$ 检验则是检验各个回归系数的显著性。

表 11-4 中给出的 Excel 输出的回归分析结果, 有些已在上述内容中做了介绍, 有些则没有。为使读者能够完全理解 Excel 输出的结果, 这里给出上面未涉及的一些结果的计算公式 (见表 $11-5$ )。对这些内容的解释可参考相关的图书。

表 11-5 Excel 输出的部分结果的计算公式

\begin{center}
\begin{tabular}{ccc}
\hline
名称 & 计算公式 & 备注 \\
\hline
\begin{tabular}{c}
Adjusted R Square \\
(调整的 $\left.R^{2}\right)$ \\
\end{tabular} & $R_{a}^{2}=1-\left(1-R^{2}\right)\left(\frac{n-1}{n-k-1}\right)$ & $k$ 为自变量的个数 \\
\begin{tabular}{c}
Intercept (截距) 的 \\
抽样标准误差 \\
\end{tabular} & $s_{\beta_{0}}=s_{e} \sqrt{\frac{1}{n}+\frac{\bar{x}}{\sum_{i=1}^{n}\left(x_{i}-\bar{x}\right)^{2}}}$ & $s_{e}$ 为估计标准误差 \\
\begin{tabular}{c}
Intercept 的置信区间 \\
(Lower 95\% 和 Upper 95\%) \\
\end{tabular} & $\hat{\beta}_{0} \pm t_{\alpha / 2}(n-2) s_{e} \sqrt{\frac{1}{n}+\frac{\bar{x}}{\sum_{i=1}^{n}\left(x_{i}-\bar{x}\right)^{2}}}$ &  \\
\begin{tabular}{c}
斜率的置信区间 \\
(Lower 95\% 和 Upper 95\%) \\
\end{tabular} & $\hat{\beta}_{1} \pm t_{\alpha / 2}(n-2) \frac{s_{e}}{\sqrt{\sum_{i=1}^{n}\left(x_{i}-\bar{x}\right)^{2}}}$ &  \\
\hline
\end{tabular}
\end{center}

\subsection*{11.2.5 回归分析结果的评价}
前面讨论了建立一元线性回归模型的方法。现在的问题是: 已经建立的模型是否合适? 或者说, 这个拟合的模型有多好? 要回答这些问题, 可从以下几个方面人手。

(1) 所估计的回归系数 $\hat{\beta}_{1}$ 的符号是否与理论或事先预期的相一致。例如, 在不良贷款与贷款余额的回归中, 贷款余额越多, 不良贷款也可能越多; 也就是说, 回归系数 $\hat{\beta}_{1}$的值应该是正的, 在上面建立的回归方程中, 得到的回归系数 $\hat{\beta}_{1}=0.037895$, 为正值。

(2) 如果理论上认为 $y$ 与 $x$ 之间的关系不仅是正的, 而且是统计上显著的, 那么所建立的回归方程也应该如此。例如, 在不良贷款与贷款余额的回归中, 二者之间为正的线性关系,而且对回归系数 $\hat{\beta}_{1}$ 的 $t$ 检验结果表明,二者之间的线性关系是统计上显著的。

(3) 回归模型在多大程度上解释了因变量 $y$ 取值的差异? 可以用判定系数 $R^{2}$ 来回答这一问题。例如, 在不良贷款与贷款余额的回归中, 得到的 $R^{2}=71.16 \%$, 解释了不良贷款变差的 $2 / 3$ 以上, 说明拟合的效果还算不错。

(4) 考察关于误差项 $\varepsilon$ 的正态性假定是否成立。在对线性关系进行 $F$ 检验和对回归系数进行 $t$ 检验时, 都要求误差项 $\varepsilon$ 服从正态分布, 否则, 所用的检验程序将是无效的。检验 $\varepsilon$ 正态性的简单方法是画出残差的直方图或正态概率图。

\section*{11.3 利用回归方程进行预测}
回归模型经过各种检验并证实符合预定的要求后, 就可以利用它来预测因变量了。所谓预测 (predict) 是指通过自变量 $x$ 的取值来预测因变量 $y$ 的取值, 例如, 根据前面建立的不良贷款与贷款余额的估计方程, 给出一个贷款余额的数值, 就可以得到不良贷款的一个预测值。

\section*{11. 3 . 1 点估计}
利用估计的回归方程, 对于 $x$ 的一个特定值 $x_{0}$, 求出 $y$ 的一个估计值就是点估计。点估计可分为两种: 一是平均值的点估计; 二是个别值的点估计。(1)

平均值的点估计是利用估计的回归方程, 对于 $x$ 的一个特定值 $x_{0}$, 求出 $y$ 的平均值的一个估计值 $E\left(y_{0}\right)$ 。

例如, 在例 11.6 中, 得到的估计的回归方程为 $\hat{y}=-0.8295+0.037895 x$, 如果估计贷款余额为 100 亿元, 所有分行不良贷款的平均值就是平均值的点估计。根据估计的回归方程, 得

$$
\begin{aligned}
E\left(y_{0}\right) & =-0.8295+0.037895 \times 100 \\
& =2.96(\text { 亿元) }
\end{aligned}
$$

个别值的点估计是利用估计的回归方程, 对于 $x$ 的一个特定值 $x_{0}$, 求出 $y$ 的一个个别值的估计值 $\hat{y}_{0}$ 。

例如, 如果只想知道贷款余额为 72.8 亿元的那个分行 (这里是编号为 10 的那个分行)的不良贷款是多少, 则属于个别值的点估计。根据估计的回归方程, 得
\footnotetext{(1) 平均值的点估计实际上是对总体参数的估计, 而个别值的点估计则是对因变量的某个具体取值的估计。
}

$$
\begin{aligned}
\hat{y} & =-0.8295+0.037895 \times 72.8 \\
& =1.93 \text { (亿元) }
\end{aligned}
$$

这就是说,贷款余额为 72.8 亿元的那个分行的不良贷款估计值为 1.93 亿元。

在点估计条件下, 对于同一个 $x_{0}$, 平均值的点估计和个别值的点估计的结果是一样的,但在区间估计中则有所不同。

\subsection*{11.3.2区间估计}
利用估计的回归方程, 对于 $x$ 的一个特定值 $x_{0}$, 求出 $y$ 的一个估计值的区间就是区间估计。区间估计也有两种类型: 一是置信区间估计, 它是对 $x$ 的一个给定值 $x_{0}$, 求出 $y$的平均值的估计区间,这一区间称为置信区间(confidence interval); 二是预测区间估计,它是对 $x$ 的一个给定值 $x_{0}$, 求出 $y$ 的一个个别值的估计区间,这一区间称为预测区间 (prediction interval)。

\section*{1. $y$ 的平均值的置信区间估计}
置信区间估计 (confidence interval estimate) 是对 $x$ 的一个给定值 $x_{0}$, 求出 $y$ 的平均值的区间估计。

设 $x_{0}$ 为自变量 $x$ 的一个特定值或给定值; $E\left(y_{0}\right)$ 为给定 $x_{0}$ 时因变量 $y$ 的平均值或期望值。当 $x=x_{0}$ 时, $\hat{y}_{0}=\hat{\beta}_{0}+\hat{\beta}_{1} x_{0}$ 为 $E\left(y_{0}\right)$ 的估计值。

一般来说, 不能期望估计值 $\hat{y}_{0}$ 精确地等于 $E\left(y_{0}\right)$ 。因此, 要想用 $\hat{y}_{0}$ 推断 $E\left(y_{0}\right)$, 必须考虑根据估计的回归方程得到的 $\hat{y}_{0}$ 的方差。对于给定的 $x_{0}$, 统计学家给出了估计 $\hat{y}_{0}$ 的标准差的公式, 用 $s_{y_{0}}$ 表示 $\hat{y}_{0}$ 的标准差的估计量, 其计算公式为:


\begin{equation*}
s_{y_{0}}=s_{e} \sqrt{\frac{1}{n}+\frac{\left(x_{0}-\bar{x}\right)^{2}}{\sum_{i=1}^{n}\left(x_{i}-\bar{x}\right)^{2}}} \tag{11.22}
\end{equation*}


有了 $\hat{y}_{0}$ 的标准差之后, 对于给定的 $x_{0}, E\left(y_{0}\right)$ 在 $1-\alpha$ 置信水平下的置信区间可表示为:


\begin{equation*}
\hat{y}_{0} \pm t_{\alpha / 2} s_{e} \sqrt{\frac{1}{n}+\frac{\left(x_{0}-\bar{x}\right)^{2}}{\sum_{i=1}^{n}\left(x_{i}-\bar{x}\right)^{2}}} \tag{11.23}
\end{equation*}


\section*{例 11.14}
根据例 11.9 所求得的估计方程,取 $x_{0}=100$ ,建立不良货款的 $95 \%$ 的置信区间。

解 根据前面的计算结果, 已知 $n=25, s_{e}=1.9799$, 查表得 $t_{\alpha / 2}(n-2)=t_{0.025}(25-$ 2) $=2.0687$ 。

当贷款余额为 100 亿元时,不良贷款的点估计值为:

$$
E\left(y_{0}\right)=-0.8295+0.037895 \times 100=2.96(\text { 亿元 })
$$

根据式(11.23) 得 $E\left(y_{0}\right)$ 的置信区间为:

$2.96 \pm 2.0687 \times 1.9799 \times \sqrt{\frac{1}{25}+\frac{(100-120.268)^{2}}{154933.5744}}=2.96 \pm 0.8459$

即 $2.1141 \leqslant E\left(y_{0}\right) \leqslant 3.8059$ 。这就是说, 当贷款余额为 100 亿元时, 不良贷款的平均值在 2.1141 亿 3.8059 亿元之间。

当 $x_{0}=\bar{x}$ 时, $\hat{y}_{0}$ 的标准差的估计量最小, 此时有 $s_{y_{0}}=s_{e} \sqrt{1 / n}$ 。这就是说, 当 $x_{0}=\bar{x}$时, 估计是最准确的。 $x_{0}$ 偏离 $\bar{x}$ 越远, $y$ 的平均值的置信区间就变得越宽, 估计的效果就越不好。\\
2. $y$ 的个别值的预测区间估计

预测区间估计 (prediction interval estimate) 是对 $x$ 的一个给定值 $x_{0}$, 求出 $y$ 的一个个别值的区间估计。

假如不是估计贷款余额为 100 亿元时所有分行的平均不良贷款, 而只希望估计贷款余额为 72.8 亿元的那个分行的不良贷款的区间是多少,这个区间称为预测区间。

为求出预测区间, 首先必须知道用于估计的标准差。统计学家给出了 $y$ 的一个个别值 $y_{0}$ 的标准差的估计量, 用 $s_{\text {ind }}$ 表示, 其计算公式为:


\begin{equation*}
s_{\text {ind }}=s_{e} \sqrt{1+\frac{1}{n}+\frac{\left(x_{0}-\bar{x}\right)^{2}}{\sum_{i=1}^{n}\left(x_{i}-\bar{x}\right)^{2}}} \tag{11.24}
\end{equation*}


因此, 对于给定的 $x_{0}, y$ 的一个个别值 $y_{0}$ 在 $1-\alpha$ 置信水平下的预测区间可表示为:


\begin{equation*}
\hat{y}_{0} \pm t_{\alpha / 2} s_{e} \sqrt{1+\frac{1}{n}+\frac{\left(x_{0}-\bar{x}\right)^{2}}{\sum_{i=1}^{n}\left(x_{i}-\bar{x}\right)^{2}}} \tag{11.25}
\end{equation*}


与式 (11.23) 相比, 式 (11.25) 的根号内多了一个 1 。因此, 即使是对同一个 $x_{0}$,这两个区间的宽度也是不一样的, 预测区间要比置信区间宽一些。

\section*{例 11.15}
根据例 11.9 所求得的估计方程, 建立贷款余额为 72.8 亿元的那个分行不良货款的 $95 \%$ 的预测区间。

根据前面的计算结果, 已知 $n=25, s_{e}=1.9799$, 查表得 $t_{\alpha / 2}(n-2)=t_{0.025}(25-2)=$ 2. 0687 。

当贷款余额为 72.8 亿元时, 不良贷款的点估计值为:

$$
\begin{aligned}
\hat{y} & =-0.8295+0.037895 \times 72.8 \\
& =1.93(\text { 亿元) }
\end{aligned}
$$

不良货款的 $95 \%$ 的预测区间为:\\
$1.93 \pm 2.0687 \times 1.9799 \times \sqrt{1+\frac{1}{25}+\frac{(72.8-120.268)^{2}}{154933.5744}}=1.93 \pm 4.2066$即 $-2.2766 \leqslant \hat{y}_{0} \leqslant 6.1366$ 。 这就是说, 贷款余额为 72.8 亿元的那个分行, 其不良货款的 $95 \%$ 的预测区间在 -2.2766 亿 6.1366 亿元之间。

表 11-6 给出了 25 家分行不良贷款的置信区间和预测区间。

从表 11-6 可以看出, 两个区间的宽度不太一样, $y$ 的个别值的预测区间要宽一些。二者的差别表明,估计 $y$ 的平均值比预测 $y$ 的一个特定值或个别值更精确。(请读者想一想为什么?)当 $x_{0}=\bar{x}$ 时, 预测区间是最精确的。图 11-11 给出了置信区间和预测区间的示意图。

表 11-6 25 家分行不良贷款的置信区间和预测区间

\begin{center}
\begin{tabular}{|c|c|c|c|c|c|c|c|}
\hline
\multirow{2}{*}{}\begin{tabular}{l}
分行 \\
编号 \\
\end{tabular} & \multirow{2}{*}{}\begin{tabular}{c}
不良贷款 \\
$(y)$ \\
\end{tabular} & \multirow{2}{*}{}\begin{tabular}{c}
贷款余额 \\
$(x)$ \\
\end{tabular} & \multirow{2}{*}{预测 $Y$} & \multicolumn{2}{|c|}{置信区间} & \multicolumn{2}{|c|}{预测区间} \\
\hline
 &  &  &  & 置信下限 & 置信上限 & 预测下限 & 预测上限 \\
\hline
1 & 0.9 & 67.3 & 1.7208 & 0.7333 & 2.7083 & -2.4964 & 5.9380 \\
\hline
2 & 1.1 & 111.3 & 3. 3882 & 2.5636 & 4.2128 & -0.7939 & 7. 5702 \\
\hline
3 & 4.8 & 173 & 5.7263 & 4.7401 & 6.7124 & 1.5094 & 9. 9431 \\
\hline
4 & 3.2 & 80.8 & 2. 2324 & 1. 3159 & 3.1489 & -1.9687 & 6.4335 \\
\hline
5 & 7.8 & 199.7 & 6.7381 & 5.5742 & 7.9019 & 2.4761 & 11.0000 \\
\hline
6 & 2.7 & 16.2 & -0.2156 & -1.5737 & 1. 1424 & -4.5346 & 4. 1034 \\
\hline
7 & 1.6 & 107.4 & 3. 2404 & 2. 4102 & 4.0705 & -0.9428 & 7. 4235 \\
\hline
8 & 12.5 & 185.4 & 6.1962 & 5.1328 & 7.2595 & 1.9606 & 10.4317 \\
\hline
9 & 1.0 & 96.1 & 2.8122 & 1.9551 & 3.6692 & -1.3764 & 7. 0007 \\
\hline
10 & 2.6 & 72.8 & 1.9292 & 0.9725 & 2.8859 & -2.2809 & 6. 1393 \\
\hline
11 & 0.3 & 64.2 & 1.6033 & 0.5975 & 2.6092 & -2.6182 & 5.8248 \\
\hline
12 & 4.0 & 132.2 & 4. 1802 & 3.3515 & 5.0086 & -0.0027 & 8. 3630 \\
\hline
13 & 0.8 & 58.6 & 1. 3911 & 0.3504 & 2. 4319 & -2.8388 & 5.6211 \\
\hline
14 & 3.5 & 174.6 & 5.7869 & 4.7914 & 6.7824 & 1.5678 & 10.0059 \\
\hline
15 & 10.2 & 263.5 & 9. 1557 & 7. 4547 & 10.8567 & 4.7170 & 13.5945 \\
\hline
16 & 3.0 & 79.3 & 2. 1755 & 1. 2519 & 3.0991 & -2.0271 & 6. 3782 \\
\hline
17 & 0.2 & 14.8 & -0.2687 & -1.6384 & 1. 1010 & -4.5913 & 4. 0540 \\
\hline
18 & 0.4 & 73.5 & 1.9557 & 1.0028 & 2.9087 & -2.2535 & 6. 1650 \\
\hline
19 & 1.0 & 24.7 & 0.1065 & -1.1821 & 1. 3951 & -4.1912 & 4. 4041 \\
\hline
20 & 6.8 & 139.4 & 4.4530 & 3. 6099 & 5.2962 & 0.2673 & 8. 6387 \\
\hline
21 & 11.6 & 368.2 & 13. 1233 & 10.4160 & 15.8306 & 8.2102 & 18.036 4 \\
\hline
22 & 1.6 & 95.7 & 2.7970 & 1. 9387 & 3.6553 & -1.3918 & 6.9858 \\
\hline
23 & 1.2 & 109.6 & 3. 3237 & 2. 4969 & 4. 1505 & -0.8587 & 7. 5062 \\
\hline
24 & 7.2 & 196.2 & 6.6054 & 5.4671 & 7. 7437 & 2.3504 & 10.8604 \\
\hline
25 & 3.2 & 102.2 & 3.0433 & 2. 2027 & 3. 8839 & -1.1419 & 7.2285 \\
\hline
\end{tabular}
\end{center}

\begin{center}
\end{center}

图 11-11 置信区间和预测区间示意图

\section*{11.4 残差分析}
在回归模型 $y=\beta_{0}+\beta_{1} x+\varepsilon$ 中, 假定 $\varepsilon$ 是期望值为 0 、方差相等且服从正态分布的一个随机变量。如果关于 $\varepsilon$ 的假定不成立, 那么, 所做的检验以及估计和预测也许站不住脚。确定有关 $\varepsilon$ 的假定是否成立的方法之一就是进行残差分析 (residual analysis)。

\subsection*{11.4. 1 残差与残差图}
残差 (residual) 是因变量的观测值 $y_{i}$ 与根据估计的回归方程求出的预测值 $\hat{y}_{i}$ 之差, 用 $e$表示。它反映了用估计的回归方程去预测 $y_{i}$ 而引起的误差。第 $i$ 个观测值的残差可以写为:


\begin{equation*}
e_{i}=y_{i}-\hat{y}_{i} \tag{11.26}
\end{equation*}


根据例 11.9 求得的估计方程, 由 Excel 输出的预测值及残差如表 11-7 所示。

表 11-7 Excel 输出的预测值、残差和标准化残差(1)

\begin{center}
\begin{tabular}{ccccccc}
\hline
\begin{tabular}{c}
分行 \\
编号 \\
\end{tabular} & \begin{tabular}{c}
不良贷款 \\
$y$ \\
\end{tabular} & \begin{tabular}{c}
贷款余额 \\
$x$ \\
\end{tabular} & \begin{tabular}{c}
预测 \\
$\hat{y}_{i}$ \\
\end{tabular} & \begin{tabular}{c}
残差 \\
$\boldsymbol{e}_{i}$ \\
\end{tabular} & \begin{tabular}{c}
标准化残差 \\
$\boldsymbol{z}_{\boldsymbol{e}}$ \\
\end{tabular} & \begin{tabular}{c}
杜杆率 \\
$h_{i}$ \\
\end{tabular} \\
\hline
1 & 0.9 & 67.3 & 1.7208 & -0.8208 & -0.4146 & 0.0581 \\
2 & 1.1 & 111.3 & 3.3882 & -2.2882 & -1.1557 & 0.0405 \\
3 & 4.8 & 173 & 5.7263 & -0.9263 & -0.4678 & 0.0579 \\
\hline
\end{tabular}
\end{center}

(1) 本表给出的标准化残差是按式 (11.27) 计算的。而 Excel 给出的标准化残差的计算公式为: $z_{e_{i}}=$ $\frac{y_{i}-\hat{y}_{i}}{s_{e} \sqrt{1-\left(\frac{1}{n}+\frac{\left(x_{i}-\bar{x}\right)^{2}}{\sum\left(x_{i}-\bar{x}\right)^{2}}\right)}}$, 这实际上是学生化删除残差 (studentized deleted residuals)。对这一问题的进一步了解请参阅有关的统计书籍。

\begin{center}
\begin{tabular}{|c|c|c|c|c|c|c|}
\hline
\begin{tabular}{l}
分行 \\
编号 \\
\end{tabular} & \begin{tabular}{c}
不良贷款 \\
$y$ \\
$y$ \\
\end{tabular} & \begin{tabular}{c}
贷款余额 \\
$x$ \\
\end{tabular} & \begin{tabular}{c}
预测 \\
$\hat{y}_{i}$ \\
\end{tabular} & \begin{tabular}{c}
残差 \\
$e_{i}$ \\
\end{tabular} & \begin{tabular}{c}
标准化残差 \\
$z_{e}$ \\
\end{tabular} & \begin{tabular}{c}
杜杆率 \\
$h_{i}$ \\
\end{tabular} \\
\hline
4 & 3.2 & 80.8 & 2. 2324 & 0.9676 & 0.4887 & 0.0501 \\
\hline
5 & 7. 8 & 199.7 & 6.7381 & 1. 0619 & 0.5364 & 0.0807 \\
\hline
6 & 2.7 & 16.2 & -0.2156 & 2.9156 & 1.4726 & 0.1099 \\
\hline
7 & 1.6 & 107.4 & 3.2404 & -1.6404 & -0.8285 & 0.0411 \\
\hline
8 & 12.5 & 185.4 & 6.1962 & 6.3038 & 3. 1838 & 0.0674 \\
\hline
9 & 1.0 & 96.1 & 2.8122 & -1.8122 & -0.9153 & 0.0438 \\
\hline
10 & 2.6 & 72.8 & 1.9292 & 0.6708 & 0.3388 & 0.0545 \\
\hline
11 & 0.3 & 64.2 & 1.6033 & -1.3033 & -0.6583 & 0.0603 \\
\hline
12 & 4.0 & 132.2 & 4. 1802 & -0.1802 & -0.0910 & 0.0409 \\
\hline
13 & 0.8 & 58.6 & 1. 3911 & -0.5911 & -0.2985 & 0.0645 \\
\hline
14 & 3.5 & 174.6 & 5.7869 & -2.2869 & -1.1550 & 0.0591 \\
\hline
15 & 10.2 & 263.5 & 9. 1557 & 1. 0443 & 0.5274 & 0.1724 \\
\hline
16 & 3.0 & 79. 3 & 2.1755 & 0.8245 & 0.4164 & 0.0508 \\
\hline
17 & 0.2 & 14.8 & -0.2687 & 0.4687 & 0.2367 & 0.1118 \\
\hline
18 & 0.4 & 73.5 & 1. 9557 & -1.5557 & -0.7857 & 0.0541 \\
\hline
19 & 1.0 & 24.7 & 0.1065 & 0.8935 & 0.4513 & 0.0989 \\
\hline
20 & 6.8 & 139.4 & 4.4530 & 2. 3470 & 1. 1854 & 0.0424 \\
\hline
21 & 11.6 & 368.2 & 13. 1233 & -1.5233 & -0.7694 & 0.4368 \\
\hline
22 & 1. 6 & 95.7 & 2.7970 & -1.1970 & -0.6046 & 0.0439 \\
\hline
23 & 1. 2 & 109.6 & 3. 3237 & -2.1237 & -1.0726 & 0.0407 \\
\hline
24 & 7. 2 & 196.2 & 6.6054 & 0.5946 & 0.3003 & 0.0772 \\
\hline
25 & 3. 2 & 102.2 & 3.0433 & 0.1567 & 0.0791 & 0.0421 \\
\hline
\end{tabular}
\end{center}

可以通过对残差图的分析来判断对误差项 $\varepsilon$ 的假定是否成立。常用的残差图有关于 $x$ 的残差图、关于 $\hat{y}$ 的残差图、标准化残差图等。关于 $x$ 的残差图是用横轴表示自变量 $x$ 的值, 用纵轴表示对应的残差 $e=y-\hat{y}$, 每个 $x$ 的值与对应的残差用图上的一个点来表示。图 11-12 就是根据表 11-7 中的有关结果绘制的不良贷款与贷款余额回归的残差图。

为分析残差图, 首先考察一下残差图的形态及其反映的信息。图 11-13 给出

\begin{center}
\end{center}

图 11-12 不良贷款与贷款余额回归的残差图\\
了几种不同形态的残差图。

\begin{center}
\end{center}

(a) 满意的模型

\begin{center}
\end{center}

(b) 非常数方差

\begin{center}
\end{center}

(c) 模型不合理

图 11-13 不同形态的残差图

若对所有的 $x$ 值, $\varepsilon$ 的方差都相同, 而且假定描述变量 $x$ 和 $y$ 之间关系的回归模型是合理的, 那么残差图中的所有点都应落在一条水平带中间, 如图 11-13 (a) 所示。但如果对所有的值, $\varepsilon$ 的方差是不同的, 例如, 对于较大的 $x$ 值, 相应的残差也较大, 如图 11-13 (b) 所示, 这就意味着违背了 $\varepsilon$ 方差相等的假设。如果残差图如图 11-13 (c)所示的那样, 则表明所选择的回归模型不合理, 这时应考虑曲线回归或多元回归模型。

考察图 11-12 不良贷款与贷款余额回归的残差图, 可以看出, 各残差 (有一个点除外)基本上位于一条水平带中间, 这表明关于不良贷款与贷款余额回归的线性假定以及对误差项 $\varepsilon$ 的假定是成立的。

\subsection*{11.4. 2 标准化残差}
对 $\varepsilon$ 正态性假定的检验, 也可以通过对标准化残差的分析来完成。标准化残差 (standardized residual) 是残差除以它的标准差后得到的数值, 也称为 Pearson 残差或半学生化残差 (semi-studentized residuals), 用 $z_{e}$ 表示。第 $i$ 个观测值的标准化残差可以表示为:

\begin{center}
\end{center}

图 11-14 不良贷款与贷款余额回归的标准化残差图


\begin{equation*}
z_{e_{i}}=\frac{e_{i}}{s_{e}}=\frac{y_{i}-\hat{y}_{i}}{s_{e}} \tag{11.27}
\end{equation*}


式中, $s_{e}$ 是残差的标准差的估计。

如果误差项 $\varepsilon$ 服从正态分布这一假定成立, 那么标准化残差的分布也应服从正态分布。因此, 在标准化残差图中, 大约有 $95 \%$ 的标准化残差在 $-2 \sim 2$ 之间。表 $11-7$ 给出了根据式 (11.27)计算的标准化残差, 将其绘制成图形如图 11-14 所示。

从图 11-14 可以看出, 除了用箭头所标识的那个点外, 所有的标准化\\
残差都在 $-2 \sim 2$ 之间, 这表明误差项 $\varepsilon$ 服从正态分布的假定成立。

\section*{思考与练习}
\section*{思考题}
11.1 解释相关关系的含义, 并说明相关关系的特点。

11.2 相关分析主要解决哪些问题?

11.3 相关分析中有哪些基本假定?

11.4 简述相关系数的性质。

11.5 为什么要对相关系数进行显著性检验?

11.6 简述相关系数显著性检验的步骤。

11.7 解释回归模型、回归方程、估计的回归方程的含义。

11.8 一元线性回归模型中有哪些基本假定?

11.9 简述参数最小二乘估计的基本原理。

11.10 解释总平方和、回归平方和、残差平方和的含义,并说明它们之间的关系。

11.11 简述判定系数的含义和作用。

11.12 在回归分析中, $F$ 检验和 $t$ 检验各有什么作用?

11.13 简要说明残差分析在回归分析中的作用。

\section*{练习题}
11.1 从某一行业中随机抽取 12 家企业, 所得产量与生产费用的数据如下:

\begin{center}
\begin{tabular}{ccc||ccc}
\hline
企业编号 & 产量 (台) & 生产费用 (万元) & 企业编号 & 产量 (台) & 生产费用 (万元) \\
\hline
1 & 40 & 130 & 7 & 84 & 165 \\
2 & 42 & 150 & 8 & 100 & 170 \\
3 & 50 & 155 & 9 & 116 & 167 \\
4 & 55 & 140 & 10 & 125 & 180 \\
5 & 65 & 150 & 11 & 130 & 175 \\
6 & 78 & 154 & 12 & 140 & 185 \\
\hline
\end{tabular}
\end{center}

(1)绘制产量与生产费用的散点图,判断二者之间的关系形态。

(2)计算产量与生产费用之间的线性相关系数。

(3)对相关系数的显著性进行检验 $(\alpha=0.05)$, 并说明二者之间的关系强度。

11.2 随机抽取 10 家航空公司, 对其最近一年的航班正点率和顾客投诉次数进行调查, 所得数据如下:

\begin{center}
\begin{tabular}{ccc}
\hline
航空公司编号 & 航班正点率 $(\%)$ & 21 \\
1 & 81.8 & 58 \\
2 & 76.6 & 85 \\
3 & 76.6 & 6 \\
\hline
\end{tabular}
\end{center}

\section*{续表}
\begin{center}
\begin{tabular}{ccc}
\hline
航空公司编号 & 航班正点率 $(\%)$ & 顾客投诉次数 \\
\hline
4 & 75.7 & 68 \\
5 & 73.8 & 74 \\
6 & 72.2 & 93 \\
7 & 71.2 & 72 \\
8 & 70.8 & 122 \\
9 & 91.4 & 18 \\
10 & 68.5 & 125 \\
\hline
\end{tabular}
\end{center}

(1) 绘制散点图,说明二者之间的关系形态。

(2)用航班正点率作自变量, 顾客投诉次数作因变量, 建立估计的回归方程, 并解释回归系数的意义。

(3) 检验回归系数的显著性 $(\alpha=0.05)$ 。

(4) 如果航班正点率为 $80 \%$, 估计顾客投诉次数。

(5) 求航班正点率为 $80 \%$ 时,顾客投诉次数的 $95 \%$ 的置信区间和预测区间。

11.3 下面是 20 个城市写字楼出租率和每平方米月租金的数据:

\begin{center}
\begin{tabular}{|c|c|c|c|}
\hline
 & 地区编号 & 出租率 $(\%)$ & 每平方米月租金(元) \\
\hline
\multirow{20}{*}{.} & 1 & 70.6 & 99 \\
\hline
 & 2 & 69.8 & 74 \\
\hline
 & 3 & 73.4 & 83 \\
\hline
 & 4 & 67.1 & 70 \\
\hline
 & 5 & 70.1 & 84 \\
\hline
 & 6 & 68.7 & 65 \\
\hline
 & 7 & 63.4 & 67 \\
\hline
 & 8 & 73.5 & 105 \\
\hline
 & 9 & 71.4 & 95 \\
\hline
 & 10 & 80.7 & 107 \\
\hline
 & 11 & 71.2 & 86 \\
\hline
 & 12 & 62.0 & 66 \\
\hline
 & 13 & 78.7 & 106 \\
\hline
 & 14 & 69.5 & 70 \\
\hline
 & 15 & 68.7 & 81 \\
\hline
 & 16 & 69.5 & 75 \\
\hline
 & 17 & 67.7 & 82 \\
\hline
 & 18 & 68.4 & 94 \\
\hline
 & 19 & 72.0 & 92 \\
\hline
 & 20 & 67.9 & 76 \\
\hline
\end{tabular}
\end{center}

设月租金为自变量, 出租率为因变量, 用 Excel 进行回归, 并对结果进行解释和分析。

11.4 某汽车生产商欲了解广告费用 $(x)$ 对销售量 $(y)$ 的影响, 收集了过去 12 年的有关数据。通\\
过计算得到下面的有关结果:

方差分析表

\begin{center}
\begin{tabular}{cccccc}
\hline
变差来源 & df & SS & MS & F & Significance F \\
\hline
回归 &  &  &  &  & $2.17 \mathrm{E}-09$ \\
残差 &  & 40158.07 & - & - &  \\
总计 & 11 & 1642866.67 & - & - & - \\
\hline
\end{tabular}
\end{center}

\section*{参数估计表}
\begin{center}
\begin{tabular}{lcccc}
\hline
 & Coefficients & 标准误差 & t Stat & P-value \\
\hline
Intercept & 363.6891 & 62.45529 & 5.823191 & 0.000168 \\
X Variable 1 & 1.420211 & 0.071091 & 19.97749 & $2.17 \mathrm{E}-09$ \\
\hline
\end{tabular}
\end{center}

(1) 完成上面的方差分析表。

(2) 汽车销售量的变差中有多少是由广告费用的变动引起的?

(3) 销售量与广告费用之间的相关系数是多少?

(4)写出估计的回归方程并解释回归系数的实际意义。

(5) 检验线性关系的显著性 $(\alpha=0.05)$ 。

11.5 根据下面的数据建立回归方程, 计算残差、判定系数 $R^{2}$ 、估计标准误差 $s_{e}$, 并分析回归方程的拟合程度。

\begin{center}
\begin{tabular}{cccccc}
\hline
$x$ & 15 & 8 & 19 & 12 & 5 \\
\hline
$y$ & 47 & 36 & 56 & 44 & 21 \\
\hline
\end{tabular}
\end{center}

11.6 随机抽取 7 家超市,得到其广告费支出和销售额数据如下:

\begin{center}
\begin{tabular}{ccc}
\hline
超市 & 广告费支出 (万元) & 销售额 (万元) \\
\hline
A & 1 & 19 \\
B & 2 & 32 \\
C & 4 & 44 \\
D & 6 & 40 \\
E & 10 & 52 \\
F & 14 & 53 \\
G & 20 & 54 \\
\hline
\end{tabular}
\end{center}

(1) 用广告费支出作自变量 $x$, 销售额作因变量 $y$, 建立估计的回归方程。

(2) 检验广告费支出与销售额之间的线性关系是否显著 $(\alpha=0.05)$ 。

(3)绘制关于 $x$ 的残差图,关于误差项 $\varepsilon$ 的假定是否成立?

(4)你是选用这个模型, 还是另找一个更好的模型?\\

\section*{中国人帕金森病患病率与国际水平无差异}
由北京协和医院临床流行病教研室主任、神经内科教授张振蓖等组成的研究小组,对沿用 20 多年的中国人帕金森病患病率作出重大修正。新报告数据显示, 65 岁以上的中国人帕金森病患病率男性为 $1.7 \%$, 女性为 $1.6 \%$, 根据国际通行做法换算后, 这一患病率为 $2.1 \%$, 这个数据与国际水平相比并无差异。

该研究由张振警教授牵头, 北京、上海、西安三地神经内科专家参与, 历时 6 年,对上述三大城市的 137 个城乡居民区经抽样获取的 29454 人进行入户调查和定期随访,使用国际最新的分层多级别整群抽样误差的加权调整, 最大限度地避免了抽样误差。

研究者分析了帕金森病发病的危险因素, 发现年龄是最重要的因素。此外, 帕金森病的发病与职业高度相关, 丧偶、高文化程度、强脑力劳动等导致的精神、情绪紧张都会增加该病的发生, 而饮绿茶和咖啡可预防帕金森病。研究未发现其与基因、遗传有必然关系。专家同时指出, 沿用 20 余年的中国人帕金森病低患病率不是种族差异, 而是由流行病学调查方法和统计学方法造成的。

资料来源:段文利. 中国人帕金森患病率与国际无差异. 中华医学信息导报,2005 (13).\\
本章将讨论涉及两个及两个以上自变量的回归问题, 即多元回归, 主要介绍多元线性回归。讨论的重点放在多元回归的计算机输出结果及其应用上。

\section*{12.1 多元线性回归模型}
在许多实际问题中, 影响因变量的因素往往有多个, 这种一个因变量与多个自变量的回归问题就是多元回归, 当因变量与各自变量之间为线性关系时, 称为多元线性回归。多元线性回归分析的原理同一元线性回归基本相同, 但计算上要复杂得多, 需借助计算机来完成。

\subsection*{12.1. 1 多元回归模型与回归方程}
设因变量为 $y, k$ 个自变量分别为 $x_{1}, x_{2}, \cdots, x_{k}$, 描述因变量 $y$ 如何依赖于自变量 $x_{1}, x_{2}, \cdots, x_{k}$ 和误差项 $\varepsilon$ 的方程称为多元回归模型 (multiple regression model)。其一般形式可表示为:


\begin{equation*}
y=\beta_{0}+\beta_{1} x_{1}+\beta_{2} x_{2}+\cdots+\beta_{k} x_{k}+\varepsilon \tag{12.1}
\end{equation*}


式中, $\beta_{0}, \beta_{1}, \beta_{2}, \cdots, \beta_{k}$ 是模型的参数; $\varepsilon$ 为误差项。

式 (12.1) 表明: $y$ 是 $x_{1}, x_{2}, \cdots, x_{k}$ 的线性函数 $\left(\beta_{0}+\beta_{1} x_{1}+\beta_{2} x_{2}+\cdots+\beta_{k} x_{k}\right.$ 部分)加上误差项 $\varepsilon$ 。误差项反映了除 $x_{1}, x_{2}, \cdots, x_{k}$ 与 $y$ 的线性关系之外的随机因素对 $y$ 的影响, 是不能由 $x_{1}, x_{2}, \cdots, x_{k}$ 与 $y$ 之间的线性关系所解释的变异性。

与一元线性回归类似, 在多元线性回归模型中, 对误差项 $\varepsilon$ 同样有三个基本假定:

(1) 误差项 $\varepsilon$ 是一个期望值为 0 的随机变量, 即 $E(\varepsilon)=0$ 。这意味着对于给定 $x_{1}$, $x_{2}, \cdots, x_{k}$ 的值, $y$ 的期望值为 $E(y)=\beta_{0}+\beta_{1} x_{1}+\beta_{2} x_{2}+\cdots+\beta_{k} x_{k}$ 。

(2) 对于自变量 $x_{1}, x_{2}, \cdots, x_{k}$ 的所有值, $\varepsilon$ 的方差 $\sigma^{2}$ 都相同。

(3) 误差项 $\varepsilon$ 是一个服从正态分布的随机变量, 且相互独立, 即 $\varepsilon \sim N\left(0, \sigma^{2}\right)$ 。独立性意味着自变量 $x_{1}, x_{2}, \cdots, x_{k}$ 的一组特定值所对应的 $\varepsilon$ 与 $x_{1}, x_{2}, \cdots, x_{k}$ 任意一组其他值所对应的 $\varepsilon$ 不相关。正态性意味着对于给定的 $x_{1}, x_{2}, \cdots, x_{k}$ 的值, 因变量 $y$ 是一个服从正态分布的随机变量。

根据回归模型的假定, 有


\begin{equation*}
E(y)=\beta_{0}+\beta_{1} x_{1}+\beta_{2} x_{2}+\cdots+\beta_{k} x_{k} \tag{12.2}
\end{equation*}


式 (12.2) 称为多元回归方程 (multiple regression equation), 它描述了因变量 $y$ 的期望值与自变量 $x_{1}, x_{2}, \cdots, x_{k}$ 之间的关系。

一元回归方程在二维空间中是一条直线, 可在直角坐标系中将其画出来。但多元回归就很难做到这一点。为了对式 (12.2) 的回归方程有更全面的了解, 可考虑含有两个自变量的多元回归方程, 其形式为:

$$
E(y)=\beta_{0}+\beta_{1} x_{1}+\beta_{2} x_{2}
$$

在三维空间中可以将这个方程画出来, 二元回归方程在三维空间中是一个平面,\\
如图 12-1 所示。

\begin{center}
\end{center}

图 12-1 二元回归方程的图示

\subsection*{12.1.2 估计的多元回归方程}
回归方程中的参数 $\beta_{0}, \beta_{1}, \beta_{2}, \cdots, \beta_{k}$ 是未知的, 需要利用样本数据去估计它们。当用样本统计量 $\hat{\beta}_{0}, \hat{\beta}_{1}, \hat{\beta}_{2}, \cdots, \hat{\beta}_{k}$ 去估计回归方程中的未知参数 $\beta_{0}, \beta_{1}, \beta_{2}, \cdots, \beta_{k}$ 时,就得到了估计的多元回归方程 (estimated multiple regression equation), 其一般形式为:


\begin{equation*}
\hat{y}=\hat{\beta}_{0}+\hat{\beta}_{1} x_{1}+\hat{\beta}_{2} x_{2}+\cdots+\hat{\beta}_{k} x_{k} \tag{12.3}
\end{equation*}


式中, $\hat{\beta}_{0}, \hat{\beta}_{1}, \hat{\beta}_{2}, \cdots, \hat{\beta}_{k}$ 是参数 $\beta_{0}, \beta_{1}, \beta_{2}, \cdots, \beta_{k}$ 的估计值; $\hat{y}$ 是因变量 $y$ 的估计值。其中的 $\hat{\beta}_{1}, \hat{\beta}_{2}, \cdots, \hat{\beta}_{k}$ 称为偏回归系数。 $\hat{\beta}_{1}$ 表示当 $x_{2}, x_{3}, \cdots, x_{k}$ 不变时, $x_{1}$ 每变动一个单位因变量 $y$ 的平均变动量; $\hat{\beta}_{2}$ 表示当 $x_{1}, x_{3}, \cdots, x_{k}$ 不变时, $x_{2}$ 每变动一个单位因变量 $y$ 的平均变动量; 其余偏回归系数的含义类似。

\subsection*{12.1.3 参数的最小二乘估计}
回归方程中的 $\hat{\beta}_{0}, \hat{\beta}_{1}, \hat{\beta}_{2}, \cdots, \hat{\beta}_{k}$ 仍然是根据最小二乘法求得的, 也就是使残差平方和


\begin{equation*}
Q=\sum\left(y_{i}-\hat{y}_{i}\right)^{2}=\sum\left(y_{i}-\hat{\beta}_{0}-\hat{\beta}_{1} x_{1}-\cdots-\hat{\beta}_{k} x_{k}\right)^{2} \tag{12.4}
\end{equation*}


最小。由此可以得到求解 $\hat{\beta}_{0}, \hat{\beta}_{1}, \hat{\beta}_{2}, \cdots, \hat{\beta}_{k}$ 的标准方程组:

\[
\left\{\begin{array}{l}
\left.\frac{\partial Q}{\partial \beta_{0}}\right|_{\beta_{0}=\hat{\beta}_{0}}=0  \tag{12.5}\\
\left.\frac{\partial Q}{\partial \beta_{i}}\right|_{\beta_{i}=\hat{\beta}_{i}}=0
\end{array} \quad i=1,2, \cdots, k\right.
\]

求解上述方程组需要借助计算机, 可直接由 Excel 给出回归结果。

\section*{例 12.1}
继续沿用第 11 章中的例 11.6 。一家大型商业银行在多个地区设有分行, 其业务主要是进行基础设施建设、国家重点项目建设、固定资产投资等项目的贷款。近年来,该\\
银行的贷款额平稳增长, 但不良贷款额也有较大比例的提高, 这给银行业务的发展带来较大压力。为弄清楚不良贷款形成的原因, 管理者希望利用银行业务的有关数据做些定量分析, 以便找出控制不良货款的办法。表 12-1 就是该银行所属的 25 家分行的有关业务数据。

表 12-1 某商业银行的主要业务数据

\begin{center}
\begin{tabular}{|c|c|c|c|c|c|}
\hline
分行编号 & \begin{tabular}{c}
不良贷款 \\
(亿元) \\
\end{tabular} & \begin{tabular}{c}
贷款余额 \\
(亿元) \\
\end{tabular} & \begin{tabular}{c}
累计应收贷款 \\
(亿元) \\
\end{tabular} & \begin{tabular}{c}
贷款项目个数 \\
(个) \\
\end{tabular} & \begin{tabular}{c}
固定资产投资额 \\
(亿元) \\
\end{tabular} \\
\hline
1 & 0.9 & 67.3 & 6.8 & 5 & 51.9 \\
\hline
2 & 1.1 & 111.3 & 19.8 & 16 & 90.9 \\
\hline
3 & 4. 8 & 173.0 & 7. 7 & 17 & 73. 7 \\
\hline
4 & 3.2 & 80.8 & 7. 2 & 10 & 14.5 \\
\hline
5 & 7. 8 & 199.7 & 16.5 & 19 & 63.2 \\
\hline
6 & 2. 7 & 16. 2 & 2.2 & 1 & 2.2 \\
\hline
7 & 1.6 & 107.4 & 10.7 & 17 & 20.2 \\
\hline
8 & 12.5 & 185.4 & 27.1 & 18 & 43.8 \\
\hline
9 & 1.0 & 96.1 & 1. 7 & 10 & 55.9 \\
\hline
10 & 2.6 & 72.8 & 9.1 & 14 & 64.3 \\
\hline
11 & 0.3 & 64.2 & 2.1 & 11 & 42.7 \\
\hline
12 & 4. 0 & 132.2 & 11.2 & 23 & 76. 7 \\
\hline
13 & 0.8 & 58.6 & 6.0 & 14 & 22.8 \\
\hline
14 & 3.5 & 174.6 & 12.7 & 26 & 117.1 \\
\hline
15 & 10.2 & 263.5 & 15.6 & 34 & 146.7 \\
\hline
16 & 3.0 & 79.3 & 8.9 & 15 & 29.9 \\
\hline
17 & 0.2 & 14.8 & 0.6 & 2 & 42.1 \\
\hline
18 & 0.4 & 73.5 & 5.9 & 11 & 25.3 \\
\hline
19 & 1.0 & 24.7 & 5.0 & 4 & 13.4 \\
\hline
20 & 6.8 & 139.4 & 7.2 & 28 & 64.3 \\
\hline
21 & 11.6 & 368.2 & 16.8 & 32 & 163.9 \\
\hline
22 & 1. 6 & 95.7 & 3. 8 & 10 & 44.5 \\
\hline
23 & 1.2 & 109.6 & 10.3 & 14 & 67.9 \\
\hline
24 & 7.2 & 196.2 & 15.8 & 16 & 39.7 \\
\hline
25 & 3.2 & 102.2 & 12.0 & 10 & 97.1 \\
\hline
\end{tabular}
\end{center}

试建立不良贷款 $(y)$ 与货款余额 $\left(x_{1}\right)$ 、累计应收贷款 $\left(x_{2}\right)$ 、货款项目个数 $\left(x_{3}\right)$\\
和固定资产投资额 $\left(x_{4}\right)$ 的线性回归方程, 并解释各回归系数的含义。

由 Excel 输出的多元回归分析结果如表 12-2 所示。

根据表 $12-2$ 的结果, 得到不良贷款与贷款余额、累计应收贷款、贷款项目个数和固定资产投资额的多元线性回归方程:

$$
\hat{y}=-1.021640+0.040039 x_{1}+0.148034 x_{2}+0.014529 x_{3}-0.029193 x_{4}
$$

各回归系数的实际意义为:

$\hat{\beta}_{1}=0.040039$, 表示在累计应收贷款、贷款项目个数和固定资产投资额不变的条件下,贷款余额每增加 1 亿元,不良贷款平均增加 0.040039 亿元。

表 12-2 Excel 输出的多元回归分析结果

\section*{SUMMARY OUTPUT}
回归统计

\begin{center}
\begin{tabular}{cc}
\hline
Multiple R & 0.893087 \\
R Square & 0.797604 \\
Adjusted R Square & 0.757125 \\
标准误差 & 1.778752 \\
观测值 & 25 \\
\hline
\end{tabular}
\end{center}

方差分析

\begin{center}
\begin{tabular}{ccccccc}
\hline
 & df & SS & MS & F & Significance F &  \\
\hline
回归分析 & 4 & 249.3712 & 62.3428 & 19.7040 & $1.03539 E-06$ &  \\
残差 & 20 & 63.2792 & 3.163960 &  &  &  \\
总计 & 24 & 312.6504 &  &  &  &  \\
\hline
 & Coefficients & 标准误差 & t Stat & P-value & Lower 95\% & Upper 95\% \\
\hline
Intercept & -1.021640 & 0.782372 & -1.305823 & 0.206434 & -2.653639 & 0.610360 \\
X Variable 1 & 0.040039 & 0.010434 & 3.837495 & 0.001028 & 0.018275 & 0.061804 \\
X Variable 2 & 0.148034 & 0.078794 & 1.878738 & 0.074935 & -0.016328 & 0.312398 \\
X Variable 3 & 0.014529 & 0.083033 & 0.174983 & 0.862853 & -0.158675 & 0.167733 \\
X Variable 4 & -0.029193 & 0.015073 & -1.936769 & 0.067030 & -0.060635 & 0.002249 \\
\hline
\end{tabular}
\end{center}

$\hat{\beta}_{2}=0.148034$, 表示在货款余额、贷款项目个数和固定资产投资额不变的条件下, 累计应收贷款每增加 1 亿元,不良货款平均增加 0.148034 亿元。

$\hat{\beta}_{3}=0.014529$, 表示在贷款余额、累计应收贷款和固定资产投资额不变的条件下,贷款项目个数每增加 1 个, 不良贷款平均增加 0.014529 亿元。

$\hat{\beta}_{4}=-0.029193$, 表示在贷款余额、累计应收贷款和贷款项目个数不变的条件下, 固定资产投资额每增加 1 亿元,不良贷款平均减少 0.029193 亿元。

\section*{12.2 回归方程的拟合优度}
\section*{12. 2.1 多重判定系数}
与一元回归类似, 对多元线性回归方程, 需要用多重判定系数来评价其拟合程度。

在一元回归中曾介绍过因变量离差平方和的分解方法, 对多元回归中因变量离差平方和的分解也一样, 同样有


\begin{equation*}
S S T=S S R+S S E \tag{12.6}
\end{equation*}


式中, $S S T=\sum\left(y_{i}-\bar{y}\right)^{2}$, 为总平方和; $S S R=\sum\left(\hat{y}_{i}-\bar{y}\right)^{2}$, 为回归平方和; $S S E=$ $\sum\left(y_{i}-\hat{y}_{i}\right)^{2}$, 为残差平方和。

由于这三个平方和的计算非常麻烦, 所以可直接利用 Excel 的输出结果。由表 12-2 给出的方差分析部分可知, $S S T=312.650400 ; S S R=249.371206 ; S S E=63.279$ 194。

有了这些平方和, 可以将多重判定系数定义如下:


\begin{equation*}
R^{2}=\frac{S S R}{S S T}=1-\frac{S S E}{S S T} \tag{12.7}
\end{equation*}


多重判定系数 (multiple coefficient of determination) 是多元回归中的回归平方和占总平方和的比例, 它是度量多元回归方程拟合程度的一个统计量, 反映了因变量 $y$ 的变差中能被估计的回归方程所解释的比例。

对于多重判定系数还有一点需要注意: 自变量个数的增加将影响因变量的变差中被估计的回归方程所解释的比例。当增加自变量时, 会使预测误差变得较小, 从而减少残差平方和 $S S E$ 。由于回归平方和 $S S R=S S T-S S E$, 当 $S S E$ 变小时, $S S R$ 就会变大, 从而使 $R^{2}$ 变大。如果模型中增加一个自变量, 即使这个自变量在统计上并不显著, $R^{2}$ 也会变大。因此, 为避免增加自变量而高估 $R^{2}$, 统计学家提出用样本量 $n$ 和自变量的个数 $k$ 去调整 $R^{2}$, 计算出调整的多重判定系数 (adjusted multiple coefficient of determination), 记为 $R_{a}^{2}$, 其计算公式为:


\begin{equation*}
R_{a}^{2}=1-\left(1-R^{2}\right)\left(\frac{n-1}{n-k-1}\right) \tag{12.8}
\end{equation*}


$R_{a}^{2}$ 的解释与 $R^{2}$ 类似, 不同的是: $R_{a}^{2}$ 同时考虑了样本量 $n$ 和模型中自变量的个数 $k$ 的影响, 这就使得 $R_{a}^{2}$ 的值永远小于 $R^{2}$, 而且 $R_{a}^{2}$ 的值不会因为模型中自变量个数的增加而越来越接近 1 。因此, 在多元回归分析中, 通常用调整的多重判定系数。

$R^{2}$ 的平方根称为多重相关系数, 也称为复相关系数, 它度量了因变量同 $k$ 个自变量的相关程度。

根据表 $12-2$ 的输出结果得知: 多重判定系数 $R^{2}=0.797604=79.7604 \%$ 。其实际意义是: 在不良贷款取值的变差中, 能被不良贷款与贷款余额、累计应收贷款、贷款项目个数和固定资产投资额的多元回归方程所解释的比例为 $79.7604 \%$ 。\\
调整的多重判定系数 $R_{a}^{2}=0.757125=75.7125 \%$, 其意义与 $R^{2}$ 类似。它表示: 在用样本量和模型中自变量的个数进行调整后, 在不良贷款取值的变差中, 能被不良贷款与贷款余额、累计应收贷款、贷款项目个数和固定资产投资额的多元回归方程所解释的比例为 $75.7125 \%$ 。

\subsection*{12.2.2 估计标准误差}
同一元线性回归一样, 多元回归中的估计标准误差也是误差项 $\varepsilon$ 的方差 $\sigma^{2}$ 的一个估计值, 它在衡量多元回归方程的拟合优度方面起着重要作用。计算公式为:


\begin{equation*}
s_{e}=\sqrt{\frac{\sum\left(y_{i}-\hat{y}_{i}\right)^{2}}{n-k-1}}=\sqrt{\frac{S S E}{n-k-1}}=\sqrt{M S E} \tag{12.9}
\end{equation*}


式中, $k$ 为自变量的个数。

多元回归中对 $s_{e}$ 的解释与一元回归类似。由于 $s_{e}$ 所估计的是预测误差的标准差, 其含义是根据自变量 $x_{1}, x_{2}, \cdots, x_{k}$ 来预测因变量 $y$ 时的平均预测误差。

Excel 输出的回归结果直接给出了 $s_{e}$ 的值。根据表 $12-2$ 的输出结果, $s_{e}=1.778752$ (与根据公式计算的结果一样)。其含义是: 根据所建立的多元回归方程, 用贷款余额、累计应收贷款、贷款项目个数和固定资产投资额来预测不良贷款时, 平均预测误差为 1.778752 亿元。

\section*{12.3 显著性检验}
在一元线性回归中, 线性关系的检验 ( $F$ 检验) 与回归系数的检验 ( $t$ 检验) 是等价的, 这一点很容易理解。比如, $F$ 检验表明不良贷款与贷款余额之间有显著的线性关系,必然也意味着回归系数不会等于 0 , 因为只有一个自变量。但在多元回归中, 这两种检验不再等价。线性关系检验主要是检验因变量与多个自变量的线性关系是否显著, 在 $k$ 个自变量中, 只要有一个自变量与因变量的线性关系显著, $F$ 检验就能通过, 但这不一定意味着每个自变量与因变量的关系都显著。回归系数检验则是对每个回归系数分别进行单独的检验, 它主要用于检验每个自变量对因变量的影响是否显著。如果某个自变量没有通过检验, 就意味着这个自变量对因变量的影响不显著, 也许就没有必要将这个自变量放进回归模型中了。

\subsection*{12.3. 1 线性关系检验}
线性关系检验用于检验因变量 $y$ 与 $k$ 个自变量之间的关系是否显著, 也称为总体显著性检验。

检验的具体步骤如下。

第 1 步: 提出假设。

$$
H_{0}: \beta_{1}=\beta_{2}=\cdots=\beta_{k}=0
$$

$$
H_{1}: \beta_{1}, \beta_{2}, \cdots, \beta_{k} \text { 至少有一个不等于 } 0
$$

第 2 步: 计算检验的统计量 $F$ 。


\begin{equation*}
F=\frac{S S R / k}{\operatorname{SSE} /(n-k-1)} \sim F(k, n-k-1) \tag{12.10}
\end{equation*}


第 3 步: 作出统计决策。给定显著性水平 $\alpha$, 根据分子自由度 $=k$, 分母自由度 $=$ $n-k-1$ 查 $F$ 分布表得到 $F_{\alpha}$ 。若 $F>F_{\alpha}$, 则拒绝原假设; 若 $F<F_{\alpha}$, 则不拒绝原假设。根据计算机输出的结果, 可直接利用 $P$ 值作出决策:若 $P<\alpha$, 则拒绝原假设; 若 $P>\alpha$, 则不拒绝原假设。

\section*{例 12.2}
对例 12.1 建立的回归方程线性关系的显著性进行检验 $(\alpha=0.05)$ 。

—解 第 1 步:提出假设。

$$
H_{0}: \beta_{1}=\beta_{2}=\beta_{3}=\beta_{4}=0 ; H_{1}: \beta_{1}, \beta_{2}, \beta_{3}, \beta_{4} \text { 至少有一个不等于 } 0
$$

第 2 步: 计算检验的统计量 $F$ (可直接用表 $12-2$ Excel 输出的方差分析中的有关结果)。

$$
F=\frac{S S R / k}{S S E /(n-k-1)}=19.704044
$$

第 3 步: 作出统计决策。给定显著性水平 $\alpha=0.05$, 根据分子自由度 $=4$, 分母自由度 $=$ $25-4-1=20$, 查 $F$ 分布表得 $F_{a=0.05}(4,20)=2.87$ 。由于 $F=19.704044>F_{0.05}(4,20)=$ 2. 87 ,拒绝原假设 $H_{0}$ 。这意味着不良贷款与贷款余额、累计应收贷款、贷款项目个数和固定资产投资额之间的线性关系是显著的。

也可直接将 Excel 输出的方差分析部分中的 Significance $\mathrm{F}$ 值(即 $P$ 值)与给定的显著性水平 $\alpha=0.05$ 进行比较, 由于 Significance $\mathrm{F}=1.03539 \mathrm{E}-06<\alpha=0.05$, 拒绝原假设 $H_{0}$ 。

$F$ 检验表明:不良贷款与贷款余额、累计应收贷款、贷款项目个数和固定资产投资额之间的线性关系显著, 但这并不意味着不良贷款与每个变量之间的关系都显著, 因为 $F$ 检验说明的是总体的显著性。要判断每个自变量对不良贷款的影响是否显著, 需要对各回归系数分别进行 $t$ 检验。

\subsection*{12.3.2 回归系数检验和推断}
回归方程通过线性关系检验后, 就可以对各个回归系数 $\beta_{i}$ 有选择地进行一次或多次检验。但究竟要对哪几个回归系数进行检验, 通常需要在建立模型之前作出决定。此外, 还应对回归系数检验的个数进行限制, 以避免犯过多的第 I 类错误。

回归系数检验的具体步骤如下:

第 1 步: 提出假设。对于任意参数 $\beta_{i}(i=1,2, \cdots, k)$, 有

$$
H_{0}: \beta_{i}=0 ; \quad H_{1}: \beta_{i} \neq 0
$$

第 2 步: 计算检验的统计量 $t$ 。


\begin{equation*}
t_{i}=\frac{\hat{\beta}_{i}}{s_{\hat{\beta}_{i}}} \sim t(n-k-1) \tag{12.11}
\end{equation*}


式中, $s_{\beta_{i}}$ 是回归系数 $\hat{\beta}_{i}$ 的抽样分布的标准差, 即


\begin{equation*}
s_{\hat{\beta}_{i}}=\frac{s_{e}}{\sqrt{\sum x_{i}^{2}-\frac{1}{n}\left(\sum x_{i}\right)^{2}}} \tag{12.12}
\end{equation*}


第 3 步: 作出统计决策。给定显著性水平 $\alpha$, 根据自由度 $=n-k-1$ 查 $t$ 分布表, 得到 $t_{\alpha / 2}$ 的值。若 $|t|>t_{\alpha / 2}$, 则拒绝原假设; 若 $|t|<t_{\alpha / 2}$, 则不拒绝原假设。

\section*{例 12.3}
对例 12.1 建立的回归方程中各回归系数的显著性进行检验 $(\alpha=0.05)$ 。

・解 第 1 步:提出假设。对于任意参数 $\beta_{i}(i=1,2,3,4)$, 有

$$
H_{0}: \beta_{i}=0 ; H_{1}: \beta_{i} \neq 0
$$

第 2 步: 计算检验的统计量 $t$ 。


\begin{equation*}
t_{i}=\frac{\hat{\beta}_{i}}{s_{\hat{\beta}_{i}}} \tag{12.13}
\end{equation*}


根据表 $12-2$ 的结果可知, $t_{1}=3.837495, t_{2}=1.878738, t_{3}=0.174983, t_{4}=$ -1.936769 。

第 3 步: 作出统计决策。给定显著性水平 $\alpha=0.05$, 根据自由度 $=n-k-1=25-4-1=$ 20 , 查 $t$ 分布表, 得 $t_{\alpha / 2}=t_{0.025}=2.0860$ 。这里, 只有 $\beta_{1}$ 通过了检验, 其他 3 个自变量都没有通过检验。直接用 $P$ 值进行比较也是一样: $\beta_{1}$ 所对应的 $P$ 值小于 0.05 , 通过检验;其余 3 个系数所对应的 $P$ 值均大于 0.05 , 未通过检验。

这说明在影响不良贷款的 4 个自变量中, 只有贷款余额的影响是显著的, 而其他 3 个自变量的影响均不显著。这意味着其他 3 个自变量对预测不良贷款的作用已经不大。假定只选一个自变量来预测不良贷款,应该选贷款余额。

当然, 得出上述分析结论还需要有其他证据, 因为累计应收贷款、贷款项目个数和固定资产投资额这 3 个变量没有通过检验, 也可能是由其他原因造成的。比如, 当贷款余额、累计应收贷款、贷款项目个数和固定资产投资额这 4 个自变量之间高度相关时,有可能造成一个或几个回归系数无法通过检验, 但这并不一定意味着没通过检验的那些自变量对因变量的影响就不显著。自变量之间的相关所造成的这种问题在统计上称为多重共线性,有关这一问题将在 12.4 节中作介绍。

除了对回归系数进行检验, 还可以求出各回归系数的置信区间。回归系数 $\beta_{i}$ 在 $1-\alpha$置信水平下的置信区间为:


\begin{equation*}
\hat{\beta}_{i} \pm t_{\alpha / 2}(n-k-1) s_{\beta_{i}} \tag{12.14}
\end{equation*}


在 Excel 输出的回归结果中, 给出了各回归系数的置信区间。比如, 在表 $12-2$ 中给出的 $\beta_{1}$ 的 $95 \%$ 的置信区间为 (0.018 275,0.061 804)。这一置信区间的含义是: 在累计应收贷款 $x_{2}$ 、贷款项目个数 $x_{3}$ 和固定资产投资额 $x_{4}$ 不变的条件下, 贷款余额每增加 1 亿元, 不良贷款平均增加额在 0.018275 亿 0.061 804 亿元之间。其他几个回归系数的置信区间的含义类似。

\section*{12.4 多重共线性}
当回归模型中使用两个或两个以上的自变量时, 这些自变量往往会提供多余的信息;也就是说, 这些自变量之间彼此相关。比如, 在例 12.1 所建立的回归方程中, 使用了 4 个自变量, 即贷款余额 $\left(x_{1}\right)$ 、累计应收贷款 $\left(x_{2}\right)$ 、贷款项目个数 $\left(x_{3}\right)$ 和固定资产投资额 $\left(x_{4}\right)$ 。虽然这 4 个自变量对预测不良贷款都有作用, 但由于这 4 个自变量之间存在相关关系, 在预测中所提供的信息就是重复的。直观上看, 贷款余额 $\left(x_{1}\right)$ 与累计应收贷款 $\left(x_{2}\right)$ 之间就有较高的相关关系, 这两个变量所提供的预测信息是重复的, 或许只用其中的一个自变量就可以了。其他几个变量之间也有类似的相关情况。

\subsection*{12.4. 1 多重共线性及其产生的问题}
当回归模型中两个或两个以上的自变量彼此相关时, 则称回归模型中存在多重共线性 (multicollinearity)。在实际问题中, 所使用的自变量之间相关是一件很平常的事, 但是在回归分析中存在多重共线性会导致某些问题。

首先, 变量之间高度相关时, 可能会使回归的结果混乱, 甚至会把分析引人歧途。

在例 12.1 的回归中, 根据表 $12-2$ 的结果可知, Significance $F=1.03539 \mathrm{E}-06<\alpha=0.05$,这表明不良贷款 $y$ 与贷款余额 $\left(x_{1}\right)$ 、累计应收贷款 $\left(x_{2}\right)$ 、贷款项目个数 $\left(x_{3}\right)$ 和固定资产投资额 $\left(x_{4}\right)$ 之间的线性关系是显著的。但 4 个回归系数中, 只有 $\beta_{1}$ 通过了检验 $(P$-value $=0.001028<\alpha=0.05$ ), 而其他 3 个回归系数均未通过检验 ( $P$-value 分别为 $0.074935,0.862853,0.067030$, 均大于 0.05 )。

这种检验结果看起来矛盾, 实际上并不矛盾。因为线性关系检验 ( $F$ 检验) 表明线性关系显著时, 只是说因变量 (不良贷款) 至少同 4 个自变量中的一个自变量的线性关系是显著的, 并不意味着同每个自变量之间的关系都显著。事实上, 4 个自变量在预测不良贷款时可能都有贡献 (读者可就 4 个自变量分别进行一元回归加以验证), 只不过一些自变量的贡献与另一些自变量的贡献相互重叠了。

其次, 多重共线性可能对参数估计值的正负号产生影响, 特别是 $\beta_{i}$ 的正负号有可能同预期的正负号相反。比如, 从表 $12-2$ 的输出结果可以看出, 在 4 个回归系数中, $\hat{\beta}_{4}=$ - 0.029 193, 这意味着固定资产投资额增加时, 不良贷款是减少的。但实际情况是否如此呢? 不一定。如果仅就不良贷款与固定资产投资额作一元回归, 得到的估计方程为: $\hat{y}=0.979961+0.046586 x$, 这表明固定资产投资额每增加 1 亿元, 不良贷款平均增加 0.046586 亿元。这种情况就是由自变量之间的相关造成的, 因为 4 个自变量放在一起产\\
生了多余的信息。因此, 存在多重共线性时, 对回归系数的解释是危险的。

\subsection*{12.4.2 多重共线性的判别}
检测多重共线性的方法有多种, 其中最简单的一种办法是计算模型中各对自变量之间的相关系数, 并对各相关系数进行显著性检验。如果有一个或多个相关系数是显著的, 就表示模型中所使用的自变量之间相关,存在多重共线性问题。

具体来说, 如果出现下列情况, 则暗示存在多重共线性:

(1)模型中各对自变量之间显著相关。

(2) 当模型的线性关系检验 ( $F$ 检验) 显著时, 几乎所有回归系数 $\beta i$ 的 $t$ 检验却不显著。

(3) 回归系数的正负号与预期的相反。

(4)容忍度(tolerance)与方差扩大因子(variance inflation factor, VIF)。某个自变量的容忍度等于 1 减去该自变量为因变量而其他 $k-1$ 个自变量为预测变量时所得到的线性回归模型的判定系数, 即 $1-R_{i}^{2}$ 。容忍度越小, 多重共线性越严重。通常认为容忍度小于 0.1 时, 存在严重的多重共线性。方差扩大因子等于容忍度的倒数, 即 $\mathrm{VIF}=\frac{1}{1-R_{i}^{2}}$ 。显然, VIF 越大, 多重共线性越严重。一般认为 VIF 大于 10 时, 存在严重的多重共线性。

下面仍利用例 12.1 的数据说明上述判别方法的使用。

\section*{例 12.4}
沿用例 12.1 的数据, 按上述方法判别所建立的回归方程是否存在多重共线性。

-解 首先, 计算出 4 个自变量之间的相关矩阵, 由 Excel 输出的结果如表 $12-3$ 所示。表 12-3 贷款余额、累计应收贷款、贷款项目个数、固定资产投资额之间的相关矩阵

\begin{center}
\begin{tabular}{ccccc}
\hline
 & 贷款余额 & 累计应收贷款 & 贷款项目个数 & 固定资产投资额 \\
\hline
贷款余额 & 1 &  &  &  \\
累计应收贷款 & 0.678772 & 1 &  &  \\
贷款项目个数 & 0.848416 & 0.585831 & 1 &  \\
固定资产投资额 & 0.779702 & 0.472431 & 0.746646 & 1 \\
\hline
\end{tabular}
\end{center}

将各相关系数检验的统计量列入,如表 12-4 所示。

表 12-4 各相关系数检验的统计量

\begin{center}
\begin{tabular}{cccc}
\hline
 & 贷款余额 & 累计应收贷款 & 贷款项目个数 \\
\hline
贷款余额 & 1 &  &  \\
累计应收贷款 & 4.432870 & 1 &  \\
贷款项目个数 & 7.686824 & 3.466726 & 1 \\
固定资产投资额 & 5.971918 & 2.570663 & 5.382848 \\
\hline
\end{tabular}
\end{center}

查表得 $t_{\alpha / 2}(25-2)=2.0687$, 由于所有的统计量均大于 $t_{\alpha / 2}(25-2)=2.0687$, 所\\
以均拒绝原假设 $H_{0}$, 说明这 4 个自变量两两之间都有显著的相关关系, 这意味着在回归模型中引入 4 个自变量存在多重共线性问题。

其次, 由表 12-2 输出的结果可知, 回归模型的线性关系是显著的 (Significance $\mathrm{F}=$ $1.03539 \mathrm{E}-06<\alpha=0.05)$, 而回归系数检验时却有 3 个没有通过 $t$ 检验, 这也暗示了模型中存在多重共线性。

最后,固定资产投资额的回归系数为负号(-0.029 193),这与预期的不一致。从不良货款与固定资产投资额的一元回归中得知,固定资产投资额的增加会使不良贷款增加。

总之, 上述三点都表明回归模型中存在多重共线性。

\subsection*{12.4.3 多重共线性问题的处理}
一旦发现模型中存在多重共线性问题, 就应采取某种措施。至于采取什么样的方法来解决, 要看多重共线性的严重程度。下面给出多重共线性问题的一些解决办法。(1)

(1)将一个或多个相关的自变量从模型中剔除, 使保留的自变量尽可能不相关。

(2) 如果要在模型中保留所有的自变量, 就应该:

\begin{itemize}
  \item 避免根据 $t$ 统计量对单个参数 $\beta$ 进行检验。
  \item 对因变量 $y$ 值的推断(估计或预测)限定在自变量样本值的范围内。
\end{itemize}

\section*{例 12.5}
利用例 12.1 所建立的回归方程, 对多重共线性问题进行处理。

-一解 首先, 考虑将一些相关的自变量从模型中剔除。从表 $12-3$ 的相关矩阵中可以看出, 货款余额与贷款项目个数的相关系数最高, 从定性角度看, 贷款余额与累计应收贷款之间也有很强的相关关系。因此将贷款项目个数和累计应收贷款这两个自变量剔除, 建立不良货款 $(y)$ 与贷款余额 $\left(x_{1}\right)$ 和固定资产投资额 $\left(x_{2}\right)$ 的线性模型。由 Excel 输出的结果如表 12-5 所示。

表 12-5 Excel 输出的回归分析结果

SUMMARY OUTPUT

回归统计

Multiple R

0.872380

$R$ Square

0.761046

Adjusted R Square\\
0. 739323

标准误差

\begin{enumerate}
  \item 842787
\end{enumerate}

观测值 25

(1) 处理多重共线性问题有更好的办法, 感兴趣的读者可参考有关回归方面的书籍。\\
续表

方差分析

\begin{center}
\begin{tabular}{|c|c|c|c|c|c|c|}
\hline
 & df & SS & MS & F & Significance F &  \\
\hline
回归分析 & 2 & 237. 941432 & 118.970716 & 35.034023 & $1.45028 \mathrm{E}-07$ &  \\
\hline
残差 & 22 & 74.708968 & 3. 395862 &  &  &  \\
\hline
\multirow[t]{2}{*}{总计} & 24 & 312.6504 &  &  &  &  \\
\hline
 & Coefficients & 标准误差 & t Stat & P-value & Lower $95 \%$ & Upper $95 \%$ \\
\hline
Intercept & -0.443424 & 0.696865 & -0.636312 & 0.531138 & -1.888634 & 1.001787 \\
\hline
X Variable 1 & 0.050332 & 0.007477 & 6. 731607 & 0.000001 & 0.034826 & 0.065838 \\
\hline
$\mathrm{X}$ Variable 2 & -0.031903 & 0.014954 & -2.133368 & 0.044294 & -0.062916 & -0.000890 \\
\hline
\end{tabular}
\end{center}

从表 $12-5$ 的输出结果中可以看出, 线性关系和各回归系数在 0.05 的显著性水平下是显著的,多重共线性问题不存在了。

多重共线性问题带来的主要麻烦是对单个回归系数的解释和检验。在求因变量的置信区间和预测区间时一般不会受其影响, 但必须保证用于估计或预测的自变量的值是在样本数据的范围之内。因此, 如果仅仅是为了估计或预测, 可以将所有自变量都保留在模型中。

最后需要提醒的是: 在建立多元线性回归模型时, 不要试图引人更多的自变量, 除非确实有必要。特别是在社会科学的研究中, 由于所使用的大多数数据都是非试验性质的,因此, 在某些情况下, 得到的结果往往并不令人满意, 但这不一定是因为选择的模型不合适, 而是数据的质量不好, 或者是引人的自变量不合适。

\section*{12.5 利用回归方程进行预测}
在一元线性回归中, 我们曾介绍利用自变量来估计因变量的方法。对于多元线性回归, 同样可以利用给定的 $k$ 个自变量, 求出因变量 $y$ 的平均值的置信区间和个别值的预测区间。

置信区间和预测区间的计算公式复杂, 超出了本书的范围, 这里不再介绍。多元回归问题的求解完全依赖于计算机, 已有的统计软件, 如 SAS, SPSS, MINITAB, STATISTICA 等都有现成的回归分析程序, 可以直接给出因变量的置信区间和预测区间。

\section*{例 12.6}
沿用例 12.1 的数据。根据贷款余额 $x_{1}$ 、累计应收贷款 $x_{2}$ 、贷款项目个数 $x_{3}$ 和固定资产投资额 $x_{4}$ 建立不良贷款 $y$ 的 $95 \%$ 的置信区间和预测区间。

\section*{太用 SPSS 进行回归分析的操作步骤}
第 1 步: 选择【分析】 $\rightarrow$ 【回归-线性】, 进入主对话框。

第 2 步: 在主对话框中将因变量 (本例为不良贷款) 选入【因变量】, 将所有自变量 (本例为贷款余额、累计应收贷款、贷款项目个数、固定资产投资额)选入【自变量】,并在【方法】下选择【进入】。

第 3 步:点击【保存】。在【预测值】下选择【非标准化】(输出点预测值)。在【预测区间】下选择【均值】和【单值】(输出置信区间和预测区间)。在【置信区间】中选择所要求的置信水平 (默认值为 $95 \%$, 一般不用改变)。点击【继续】。点击【确定】。

输出结果如表 12-6 所示。

表 12-6 不良贷款的置信区间和预测区间

\begin{center}
\begin{tabular}{|c|c|c|c|c|c|c|c|c|c|c|}
\hline
\begin{tabular}{l}
分行 \\
编号 \\
\end{tabular} & \begin{tabular}{l}
不良 \\
贷款 \\
\end{tabular} & \begin{tabular}{l}
贷款 \\
余额 \\
\end{tabular} & \begin{tabular}{l}
累计应 \\
收贷款 \\
\end{tabular} & \begin{tabular}{l}
贷款项 \\
目个数 \\
\end{tabular} & \begin{tabular}{c}
固定资产 \\
投资额 \\
\end{tabular} & PRE\_ 1 & $\mathrm{LMCI}_{-} 1$ & UMCI\_ 1 & LICI\_1 & $\mathrm{LICI}_{-} 1$ \\
\hline
1 & 0.9 & 67.3 & 6.8 & 5 & 51.9 & 1. 23718 & -0.18541 & 2.65976 & -2.73660 & 5.21095 \\
\hline
2 & 1.1 & 111.3 & 19.8 & 16 & 90.9 & 3. 94465 & 1. 66643 & 6.22287 & -0.40937 & 8. 29867 \\
\hline
3 & 4.8 & 173.0 & 7. 7 & 17 & 73.7 & 5.14051 & 3. 85596 & 6. 42507 & 1. 21403 & 9. 06699 \\
\hline
4 & 3.2 & 80.8 & 7. 2 & 10 & 14.5 & 3.00138 & 1. 78174 & 4. 22102 & -0.90434 & 6.90711 \\
\hline
5 & 7.8 & 199.7 & 16.5 & 19 & 63.2 & 7. 84785 & 6.43467 & 9. 26102 & 3. 877431 & 11.81827 \\
\hline
6 & 2.7 & 16. 2 & 2.2 & 1 & 2.2 & -0.09702 & -1.62200 & 1. 42796 & -4.10860 & 3.91455 \\
\hline
7 & 1.6 & 107.4 & 10. 7 & 17 & 20.2 & 4. 51985 & 3.01816 & 6.02155 & 0.51707 & 8. 52263 \\
\hline
8 & 12.5 & 185.4 & 27.1 & 18 & 43.8 & 9. 39626 & 6.740061 & 12.05245 & 4. 833091 & 13. 95943 \\
\hline
9 & 1.0 & 96.1 & 1. 7 & 10 & 55.9 & 1.59121 & 0.16342 & 3.01901 & -2.38443 & 5. 56686 \\
\hline
10 & 2.6 & 72.8 & 9.1 & 14 & 64.3 & 1. 56664 & 0.32302 & 2.81027 & -2.34664 & 5.47992 \\
\hline
11 & 0.3 & 64.2 & 2.1 & 11 & 42.7 & 0.77305 & -0.42179 & 1.96788 & -3.12500 & 4. 67109 \\
\hline
12 & 4.0 & 132.2 & 11.2 & 23 & 76.7 & 4.02462 & 2.64760 & 5.40165 & 0.06693 & 7. 98232 \\
\hline
13 & 0.8 & 58.6 & 6.0 & 14 & 22.8 & 1.75068 & 0.40366 & 3.09771 & -2.19667 & 5.69804 \\
\hline
14 & 3.5 & 174.6 & 12.7 & 26 & 117.1 & 4.80854 & 3. 19196 & 6.42512 & 0.76126 & 8. 85582 \\
\hline
15 & 10.2 & 263.5 & 15.8 & 34 & 146.7 & 8.04946 & 6.063031 & 10.03590 & 3. 840771 & 12. 25815 \\
\hline
16 & 3.0 & 79.3 & 8.9 & 15 & 28.9 & 2.81606 & 1.62898 & 4.00313 & -1.07962 & 6.71173 \\
\hline
17 & 0.2 & 14.8 & 0.6 & 2 & 42.1 & -1.54020 & -3.24509 & 0.16470 & -5.62356 & 2. 54316 \\
\hline
18 & 0.4 & 73.5 & 5.9 & 11 & 25.3 & 2.21590 & 1. 18479 & 3.24701 & -1.63512 & 6.06691 \\
\hline
19 & 1.0 & 24.7 & 5.0 & 4 & 13. 4 & 0.37443 & -0.87232 & 1. 62119 & -3.53984 & 4. 28871 \\
\hline
20 & 6.8 & 139.4 & 7.2 & 28 & 64.3 & 4.15541 & 1. 94297 & 6.36785 & -0.16455 & 8. 47537 \\
\hline
21 & 11.6 & 368.2 & 16.8 & 32 & 163.9 & 11.88805 & 9. 066561 & 14. 70954 & 7. 226721 & 16. 54938 \\
\hline
22 & 1. 6 & 95.7 & 3. 8 & 10 & 44.5 & 2.21887 & 1. 03994 & 3. 39779 & -1.67434 & 6.11207 \\
\hline
23 & 1. 2 & 109.6 & 10.3 & 14 & 67.9 & 3.11264 & 2.22239 & 4.00289 & -0.70308 & 6.92836 \\
\hline
24 & 7.2 & 196.2 & 15.8 & 16 & 39.7 & 8. 24653 & 6. 330601 & 10.16246 & 4.070651 & 12. 42240 \\
\hline
25 & 3.2 & 102.2 & 12.0 & 10 & 97.1 & 2. 15746 & 0.17604 & 4.13887 & -2.04887 & 6.36378 \\
\hline
\end{tabular}
\end{center}

表 12-6 中的 PRE\_1 是点估计(预测)值,LMCI\_1 和 UMCI\_1 是平均值的置信区间 (SPSS 称为均值的预测区间) 的下限和上限, LICI\_1 和 UICI\_1 是个别值的预测区间的下限和上限。

\section*{12.6 变量选择与逐步回归}
根据多个自变量建立回归模型时, 若试图将所有的自变量都引人回归模型, 带来的问题往往让人无所适从, 或者是对所建立的模型不能进行有效的解释。比如, 在例 12.1 中,建立不良贷款与 4 个自变量的回归模型时, 得到的结果就很难解释。如果在建立模型之前能对所收集到的自变量进行一定的笁选, 去掉那些不必要的自变量, 不仅会使建立模型变得容易,而且使模型更具有可操作性,也更容易解释。

\subsection*{12.6. 1 变量选择过程}
在建立回归模型时, 总希望用最少的变量来建立模型。但究竟哪些自变量应该引人模型, 哪些自变量不应该引人模型, 需要对自变量进行一定的笠选。在进行回归时, 每次只增加一个变量, 并且将新变量与模型中的变量进行比较, 若新变量引人模型后以前的某个变量的 $t$ 统计量不显著, 这个变量就会被从模型中剔除, 在这种情况下, 回归分析就很难受到多重共线性的影响, 这就是回归中的搜寻过程。逐步回归是一种搜寻过程, 也是避免多重共线性的方法之一。1

选择自变量的原则通常是对统计量进行显著性检验, 检验的根据是: 将一个或一个以上的自变量引人回归模型中时, 是否使残差平方和 $(S S E)$ 显著减少。如果增加一个自变量使残差平方和显著减少, 则说明有必要将这个自变量引人回归模型, 否则, 就没有必要将这个自变量引人回归模型。确定在模型中引人自变量 $x_{i}$ 是否使残差平方和显著减少的方法, 就是使用 $F$ 统计量的值作为一个标准, 以此来确定是在模型中增加一个自变量, 还是从模型中剔除一个自变量。

变量选择的方法主要有向前选择 (forward selection)、向后剔除 (backward elimination)、逐步回归 (stepwise regression) 等。

\subsection*{12.6.2 向前选择}
向前选择法是从模型中没有自变量开始, 然后按下面的步骤选择自变量来拟合模型。

第 1 步: 对 $k$ 个自变量 $\left(x_{1}, x_{2}, \cdots, x_{k}\right)$ 分别拟合与因变量 $y$ 的一元线性回归模

(1) 除了逐步回归, 岭回归 (ridge regression) 也是一种专门用于多重共线性数据分析的回归方法。它实际上是一种改良的最小二乘法, 通过放弃最小二乘法的无偏性, 以损失部分信息、降低精度为代价来寻求效果稍差但回归系数更符合实际的回归方程。当然, 由于多重共线性, 有些重要的解释变量可能无法进人分析中。\\
型, 共有 $k$ 个, 然后找出 $F$ 统计量的值最大的模型及其自变量 $x_{i}$, 并将其首先引人模型。 (如果所有模型均无统计上的显著性, 则运算过程终止, 没有模型被拟合。)

第 2 步: 在已经引入模型的 $x_{i}$ 的基础上, 再分别拟合 $k-1$ 个自变量 $\left(x_{1}, \cdots, x_{i-1}\right.$, $\left.x_{i+1}, \cdots, x_{k}\right)$ 的线性回归模型, 即变量组合为 $x_{i}+x_{1}, \cdots, x_{i}+x_{i-1}, x_{i}+x_{i+1}, \cdots, x_{i}+x_{k}$的 $k-1$ 个线性回归模型。然后分别考察这 $k-1$ 个线性模型, 挑选出 $F$ 统计量的值最大的含有两个自变量的模型, 并将 $F$ 统计量的值最大的那个自变量 $x_{j}$ 引人模型。如果除 $x_{i}$ 之外的 $k-1$ 个自变量中没有一个是统计上显著的, 则运算过程终止。如此反复进行, 直至模型外的自变量均无统计上的显著性为止。

向前选择变量的方法是不停地向模型中增加自变量, 直至增加自变量不能导致 SSE 显著增加 (这个过程通过 $F$ 检验来完成) 为止。由此可见, 只要将某个自变量增加到模型中, 这个变量就一定会保留在模型中。

\subsection*{12.6.3 向后剔除}
与向前选择法相反, 向后剔除法的基本过程如下:

第 1 步: 先对因变量拟合包括所有 $k$ 个自变量的线性回归模型。然后考察 $p(p<k)$个去掉一个自变量的模型 (这些模型中的每一个都有 $k-1$ 个自变量), 使模型的 SSE 值减小最少的自变量被挑选出来并从模型中剔除。

第 2 步: 考察 $p-1$ 个再去掉一个自变量的模型 (这些模型中的每一个都有 $k-2$ 个自变量), 使模型的 SSE 值减小最少的自变量被挑选出来并从模型中剔除。如此反复进行,一直将自变量从模型中剔除, 直至剔除一个自变量不会使 $S S E$ 显著减小为止。这时, 模型中所剩的自变量都是显著的。上述过程可以通过 $F$ 检验的 $P$ 值来判断。

\subsection*{12.6.4 逐步回归}
逐步回归是将上述两种方法结合起来筛选自变量的方法。前两步与向前选择法相同。不过在增加了一个自变量后, 它会对模型中所有的变量进行考察, 看看有没有可能剔除某个自变量。如果在增加了一个自变量后, 前面增加的某个自变量对模型的贡献变得不显著, 这个变量就会被剔除。因此, 逐步回归是向前选择和向后剔除的结合。逐步回归过程就是按此方法不停地增加变量并考虑剔除以前增加的变量的可能性, 直至增加变量不会导致 SSE 显著减少, 这个过程可通过 $F$ 统计量来检验。逐步回归法中, 在前面步骤中增加的自变量在后面的步骤中有可能被剔除, 而在前面步骤中被剔除的自变量在后面的步骤中也可能重新进人模型。

\section*{例 12.7}
根据例 12.1 不良贷款 $y$ 与货款余额 $x_{1}$ 、累计应收贷款 $x_{2}$ 、货款项目个数 $x_{3}$ 和固定资产投资额 $x_{4}$ 的数据, 采用逐步回归方法建立回归模型。

\begin{itemize}
  \item 解 由于 Excel 没有此项功能, 因此使用 SPSS 进行逐步回归, 操作步骤如下。
\end{itemize}

$\star$ 用 SPSS 进行逐步回归的操作步骤

第 1 步: 选择【分析】 $\rightarrow$ 【回归-线性】,进入主对话框。

第 2 步: 在对话框中将因变量选入【因变量】, 将所有自变量选入【自变量】, 并在【方法】下选择【步进】。

第 3 步: 点击【选项】, 并在【步进法条件】下选中【使用 $\mathbf{F}$ 的概率】, 并在【进入】框中输入增加变量所要求的显著性水平 (默认值为 0.05, 一般不用改变); 在【除去】输入剔除变量所要求的显著性水平 (默认值为 0.10 , 一般不用改变)。点击【继续】回到主对话框。点击【确定】。

(注: 需要预测时, 点击【保存】, 在【预测值】下选中【未标准化】(输出点预测值); 在【预测区间】下选中【平均值】和【单值】(输出置信区间和预测区间);在【置信区间】中选择所要求的置信水平 (默认值为 $95 \%$, 一般不用改变)。需要残差分析时, 在【残差】下选中所需的残差, 如【未标准化】。需要输出标准化残差的直方图和正态概率图时, 点击【图】, 在【标准化残差图】下选中【直方图】和【正态概率图】。)

按上述步骤得到的部分输出结果如下。

表 12-7 最先引入的自变量是货款余额(Model 1),其次是固定资产投资额(Model 2),其他两个自变量累计应收贷款和贷款项目个数均被剔除出模型。

\section*{表 12-7 变量的进入和移出标准}
输人/除去的变量 ${ }^{a}$

\begin{center}
\begin{tabular}{|c|c|c|c|}
\hline
模型 & 输人的变量 & 除去的变量 & 方法 \\
\hline
1 & 贷款余额 &  & \begin{tabular}{l}
步进 (条件: 要输人的 $F$ 的概率 $<=$ \\
0.050 , 要除去的 $F$ 的概率 $>=0.100$ ) \\
\end{tabular} \\
\hline
2 & 固定资产投资额 &  & \begin{tabular}{l}
步进 (条件: 要输人的 $F$ 的概率 $<=$ \\
0.050 , 要除去的 $F$ 的概率 $>=0.100$ ) \\
\end{tabular} \\
\hline
\end{tabular}
\end{center}

a. 因变量: 不良贷款。

表 $12-8$ 给出了两个回归模型的一些主要统计量, 包括复相关系数 $R$ 、判定系数 $R^{2} 、$调整的判定系数 $R_{a}^{2}$ 以及估计标准误差 $s_{e}$ 等。

表 12-8 两个模型的主要统计量

模型摘要

\begin{center}
\begin{tabular}{ccccc}
\hline
模型 & $R$ & $R$ 方 & 调整后 $R$ 方 & 标准估算的错误 \\
\hline
1 & $0.844^{\mathrm{a}}$ & 0.712 & 0.699 & 1.9799 \\
2 & $0.872^{\mathrm{b}}$ & 0.761 & 0.739 & 1.8428 \\
\hline
\end{tabular}
\end{center}

a. 预测变量: (常量), 贷款余额。

b. 预测变量: (常量), 贷款余额, 固定资产投资额。

表 $12-9$ 给出了回归分析中的方差分析表。\\
表 12-9 两个模型的方差分析表

ANOVA $^{a}$

\begin{center}
\begin{tabular}{rrrrrrr}
\hline
模型 &  & 平方和 & 自由度 & \multicolumn{1}{c}{均方} & \multicolumn{1}{c}{$F$} & 显著性 \\
\hline
\multirow{2}{*}{1} & 回归 & 222.486 & 1 & 222.486 & 56.754 & $0.000^{\mathrm{b}}$ \\
 & 残差 & 90.164 & 23 & 3.920 &  &  \\
 & 总计 & 312.650 & 24 &  &  &  \\
 & 回归 & 237.941 & 2 & 118.971 & 35.034 & $0.000^{\mathrm{c}}$ \\
 & 残差 & 74.709 & 22 & 3.396 &  &  \\
 & 总计 & 312.650 & 24 &  &  &  \\
\hline
\end{tabular}
\end{center}

a. 因变量: 不良贷款。

b. 预测变量: (常量), 贷款余额。

c. 预测变量: (常量), 贷款余额, 固定资产投资额。

表 12-10 给出了参数的估计值和用于检验的 $t$ 统计量和 $P$ 值。

表 12-10 模型参数的估计和检验

系数

\begin{center}
\begin{tabular}{|c|c|c|c|c|c|c|}
\hline
\multirow{2}{*}{模型} &  & \multicolumn{2}{|c|}{非标准化系数} & \multirow{2}{*}{$\frac{\text { 标准化系数 }}{\text { Beta }}$} & \multirow{2}{*}{$t$} & \multirow{2}{*}{显著性} \\
\hline
 &  & $B$ & 标准错误 &  &  &  \\
\hline
\multirow{3}{*}{1} & (常量) & -0.830 & 0.723 &  & -1.147 & 0.263 \\
\hline
 & 贷款余额 & 0.038 & 0.005 & 0.844 & 7.534 & 0.000 \\
\hline
 & (常量) & -0.443 & 0.697 &  & -0.636 & 0.531 \\
\hline
\multirow[t]{2}{*}{2} & 贷款余额 & 0.050 & 0.007 & 1. 120 & 6.732 & 0.000 \\
\hline
 & 固定资产投资 & -0.032 & 0.015 & -0.355 & -2.133 & 0.044 \\
\hline
\end{tabular}
\end{center}

a. 因变量: 不良贷款。

由逐步回归的结果可知, 最后得到的模型为:

$$
\hat{y}=-0.443+0.050 x_{1}-0.032 x_{4}
$$

读者可能注意到, 采用逐步回归方法得到的表 12-10 中的结果与表 $12-5$ 中的结果是一致的。

\section*{思考与练习}
思考题

12.1 解释多元回归模型、多元回归方程、估计的多元回归方程的含义。

12.2 多元线性回归模型中有哪些基本假定?

12.3 解释多重判定系数和调整的多重判定系数的含义和作用。

12.4 解释多重共线性的含义。

12.5 多重共线性对回归分析有哪些影响?

12.6 多重共线性的判别方法主要有哪些?

12.7 多重共线性的处理方法有哪些?

12.8 在多元线性回归中, 选择自变量的方法有哪些?

\section*{练习题}
12.1 根据下面的数据用 Excel 进行回归, 并对回归结果进行讨论, 计算 $x_{1}=200, x_{2}=7$ 时 $y$ 的预测值。

\begin{center}
\begin{tabular}{ccc}
\hline
$y$ & $x_{1}$ & $x_{2}$ \\
\hline
12 & 174 & 3 \\
18 & 281 & 9 \\
31 & 189 & 4 \\
28 & 202 & 8 \\
52 & 149 & 9 \\
47 & 188 & 12 \\
38 & 215 & 5 \\
22 & 150 & 11 \\
36 & 167 & 8 \\
17 & 135 & 5 \\
\hline
\end{tabular}
\end{center}

12.2 一家电器销售公司的管理人员认为, 月销售收入是广告费用的函数, 想通过广告费用对月销售收入作出估计。下面是近 8 个月的月销售收入与广告费用数据(单位:万元):

\begin{center}
\begin{tabular}{ccc}
\hline
月销售收人 $y$ & 电视广告费用 $x_{1}$ & 报纸广告费用 $x_{2}$ \\
\hline
96 & 5.0 & 1.5 \\
90 & 2.0 & 2.0 \\
95 & 4.0 & 1.5 \\
92 & 2.5 & 2.5 \\
95 & 3.0 & 3.3 \\
94 & 3.5 & 2.3 \\
94 & 2.5 & 4.2 \\
94 & 3.0 & 2.5 \\
\hline
\end{tabular}
\end{center}

(1)用电视广告费用作自变量,月销售收入作因变量,建立估计的回归方程。

(2)用电视广告费用和报纸广告费用作自变量,月销售收入作因变量,建立估计的回归方程。

(3)上述(1)和(2)所建立的估计的回归方程中, 电视广告费用的系数是否相同? 对其回归系数分别进行解释。

(4) 在根据问题(2)所建立的估计的回归方程中, 月销售收入的总变差中被估计的回归方程所解释的比例是多少?

(5)针对根据问题(2)所建立的估计的回归方程,检验回归系数是否显著 $(\alpha=0.05)$ 。

12.3 某农场通过试验取得早稻收获量与春季降雨量和春季温度的数据如下:

\begin{center}
\begin{tabular}{ccc}
\hline
收获量 $y\left(\mathrm{~kg} / \mathrm{hm}^{2}\right)$ & 降雨量 $x_{1}(\mathrm{~mm})$ & 温度 $x_{2}\left({ }^{\circ} \mathrm{C}\right)$ \\
\hline
2250 & 25 & 6 \\
3450 & 33 & 8 \\
4500 & 45 & 10 \\
6750 & 105 & 13 \\
7200 & 110 & 14 \\
7500 & 115 & 16 \\
8250 & 120 & 17 \\
\hline
\end{tabular}
\end{center}

(1) 确定早稻收获量对春季降雨量和春季温度的二元线性回归方程。

(2) 解释回归系数的实际意义。

(3)根据你的判断,模型中是否存在多重共线性?

12.4 一家房地产评估公司想对某城市的房地产销售价格 $(y)$ 与地产估价 $\left(x_{1}\right)$ 、房产估价 $\left(x_{2}\right)$和使用面积 $\left(x_{3}\right)$ 建立一个模型, 以便对销售价格作出合理预测。为此, 它收集了 20 栋住宅的房地产评估数据。

\begin{center}
\begin{tabular}{|c|c|c|c|c|}
\hline
房地产编号 & \begin{tabular}{l}
销售价格 y \\
(元/平方米) \\
\end{tabular} & \begin{tabular}{c}
地产估价 $x_{1}$ \\
(万元) \\
\end{tabular} & \begin{tabular}{c}
房产估价 $x_{2}$ \\
(万元) \\
\end{tabular} & \begin{tabular}{c}
使用面积 $x_{3}$ \\
(平方米) \\
\end{tabular} \\
\hline
1 & 6890 & 596 & 4497 & 18730 \\
\hline
2 & 4850 & 900 & 2780 & 9280 \\
\hline
3 & 5550 & 950 & 3144 & 11260 \\
\hline
4 & 6200 & 1000 & 3959 & 12650 \\
\hline
5 & 11650 & 1800 & 7283 & 22140 \\
\hline
6 & 4500 & 850 & 2732 & 9120 \\
\hline
7 & 3800 & 800 & 2986 & 8990 \\
\hline
8 & 8300 & 2300 & 4775 & 18030 \\
\hline
9 & 5900 & 810 & 3912 & 12040 \\
\hline
10 & 4750 & 900 & 2935 & 17250 \\
\hline
11 & 4050 & 730 & 4012 & 10800 \\
\hline
12 & 4000 & 800 & 3168 & 15290 \\
\hline
13 & 9700 & 2000 & 5851 & 24550 \\
\hline
14 & 4550 & 800 & 2345 & 11510 \\
\hline
15 & 4090 & 800 & 2089 & 11730 \\
\hline
16 & 8000 & 1050 & 5625 & 19600 \\
\hline
17 & 5600 & 400 & 2086 & 13440 \\
\hline
18 & 3700 & 450 & 2261 & 9880 \\
\hline
19 & 5000 & 340 & 3595 & 10760 \\
\hline
20 & 2240 & 150 & 578 & 9620 \\
\hline
\end{tabular}
\end{center}

用 Excel 进行回归,并回答下面的问题:

(1)写出估计的多元回归方程。

(2)销售价格的总变差中被估计的回归方程所解释的比例是多少?

(3) 检验回归方程的线性关系是否显著 $(\alpha=0.05)$ 。

(4) 检验各回归系数是否显著 $(\alpha=0.05)$ 。

12.5 下面是随机抽取的 15 家大型商场销售的同类产品的有关数据(单位:元):

\begin{center}
\begin{tabular}{cccc}
\hline
企业编号 & 销售价格 $y$ & 购进价格 $x_{1}$ & 销售费用 $x_{2}$ \\
\hline
1 & 1238 & 966 & 223 \\
2 & 1266 & 894 & 257 \\
3 & 1200 & 440 & 387 \\
4 & 1193 & 664 & 310 \\
5 & 1106 & 791 & 339 \\
6 & 1303 & 852 & 283 \\
7 & 1313 & 804 & 302 \\
8 & 1144 & 905 & 214 \\
9 & 1286 & 771 & 304 \\
10 & 1084 & 511 & 326 \\
11 & 1120 & 505 & 339 \\
12 & 1156 & 851 & 235 \\
13 & 1083 & 659 & 276 \\
14 & 1263 & 490 & 390 \\
15 & 1246 & 696 & 316 \\
\hline
\end{tabular}
\end{center}

(1)计算 $y$ 与 $x_{1} 、 y$ 与 $x_{2}$ 之间的相关系数, 是否有证据表明销售价格与购进价格、销售价格与销售费用之间存在线性关系?

(2)根据上述结果,你认为用购进价格和销售费用来预测销售价格是否有效?

(3) 用 Excel 进行回归, 并检验模型的线性关系是否显著 $(\alpha=0.05)$ 。

(4)解释判定系数 $R^{2}$, 所得的结论与 (2) 是否一致?

(5) 计算 $x_{1}$ 与 $x_{2}$ 之间的相关系数, 所得结果意味着什么?

(6)模型中是否存在多重共线性? 你对模型有何建议?

\section*{第13章 时间序列分析和预测}
\begin{center}
\end{center}

在生活和工作中经常需要作出预测。比如, 预测一只股票的价格走势, 预测下一年度的销售额, 等等。研究时间序列的主要目的之一就是进行预测, 主要是根据已有的时间序列数据预测未来的变化。时间序列预测的关键是确定已有时间序列的变化模式, 并假定这种模式会延续到未来。下面是我国 2015 年 1 季度至 2019 年 4 季度国内生产总值的变化趋势图。

\begin{center}
\end{center}

年/季度

根据上面的图形, 你认为国内生产总值包含什么样的成分? 应该选择什么方法来预测下一年各季度的国内生产总值? 本章内容将回答这些问题。\\
时间序列数据用于描述现象随时间发展变化的特征。时间序列分析就其发展的历史阶段和所使用的统计分析方法来看, 有传统的时间序列分析和现代的时间序列分析。本章主要介绍传统的时间序列分析方法, 内容包括时间序列数据的统计描述和预测方法。

\section*{13.1 时间序列及其分解}
时间序列 (time series) 是同一现象在不同时间的相继观察值排列而成的序列。经济数据大多数以时间序列的形式给出。根据观察时间的不同, 时间序列中的时间可以是年份、季度、月份或其他形式。为便于表述, 本书中用 $t$ 表示所观察的时间, $Y$ 表示观察值, $Y_{i}(i=1,2, \cdots, n)$ 为时间 $t_{i}$ 的观察值。

时间序列可以分为平稳序列和非平稳序列两大类。平稳序列 (stationary series) 是基本上不存在趋势的序列。这类序列中的各观察值基本上在某个固定的水平上波动, 虽然在不同的时间段波动的程度不同, 但并不存在某种规律, 波动可以看成是随机的。非平稳序列 (non-stationary series) 是包含趋势、季节性或周期性的序列, 它可能只含有其中一种成分, 也可能含有几种成分。因此, 非平稳序列又可以分为有趋势的序列、有趋势和季节性的序列、几种成分混合而成的复合型序列。

趋势 (trend) 是时间序列在长期内呈现出来的某种持续上升或持续下降的变动。时间序列中的趋势可以是线性的, 也可以是非线性的。图 13-1 是一种线性趋势的时间序列。

\begin{center}
\end{center}

图 13-1 有趋势的序列

季节性 (seasonality) 也称季节变动 (seasonal fluctuation), 它是时间序列在一年内重复出现的周期性波动。比如, 在商业活动中, 常常听到 “销售旺季” 或 “销售淡季” 这类术语; 在旅游业中, 常常使用 “旅游旺季” 或 “旅游淡季” 这类术语; 等等。这些术语表明, 活动因季节的不同而发生变化。当然, 季节性中的 “季节” 一词是广义的, 它不仅\\
仅是指一年中的四季, 其实是指任何一种周期性的变化。在现实生活中, 季节变动是一种极为普遍的现象, 是气候条件、生产条件、节假日或人们的风俗习惯等各种因素作用的结果。比如, 农业生产、交通运输、建筑业、旅游业、商品销售以及工业生产中都有明显的季节性。

含有季节成分的序列可能含有趋势, 也可能不含有趋势。图 13-2 是含季节成分和趋势的时间序列。

\begin{center}
\end{center}

图 13-2 含有季节成分和趋势的序列

周期性 (cyclicity) 也称循环波动 (cyclical fluctuation), 是时间序列中呈现出来的围绕长期趋势的一种波浪形或振荡式变动。周期性通常是由商业和经济活动引起的, 它不同于趋势变动, 不是朝着单一方向的持续运动, 而是涨落相间的交替波动; 它也不同于季节变动, 季节变动有比较固定的规律, 且变动周期大多为一年, 循环波动则无固定规律, 变动周期多在一年以上, 且周期长短不一。周期性通常是由经济环境的变化引起的。由于分析周期需要较长的时间序列数据, 当序列较短时则难以发现周期。图 13-3 是含有周期成分的序列。

除此以外, 还有些偶然性因素对时间序列产生影响, 致使时间序列呈现出某种随机波动。时间序列中除去趋势、周期性和季节性之后的偶然性波动称为 随机性 (randomness),也称不规则波动 (irregular variations)。图 13-4 就是含有随机成分的序列。

这样, 时间序列的成分可以分为四种, 即趋势 $(T)$ 、季节性或季节变动 $(S)$ 、周期性或循环波动 $(C)$ 、随机性或不规则波动 $(I)$ 。传统时间序列分析的一项主要内容就是把这些成分从时间序列中分离出来, 并用一定的数学关系式表示它们之间的关系, 而后分别进行分析。按四种成分对时间序列的影响方式不同, 时间序列可分解为加法模型 (additive model)或乘法模型(multiplicative model)等。本章所介绍的时间序列分解方法都是以乘法模型为基础, 其形式为:

\begin{center}
\end{center}

图 13-3 含有周期成分的序列

\begin{center}
\end{center}

图 13-4 含有随机成分的序列


\begin{equation*}
Y_{t}=T_{t} \times S_{t} \times C_{t} \times I_{t} \tag{13.1}
\end{equation*}


\section*{13.2}
时间序列的描䢡性分析

\subsection*{13.2. 1 图形描述}
在对时间序列进行分析时, 最好是先作一个图, 然后通过图形观察数据随时间变化的模式及趋势。作图是观察时间序列形态的一种有效方法, 它对于进一步分析和预测有很大帮助。下面我们给出几个时间序列,并通过图形进行观察和分析。

\section*{例 13.1}
表 13-1 是 2000-2019 年我国发电量、人均 GDP、轿车产量和 CPI(居民消费价格指数)的时间序列。绘制折线图判断时间序列的成分。

表 13-1 发电量等的时间序列

\begin{center}
\begin{tabular}{ccccc}
\hline
年份 & \begin{tabular}{c}
发电量 \\
(亿千瓦时) \\
\end{tabular} & \begin{tabular}{c}
人均 GDP \\
(元) \\
\end{tabular} & \begin{tabular}{c}
轿车产量 \\
(万辆 $)$ \\
\end{tabular} & \begin{tabular}{c}
CPI \\
(上年 $=100)$ \\
\end{tabular} \\
\hline
2000 & 13556.0 & 7942 & 60.7 & 100.4 \\
2001 & 14808.0 & 8717 & 70.4 & 100.7 \\
2002 & 16540.0 & 9506 & 109.2 & 99.2 \\
2003 & 19105.8 & 10666 & 207.1 & 101.2 \\
2004 & 22033.1 & 12487 & 227.6 & 103.9 \\
2005 & 25002.6 & 14368 & 277.0 & 101.8 \\
2006 & 28657.3 & 16738 & 386.9 & 101.5 \\
2007 & 32815.5 & 20494 & 479.8 & 104.8 \\
2008 & 34668.8 & 24100 & 503.8 & 105.9 \\
2009 & 37146.5 & 26180 & 748.5 & 99.3 \\
2010 & 42071.6 & 30808 & 957.6 & 103.3 \\
2011 & 47130.2 & 36302 & 1012.7 & 105.4 \\
2012 & 49875.5 & 39874 & 1077.0 & 102.6 \\
2013 & 54316.4 & 43684 & 1210.4 & 102.6 \\
2014 & 57944.6 & 47173 & 1248.3 & 102.0 \\
2015 & 58145.7 & 50237 & 1163.0 & 101.4 \\
2016 & 61331.6 & 54139 & 1211.1 & 102.0 \\
2017 & 66044.5 & 60014 & 194.5 & 1217.4 \\
2018 & 71661.3 & 66006 & 1028.5 &  \\
2019 & 75034.3 & 70892 &  & 102.6 \\
\hline
\end{tabular}
\end{center}

资料来源:国家统计局网站.

\begin{itemize}
  \item 解 4 个时间序列的折线图如图 13-5 所示。
\end{itemize}

\begin{center}
\end{center}

年份

(a) 发电量

\begin{center}
\end{center}

年份

(c) 轿车产量

\begin{center}
\end{center}

(b) 人均GDP

\begin{center}
\end{center}

年份

(d) CPI

图 13-54 个时间序列的折线图

\subsection*{13.2.2 增长率分析}
在一些经济报道中常使用 “增长率”一词。增长率是对现象在不同时间里的变化状况所作的描述。由于对比的基期不同, 增长率有不同的计算方法。这里主要介绍增长率和平均增长率的计算方法。

\section*{1. 增长率}
增长率 (growth rate) 也称增长速度, 它是时间序列中报告期观察值与基期观察值之比减 1 后的结果, 用 $\%$ 表示。由于对比的基期不同, 增长率可以分为环比增长率和定基增长率。环比增长率是报告期观察值与前一时期观察值之比减 1 的结果, 说明现象逐期增长变化的程度; 定基增长率是报告期观察值与某一固定时期观察值之比减 1 的结果, 说明现象在整个观察期内总的增长变化程度。设增长率为 $G$, 环比增长率和定基增长率可表示为:\\
环比增长率: $G_{i}=\frac{Y_{i}-Y_{i-1}}{Y_{i-1}}=\frac{Y_{i}}{Y_{i-1}}-1, i=1,2, \cdots, n$

定基增长率: $G_{i}=\frac{Y_{i}-Y_{0}}{Y_{0}}=\frac{Y_{i}}{Y_{0}}-1, i=1,2, \cdots, n$

式中, $Y_{0}$ 表示用于对比的固定基期的观察值。

\section*{2. 平均增长率}
平均增长率 (average rate of increase) 也称平均增长速度, 它是时间序列中逐期环比值 (也称环比发展速度) 的几何平均数减 1 后的结果, 计算公式为:


\begin{equation*}
\bar{G}=\sqrt[n]{\left(\frac{Y_{1}}{Y_{0}}\right)\left(\frac{Y_{2}}{Y_{1}}\right) \cdots\left(\frac{Y_{n}}{Y_{n-1}}\right)}-1=\sqrt[n]{\frac{Y_{n}}{Y_{0}}}-1 \tag{13.4}
\end{equation*}


式中, $\bar{G}$ 表示平均增长率; $n$ 表示环比值的个数。

\section*{例 13.2}
根据表 13-1 中的发电量数据, 计算 2000-2019 年的平均增长率, 并根据平均增长率预测 2020 年和 2021 年的发电量。

根据式 (13.4) 得发电量的年平均增长率为:

$$
\bar{G}=\sqrt[n]{\frac{Y_{n}}{Y_{0}}}-1=\sqrt[19]{\frac{175034.3}{13556.0}}-1=9.42 \%
$$

2020 年和 2021 年的发电量预测值分别为:

$$
\begin{aligned}
\hat{Y}_{2020} & =2019 \text { 年发电量 } \times(1+\text { 年平均增长率 })=75034.3 \times(1+9.42 \%) \\
& =82102.53(\text { 亿千瓦时) } \\
\hat{Y}_{2021} & =2019 \text { 年发电量 } \times(1+\text { 年平均增长率 })^{2}=75034.3 \times(1+9.42 \%)^{2} \\
& =89836.59(\text { 亿千瓦时) }
\end{aligned}
$$

\section*{3. 增长率分析中应注意的问题}
对于大多数时间序列, 特别是有关社会经济现象的时间序列, 经常利用增长率来描述其增长状况。尽管增长率的计算与分析都比较简单, 但实际应用中有时会出现误用乃至滥用的情况。因此, 在应用增长率分析实际问题时应注意以下几点:

第一, 当时间序列中的观察值出现 0 或负数时, 不宜计算增长率。例如, 假定某企业连续 5 年的利润额 (单位: 万元) 分别为 $5,2,0,-3,2$, 对这一序列计算增长率, 要么不符合数学公理, 要么无法解释其实际意义。在这种情况下, 适宜直接用绝对数进行分析。

第二, 在有些情况下, 不能单纯就增长率论增长率, 要注意将增长率与绝对水平结合起来分析。先看一个例子。

\section*{例 13.3}
假定甲、乙两个企业各年的利润额及增长率数据如表 13-2 所示。

表 13-2 甲、乙两个企业的有关数据

\begin{center}
\begin{tabular}{cccccc}
\hline
\multirow{2}{*}{年份} & \multicolumn{2}{c}{甲企业} &  & \multicolumn{2}{c}{乙企业} \\
\cline { 2 - 3 }\cline { 5 - 6 }
 & 利润额(万元) & 增长率 $(\%)$ &  & 利润额(万元) & 增长率 $(\%)$ \\
\hline
上年 & 500 & - &  & 60 & - \\
本年 & 600 & 20 & 84 & 40 &  \\
\hline
\end{tabular}
\end{center}

一解 如果不看利润额的绝对值, 仅就增长率对甲、乙两个企业进行分析, 可以看出乙企业的利润增长率比甲企业高出一倍。如果就此得出乙企业的经营业绩比甲企业好得多这样的结论, 就是不切实际的。因为增长率是一个相对值, 它与对比的基期值的大小有很大关系。大的增长率背后, 其隐含的绝对值可能很小, 小的增长率背后, 其隐含的绝对值可能很大。这就是说, 由于对比的基点不同, 可能会造成增长率数值的较大差异。

本例表明, 两个企业的生产起点不同, 基期的利润额不同, 造成了二者增长率的较大差异。从利润的绝对额来看, 两个企业的增长率每增长一个百分点所增加的利润绝对额是不同的。在这种情况下, 需要将增长率与绝对数量结合起来进行分析, 通常要计算增长 $1 \%$ 的绝对值来克服增长率分析的局限性。

增长 $1 \%$ 的绝对值表示增长率每增长一个百分点而增加的绝对数量, 其计算公式为:


\begin{equation*}
\text { 增长 } 1 \% \text { 的绝对值 }=\frac{\text { 前期水平 }}{100} \tag{13.5}
\end{equation*}


根据表 13-2 的数据计算, 甲企业利润增长一个百分点增加的利润额为 5 万元, 而乙企业则为 0.6 万元, 甲企业远高于乙企业。这说明甲企业的经营业绩并不比乙企业差, 而是更好。

\section*{13.3}
预测方法的选择时间序列分析的一个主要目的就是根据已有的历史数据对未来进行预测。在对时间序列进行预测时, 首先可以通过绘制折线图来观察时间序列所包含的成分, 然后找出适合此类时间序列的预测方法, 再对可能的预测方法进行评估, 以确定最佳预测方案进行预测。

\subsection*{13.3. 1 选择预测方法}
确定了时间序列的成分后, 接下来就是找出适合此时间序列的预测方法。利用时间序列数据进行预测时, 通常假定过去的变化趋势会延续到未来, 这样就可以根据过去已有的形态或模式进行预测。时间序列的预测方法既有传统方法, 如简单平均法、移动平均法、指数平滑法等, 也有较为精准的现代方法, 如 Box-Jenkins 的自回归模型 (ARMA)。\\
一般来说, 任何时间序列中都会有不规则成分存在, 而商务与管理数据中通常不考虑周期性, 所以只剩下趋势成分和季节成分。本章所介绍的预测方法主要是针对平稳序列以及含有趋势或季节成分的时间序列。图 13-6 给出了时间序列的类型和可供选择的预测方法。

\begin{center}
\end{center}

图 13-6 时间序列的类型和预测方法的选择

不含趋势和季节成分的时间序列, 即平稳时间序列只含随机成分, 只要通过平滑就可以消除随机波动, 因此, 这类预测方法也称为平滑预测法, 将在 13.4 节中讨论。对于只含有趋势成分的时间序列, 可以采用趋势预测法, 将在 13.5 节中讨论。对于既含有趋势又含有季节成分的时间序列, 则采用季节性预测法, 该方法对于既含有季节成分也含有随机成分的时间序列同样适用, 将在 13.6 节中介绍。

\subsection*{13.3. 2 预测方法的评估}
在选择某种特定的方法进行预测时, 需要评价该方法的预测效果或准确性。评价的方法就是找出预测值与实际值的差距, 这个差值就是预测误差。最优的预测方法也就是使预测误差最小的方法。预测误差的计算方法有几种, 包括平均误差、平均绝对误差、均方误差、平均百分比误差和平均绝对百分比误差等。选择哪种方法取决于预测者的目标、对方法的熟悉程度等。

\begin{enumerate}
  \item 平均误差
\end{enumerate}

设时间序列的第 $i$ 个观察值为 $Y_{i}$, 预测值为 $F_{i}$, 所有预测误差 $Y_{i}-F_{i}$ 的平均数就是平均误差(mean error), 用 $M E$ 表示, 其计算公式为:


\begin{equation*}
M E=\frac{\sum_{i=1}^{n}\left(Y_{i}-F_{i}\right)}{n} \tag{13.6}
\end{equation*}


式中, $n$ 为预测值的个数。

由于预测误差的数值可能有正有负, 求和的结果就会相互抵消, 在这种情况下, 平均误差可能会低估误差。

\section*{2. 平均绝对误差}
平均绝对误差 (mean absolute deviation) 是将预测误差取绝对值后计算的平均误差,用 $M A D$ 表示,其计算公式为:


\begin{equation*}
M A D=\frac{\sum_{i=1}^{n}\left|Y_{i}-F_{i}\right|}{n} \tag{13.7}
\end{equation*}


平均绝对误差可以避免误差相互抵消的问题, 因而可以准确反映实际预测误差的大小。

\section*{3. 均方误差}
均方误差(mean square error)是通过平方消去误差的正负号后计算的平均误差, 用 $M S E$ 表示, 其计算公式为:


\begin{equation*}
M S E=\frac{\sum_{i=1}^{n}\left(Y_{i}-F_{i}\right)^{2}}{n} \tag{13.8}
\end{equation*}


\section*{4. 平均百分比误差和平均绝对百分比误差}
$M E, M A D$ 和 $M S E$ 的大小受时间序列数据的水平和计量单位的影响, 有时并不能真正反映预测模型的好坏, 它们只有在比较不同模型对同一数据的预测时才有意义。而平均百分比误差 (mean percentage error) 和平均绝对百分比误差 (mean absolute percentage error) 则不同, 它们消除了时间序列数据的水平和计量单位的影响, 是反映误差大小的相对值。平均百分比误差用 $M P E$ 表示, 其计算公式为:


\begin{equation*}
M P E=\frac{\sum_{i=1}^{n}\left(\frac{Y_{i}-F_{i}}{Y_{i}} \times 100\right)}{n} \tag{13.9}
\end{equation*}


平均绝对百分比误差用 $M A P E$ 表示, 其计算公式为:


\begin{equation*}
M A P E=\frac{\sum_{i=1}^{n}\left(\frac{\left|Y_{i}-F_{i}\right|}{Y_{i}} \times 100\right)}{n} \tag{13.10}
\end{equation*}


上面介绍的预测误差的计算方法中哪种最优, 还没有一致的看法。本章采用均方误差 (MSE) 来评价预测方法的优劣。

\section*{13.4}
\section*{平稳序列的预测}
平稳时间序列通常只含有随机成分, 其预测方法主要有简单平均法、移动平均法和指数平滑法等, 这些方法主要是通过对时间序列进行平滑以消除随机波动, 因而也称为平滑法。平滑法既可用于对平稳时间序列进行短期预测,也可用于对时间序列进行平滑以描述序列的趋势(包括线性趋势和非线性趋势)。

\subsection*{13.4. 1 简单平均法}
简单平均法是根据已有的 $t$ 期观察值通过简单平均来预测下一期的数值。设时间序列已有的 $t$ 期观察值分别为 $Y_{1}, Y_{2}, \cdots, Y_{t}$, 则 $t+1$ 期的预测值 $F_{t+1}$ 为:


\begin{equation*}
F_{t+1}=\frac{1}{t}\left(Y_{1}+Y_{2}+\cdots+Y_{t}\right)=\frac{1}{t} \sum_{i=1}^{t} Y_{i} \tag{13.11}
\end{equation*}


到了 $t+1$ 期后,有了 $t+1$ 期的实际值,便可计算出 $t+1$ 期的预测误差 $e_{t+1}$ :


\begin{equation*}
e_{t+1}=Y_{t+1}-F_{t+1} \tag{13.12}
\end{equation*}


于是, $t+2$ 期的预测值为:


\begin{equation*}
F_{t+2}=\frac{1}{t+1}\left(Y_{1}+Y_{2}+\cdots+Y_{t}+Y_{t+1}\right)=\frac{1}{t+1} \sum_{i=1}^{t+1} Y_{i} \tag{13.13}
\end{equation*}


依此类推。

\section*{例 13.4}
根据表 13-1 中的 CPI 数据, 采用简单平均法预测 2020 年的 CPI。

根据式 (13.11) 得

$$
F_{2020}=\frac{1}{20} \sum_{i=1}^{20} Y_{i}=\frac{1}{20}(100.4 \%+100.7 \%+\cdots+102.9 \%)=102.23 \%
$$

简单平均法适合对较为平稳的时间序列进行预测, 即当时间序列没有趋势时, 用该方法比较好。但如果时间序列有趋势或季节成分, 该方法的预测则不够准确。此外, 简单平均法将远期的数值和近期的数值看作对未来同等重要。但从预测角度看, 近期的数值比远期的数值对未来有更大的作用,因此简单平均法预测的结果不够准确。

\subsection*{13.4. 2 移动平均法}
移动平均法 (moving average) 是通过对时间序列逐期递移求得平均数作为预测值的一种预测方法, 其方法有简单移动平均法 (simple moving average) 和加权移动平均法 (weighted moving average)两种。这里主要介绍简单移动平均法。

简单移动平均是将最近的 $k$ 期数据加以平均, 作为下一期的预测值。设移动间隔为 $k$ $(1<k<t)$ ,则 $t$ 期的移动平均值为:


\begin{equation*}
\bar{Y}_{t}=\frac{Y_{t-k+1}+Y_{t-k+2}+\cdots+Y_{t-1}+Y_{t}}{k} \tag{13.14}
\end{equation*}


式 (13.14) 是对时间序列的平滑结果, 通过这些平滑值可以描述出时间序列的变化形态或趋势。当然, 也可以用它们来进行预测。

$t+1$ 期的简单移动平均预测值为:


\begin{equation*}
F_{t+1}=\bar{Y}_{t}=\frac{Y_{t-k+1}+Y_{t-k+2}+\cdots+Y_{t-1}+Y_{t}}{k} \tag{13.15}
\end{equation*}


同样, $t+2$ 期的预测值为:


\begin{equation*}
F_{t+2}=\bar{Y}_{t+1}=\frac{Y_{t-k+2}+Y_{t-k+3}+\cdots+Y_{t}+Y_{t+1}}{k} \tag{13.16}
\end{equation*}


依此类推。

移动平均法只使用最近 $k$ 期的数据, 在每次计算移动平均值时, 移动的间隔都为 $k$ 。该方法也适合对较为平稳的时间序列进行预测。应用时, 关键是确定合理的移动间隔 $k$ 。对于同一个时间序列, 采用不同的移动间隔, 预测的准确性是不同的。可通过试验选择一个使均方误差最小的移动间隔。

\section*{例 13.5}
根据表 13-1 中的 CPI 数据, 分别取移动间隔 $k=3$ 和 $k=5$, 预测历史各年份和 2020 年的 CPI, 计算出预测误差, 并将原序列和预测后的序列绘制成图形进行比较。

解 采用 Excel 进行移动平均预测时, 在【数据分析】选项中选择【移动平均】, 并在

\begin{center}
\begin{tabular}{|c|c|c|c|c|c|c|c|}
\hline
年份 & CPI & \begin{tabular}{c}
移动平均预测 \\
$k=3$ \\
\end{tabular} & 预测误差 & 误差平方 & \begin{tabular}{c}
移动平均预测 \\
$k=5$ \\
\end{tabular} & 预测误差 & 误差平方 \\
\hline
2000 & 100.4 & $\# \mathrm{~N} / \mathrm{A}$ & $\# \mathrm{~N} / \mathrm{A}$ & $\# \mathrm{~N} / \mathrm{A}$ & $\# \mathrm{~N} / \mathrm{A}$ & $\# \mathrm{~N} / \mathrm{A}$ & $\#$ N/A \\
\hline
2001 & 100.7 & $\# \mathrm{~N} / \mathrm{A}$ & $\# \mathrm{~N} / \mathrm{A}$ & $\# \mathrm{~N} / \mathrm{A}$ & $\# \mathrm{~N} / \mathrm{A}$ & $\#$ N/A & $\# \mathrm{~N} / \mathrm{A}$ \\
\hline
2002 & 99.2 & $\# \mathrm{~N} / \mathrm{A}$ & $\# \mathrm{~N} / \mathrm{A}$ & $\# \mathrm{~N} / \mathrm{A}$ & $\# \mathrm{~N} / \mathrm{A}$ & $\# \mathrm{~N} / \mathrm{A}$ & $\# \mathrm{~N} / \mathrm{A}$ \\
\hline
2003 & 101.2 & 100.1 & 1. 10 & 1.21 & $\# \mathrm{~N} / \mathrm{A}$ & $\# \mathrm{~N} / \mathrm{A}$ & $\#$ N/A \\
\hline
2004 & 103.9 & 100.4 & 3. 53 & 12. 48 & $\# \mathrm{~N} / \mathrm{A}$ & $\# \mathrm{~N} / \mathrm{A}$ & $\#$ N/A \\
\hline
2005 & 101.8 & 101.4 & 0.37 & 0.13 & 100.08 & 0.72 & 0.52 \\
\hline
2006 & 101.5 & 102.3 & -0.80 & 0.64 & 101.36 & 0.14 & 0.02 \\
\hline
2007 & 104.8 & 102.4 & 2. 40 & 5. 76 & 101.52 & 3.28 & 10.76 \\
\hline
2008 & 105.9 & 102.7 & 3. 20 & 10.24 & 102.64 & 3.26 & 10.63 \\
\hline
\end{tabular}
\end{center}

对话框中输入数据区域和移动间隔即可。输出的结果如表 13-3 所示。

表 13-3 CPI 的移动平均预测(1)

(1) 使用 Excel 进行移动平均预测时, 为了使预测值与实际值相对应, 在选择输出区域时, 应将输出区域的第一个单元格设置在第一个数值的下一行。

\begin{center}
\begin{tabular}{rrrrrrrr}
\hline
年份 & CPI & \begin{tabular}{c}
移动平均预测 \\
$k=3$ \\
\end{tabular} & 预测误差 & 误差平方 & \begin{tabular}{c}
移动平均预测 \\
$k=5$ \\
\end{tabular} & 预测误差 & 误差平方 \\
\hline
2009 & 99.3 & 104.1 & -4.77 & 22.72 & 103.58 & -4.28 & 18.32 \\
2010 & 103.3 & 103.3 & -0.03 & 0.00 & 102.66 & 0.64 & 0.41 \\
2011 & 105.4 & 102.8 & 2.57 & 6.59 & 102.96 & 2.44 & 5.95 \\
2012 & 102.6 & 102.7 & -0.07 & 0.00 & 103.74 & -1.14 & 1.30 \\
2013 & 102.6 & 103.8 & -1.17 & 1.36 & 103.30 & -0.70 & 0.49 \\
2014 & 102.0 & 103.5 & -1.53 & 2.35 & 102.64 & -0.64 & 0.41 \\
2015 & 101.4 & 102.4 & -1.00 & 1.00 & 103.18 & -1.78 & 3.17 \\
2016 & 102.0 & 102.0 & 0.00 & 0.00 & 102.80 & -0.80 & 0.64 \\
2017 & 101.6 & 101.8 & -0.20 & 0.04 & 102.12 & -0.52 & 0.27 \\
2018 & 102.1 & 101.7 & 0.43 & 0.19 & 101.92 & 0.18 & 0.03 \\
2019 & 102.9 & 101.9 & 1.00 & 1.00 & 101.82 & 1.08 & 1.17 \\
2020 & - & 102.2 & - & - & 102.00 & - & - \\
合计 & - & - & - & 65.72 & - & - & 54.08 \\
\hline
\end{tabular}
\end{center}

以 3 期移动平均为例, 表 $13-3$ 中的 102.2 就是 2017 年、2018 年和 2019 年 3 年的平均值, 用它作为 2020 年的预测值。从预测效果看, 3 期移动平均的均方误差为 3.87 $(65.72 \div 17)$, 而 5 期移动平均的均方误差为 $3.61(54.08 \div 15)$ 。因此, 就本序列而言,采用 3 期移动平均和 5 期移动平均, 预测的效果相差不大。

各年 CPI 的走势及其预测值如图 13-7 所示。

\begin{center}
\end{center}

图 13-7 CPI 的 3 期移动平均和 5 期移动平均的预测

\subsection*{13.4.3 指数平滑法}
指数平滑法 (exponential smoothing) 是通过对过去的观察值加权平均进行预测的一种方法,该方法使 $t+1$ 期的预测值等于 $t$ 期的实际观察值与 $t$ 期的预测值的加权平均值。指数平滑法是加权平均的一种特殊形式,观察值的时间越久远,其权数呈现指数下降,因而称为指数平滑。指数平滑法有一次指数平滑法、二次指数平滑法、三次指数平滑法等,本节主要介绍一次指数平滑法。

一次指数平滑法也称单一指数平滑法 (single exponential smoothing), 它只有一个平滑系数, 而且观察值离预测时期越久远, 权数越小。一次指数平滑法将一段时期内的预测值与观察值的线性组合作为 $t+1$ 期的预测值, 其预测模型为:


\begin{equation*}
F_{t+1}=\alpha Y_{t}+(1-\alpha) F_{t} \tag{13.17}
\end{equation*}


式中, $Y_{t}$ 为 $t$ 期的实际观察值; $F_{t}$ 为 $t$ 期的预测值; $\alpha(0<\alpha<1)$ 为平滑系数。

由式 (13.17) 可以看出, $t+1$ 期的预测值是 $t$ 期的实际观察值与 $t$ 期的预测值的加权平均值。由于开始计算时还没有 1 期的预测值 $F_{1}$, 通常可以设 $F_{1}$ 等于 1 期的实际观察值,即 $F_{1}=Y_{1}$ 。因此 2 期的预测值为:

$$
F_{2}=\alpha Y_{1}+(1-\alpha) F_{1}=\alpha Y_{1}+(1-\alpha) Y_{1}=Y_{1}
$$

3 期的预测值为:

$$
F_{3}=\alpha Y_{2}+(1-\alpha) F_{2}=\alpha Y_{2}+(1-\alpha) Y_{1}
$$

4 期的预测值为:

$$
F_{4}=\alpha Y_{3}+(1-\alpha) F_{3}=\alpha Y_{3}+\alpha(1-\alpha) Y_{2}+(1-\alpha)^{2} Y_{1}
$$

依此类推。可见任何预测值 $F_{t+1}$ 都是以前所有的实际观察值的加权平均。尽管如此, 并非所有过去的观察值都需要保留, 以计算下一期的预测值。实际上, 一旦选定平滑系数 $\alpha$,只需要两项信息就可以计算预测值。式 (13.17) 表明, 只要知道 $t$ 期的实际观察值 $Y_{t}$ 与 $t$期的预测值 $F_{t}$, 就可以计算 $t+1$ 期的预测值。

对指数平滑法的预测精度, 同样用均方误差来衡量。为此, 将式 (13.17) 写成下面的形式:


\begin{align*}
F_{t+1} & =\alpha Y_{t}+(1-\alpha) F_{t} \\
& =\alpha Y_{t}+F_{t}-\alpha F_{t} \\
& =F_{t}+\alpha\left(Y_{t}-F_{t}\right) \tag{13.18}
\end{align*}


可见, $F_{t+1}$ 是 $t$ 期的预测值 $F_{t}$ 加上用 $\alpha$ 调整的 $t$ 期的预测误差 $Y_{t}-F_{t}$ 。

使用指数平滑法时, 关键的问题是确定一个合适的平滑系数 $\alpha$, 因为不同的 $\alpha$ 会对预测结果产生不同的影响。例如, 当 $\alpha=0$ 时, 预测值仅仅是重复上一期的预测结果; 当 $\alpha=$ 1 时, 预测值就是上一期的实际值。预测时, 可选择不同的 $\alpha$, 然后找出预测误差最小的作为最后的 $\alpha$ 值。

此外, 与移动平均法一样, 一次指数平滑法也可以用于对时间序列进行修匀, 以消除随机波动, 找出序列的变化趋势。

\section*{例 13.6}
根据表 13-1 中的 CPI 数据, 选择平滑系数 $\alpha=0.3$ 和 $\alpha=0.5$, 用指数平滑法预测历史各年份的 CPI 和 2020 年的 CPI, 计算出预测误差, 并将原序列和预测后的序列绘制成图形进行比较。

用 Excel 进行指数平滑预测的步骤如下:

太用 Excel 进行指数平滑预测

第 1 步: 点击【数据】 $\rightarrow$ 【数据分析】。在出现的对话框中选择【指数平滑】, 点击【确定】。

第 2 步: 在出现的对话框中, 在【输入区域】中输入要预测的数据所在的区域。在【阻尼系数】中输入 $1-\alpha$ 的值 (如果 $\alpha=0.3$, 则阻尼系数 $=0.7$ )。在【输出区域】中选择结果的输出位置(选择与第 1 期数值对应的右侧单元格)。选择【图表输出】。点击【确定】。

表 13-4 是用 $\alpha=0.3, \alpha=0.5$ 对各期 CPI 进行平滑预测的结果, 并给出了预测误差以及 2020 年的预测值。

表 13-4 CPI 的指数平滑预测

\begin{center}
\begin{tabular}{rrrrrrrr}
\hline
年份 & CPI & \begin{tabular}{c}
指数平滑预测 \\
$\alpha=0.3$ \\
\end{tabular} & 预测误差 & 误差平方 & \begin{tabular}{c}
指数平滑预测 \\
$\alpha=0.5$ \\
\end{tabular} & 预测误差 & 误差平方 \\
\hline
2000 & 100.4 & $\# \mathrm{~N} / \mathrm{A}$ & \# N/A & \# N/A & \# N/A & \# N/A & \# N/A \\
2001 & 100.7 & 100.40 & 0.30 & 0.09 & 100.40 & 0.30 & 0.09 \\
2002 & 99.2 & 100.49 & -1.29 & 1.66 & 100.55 & -1.35 & 1.82 \\
2003 & 101.2 & 100.10 & 1.10 & 1.20 & 99.88 & 1.33 & 1.76 \\
2004 & 103.9 & 100.43 & 3.47 & 12.03 & 100.54 & 3.36 & 11.31 \\
2005 & 101.8 & 101.47 & 0.33 & 0.11 & 102.22 & -0.42 & 0.18 \\
2006 & 101.5 & 101.57 & -0.07 & 0.01 & 102.01 & -0.51 & 0.26 \\
2007 & 104.8 & 101.55 & 3.25 & 10.57 & 101.75 & 3.05 & 9.27 \\
2008 & 105.9 & 102.52 & 3.38 & 11.39 & 103.28 & 2.62 & 6.88 \\
2009 & 99.3 & 103.54 & -4.24 & 17.95 & 104.59 & -5.29 & 27.97 \\
2010 & 103.3 & 102.27 & 1.03 & 1.07 & 101.94 & 1.36 & 1.84 \\
2011 & 105.4 & 102.58 & 2.82 & 7.97 & 102.62 & 2.78 & 7.72 \\
2012 & 102.6 & 103.42 & -0.82 & 0.68 & 104.01 & -1.41 & 1.99 \\
2013 & 102.6 & 103.18 & -0.58 & 0.33 & 103.31 & -0.71 & 0.50 \\
2014 & 102.0 & 103.00 & -1.00 & 1.01 & 102.95 & -0.95 & 0.91 \\
2015 & 101.4 & 102.70 & -1.30 & 1.70 & 102.48 & -1.08 & 1.16 \\
\hline
 &  &  &  &  &  &  &  \\
\hline
\end{tabular}
\end{center}

\begin{center}
\begin{tabular}{rrrrrrrr}
\hline
年份 & CPI & \begin{tabular}{c}
指数平滑预测 \\
$\alpha=0.3$ \\
\end{tabular} & 预测误差 & 误差平方 & \begin{tabular}{c}
指数平滑预测 \\
$\alpha=0.5$ \\
\end{tabular} & 预测误差 & 误差平方 \\
\hline
2016 & 102.0 & 102.31 & -0.31 & 0.10 & 101.94 & 0.06 & 0.00 \\
2017 & 101.6 & 102.22 & -0.62 & 0.38 & 101.97 & -0.37 & 0.14 \\
2018 & 102.1 & 102.03 & 0.07 & 0.00 & 101.78 & 0.32 & 0.10 \\
2019 & 102.9 & 102.05 & 0.85 & 0.72 & 101.94 & 0.96 & 0.92 \\
2020 & - & 102.31 & - & - & 102.42 & - & - \\
合计 & - & - & - & 68.97 & - & - & 74.80 \\
\hline
\end{tabular}
\end{center}

比较各误差平方可知, $\alpha=0.3$ 时预测效果较好。但用简单指数平滑法进行预测时, 一般 $\alpha$ 值不大于 0.5。若 $\alpha$ 大于 0.5 才能接近实际值, 通常说明序列有某种趋势或波动过大,这时不适合用指数平滑法进行预测。不同 $\alpha$ 下的预测值和实际观察值的图形如图 13-8 所示。

\begin{center}
\end{center}

图 13-8 CPI 的指数平滑预测

\section*{13.5 趋势型序列的预测}
上面介绍的平滑法都可以用于描述时间序列的趋势, 包括线性趋势和非线性趋势。当用这些方法进行预测时, 要注意它们一般只适合平稳时间序列, 当序列存在明显的趋势或季节成分时, 这些方法就不再适用。本节将介绍含有趋势成分的时间序列的预测方法。\\
时间序列的趋势可以分为线性趋势和非线性趋势两大类,如果这种趋势能够延续到未来, 就可以利用趋势进行外推预测。有趋势序列的预测方法主要有线性趋势预测、非线性趋势预测和自回归模型预测等。本节主要介绍线性趋势和非线性趋势的预测方法。

\subsection*{13.5. 1 线性趋势预测}
线性趋势 (linear trend) 是指现象随着时间的推移而呈现出稳定增长或下降的线性变化规律。例如, 观察图 13-5 中的发电量序列 (见图 13-5 (a)), 就有明显的线性趋势。如果这种趋势能延续到未来, 就可以利用这种趋势预测未来的发电产量。

当现象按线性趋势发展变化时, 可以用下列线性趋势方程来描述:


\begin{equation*}
\hat{Y}_{t}=b_{0}+b_{1} t \tag{13.19}
\end{equation*}


式中, $Y_{t}$ 代表时间序列 $Y_{t}$ 的预测值; $t$ 代表时间; $b_{0}$ 代表趋势线在 $Y$ 轴上的截距, 是当 $t=0$ 时 $\hat{Y}_{t}$ 的数值; $b_{1}$ 是趋势线的斜率, 表示时间 $t$ 变动一个单位, 观察值的平均变动数量。

趋势方程中的两个待定系数 $b_{0}$ 和 $b_{1}$ 通常按回归中的最小二乘法求得(该方法已在第 11 章中作了介绍, 这里不再赘述)。根据最小二乘法得到求解 $b_{0}$ 和 $b_{1}$ 的公式如下:

\[
\left\{\begin{array}{l}
b_{1}=\frac{n \sum t Y-\sum t \sum Y}{n \sum t^{2}-\left(\sum t\right)^{2}}  \tag{13.20}\\
b_{0}=\bar{Y}-b_{1} t
\end{array}\right.
\]

通过趋势方程可以计算出各期的预测值, 然后通过这些预测值来分析序列的变化趋势及模式。此外, 也可以利用趋势方程进行外推预测。趋势预测的误差可用线性回归中的估计标准误差来衡量, 其计算公式为:


\begin{equation*}
s_{e}=\sqrt{\frac{\sum_{i=1}^{n}\left(Y_{i}-\hat{Y}_{i}\right)^{2}}{n-m}} \tag{13.21}
\end{equation*}


式中, $m$ 为趋势方程中待确定的未知数的个数, 对于直线趋势方程, $m=2$ 。

\section*{例 13.7}
沿用例 13.1, 根据最小二乘法确定发电量的线性趋势方程, 计算各期的预测值, 预测 2020 年的发电量, 并将原序列和各期的预测值序列绘制成图形进行比较。

一 解 令年份编号为 $t_{i}(i=1,2, \cdots, 20)$, 根据最小二乘法求得的线性趋势方程为:

$$
\hat{Y}_{t}=6444.178+3328.599 t
$$

$b_{1}=3328.599$, 表示时间每增加一年, 发电量平均增加 3328.599 亿千瓦时。将 $t=1,2$, $\cdots, 20$ 代入趋势方程得到各期的预测值(见表 13-5)。预测的估计标准误差为 $s_{e}=1678.47$ 。将 $t=21$ 代入趋势方程即可得到 2020 年发电量的预测值, 即

$$
\hat{Y}_{21}=6444.178+3328.599 \times 21=76344.75 \text { (亿千瓦时) }
$$

\begin{center}
\begin{tabular}{cccrr}
\hline
年份 & 时间代码 & \begin{tabular}{c}
发电量 \\
(亿千瓦时) \\
\end{tabular} & 预测值 & \multicolumn{1}{c}{残差} \\
\hline
2000 & 1 & 13556.0 & 9772.78 & 3783.22 \\
2001 & 2 & 14808.0 & 13101.38 & 1706.62 \\
2002 & 3 & 16540.0 & 16429.97 & 110.03 \\
2003 & 4 & 19105.8 & 19758.57 & -652.77 \\
2004 & 5 & 22033.1 & 23087.17 & -1054.07 \\
2005 & 6 & 25002.6 & 26415.77 & -1413.17 \\
2006 & 7 & 28657.3 & 29744.37 & -1087.07 \\
2007 & 8 & 32815.5 & 33072.97 & -257.47 \\
2008 & 9 & 34668.8 & 36401.57 & -1732.77 \\
2009 & 10 & 37146.5 & 39730.17 & -2583.67 \\
2010 & 11 & 42071.6 & 43058.76 & -987.16 \\
2011 & 12 & 47130.2 & 46387.36 & 742.84 \\
2012 & 13 & 49875.5 & 49715.96 & 159.54 \\
2013 & 14 & 54316.4 & 53044.56 & 1271.84 \\
2014 & 15 & 57944.6 & 56373.16 & 1571.44 \\
2015 & 16 & 58145.7 & 59701.76 & -1556.06 \\
2016 & 17 & 61331.6 & 63030.36 & -1698.76 \\
2017 & 18 & 66044.5 & 66358.96 & -314.46 \\
2018 & 19 & 71661.3 & 69687.55 & 1973.75 \\
2019 & 20 & 75034.3 & 73016.15 & 2018.15 \\
2020 & 21 & - & 76344.75 & - \\
\hline
\end{tabular}
\end{center}

将各期的预测值序列与原序列绘制成图 13-9, 可以看出发电量的变化趋势。

\begin{center}
\end{center}

\begin{center}
\end{center}

年份

$\rightarrow$ 一发电量 一-预测值

图 13-9 发电量的线性趋势预测图

\subsection*{13.5.2 非线性趋势预测}
序列中的趋势通常认为是由于某种固定的因素作用于同一方向而形成的。若这些因素随着时间的推移呈现线性变化, 则可以对时间序列拟合趋势直线; 若呈现出某种非线性趋势 (non-linear trend), 则需要拟合适当的趋势曲线。例如, 图 13-5 (b) 和图 13-5 (c)就有明显的非线性趋势。下面介绍几种常用的趋势曲线。

\section*{1. 指数曲线}
指数曲线 (exponential curve) 用于描述以几何级数递增或递减的现象, 即时间序列的观察值 $Y_{t}$ 按指数规律变化, 或者说时间序列的逐期观察值按一定的比率增长或衰减。图 13-5 (b) 中的人均 GDP 就呈现出某种指数形态。一般的自然增长及大多数经济序列都有指数变化趋势。指数曲线的趋势方程为:


\begin{equation*}
\hat{Y}_{t}=b_{0} b_{1}^{t} \tag{13.22}
\end{equation*}


式中, $b_{0}, b_{1}$ 为待定系数。

若 $b_{1}>1$, 则增长率随着时间 $t$ 的增加而增加; 若 $b_{1}<1$, 则增长率随着时间 $t$ 的增加而降低; 若 $b_{0}>0, b_{1}<1$, 则预测值 $\hat{Y}_{t}$ 逐渐降低到以 0 为限。

为确定指数曲线中的常数 $b_{0}$ 和 $b_{1}$, 可采取线性化手段将其化为对数直线形式, 即两端取对数得:


\begin{equation*}
\lg \hat{Y}_{t}=\lg b_{0}+t \lg b_{1} \tag{13.23}
\end{equation*}


然后根据最小二乘法原理, 按直线形式的常数确定方法, 得到求解 $\lg b_{0}$ 和 $\lg b_{1}$ 的标准方程如下:

\[
\left\{\begin{array}{l}
\sum \lg Y=n \lg b_{0}+\lg b_{1} \sum t  \tag{13.24}\\
\sum \operatorname{tg} Y=\lg b_{0} \sum t+\lg b_{1} \sum t^{2}
\end{array}\right.
\]

求出 $\lg b_{0}$ 和 $\lg b_{1}$ 后, 再取其反对数, 即得到 $b_{0}$ 和 $b_{1}$ 。

\section*{例 13.8}
沿用例 13.1。根据人均 GDP 数据, 确定指数曲线方程, 计算出各期的预测值, 预测 2020 年的人均 GDP, 并将原序列和各期的预测值序列绘制成图形进行比较。

一解 首先将指数曲线线性化, 再按线性回归求得指数曲线方程:

$$
\hat{Y}_{t}=7290.749 \times 1.130^{t}
$$

将 $t=1,2, \cdots, 20$ 代入趋势方程得到各期的预测值, 将 $t=21$ 代入趋势方程即可得到 2020 年人均 GDP 的预测值。有关输出结果见表 13-6。\\
表 13-6 人均 GDP 的指数趋势预测

\begin{center}
\begin{tabular}{ccccc}
\hline
年份 & 时间代码 & 人均 GPD (元) & 预测值 & 残差 \\
\hline
2000 & 1 & 7942 & 8144.95 & -202.95 \\
2001 & 2 & 8717 & 9201.46 & -484.46 \\
2002 & 3 & 9506 & 10395.01 & -889.01 \\
2003 & 4 & 10666 & 11743.38 & -1077.38 \\
2004 & 5 & 12487 & 13266.65 & -779.65 \\
2005 & 6 & 14368 & 14987.51 & -619.51 \\
2006 & 7 & 16738 & 16931.59 & -193.59 \\
2007 & 8 & 20494 & 19127.85 & 1366.15 \\
2008 & 9 & 24100 & 21608.98 & 2491.02 \\
2009 & 10 & 26180 & 24411.96 & 1768.04 \\
2010 & 11 & 30808 & 27578.51 & 3229.49 \\
2011 & 12 & 36302 & 31155.81 & 5146.19 \\
2012 & 13 & 39874 & 35197.14 & 4676.86 \\
2013 & 14 & 43684 & 39762.68 & 3921.32 \\
2014 & 15 & 47173 & 44920.42 & 2252.58 \\
2015 & 16 & 50237 & 50747.20 & -510.20 \\
2016 & 17 & 54139 & 57329.79 & -3190.79 \\
2017 & 18 & 60014 & 64766.23 & -4752.23 \\
2018 & 19 & 66006 & 73167.27 & -7161.27 \\
2019 & 20 & 70892 & 82658.04 & -11766.04 \\
2020 & 21 &  & 93379.88 &  \\
\hline
\end{tabular}
\end{center}

将各期预测值及原序列绘成图 13-10, 可以看出人均 GDP 的变化趋势。

\begin{center}
\end{center}

年份

$\rightarrow$ 人均GDP $\rightarrow$ 预测值

图 13-10 人均 GDP 的指数趋势预测\\
指数曲线在描述序列的趋势形态时, 比一般的趋势直线有更广泛的应用。因为它可以反映出现象的相对发展变化程度, 对不同序列的指数曲线进行比较, 以分析各自的相对增长程度。

\section*{2. 多阶曲线}
有些现象的变化形态比较复杂, 它们不是按照某种固定的形态变化, 而是有升有降,在变化过程中可能有几个拐点。这时就需要拟合多项式函数。只有一个拐点时, 可以拟合二阶曲线, 即抛物线; 有两个拐点时, 需要拟合三阶曲线; 有 $k-1$ 个拐点时, 需要拟合 $k$阶曲线。 $k$ 阶曲线函数的一般形式为:


\begin{equation*}
\hat{Y}_{t}=b_{0}+b_{1} t+b_{2} t^{2}+\cdots+b_{k} t^{k} \tag{13.25}
\end{equation*}


函数中的系数 $b_{0}, b_{1}, b_{2}, \cdots, b_{k}$ 仍然可以根据最小二乘法求得, 只需要将式 (13.25)线性化, 即可按多元回归分析中的最小二乘法求得。

\section*{例 13.9}
沿用例 13.1。根据轿车产量数据, 拟合适当的趋势曲线, 计算出各期的预测值, 预测 2020 年的轿车产量, 并将原序列和各期的预测值序列绘制成图形进行比较。

从图 13-5 (c) 可以看出, 轿车产量的变化有 1 个明显的拐点, 因此可拟合二阶曲线 (即抛物线) 函数,也可考虑拟合三阶曲线函数进行预测。这里分别拟合二阶曲线和三阶曲线, 以便进行比较。

将式 (13.25) 线性化后, 根据最小二乘法求得的二阶曲线和三阶曲线方程为:

二阶曲线: $\hat{Y}_{t}=-253.92+129.95 t-2.7246 t^{2}$

三阶曲线: $\hat{Y}_{t}=139.64-70.868 t+20.609 t^{2}-0.7408 t^{3}$

将 $t=1,2, \cdots, 20$ 代入趋势方程得到各期的预测值, 将 $t=21$ 代入趋势方程即可得到 2020 年轿车产量的预测值。有关输出结果及其残差见表 13-7。

表 13-7 轿车产量的二阶曲线和三阶曲线预测

\begin{center}
\begin{tabular}{|c|c|c|c|c|c|c|}
\hline
年份 & 时间代码 & \begin{tabular}{c}
轿车产量 \\
(万辆) \\
\end{tabular} & 二阶曲线预测 & 二阶曲线预测残差 & 三阶曲线预测 & 三阶曲线预测残差 \\
\hline
2000 & 1 & 60.7 & -126.70 & 187.40 & 88.64 & -27.94 \\
\hline
2001 & 2 & 70.4 & -4.92 & 75.32 & 74.41 & -4.01 \\
\hline
2002 & 3 & 109.2 & 111.41 & -2.21 & 92.52 & 16.68 \\
\hline
2003 & 4 & 207.1 & 222.28 & -15.18 & 138.50 & 68.60 \\
\hline
2004 & 5 & 227.6 & 327.71 & -100.11 & 207.93 & 19.67 \\
\hline
2005 & 6 & 277.0 & 427.69 & -150.69 & 296.36 & -19.36 \\
\hline
\end{tabular}
\end{center}

\begin{center}
\begin{tabular}{|c|c|c|c|c|c|c|}
\hline
年份 & 时间代码 & \begin{tabular}{c}
轿车产量 \\
(万辆) \\
\end{tabular} & 二阶曲线预测 & 二阶曲线预测残差 & 三阶曲线预测 & 三阶曲线预测残差 \\
\hline
2006 & 7 & 386.9 & 522.22 & -135.32 & 399.33 & -12.43 \\
\hline
2007 & 8 & 479.8 & 611.30 & -131.50 & 512. 41 & -32.61 \\
\hline
2008 & 9 & 503.8 & 694.94 & -191.14 & 631.16 & -127.36 \\
\hline
2009 & 10 & 748.5 & 773. 12 & -24.62 & 751.12 & -2.62 \\
\hline
2010 & 11 & 957.6 & 845.85 & 111.75 & 867.85 & 89.75 \\
\hline
2011 & 12 & 1012.7 & 913.14 & 99.56 & 976.92 & 35.78 \\
\hline
2012 & 13 & 1077.0 & 974.97 & 102.03 & 1073.86 & 3. 14 \\
\hline
2013 & 14 & 1210.4 & 1031.36 & 179.04 & 1154.25 & 56.15 \\
\hline
2014 & 15 & 1248.3 & 1082.30 & 166.00 & 1213.63 & 34.67 \\
\hline
2015 & 16 & 1163.0 & 1127.79 & 35.21 & 1247.57 & -84.57 \\
\hline
2016 & 17 & 1211.1 & 1167.82 & 43.28 & 1251.60 & -40.50 \\
\hline
2017 & 18 & 1194.5 & 1202.41 & -7.91 & 1221.30 & -26.80 \\
\hline
2018 & 19 & 1217.4 & 1231.55 & -14.15 & 1152.22 & 65.18 \\
\hline
2019 & 20 & 1028.5 & 1255.25 & -226.75 & 1039.91 & -11.41 \\
\hline
2020 & 21 & - & 1273.49 & - & 879.93 & - \\
\hline
\end{tabular}
\end{center}

将各期预测值及原序列绘成图 13-11, 可以看出轿车产量的预测效果。

\begin{center}
\end{center}

图 13-11 轿车产量的二阶曲线和三阶曲线预测

\section*{13.6 复合型序列的分解预测}
复合型序列是指含有趋势、季节、周期和随机成分的序列。由于周期成分的分析需要有多年的数据, 实际中很难得到多年的数据, 因此采用的分解模型为: $Y_{t}=T_{t} \times S_{t} \times I_{t}$ 。复合型序列的预测方法有多种, 这里只介绍分解法预测, 该方法通常是将时间序列的各个因素依次分解出来, 然后进行预测。由于该方法相对来说容易理解, 结果易于解释, 而且在很多情况下能给出很好的预测结果, 因此至今仍有应用。

采用分解法进行预测时, 需要先找出季节成分并将其从序列中分离出去, 然后建立预测模型再进行预测。分解法预测通常按下列步骤进行:

第 1 步: 确定并分离季节成分。季节成分一般用季节指数(seasonal index)来表示(1), 然后将季节成分从时间序列中分离出去, 即用序列的每一个观测值除以相应的季节指数, 以消除季节成分。

第 2 步: 建立预测模型并进行预测。根据消除季节成分后的序列建立预测模型。当消除季节成分后的序列为线性趋势时, 可用一元线性回归模型预测, 为非线性趋势时, 可选择适当的非线性模型进行预测。

第 3 步: 计算最后的预测值。将第 2 步得到的预测值乘以相应的季节指数, 得到最终的预测值。

由于分解预测的计算过程比较麻烦, 实际应用时, 建议使用统计软件来实现。下面通过一个例子说明用 SPSS 进行分解预测的过程。

\section*{例 13.10}
表 13-8 是一家啤酒生产企业 2015-2020 年各季度的啤酒销售量数据(单位:万吨)。采用分解法预测 2021 年各季度的啤酒销售量。

表 13-8 某啤酒生产企业各季度的销售量数据

\begin{center}
\begin{tabular}{ccccc}
\hline
\multirow{2}{*}{年份} & \multicolumn{3}{c}{季度} &  \\
\cline { 2 - 5 }
 & 1 & 2 & 3 & 4 \\
\hline
2015 & 25 & 32 & 37 & 26 \\
2016 & 30 & 38 & 42 & 30 \\
2017 & 29 & 39 & 50 & 35 \\
2018 & 30 & 39 & 51 & 37 \\
\hline
\end{tabular}
\end{center}

(1) 季节指数可按移动平均趋势剔除法计算, 其基本步骤是: (1) 计算移动平均值(季度数据采用 4 项移动平均,月份数据则采用 12 项移动平均), 并将其结果进行 “中心化” 处理, 也就是将移动平均的结果再进行一次 2 项的移动平均, 即得出 “中心化移动平均值” (CMA)。(2) 计算移动平均的比值, 也称为季节比率, 即将序列的各观测值除以相应的中心化移动平均值, 然后再计算出各比值的季度平均值, 即为季节指数。\\
续表

\begin{center}
\begin{tabular}{ccccc}
\hline
\multirow{2}{*}{年份} & \multicolumn{4}{c}{季度} \\
\cline { 2 - 5 }
 & 1 & 2 & 3 & 4 \\
\hline
2019 & 29 & 42 & 55 & 38 \\
2020 & 31 & 43 & 54 & 41 \\
\hline
\end{tabular}
\end{center}

\begin{itemize}
  \item 解 首先画出啤酒销售量的折线图, 观察其所包含的成分, 如图 13-12 所示。
\end{itemize}

\begin{center}
\end{center}

图 13-12 某啤酒生产企业各季度销售量的折线图

图 13-12 显示, 啤酒销售量包含趋势、季节成分和随机成分。因此, 可采用分解法进行, 具体步骤如下。

第 1 步: 确定并分离季节成分。使用 SPSS 进行成分分解的步骤如下:

太使用 SPSS 进行成分分解的操作步骤

先将观测值序列定义为【年, 季度】的时间序列形式。然后按下列步骤操作:

第 1 步: 选择【分析一时间序列预测】 【【季节性分解】, 进入主对话框。

第 2 步: 将要分解的变量 (本例为销售量) 选入【变量】。在【模型类型】下选择【乘法】或【加法】(默认为乘法), 点击【确定】。

按上述步骤, 采用乘法模型得到的分解结果如表 13-9 所示。

表 13-9 啤酒销售量的分解

\begin{center}
\begin{tabular}{|c|c|c|c|c|c|c|c|c|}
\hline
 & \begin{tabular}{l}
时间 \\
编号 \\
\end{tabular} & $\dot{z}$ 年份 & 季度 & 8 年/季 & \& ERR\_1 & $\Leftrightarrow$ SAS\_1 & $\dot{\theta} \mathrm{SAF}_{-} 1$ & $\dot{\theta}$ STC\_1 \\
\hline
25 & 1 & 2015 & 1 & Q1 2015 & 1. 04857 & 31.90471 & 0.78358 & 30.42695 \\
\hline
32 & 2 & 2015 & 2 & Q2 2015 & 1.00757 & 30.75482 & 1. 04049 & 30.52389 \\
\hline
\end{tabular}
\end{center}

\begin{center}
\begin{tabular}{|c|c|c|c|c|c|c|c|c|}
\hline
 & \begin{tabular}{l}
时间 \\
编号 \\
\end{tabular} & $\dot{z}$ 年份 & 多季度 & 8 年/季 & ERR\_1 & $\Leftrightarrow$ SAS\_1 & $\Leftrightarrow$ SAF\_1 & $\Rightarrow$ STC\_1 \\
\hline
37 & 3 & 2015 & 3 & Q3 2015 & 0.94122 & 28.91214 & 1.27974 & 30.71776 \\
\hline
26 & 4 & 2015 & 4 & Q4 2015 & 0.90440 & 29.01171 & 0.89619 & 32.07854 \\
\hline
30 & 5 & 2016 & 1 & Q1 2016 & 1. 11999 & 38.28565 & 0.78358 & 34.18382 \\
\hline
38 & 6 & 2016 & 2 & Q2 2016 & 1. 04592 & 36.52135 & 1. 04049 & 34.91783 \\
\hline
42 & 7 & 2016 & 3 & Q3 2016 & 0.94144 & 32.81918 & 1. 27974 & 34.86061 \\
\hline
30 & 8 & 2016 & 4 & Q4 2016 & 0.95921 & 33.47505 & 0.89619 & 34.89847 \\
\hline
29 & 9 & 2017 & 1 & Q1 2017 & 1. 02540 & 37.00946 & 0.78358 & 36.09255 \\
\hline
39 & 10 & 2017 & 2 & Q2 2017 & 1.00061 & 37.48244 & 1. 04049 & 37.45960 \\
\hline
50 & 11 & 2017 & 3 & Q3 2017 & 1. 01752 & 39.07046 & 1.27974 & 38. 39776 \\
\hline
35 & 12 & 2017 & 4 & Q4 2017 & 1. 01340 & 39.05422 & 0.89619 & 38.53775 \\
\hline
30 & 13 & 2018 & 1 & Q1 2018 & 0.99342 & 38.28565 & 0.78358 & 38.53918 \\
\hline
39 & 14 & 2018 & 2 & Q2 2018 & 0.96642 & 37.48244 & 1. 04049 & 38.78472 \\
\hline
51 & 15 & 2018 & 3 & Q3 2018 & 1. 01782 & 39.85187 & 1. 27974 & 39.15415 \\
\hline
37 & 16 & 2018 & 4 & Q4 2018 & 1. 04542 & 41.28589 & 0.89619 & 39.49205 \\
\hline
29 & 17 & 2019 & 1 & Q1 2019 & 0.93259 & 37.00946 & 0.78358 & 39.68455 \\
\hline
42 & 18 & 2019 & 2 & Q2 2019 & 0.99598 & 40.36570 & 1. 04049 & 40.52874 \\
\hline
55 & 19 & 2019 & 3 & Q3 2019 & 1. 04247 & 42.97750 & 1. 27974 & 41.22652 \\
\hline
38 & 20 & 2019 & 4 & Q4 2019 & 1.02043 & 42.40173 & 0.89619 & 41.55293 \\
\hline
31 & 21 & 2020 & 1 & Q1 2020 & 0.95890 & 39.56184 & 0.78358 & 41.25735 \\
\hline
43 & 22 & 2020 & 2 & Q2 2020 & 0.99013 & 41.32679 & 1. 04049 & 41.73858 \\
\hline
54 & 23 & 2020 & 3 & Q3 2020 & 0.97924 & 42.19609 & 1. 27974 & 43.09070 \\
\hline
41 & 24 & 2020 & 4 & Q4 2020 & 1.04530 & 45.74923 & 0.89619 & 43.76677 \\
\hline
\end{tabular}
\end{center}

表 13-9 给出了分解后的随机误差 (ERR\_1)、季节性调整序列(SAS\_1)、季节因子 (SAF\_1) 以及趋势和循环成分 (STC\_1)。图 13-13 展示了分解后的各成分图。

第 2 步: 建立预测模型并进行预测。为了进行预测, 需要根据季节性调整序列 (SAS\_1)观察变化形态。从图 13-13 (b) 可以看出, 经季节调整后, 啤酒销售量具有线性趋势, 因此可建立一元线性回归模型进行预测 (也可以根据趋势和循环成分 (STC\_1) 序列建立线性模型, 预测效果差不多)。以时间代码为自变量、SAS\_1 为因变量, 根据最小二乘法得到的线性趋势方程为 $\hat{Y}=30.711+0.552 t$ 。根据这一方程可以得到回归预测值。最后, 再将回归预测值乘以相应的季节因子(SAF\_1), 即可得到最终预测值(1), 结果如表 13-10 所示。
\footnotetext{(1) 由于回归预测值是不含季节性因素的, 它表示在没有季节因素的影响下啤酒销售量的预测值。如果要求算出含有季节性因素的销售量的最终预测值, 则需要将回归预测值乘以相应的季节指数。
}

\begin{center}
\end{center}

(a) 季节因子

\begin{center}
\end{center}

(c) 随机误差

\begin{center}
\end{center}

(b)季节性调整序列

\begin{center}
\end{center}

(b)趋势和循环成分

图 13-13 啤酒销售量的分解图

表 13-10 啤酒销售量的分解预测

\begin{center}
\begin{tabular}{|c|c|c|c|c|c|c|c|c|c|}
\hline
$\underbrace{\text { 啤酒 }}_{\text {销售量 }}$ & \begin{tabular}{l}
时间 \\
编号 \\
\end{tabular} & $\Leftrightarrow$ & $\Leftrightarrow \mathrm{ER}$ & $\Leftrightarrow \mathrm{S}$ & $\theta \mathrm{S}$ & $\Leftrightarrow \mathrm{s} 1$ & $\dot{\theta} \underset{\text { 预测 }}{\text { 回归 }}$ & \begin{tabular}{l}
最终预 \\
测值 \\
\end{tabular} & \begin{tabular}{l}
预测 \\
误差 \\
\end{tabular} \\
\hline
25 & 1 & Q1 2015 & 1. 0486 & 31.9047 & 0.7836 & 30.4270 & 31.26 & 24.50 & 0.50 \\
\hline
32 & 2 & Q2 2015 & 1. 0076 & 30.7548 & 1. 0405 & 30.5239 & 31.82 & 33.10 & -1.10 \\
\hline
37 & 3 & Q3 2015 & 0.9412 & 28.9121 & 1. 2797 & 30.7178 & 32.37 & 41.42 & -4.42 \\
\hline
26 & 4 & Q4 2015 & 0.9044 & 29.0117 & 0.8962 & 32.0785 & 32.92 & 29.50 & -3.50 \\
\hline
30 & 5 & Q1 2016 & 1. 1200 & 38.2856 & 0.7836 & 34.1838 & 33.47 & 26.23 & 3. 77 \\
\hline
38 & 6 & Q2 2016 & 1. 0459 & 36.5213 & 1.0405 & 34.9178 & 34.03 & 35.40 & 2. 60 \\
\hline
42 & 7 & Q3 2016 & 0.9414 & 32.8192 & 1. 2797 & 34.8606 & 34.58 & 44.25 & -2.25 \\
\hline
30 & 8 & Q4 2016 & 0.9592 & 33.4750 & 0.8962 & 34.8985 & 35.13 & 31.48 & -1.48 \\
\hline
29 & 9 & Q1 2017 & 1. 0254 & 37.0095 & 0.7836 & 36.0926 & 35.68 & 27.96 & 1.04 \\
\hline
39 & 10 & Q2 2017 & 1.0006 & 37. 4824 & 1.0405 & 37.4596 & 36.24 & 37.70 & 1. 30 \\
\hline
50 & 11 & Q3 2017 & 1. 0175 & 39.0705 & 1.2797 & 38.3978 & 36. 79 & 47.08 & 2.92 \\
\hline
35 & 12 & Q4 2017 & 1.0134 & 39.0542 & 0.8962 & 38.5378 & 37.34 & 33.46 & 1. 54 \\
\hline
30 & 13 & Q1 2018 & 0.9934 & 38.2856 & 0.7836 & 38.5392 & 37.89 & 29.69 & 0.31 \\
\hline
\end{tabular}
\end{center}

\begin{center}
\begin{tabular}{|c|c|c|c|c|c|c|c|c|c|}
\hline
$\underbrace{\text { 啤酒 }}_{\text {销售量 }}$ & \begin{tabular}{l}
时间 \\
编号 \\
\end{tabular} & $Q^{\text {年 }}$ & $\Leftrightarrow \mathrm{ER}$ & $\Rightarrow \mathrm{S}$ & $\Leftrightarrow \mathrm{s}$ & $\Leftrightarrow \mathrm{S}$ & $\theta_{\text {预测 }}^{\text {回归 }}$ & \begin{tabular}{l}
最终预 \\
测值 \\
\end{tabular} & \begin{tabular}{l}
预测 \\
误差 \\
\end{tabular} \\
\hline
39 & 14 & Q2 2018 & 0.9664 & 37.4824 & 1.0405 & 38.7847 & 38.45 & 40.00 & -1.00 \\
\hline
51 & 15 & Q3 2018 & 1. 0178 & 39.8519 & 1.2797 & 39.1542 & 39.00 & 49.91 & 1.09 \\
\hline
37 & 16 & Q4 2018 & 1. 0454 & 41.2859 & 0.8962 & 39.4921 & 39.55 & 35.44 & 1. 56 \\
\hline
29 & 17 & Q1 2019 & 0.9326 & 37.0095 & 0.7836 & 39.6845 & 40.10 & 31.42 & -2.42 \\
\hline
42 & 18 & Q2 2019 & 0.9960 & 40.3657 & 1.0405 & 40.5287 & 40.66 & 42.30 & -0.30 \\
\hline
55 & 19 & Q3 2019 & 1. 0425 & 42.9775 & 1.2797 & 41.2265 & 41.21 & 42.73 & 2. 27 \\
\hline
38 & 20 & Q4 2019 & 1. 0204 & 42.4017 & 0.8962 & 41.5529 & 41.76 & 37.42 & 0.58 \\
\hline
31 & 21 & Q1 2020 & 0.9589 & 39.5618 & 0.7836 & 41.2573 & 42.31 & 33. 16 & -2.16 \\
\hline
43 & 22 & Q2 2020 & 0.9901 & 41.3268 & 1.0405 & 41.7386 & 42.86 & 44.60 & -1.60 \\
\hline
54 & 23 & Q3 2020 & 0.9792 & 42.1961 & 1. 2797 & 43.0907 & 43.42 & 55.56 & -1.56 \\
\hline
\multirow[t]{5}{*}{41} & 24 & Q4 2020 & 1.0453 & 45.7492 & 0.8962 & 43.7668 & 43.97 & 39.41 & 1. 59 \\
\hline
 & 25 & Q1 2021 &  &  & 0.7836 &  & 44.52 & 34.89 &  \\
\hline
 & 26 & Q2 2021 &  &  & 1.0405 &  & 45.07 & 46.90 &  \\
\hline
 & 27 & Q3 2021 &  &  & 1. 2797 &  & 45.63 & 58.39 &  \\
\hline
 & 28 & Q4 2021 &  &  & 0.8962 &  & 46.18 & 41.39 &  \\
\hline
\end{tabular}
\end{center}

图 13-14 给出了实际值和预测值的对比, 图形显示预测效果比较好。

\begin{center}
\end{center}

图 13-14 啤酒销售量的分解预测

\section*{思考与练习}
\section*{- 思考题}
13.1 简述时间序列的构成要素。

13.2 利用增长率分析时间序列时应注意哪些问题?

13.3 简述平稳序列和非平稳序列的含义。

13.4 简述时间序列的预测程序。

13.5 简述指数平滑法的基本含义。

13.6 简述复合型时间序列的预测步㵵。

\section*{二 练习题}
13.1 下面是一家旅馆过去 18 个月的营业额数据:

\begin{center}
\begin{tabular}{|c|c|c|c|}
\hline
月份 & 营业额(万元) & 月份 & 营业额(万元) \\
\hline
1 & 295 & 10 & 473 \\
\hline
2 & 283 & 11 & 470 \\
\hline
3 & 322 & 12 & 481 \\
\hline
4 & 355 & 13 & 449 \\
\hline
5 & 286 & 14 & 544 \\
\hline
6 & 379 & 15 & 601 \\
\hline
7 & 381 & 16 & 587 \\
\hline
8 & 431 & 17 & 644 \\
\hline
9 & 424 & 18 & 660 \\
\hline
\end{tabular}
\end{center}

(1)用 3 期移动平均法预测第 19 个月的营业额。

(2)采用指数平滑法, 分别用平滑系数 $\alpha=0.3, \alpha=0.4$ 和 $\alpha=0.5$ 预测各月的营业额, 分析预测误差, 说明用哪个平滑系数预测更合适。

(3)建立一个趋势方程预测各月的营业额, 计算出估计标准误差。

13.2 我国 1964-1999 年的纱产量数据(单位: 万吨)如下:

\begin{center}
\begin{tabular}{cc||cc||cc}
\hline
年份 & 纱产量 & 年份 & 纱产量 & 年份 & 纱产量 \\
\hline
1964 & 97.0 & 1968 & 137.7 & 1972 & 188.6 \\
1965 & 130.0 & 1969 & 180.5 & 1973 & 196.7 \\
1966 & 156.5 & 1970 & 205.2 & 1974 & 180.3 \\
1967 & 135.2 & 1971 & 190.0 & 1975 & 210.8 \\
\hline
\end{tabular}
\end{center}

\begin{center}
\begin{tabular}{ll||c|c||cc}
\hline
年份 & 纱产量 & 年份 & 纱产量 & 年份 & 纱产量 \\
\hline
1976 & 196.0 & 1984 & 321.9 & 1992 & 501.8 \\
1977 & 223.0 & 1985 & 353.5 & 1993 & 501.5 \\
1978 & 238.2 & 1986 & 397.8 & 1994 & 489.5 \\
1979 & 263.5 & 1987 & 436.8 & 1995 & 542.3 \\
1980 & 292.6 & 1988 & 465.7 & 1996 & 512.2 \\
1981 & 317.0 & 1989 & 476.7 & 1997 & 559.8 \\
1982 & 335.4 & 1990 & 462.6 & 1998 & 542.0 \\
1983 & 327.0 & 1991 & 460.8 & 1999 & 567.0 \\
\hline
\end{tabular}
\end{center}

(1)绘制时间序列图描述其趋势。

(2)选择一条适合的趋势线拟合数据,并根据趋势线预测 2000 年的纱产量。

13.3 对下面的数据分别拟合线性趋势线 $Y_{t}=b_{0}+b_{1} t$, 二阶曲线 $\hat{Y}_{t}=b_{0}+b_{1} t+b_{2} t^{2}$ 和三阶曲线 $\hat{Y}_{t}=b_{0}+b_{1} t+b_{2} t^{2}+b_{3} t^{3}$, 并对结果进行比较。

\begin{center}
\begin{tabular}{|c|c|c|c|}
\hline
时间 $t$ & 观测值 $Y$ & 时间 $t$ & 观测值 $Y$ \\
\hline
1 & 372 & 19 & 360 \\
\hline
2 & 370 & 20 & 357 \\
\hline
3 & 374 & 21 & 356 \\
\hline
4 & 375 & 22 & 352 \\
\hline
5 & 377 & 23 & 348 \\
\hline
6 & 377 & 24 & 353 \\
\hline
7 & 374 & 25 & 356 \\
\hline
8 & 372 & 26 & 356 \\
\hline
9 & 373 & 27 & 356 \\
\hline
10 & 372 & 28 & 359 \\
\hline
11 & 369 & 29 & 360 \\
\hline
12 & 367 & 30 & 357 \\
\hline
13 & 367 & 31 & 357 \\
\hline
14 & 365 & 32 & 355 \\
\hline
15 & 363 & 33 & 356 \\
\hline
16 & 359 & 34 & 363 \\
\hline
17 & 358 & 35 & 365 \\
\hline
18 & 359 &  &  \\
\hline
\end{tabular}
\end{center}

13.4 一家贸易公司主要经营产品的外销业务, 为了合理地组织货源, 需要了解外销订单的变化状\\
况 (单位: 万元)。下表是 2011-2015 年各月份的外销订单金额。

\begin{center}
\begin{tabular}{cccccc}
\hline
月份 & 2011 年 & 2012 年 & 2013 年 & 2014 年 & 2015 年 \\
\hline
1 & 54.3 & 49.1 & 56.7 & 64.4 & 61.1 \\
2 & 46.6 & 50.4 & 52.0 & 54.5 & 69.4 \\
3 & 62.6 & 59.3 & 61.7 & 68.0 & 76.5 \\
4 & 58.2 & 58.5 & 61.4 & 71.9 & 71.6 \\
5 & 57.4 & 60.0 & 62.4 & 69.4 & 74.6 \\
6 & 56.6 & 55.6 & 63.6 & 67.7 & 69.9 \\
7 & 56.1 & 58.0 & 63.2 & 68.0 & 71.4 \\
8 & 52.9 & 55.8 & 63.9 & 66.3 & 72.7 \\
9 & 54.6 & 55.8 & 63.2 & 67.8 & 69.9 \\
10 & 51.3 & 59.8 & 63.4 & 71.5 & 74.2 \\
11 & 54.8 & 59.4 & 64.4 & 70.5 & 72.7 \\
12 & 52.1 & 55.5 & 63.8 & 69.4 & 72.5 \\
\hline
\end{tabular}
\end{center}

(1)根据各年的月份数据绘制趋势图,说明该时间序列的特点。

(2)如果要计算各月份的预测值,你认为应该采取什么方法?

(3) 选择你认为合适的方法预测 2016 年 1 月份的外销订单金额。

13.5 下表是一家大型百货公司 2006-2015 年各季度的销售额数据(单位:万元)。对这一时间序列的构成要素进行分解, 计算季节指数, 建立剔除季节变动后的趋势方程。

\begin{center}
\begin{tabular}{rrrrc}
\hline
年份 & 1 季度 & 2 季度 & 3 季度 & 4 季度 \\
\hline
2006 & 993.1 & 971.2 & 2264.1 & 1943.3 \\
2007 & 1673.6 & 1931.5 & 3927.8 & 3079.6 \\
2008 & 2342.4 & 2552.6 & 3747.5 & 4472.8 \\
2009 & 3254.4 & 4245.2 & 5951.1 & 6373.1 \\
2010 & 3904.2 & 5105.9 & 7252.6 & 8630.5 \\
2011 & 5483.2 & 5997.3 & 8776.1 & 8720.6 \\
2012 & 5123.6 & 6051.0 & 9592.2 & 8341.2 \\
2013 & 4942.4 & 6825.5 & 8900.1 & 8723.1 \\
2014 & 5009.9 & 6257.9 & 8016.8 & 7865.6 \\
2015 & 6059.3 & 5819.7 & 7758.8 & 8128.2 \\
\hline
\end{tabular}
\end{center}

\includegraphics[max width=\textwidth, center]{2024_03_13_d99164feef91279bea73g-304}\\
人类发展指数: 社会综合发展的指示器

在联合国开发计划署 (UNDP) 于 1990 年首次发布的《人类发展报告》中,第一次使用人类发展指数(Human Development Index,HDI)来综合测量世界各国的人文!发展状况。该指数由三个类指数共 5 个指标复合组成, 三个类指数分别是: 平均寿命指!数(也称健康指数)、教育水平指数(也称文化指数)和人均 GDP 指数(也称生活水平指数)。该指数采用标准化方法处理数据, 指数取值在 $0 \sim 1$ 之间, 取值越高, 表明人文发展水平越高。此后, 联合国开发计划署每年发布一次全世界的《人类发展报告》, 并将评估结果分为三级: HDI 取值在 0.800 及以上的国家和地区属于人文发展高度水平, HDI 取值在 $0.500 \sim 0.799$ 之间的国家和地区属于人文发展中度水平, HDI 取值在 0.500 及以下的国家和地区属于人文发展低度水平。目前, 联合国开发计划署编制的人!文发展指数及其每年发布的《人类发展报告》已经得到普遍认可,成为评价世界各国人!文发展综合水平的重要依据。

仅采用 5 个指标勾画社会发展也是无奈之举, 联合国必须考虑数据的国际可比性和可获取性。人类发展指数低的大多数国家往往缺乏精确的统计数据, 因此在选用指标编制指数时, 联合国的此项测评受到很多限制, 只能选取大多数国家都可能有的统计数据, 这使得在利用 HDI 指数综合测量社会经济发展水平时, 存在结构性缺失与测评上的缺陷。

为此, 中国人民大学中国调查与数据中心在深入研究的基础上, 根据中国国情, 对联合国的人类发展指数进行拓展, 建立了中国人民大学中国发展指数(RCDI)。该指数包括 4 个类指数共 15 个指标,综合反映了全体国民的健康(健康指数)、受教育程度 (教育指数)、经济发展和生活水平(生活水平指数)以及社会环境(社会环境指数)的变化情况, 从 2006 年开始每年发布。该指数的建立和发布在社会上引起很大反响, 受到政府有关部门的高度重视。这是用指数方法反映社会综合发展的成功案例。\\
在日常工作生活中, 我们经常遇到或者需要使用各种指数, 例如, 居民消费价格指数 (CPI)、股票价格指数、房地产价格指数等, 这些指数同社会经济生活关系非常密切。了解指数是如何编制的, 有助于我们更好地认识指数的功能与作用。

\section*{14.1 基本问题}
\section*{14. 1.1 指数概念}
指数, 或称统计指数, 是分析社会经济现象数量变化的一种重要统计方法。它产生于 18 世纪后半叶, 当时美洲新大陆开采的金银源源不断地流入欧洲, 使欧洲物价骤然上涨,引起了社会的普遍关注。经济学家为了测定物价的变动, 开始尝试编制物价指数。此后 200 多年, 指数的理论和应用不断发展, 逐步扩展到工业生产、进出口贸易、铁路运输、工资、成本、生活费用、证券等各个方面。其中, 有些指数, 如零售商品物价指数、生活费用价格指数等, 同人们的生活休戚相关; 有些指数, 如生产资料价格指数、股票价格指数等, 则直接影响到人们的投资活动, 成为社会经济的晴雨表。目前, 指数已成为分析社会经济和预测景气度的重要工具。

指数是测定多项内容数量综合变动的相对数。这个概念中包含两个要点: 第一个要点是指数的实质是测定多项内容, 例如, 零售价格指数反映的是零售市场几百万种商品价格变化的整体状况。单一商品价格指数也是有的, 例如国家提高烟酒税后, 茅台酒价格大幅上升。单一项目的指数计算简单, 不是指数方法论中的核心内容。指数方法论研究如何将多项内容合在一起, 从整体上进行反映。指数概念的第二个要点是其表现形式为动态相对数, 既然是动态相对数, 就涉及指标的基期对比, 不同要素基期的选择就成为指数方法需要讨论的问题。编制指数的方法就是围绕上述两个问题展开的。

\section*{14. 1.2 指数分类}
从不同的角度出发, 统计指数可以划分为以下几种主要类型。

\section*{1. 按照考察对象的范围不同, 可分为个体指数和总指数}
个体指数是反映总体中个别现象或个别项目数量变动的相对数, 如某种产品的产量指数、某种商品的价格指数等。个体指数是计算总指数的基础。

总指数是综合反映多种项目数量变动的相对数, 如多种产品的产量指数、多种商品的价格指数等。由于多种事物的使用价值不同, 其数量不具有直接综合的性质, 所以总指数的计算不能使用个体指数直接对比的方法, 而需要使用专门的方法。总指数和个体指数的区别不仅在于考察范围不同, 也在于计算方法不同。

\section*{2. 按照所反映指标的性质不同, 可分为数量指标指数和质量指标指数}
数量指标指数是反映数量指标变动程度的相对数, 如商品销售量指数、工业产品产量指数等, 数量指标通常采用实物计量单位。质量指标指数是反映品质指标变动程度的相对数, 如产品价格指数、产品单位成本指数等, 质量指标通常采用货币计量单位。

世界上没有绝对的东西, 数量指标和质量指标的划分具有相对性。如单位产品原材料消耗量指标, 相对于产品产量指标, 它是质量指标; 但相对于单位原材料价格指标, 它又是数量指标。把指标区分为数量指标和质量指标, 更多的是为了讨论问题的方便, 而不是真要把指标分成不同类型。

\section*{3. 按照计算形式不同, 可分为简单指数和加权指数}
简单指数把计人指数的各个项目的重要性视为相同; 加权指数则对计人指数的各个项目依据重要程度赋予不同的权数, 再进行计算。实际应用中, 有时由于缺少必要的权数资料, 或者指数编制的频率或时效性要求较高, 会采用适当的简单指数。加权指数可分为两种, 即综合形式和平均形式。采用综合形式编制的加权指数可称为加权综合指数; 采用平均形式编制的加权指数可称为加权平均指数。

上述各种分类是从不同的角度对统计指数所作的一般分类, 也可以交叉进行复合分类, 如在个体指数和总指数中再区分数量指标指数和质量指标指数等。

\section*{14. 1.3 指数编制中的问题}
指数编制过程中, 需要解决的问题包括选择项目、确定权数以及指数计算方法等。

\section*{1. 选择项目}
理论上讲, 指数是反映总体数量变动的相对数, 但实际中将总体全部项目都计算在内往往不可能, 也没必要。例如, 编制消费者价格指数时不可能将消费者所消费的所有商品和服务价格全部纳人价格指数, 而需要进行项目选择。在计算价格指数时, 那些被选中的项目称为 “代表规格品”, 我们使用代表规格品的价格变化来反映所有商品价格的变化。代表规格品需要具有良好的价格变动趋势代表性, 代表规格品的数量要有保证, 不能品种过少, 并注意不断更新。代表规格品的更新过程中, 价格也在不断变化, 这里面既有商品本身价格的变化, 也包含由商品质量引起的价格变化。如何进行质价分解, 是当代指数理论不断研究的课题, 国外学者尝试用不同模型进行分析, 取得了突破性成果。

\section*{2. 确定权数}
指数是对代表项目进行加权得到的结果, 如何确定权数是编制指数时必须解决的问题。确定权数的途径大体有两种: 一种是利用已有的信息构造权数。例如, 计算零售价格指数时, 每个代表规格品用其代表的商品零售额在全部零售额中的比重做权数。是否具有\\
构造权数的数据, 以及这些数据的质量如何是关键。另一种是主观权数, 常见于社会现象的指数编制。例如, 编制幸福感指数, 是将反映幸福感不同侧面的类指数综合, 最后得到总指数。每个类指数的权重是多少, 一般由指数编制人员主观决定 (尽管可能经过多次研讨, 广泛征求意见), 因为没有公认的确定权数的标准。对于第一种确定权数的途径, 指数理论要回答选择什么样的数据做权数, 以及用什么时期的数据构造权数; 对于后一种确定权数的途径, 实际上是将指数方法拓展到多指标的综合评价, 从而形成一系列的综合评价方法。

\section*{3. 指数计算方法}
总指数的计算方法有许多种, 因为利用指数测定的研究对象不同, 编制指数的数据来源不同, 本章后面的部分将介绍一些总指数的不同计算方法。每种方法都有自己的特点,适用于不同场合。指数计算方法一直备受争议, 经济学家和统计学家试图从不同角度、用不同方式对这些指数进行改造和完善。学习指数, 并不在于掌握某种指数的具体计算方法, 重要的是体会方法背后蕴藏的统计思想, 以便针对具体的研究对象, 依据编制指数的主要目的, 选择甚至创造最恰当的计算指数的方法。

\section*{14.2 总指数编制方法}
总指数是对个体指数的综合, 将个体指数综合有两个途径: 一是对个体指数的简单汇总, 不考虑权数, 我们把这类指数称为简单指数; 二是编制总指数时考虑权数的作用, 我们把这类指数称为加权指数。在加权指数中, 根据计算方式不同, 又可以分为加权综合指数和加权平均指数。

\subsection*{14.2.1 简单指数}
简单指数就是不加权的指数, 主要有两种计算方法: 简单综合指数和简单平均指数。

\section*{1. 简单综合指数}
简单综合指数是将报告期的指标总和与基期的指标总和相对比的指数, 该方法的特点是先综合, 后对比, 计算公式为:


\begin{align*}
& I_{p}=\frac{\sum p_{1}}{\sum p_{0}}  \tag{14.1}\\
& I_{q}=\frac{\sum q_{1}}{\sum q_{0}} \tag{14.2}
\end{align*}


式中: $p$ 一质量指标;

$q$ 一数量指标;\\
$I_{p}$ 一质量指标指数;

$I_{q}$ 一数量指标指数;

下标 1 一一报告期;

下标 0 一一 基期。

\section*{例 14.1}
如图 14-1所示, 菜篮里有面包、牛奶等食品的报告期和基期价格数据(单位:元),采用简单综合指数的方法计算价格指数。

\begin{center}
\end{center}

基期

\begin{center}
\end{center}

报告期

图 14-1 食品的价格数据

・解 $I_{p}=\frac{\sum p_{1}}{\sum p_{0}}=\frac{12.50}{10.00}=125 \%$

计算结果表明, 菜篮内商品的报告期价格比基期价格上涨了 $25 \%$ 。

简单综合指数的优点在于操作简单, 对数据要求少。它的一个显著缺点是, 以价格指数为例, 在参与计算的商品价格水平有较大差异时, 价格低的商品的价格波动会被价格高的商品掩盖。

\section*{例 14.2}
现有彩电和蔬菜两种商品, 基期和报告期的价格如表 14-1 所示, 采用简单综合指数的方法计算价格指数。

表 14-1 彩电和蔬菜的价格数据

单位: 元

\begin{center}
\begin{tabular}{rcrr}
\hline
商品 & 计量单位 & $p_{0}$ & $p_{1}$ \\
\hline
彩电 & 台 & 8000 & 4000 \\
蔬菜 & 千克 & 1 & 2 \\
\hline
\end{tabular}
\end{center}

・解 $I_{p}=\frac{\sum p_{1}}{\sum p_{0}}=\frac{4002}{8001} \approx 0.5=50 \%$

结果表明,报告期与基期相比,价格下降了 $50 \%$ 。由于蔬菜与彩电的价格相差太大,综合指数反映不出蔬菜价格的变动。由此看出, 简单综合指数只能用于指标值相差不大的\\
商品, 在商品价格差异大且变动幅度差异大的情况下, 这种方法不能反映实际变动水平。

\section*{2. 简单平均指数}
简单平均指数是将个体指数进行简单平均得到的总指数。该方法的计算过程是先对比, 后综合, 计算公式为:


\begin{align*}
& I_{p}=\frac{\sum \frac{p_{1}}{p_{0}}}{n}  \tag{14.3}\\
& I_{q}=\frac{\sum \frac{q_{1}}{q_{0}}}{n} \tag{14.4}
\end{align*}


\section*{例 14.3}
同样根据表 14-1 的数据, 采用简单平均的方法计算价格指数。

—解 $I_{p}=\frac{\sum \frac{p_{1}}{p_{0}}}{n}=\frac{\frac{4000}{8000}+\frac{2}{1}}{2}=1.25=125 \%$

计算结果表明, 报告期价格比基期价格提高了 $25 \%$ 。显然, 这个计算结果比前面的结果更合理。在本例中, 简单平均指数消除了不同商品价格水平的影响, 可以反映各种商品的价格变动情况。但该指数也有欠缺, 因为不同商品对市场价格总水平的影响是不同的,而简单平均指数法平等看待各种商品。

总的来说, 简单综合指数和简单平均指数都存在方法上的缺陷, 没有考虑权数的影响, 计算结果难以反映实际情况。另外, 将使用价值不同的产品个体指数或价格(指标值)相加,既缺乏实际意义,又缺少理论依据。编制指数时需要考虑权数的作用。

\subsection*{14.2.2 加权指数}
加权指数因所采用的权数不同分为加权综合指数和加权平均指数。编制加权指数首先要确定合理的权数, 然后根据实际需要确定适当的计算公式。

\section*{1. 加权综合指数}
我们用一个例子说明加权综合指数的编制原理和方法。

\section*{例 14.4}
表 14-2 是某商场甲、乙、丙三种商品 2007 年和 2008 年的资料, 其中, 下标 0 表示 2007 年, 下标 1 表示 2008 年, $p$ 表示价格, $q$ 表示销售量。计算三种商品的销售量总指数, 以综合反映商场商品销售量的变化。\\
表 14-2 某商场甲、乙、丙三种商品的销售量及销售价格资料

\begin{center}
\begin{tabular}{|c|c|c|c|c|c|c|c|c|c|}
\hline
\multirow{3}{*}{}\begin{tabular}{l}
商品 \\
名称 \\
\end{tabular} & \multirow{3}{*}{}\begin{tabular}{l}
计量 \\
单位 \\
\end{tabular} & \multicolumn{2}{|c|}{销售量} & \multicolumn{2}{|c|}{价格(元)} & \multicolumn{4}{|c|}{销售额(万元)} \\
\hline
 &  & \multirow{2}{*}{$q_{0}$} & \multirow{2}{*}{$q_{1}$} & \multirow{2}{*}{$p_{0}$} & \multirow{2}{*}{$p_{1}$} & \multirow{2}{*}{$p_{0} q_{0}$} & \multirow{2}{*}{$p_{1} q_{1}$} & \multicolumn{2}{|c|}{假定} \\
\hline
 &  &  &  &  &  &  &  & $p_{0} q_{1}$ & $p_{1} q_{0}$ \\
\hline
甲 & 件 & 200 & 300 & 60 & 60 & 1.2 & 1.8 & 1.8 & 1.2 \\
\hline
乙 & 双 & 400 & 500 & 20 & 30 & 0.8 & 1.5 & 1.0 & 1.2 \\
\hline
丙 & 米 & 500 & 600 & 70 & 80 & 3.5 & 4.8 & 4.2 & 4.0 \\
\hline
合计 & - & - & - & - & - & 5.5 & 8.1 & 7.0 & 6.4 \\
\hline
\end{tabular}
\end{center}

・・解 在具体求解前需要作些分析。

若编制甲、乙、丙三种商品的销售量总指数, 需要把三种商品报告期和基期的销售量分别加总, 再将两个时期的销售量进行对比。然而, 这三种商品的使用价值不同, 计量单位也不一样, 如果销售量直接加总, 没有实际意义。同样, 若编制这三种商品的价格总指数, 把各商品的价格加总也是没有意义的。

该如何处理呢? 需要掌握两个要点: 第一个要点, 引进媒介因素。在本例中, 不同商品的销售量和价格都不能直接加总, 因为它们都是不同度量因素。然而每种商品的销售量和价格的乘积即销售额是可以加总的。而且从分析的角度看, 销售额的变化恰好反映了销售量增减和价格涨跌两个因素的影响。因此在编制销售量总指数时, 可以通过价格这个媒介因素, 将销售量转化为可以加总的销售额; 而在编制价格总指数时, 则可以通过销售量这个媒介因素, 将价格转化为可以加总的销售额。第二个要点, 将媒介因素固定下来, 以单纯反映被研究指标的变动情况。

将上述两个要点结合, 得到加权综合指数的基本公式:

销售量指数: $I_{q}=\frac{\sum q_{1} p}{\sum q_{0} p}$

价格指数: $I_{p}=\frac{\sum q p_{1}}{\sum q p_{0}}$

显然, 在销售量指数中, 价格 $p$ 是权数; 在价格指数中, 销售量 $q$ 是权数。由此我们得到第一个结论: 在加权综合指数中, 媒介因素 (也称同度量因素) 同时起着权数的作用。

接下来的问题是权数固定在什么时期, 由此产生著名的拉氏指数和帕氏指数。

(1) 拉氏指数。

拉氏指数 (Laspeyres index) 是统计学家拉斯贝尔斯 (Laspeyres) 于 1864 年提出的一种指数计算方法。它在计算综合指数时将作为权数的同度量因素固定在基期。相应的计算公式为:


\begin{equation*}
\text { 拉氏数量指标指数: } I_{q}=\frac{\sum q_{1} p_{0}}{\sum q_{0} p_{0}} \tag{14.7}
\end{equation*}


拉氏质量指标指数: $I_{p}=\frac{\sum q_{0} p_{1}}{\sum q_{0} p_{0}}$

式中, $I_{q}$ 表示数量指标指数; $I_{p}$ 表示质量指标指数; $p_{0}$ 和 $p_{1}$ 分别表示基期和报告期的质量指标值; $q_{0}$ 和 $q_{1}$ 分别表示基期和报告期的数量指标值。

(2)帕氏指数。

帕氏指数 (Paasche index) 是统计学家帕舍 (H. Paasche) 于 1874 年提出的一种指数计算方法。它在计算综合指数时将作为权数的同度量因素固定在报告期。相应的计算公式为:

帕氏数量指标指数: $I_{q}=\frac{\sum q_{1} p_{1}}{\sum q_{0} p_{1}}$

帕氏质量指标指数: $I_{p}=\frac{\sum q_{1} p_{1}}{\sum q_{1} p_{0}}$

现在回到本例, 若采用拉氏指数, 得到如下计算结果:

由式 (14.7), 有

$$
I_{q}=\frac{\sum q_{1} p_{0}}{\sum q_{0} p_{0}}=\frac{7.0}{5.5}=127.27 \%
$$

由式 (14.8), 有

$$
I_{p}=\frac{\sum q_{0} p_{1}}{\sum q_{0} p_{0}}=\frac{6.4}{5.5}=116.36 \%
$$

若采用帕氏指数, 计算结果为:

由式 (14.9), 有

$$
I_{q}=\frac{\sum q_{1} p_{1}}{\sum q_{0} p_{1}}=\frac{8.1}{6.4}=126.56 \%
$$

由式 (14.10), 有

$$
I_{p}=\frac{\sum q_{1} p_{1}}{\sum q_{1} p_{0}}=\frac{8.1}{7.0}=115.71 \%
$$

可以看出, 权数定在不同的时期, 计算结果不同。由此提出权数应该定在什么时期这个问题, 指数理论的许多研究围绕这个问题展开。

大多数的看法是, 计算数量指数(如生产量指数)时, 权数(价格)应该定在基期,这样才能别除价格变动的影响, 准确反映生产量的变化, 按不变价计算产量指数就是出于这个原因。计算质量指数(如价格指数)时, 不同时期的权数含义不同: 若权数定在基期, 反映的是在基期商品 (产品) 结构下价格的整体变动, 更能揭示价格变动的内容; 若权数定在报告期, 反映的是在现实商品 (产品) 结构下价格的整体变动, 商品 (产品) 结构变化的影响会融入价格指数, 更能揭示价格变动的实际影响。编制指数的目的不同, 权数确定的时期就可以不同。\\
由此我们得到第二个结论: 权数时期的选择主要取决于编制指数的目的, 取决于用指数要说明的问题。

\section*{2. 加权平均指数}
加权平均指数是以个体指数为基础, 通过对个体指数进行加权平均来编制的指数。具体步骤为, 先计算所研究现象各个项目的个体指数, 然后将所给的价值量指标 (产值或销售额)作为权数对个体指数进行加权平均。公式为:


\begin{equation*}
A_{p}=\frac{\sum \frac{p_{1}}{p_{0}} q p}{\sum q p}, \quad A_{q}=\frac{\sum \frac{q_{1}}{q_{0}} q p}{\sum q p} \tag{14.11}
\end{equation*}


和


\begin{equation*}
H_{p}=\frac{\sum q p}{\sum \frac{p_{0}}{p_{1}} q p}, \quad H_{q}=\frac{\sum q p}{\sum \frac{q_{0}}{q_{1}} q p} \tag{14.12}
\end{equation*}


一些教材将式 (14.11) 称为加权算术平均指数, 将式 (14.12) 称为加权调和平均指数。式 (14.11) 和式 (14.12) 没有本质区别, 特定条件下在形式上可以互相转换。

式中的 $A_{p}, A_{q}, H_{p}, H_{q}$ 只是表示指数计算方法不同, 以便区分。这里的核心是权数 $q p$, 由于权数可以取不同时期, 可以用作权数的就有 $q_{0} p_{0}$ 和 $q_{1} p_{1}$ 。用基期权数 $q_{0} p_{0}$ 类似于前面的拉氏指数, 例如

$$
A_{q}=\frac{\sum \frac{q_{1}}{q_{0}} q_{0} p_{0}}{\sum q_{0} p_{0}}=\frac{\sum q_{1} p_{0}}{\sum q_{0} p_{0}}
$$

与式 (14.7) 相同。

$$
A_{p}=\frac{\sum \frac{p_{1}}{p_{0}} q_{0} p_{0}}{\sum q_{0} p_{0}}=\frac{\sum q_{0} p_{1}}{\sum q_{0} p_{0}}
$$

与式 (14.8) 相同。

用报告期权数 $q_{1} p_{1}$ 类似于前面的帕氏指数, 例如

$$
H_{q}=\frac{\sum q_{1} p_{1}}{\sum \frac{q_{0}}{q_{1}} q_{1} p_{1}}=\frac{\sum q_{1} p_{1}}{\sum q_{0} p_{1}}
$$

与式 (14.9) 相同。

$$
H_{p}=\frac{\sum q_{1} p_{1}}{\sum \frac{p_{0}}{p_{1}} q_{1} p_{1}}=\frac{\sum q_{1} p_{1}}{\sum q_{1} p_{0}}
$$

与式 (14.10) 相同。\\
需要指出, 加权综合指数和加权平均指数上述的相同只是形式上的, 本质上还是有区别的, 主要表现在是全面资料还是样本资料。如果是全面资料, 可以采用加权综合指数,计算生产量指数一般属于这种情况, 因为生产量指数要包含所有产品的生产情况; 而计算价格指数时是无法得到全面资料的, 因为市场商品的项目成千上万, 全面统计做不到, 只能采取选种方法, 挑选代表规格品, 在这种背景下, 若采用加权综合指数, 其结果就是仅仅计算了代表规格品的价格变化。价格指数要反映市场所有商品价格的变化, 代表规格品是样本, 其中的每一项都代表一类商品, 每一项代表规格品要有自己的权数。在加权平均指数中, 权数的本质是:

$$
\text { 基期加权: } \frac{q_{0} p_{0}}{\sum q_{0} p_{0}}
$$

报告期加权: $\frac{q_{1} p_{1}}{\sum q_{1} p_{1}}$

其实就是用代表规格品所代表的那一类商品的销售额在全部销售额中的比重作为权数。在这样的背景下计算指数, 只能采取加权平均指数方法。所以, 加权平均指数方法主要用于价格指数的计算。

加权平均指数方法给我们的进一步启示是, 如果权数 $\frac{q p}{\sum q p}$ 相对稳定, 在计算指数时就不必去搜集 $\frac{q_{0} p_{0}}{\sum q_{0} p_{0}}$ 或 $\frac{q_{1} p_{1}}{\sum q_{1} p_{1}}$, 而可以采用固定权数的方法。固定权数多采用比重方法, 计算公式为:


\begin{equation*}
I=\frac{\sum i W}{\sum W} \tag{14.13}
\end{equation*}


式中, $i$ 为个体指数或类指数; $W$ 为权数。

目前, 消费价格指数和零售价格指数都是采用这种方法编制的。

\section*{例 14.5}
试计算某市居民消费价格总指数, 统计资料见表 14-3。

表 14-3 某市某年居民消费价格统计资料

\begin{center}
\begin{tabular}{lccc}
\hline
\multicolumn{1}{c}{商品类别} & 类指数 $i(\%)$ & 固定权数 $W(\%)$ & $i W(\%)$ \\
\hline
一、食品类 & 104.15 & 42 & 43.743 \\
二、衣着类 & 95.46 & 15 & 14.319 \\
三、家庭设备用品及服务类 & 102.70 & 11 & 11.297 \\
四、医疗保健和个人用品类 & 110.43 & 3 & 3.313 \\
五、交通和通信工具类 & 98.53 & 4 & 3.941 \\
\hline
\end{tabular}
\end{center}

\begin{center}
\begin{tabular}{lccc}
\hline
\multicolumn{1}{c}{商品类别} & 类指数 $i(\%)$ & 固定权数 $W(\%)$ & $i W(\%)$ \\
\hline
六、娱乐教育用品及服务类 & 101.26 & 5 & 5.063 \\
七、烟酒及用品类 & 103.50 & 14 & 14.490 \\
八、居住类 & 108.74 & 6 & 6.524 \\
\hline
合计 & - & 100 & 102.69 \\
\hline
\end{tabular}
\end{center}

——解 居民消费价格总指数采用固定加权平均指数公式, 即

$$
I=\frac{\sum i W}{\sum W}=\frac{102.69}{100}=102.69 \%
$$

计算结果表明,该市居民消费价格总指数报告期比基期平均上涨了 $2.69 \%$ 。

\section*{14.3 指数体系}
前面我们介绍了加权指数编制的一般方法。在实际应用中, 不仅可以利用指数反映社会经济现象的数量变动程度, 而且还能借助由几个指数组成的指数体系, 对社会经济现象之间的相互联系作更深人的分析。分析方法的基础是进行因素分解, 因素分解的对象可以是总量指数, 也可以是平均数指数。

\subsection*{14.3. 1 总量指数体系分析}
总量指数体系是指, 一个总量往往可以分解为若干个构成因素, 其数量关系可以用指标体系的形式表现出来。例如:

销售额=销售量 $\times$ 销售价格

总产值 $=$ 产量 $\times$ 产品价格

总成本 $=$ 产量 $\times$ 单位产品成本

销售利润 $=$ 销售量 $\times$ 销售价格 $\times$ 销售利润率

这种指标体系反映了总量指标与因素指标之间的相互关系。它们之间的这种联系同样可以表现为各指标指数之间的联系, 即

销售额指数=销售量指数 $\times$ 销售价格指数

总产值指数 $=$ 产量指数 $\times$ 产品价格指数

总成本指数 $=$ 产量指数 $\times$ 单位产品成本指数

销售利润指数=销售量指数 $\times$ 销售价格指数 $\times$ 销售利润率指数

我们把这种由总量指数及其若干个因素指数构成的数量关系式称为指数体系。上面列举了两因素指数和三因素指数的体系框架, 当然还可以进一步分解更多的因素。这些指数体系是在一定的经济联系基础上所形成的较为严密的数量关系式, 因而具有实际的经济意义。作为方法的说明, 这里只介绍总量指标的两因素分析。\\
在加权综合指数体系中 (加权平均指数相同), 为使总量指数等于各因素指数的乘积,两个因素指数中通常一个为数量指数, 另一个为质量指数, 而且各因素指数中权数必须是不同时期的, 比如数量指数用基期权数加权, 质量指数则必须用报告期权数加权, 反之亦然。

加权综合指数由于所用权数所属时期不同, 可以形成不同的指数体系。但实际分析中比较常用的是基期权数加权的数量指数(拉氏指数)和报告期权数加权的质量指数(帕氏指数) 形成的指数体系。该指数体系可表示为:


\begin{equation*}
\frac{\sum q_{1} p_{1}}{\sum q_{0} p_{0}}=\frac{\sum q_{1} p_{0}}{\sum q_{0} p_{0}} \times \frac{\sum q_{1} p_{1}}{\sum q_{1} p_{0}} \tag{14.14}
\end{equation*}


因素影响差额之间的关系为:


\begin{equation*}
\sum q_{1} p_{1}-\sum q_{0} p_{0}=\left(\sum q_{1} p_{0}-\sum q_{0} p_{0}\right)+\left(\sum q_{1} p_{1}-\sum q_{1} p_{0}\right) \tag{14.15}
\end{equation*}


式中, $\sum q_{1} p_{1}$ 为报告期总量指标; $\sum q_{0} p_{0}$ 为基期总量指标; $q, p$ 为因素指标,其中 $q$ 为数量指标, $p$ 为质量指标。

\section*{例 14.6}
表 14 - 2 是某商场甲、乙、丙三种商品 2007 年和 2008 年的资料, 采用指数体系对该数据进行因素分析。

$\cdot$・解 计算三种商品销售额的变动:

销售额指数 $I_{p q}=\frac{\sum q_{1} p_{1}}{\sum q_{0} p_{0}}=\frac{8.1}{5.5}=147.27 \%$

2008 年与 2007 年相比, 三种商品销售额增长了 $47.27 \%$, 增加的绝对值为:

$$
\sum q_{1} p_{1}-\sum q_{0} p_{0}=8.1-5.5=2.6(\text { 万元 })
$$

其中:

(1)销售量变动的影响:

销售量指数 $I_{q}=\frac{\sum q_{1} p_{0}}{\sum q_{0} p_{0}}=\frac{7.0}{5.5}=127.27 \%$

计算结果表明, 2008 年与 2007 年相比, 该商场三种商品销售量平均增长了 $27.27 \%$,销售量的上升使销售额增加的绝对值为:

$$
\sum q_{1} p_{0}-\sum q_{0} p_{0}=7.0-5.5=1.5 \text { (万元) }
$$

(2)销售价格变动的影响:

销售价格指数 $I_{p}=\frac{\sum q_{1} p_{1}}{\sum q_{1} p_{0}}=\frac{8.1}{7.0}=115.71 \%$\\
计算结果表明, 2008 年与 2007 年相比, 三种商品销售价格平均上升了 $15.71 \%$, 销售价格的上升使销售额增加的绝对值为:

$$
\sum q_{1} p_{1}-\sum q_{1} p_{0}=8.1-7.0=1.1 \text { (万元) }
$$

由此可见, 销售额增长了 $47.27 \%$, 是销售量平均增长 $27.27 \%$ 和销售价格平均增长 $15.71 \%$ 共同影响的结果, 即

$$
147.27 \%=127.27 \% \times 115.71 \%
$$

而销售额增加了 2.6 万元, 是销售量增长使其增加 1.5 万元和销售价格上升使其增加 1.1 万元共同影响的结果, 即

$$
2.6=1.5+1.1
$$

\subsection*{14.3. 2 平均数变动因素分解}
因素分解的思想同样可以用到平均数变动分析中。

在分组数据情况下, 加权算术平均数的计算公式为:

$$
\bar{x}=\frac{\sum x f}{\sum f}=\sum\left(x \frac{f}{\sum f}\right)
$$

由此看出, 平均数的变动受两个因素的影响: 一个是各组的变量水平 $x$; 另一个是各组的结构 $\frac{f}{\sum f}$ 。指数体系在这里仍然存在。

(1)总平均水平指数。


\begin{equation*}
I_{x f}=\frac{\bar{x}_{1}}{\bar{x}_{0}}=\frac{\sum x_{1} f_{1} / \sum f_{1}}{\sum x_{0} f_{0} / \sum f_{0}} \tag{14.16}
\end{equation*}


(2)组水平变动指数。


\begin{equation*}
I_{x}=\frac{\bar{x}_{1}}{\bar{x}_{n}}=\frac{\sum x_{1} f_{1} / \sum f_{1}}{\sum x_{0} f_{1} / \sum f_{1}} \tag{14.17}
\end{equation*}


(3)结构变动指数。


\begin{equation*}
I_{f}=\frac{\bar{x}_{n}}{\bar{x}_{0}}=\frac{\sum x_{0} f_{1} / \sum f_{1}}{\sum x_{0} f_{0} / \sum f_{0}} \tag{14.18}
\end{equation*}


此时, 指数体系的具体表现为:

总平均水平指数 $=$ 组水平变动指数 $\times$ 结构变动指数

即


\begin{equation*}
\frac{\sum x_{1} f_{1} / \sum f_{1}}{\sum x_{0} f_{0} / \sum f_{0}}=\frac{\sum x_{1} f_{1} / \sum f_{1}}{\sum x_{0} f_{1} / \sum f_{1}} \times \frac{\sum x_{0} f_{1} / \sum f_{1}}{\sum x_{0} f_{0} / \sum f_{0}} \tag{14.19}
\end{equation*}


简写为: $I_{x f}=I_{x} \times I_{f}$

总平均水平变动额 $=$ 各组水平变动影响额 + 结构变动影响额\\
即


\begin{align*}
& \sum x_{1} f_{1} / \sum f_{1}-\sum x_{0} f_{0} / \sum f_{0} \\
& =\left(\sum x_{1} f_{1} / \sum f_{1}-\sum x_{0} f_{1} / \sum f_{1}\right)+\left(\sum x_{0} f_{1} / \sum f_{1}-\sum x_{0} f_{0} / \sum f_{0}\right) \tag{14.20}
\end{align*}


简写为: $\bar{x}_{1}-\bar{x}_{0}=\left(\bar{x}_{n}-\bar{x}_{0}\right)+\left(\bar{x}_{1}-\bar{x}_{n}\right)$

仿照加权综合指数体系中的解释, 我们可以把总体平均数的变动分解为组水平的影响和结构变动的影响。进行分析时, 将总体结构看成数量指标, 将各组变量值看成质量指标。在研究结构的变动对平均数的影响时, 将各组变量值固定在基期; 在研究各组变量值的变动对平均数的影响时, 将结构固定在报告期。

\section*{例 14.7}
某机械厂所属两个分厂的某机器产品成本资料如表 14-4 所示, 试分析该厂某产品总平均单位成本的变动受各分厂成本水平变动以及产量结构变动的影响情况。

\begin{center}
\begin{tabular}{|c|c|c|c|c|c|c|c|}
\hline
\multirow{2}{*}{分厂} & \multicolumn{2}{|c|}{单位成本(元)} & \multicolumn{2}{|c|}{产量(台)} & \multicolumn{3}{|c|}{总成本(元)} \\
\hline
 & $x_{0}$ & $x_{1}$ & $f_{0}$ & $f_{1}$ & $x_{0} f_{0}$ & $x_{1} f_{1}$ & $x_{0} f_{1}$ \\
\hline
一分厂 & 1200.0 & 1220.0 & 70 & 70 & 84000 & 85400 & 84000 \\
\hline
二分厂 & 1000.0 & 900.0 & 30 & 130 & 30000 & 117000 & 130000 \\
\hline
全厂 & - & - & 100 & 200 & 114000 & 202400 & 214000 \\
\hline
\end{tabular}
\end{center}

表 14-4 某机械厂某机器产品成本资料

・解 基期平均单位成本 $\bar{x}_{0}=\frac{\sum x_{0} f_{0}}{\sum f_{0}}=\frac{114000}{100}=1140$ (元/台)

报告期平均单位成本 $\bar{x}_{1}=\frac{\sum x_{1} f_{1}}{\sum f_{1}}=\frac{202400}{200}=1012$ (元/台)

假定平均单位成本 $\bar{x}_{n}=\frac{\sum x_{0} f_{1}}{\sum f_{1}}=\frac{214000}{200}=1070$ (元/台)

该机械厂平均单位成本变动分析:

总平均水平指数 $I_{x f}=\frac{\bar{x}_{1}}{\bar{x}_{0}}=\frac{1012}{1140}=88.8 \%$

总平均水平变动额 $=\bar{x}_{1}-\bar{x}_{0}=1012-1140=-128$ (元/台)

其中:

(1)各分厂单位成本变动对全厂平均单位成本变动的影响:

组水平变动指数 $I_{x}=\frac{\bar{x}_{1}}{\bar{x}_{n}}=\frac{1012}{1070}=94.6 \%$

各分厂单位成本变动使全厂单位成本变化为:

$$
\bar{x}_{1}-\bar{x}_{n}=1012-1070=-58 \text { (元 } / \text { 台) }
$$

(2)各分厂产量结构变动对全厂平均单位成本变动的影响:

结构变动指数 $I_{f}=\frac{\bar{x}_{n}}{\bar{x}_{0}}=\frac{1070}{1140}=93.9 \%$

各分厂产量结构变动使全厂单位成本变化为:

$$
\bar{x}_{n}-\bar{x}_{0}=1070-1140=-70 \text { (元/台) }
$$

计算结果表明, 全厂平均单位成本下降 $11.2 \%$, 各分厂单位成本下降使全厂单位成本下降 $5.4 \%$, 产量结构变动使全厂单位成本下降 $6.1 \%$, 即

$$
88.8 \%=94.6 \% \times 93.9 \%
$$

从绝对数上看, 全厂平均单位成本降低了 128 元, 各分厂单位成本下降使全厂单位成本降低 58 元, 产量结构变动使全厂单位成本降低 70 元, 即

$-128=-58+(-70)$

\section*{14.4 几种典型的指数}
作为一种重要的测评和分析方法, 指数在实践中获得了广泛的应用。指数最初是用来反映物价变化, 随后应用的领域不断扩展, 从经济领域拓展到社会领域。人们用指数描述社会发展状况, 用指数测定人们的感受。这里选择几个典型领域的指数: 物价指数领域 (以居民消费价格指数为例) 、金融指数领域(以股票价格指数为例)、社会心理领域(以消费者满意度指数为例), 旨在说明指数应用的领域, 介绍指数应用的不断发展。

\section*{14. 4. 1 居民消费价格指数}
居民消费价格指数 (consumer price index, CPI) 是度量居民消费品和服务项目价格水平随时间变动的相对数, 反映居民家庭购买的消费品和服务价格水平的变动情况。该指数是分析经济形势、检测物价水平、进行国民经济核算的重要指标, 也常用于测定通货膨胀,在国民经济生活中有着十分重要的作用。

我国居民消费价格指数于 1926 年开始编制, 当时在上海、天津等地编制工人生活费用指数, 编制者是南开大学社会经济研究委员会, 指数的分类有食物、衣着、房租、燃料、杂项共 5 类, 包括 37 种代表品, 采用加权平均指数方法。这是我国物价指数编制的起源。

新中国成立后,特别是改革开放后,我国居民消费价格指数的编制不断完善,表现在:分类更细, 代表品的数量不断增加, 权数的确定更贴近实际。目前, 居民消费价格指数按城乡分别编制, 分为 8 大类别, 每个大类中又分为中类和小类, 指数中共有 251 个小类近千种代表品, 权数的确定分别依据城市样本的约 40000 个家庭和农村样本的约 60000 个家庭的实际消费构成。指数编制的过程包括以下几个步骤。

\section*{1. 选择代表规格品}
代表规格品的选择是在商品分类基础上进行的, 选择的原则是:(1)销售数量(金额)大;(2)价格变动趋势和变动程度有代表性, 即中选规格品的价格变动与未中选商品的价格变动高度相关; (3) 所选的代表规格品之间性质相隔要远, 价格变动特征的相关性低;(4)选中的工业消费品必须是合格品,有注册商标、产地、规格等级等标识。

代表规格品每年可适当更换, 但更换数量的比例有限制, 以保证代表规格品的稳定性。

\section*{2. 选择调查市县和调查点}
选择的方法是划类选点。地区的选择既要考虑其代表性, 也要注意类型上的多样性以及地区分布上的合理性和稳定性。例如, 1992 年全国共选取 146 个市 80 个县作为取得数据的基层填报单位,在此基础上选定经营规模大、商品种类多的商场(包括集市)作为调查点。调查市县和调查点都是采用按有关标志排队、等距抽取的方法确定的。

\section*{3. 价格的调查与计算}
对代表规格品的采价原则是:(1)同一规格品的价格必须同质可比;(2)如挂牌价与成交价不同, 按成交价计;(3)与居民生活密切相关,价格变动频繁的商品至少每 5 天调查一次, 一般商品每月调查 $2 \sim 3$ 次。

代表规格品的平均价采用简单算术平均法计算。

\section*{4. 权数的确定}
居民消费价格指数的权数由全国样本的 10 万多个城乡居民家庭消费支出构成确定。其中各省(自治区、直辖市)城市和农村权数分别根据全省(自治区、直辖市)城镇居民家庭生活消费支出和农村居民家庭生活消费支出的现金支出资料整理计算。全国权数根据各省 (自治区、直辖市) 的权数按各地人均消费支出金额和人口数加权平均计算。大类、中类和小类的权数依次分层计算。

\section*{5. 指数计算}
总指数的计算采用加权平均方法, 计算公式为:


\begin{equation*}
I_{p}=\frac{\sum i W}{\sum W} \tag{14.21}
\end{equation*}


式中, $i$ 为代表规格品个体指数或各层的类指数; $W$ 为相应的消费支出比重。

具体计算过程是, 先分别计算出各代表规格品基期和报告期的全社会综合平均价, 并计算出相应的价格指数, 然后分层逐级计算小类、中类、大类和总指数。\\
1.1

\section*{例 14.8}
现以部分资料(见表 $14-5$ )说明消费品部分价格总指数的编制和计算过程。

表 14-5 零售价格总指数计算表

\begin{center}
\begin{tabular}{|c|c|c|c|c|c|c|c|}
\hline
\multirow{2}{*}{商品类别及名称} & \multirow{2}{*}{}\begin{tabular}{c}
代表 \\
规格品 \\
\end{tabular} & \multirow{2}{*}{}\begin{tabular}{l}
计量 \\
单位 \\
\end{tabular} & \multicolumn{2}{|c|}{平均价格(元)} & \multirow{2}{*}{}\begin{tabular}{c}
权数 $W$ \\
$(\%)$ \\
\end{tabular} & \multirow{2}{*}{}\begin{tabular}{c}
指数 $i$ \\
$(\%)$ \\
\end{tabular} & \multirow{2}{*}{$i W(\%)$} \\
\hline
 &  &  & $p_{0}$ & $p_{1}$ &  &  &  \\
\hline
总指数 &  &  &  &  & 100 & 111.6 & 111.60 \\
\hline
一、食品类 &  &  &  &  & 38 & 116.2 & 44.16 \\
\hline
1. 粮食 &  &  &  &  & 35 & 105.3 & 36.86 \\
\hline
细粮 &  &  &  &  & 65 & 105.6 & 68.64 \\
\hline
面粉 & 标准 & $\mathrm{kg}$ & 2.40 & 2.52 & 40 & 105.0 & 42.00 \\
\hline
大米 & 粳米标一 & $\mathrm{kg}$ & 3. 50 & 3.71 & 60 & 106.0 & 63.60 \\
\hline
粗粮 &  &  &  &  & 35 & 104.8 & 36.68 \\
\hline
2. 副食品 &  &  &  &  & 45 & 125.4 & 56.43 \\
\hline
3. 其他食品 &  &  &  &  & 20 & 114.8 & 22.96 \\
\hline
二、饮料、烟酒 &  &  &  &  & 5 & 126.0 & 6. 30 \\
\hline
三、服装、鞋帽 &  &  &  &  & 10 & 115.2 & 11. 52 \\
\hline
四、纺织品 &  &  &  &  & 3 & 99.3 & 2. 97 \\
\hline
五、家用电器及音像器材 &  &  &  &  & 8 & 94.2 & 7.53 \\
\hline
六、文化办公用品 &  &  &  &  & 2 & 110.4 & 2.21 \\
\hline
七、日用品 &  &  &  &  & 11 & 109.5 & 12.05 \\
\hline
八、体育娱乐用品 &  &  &  &  & 2 & 98.1 & 1. 96 \\
\hline
九、交通、通信用品 &  &  &  &  & 1 & 91.1 & 0.91 \\
\hline
十、家具 &  &  &  &  & 2 & 97.8 & 1.96 \\
\hline
十一、化妆品 &  &  &  &  & 1 & 98.9 & 0.99 \\
\hline
十二、金银珠宝 &  &  &  &  & 3 & 108.6 & 3. 26 \\
\hline
十三、中西药品及医疗保健用品 &  &  &  &  & 7 & 116.4 & 8. 15 \\
\hline
十四、书报杂志及电子出版物 &  &  &  &  & 2 & 108.6 & 2. 17 \\
\hline
十五、燃料 &  &  &  &  & 3 & 105.6 & 3. 17 \\
\hline
十六、建筑材料及五金电料 &  &  &  &  & 2 & 114.5 & 2.29 \\
\hline
\end{tabular}
\end{center}

\begin{itemize}
  \item —解 (1) 计算出各代表规格品的价格指数。如面粉价格指数为:
\end{itemize}

$$
i=\frac{p_{1}}{p_{0}}=\frac{2.52}{2.40}=105.0 \%
$$

(2)根据各代表规格品的价格指数及给出的相应权数, 采用加权算术平均法计算小类指数。如细粮类价格指数为:

$$
i_{p}=\frac{\sum i W}{\sum W}=\frac{105 \times 40+106 \times 60}{100}=105.6 \%
$$

(3)根据各小类指数及相应的权数,采用加权算术平均法计算中类指数。如粮食类价格指数为:

$$
i_{p}=\frac{\sum i W}{\sum W}=\frac{105.6 \times 65+104.8 \times 35}{100}=105.3 \%
$$

(4)根据各中类指数及相应的权数, 采用加权算术平均法计算大类指数。如食品类价格指数为:

$$
i_{p}=\frac{\sum i W}{\sum W}=\frac{105.3 \times 35+125.4 \times 45+114.8 \times 20}{100}=116.2 \%
$$

(5)根据各大类指数及相应的权数, 采用加权算术平均法计算总指数。有

$$
\begin{aligned}
I_{p} & =\frac{\sum i W}{\sum W} \\
& =\frac{116.2 \times 38+126 \times 5+115.2 \times 10+99.3 \times 3+\cdots+114.5 \times 2}{100} \\
& =\frac{11159.8}{100}=111.598 \%
\end{aligned}
$$

居民消费价格指数除了能反映城乡居民所购买的生活消费品和服务项目价格的变动趋势和程度, 还有以下几个方面的作用:

第一, 反映通货膨胀状况。通货膨胀的严重程度是用通货膨胀率来反映的, 它说明了一定时期内商品价格持续上升的幅度。通货膨胀率一般以居民消费价格指数来表示。计算公式为:

通货膨胀率 $=\frac{\text { 报告期居民消费价格指数一基期居民消费价格指数 }}{\text { 基期居民消费价格指数 }} \times 100 \%(14.22)$

$$
\text { 基期居民消费价格指数 }
$$

第二, 反映居民购买力水平。货币购买力是指单位货币购买到的消费品和服务的数量。居民消费价格指数上涨, 货币购买力则下降, 反之则上升, 因此, 居民消费价格指数的倒数就是货币购买力指数。计算公式为:


\begin{equation*}
\text { 货币购买力指数 }=\frac{1}{\text { 居民消费价格指数 }} \times 100 \% \tag{14.23}
\end{equation*}


第三,测定职工实际工资水平。居民消费价格指数提高意味着实际工资的减少,居民消费价格指数下降则意味着实际工资的提高。因此, 利用居民消费价格指数可以将名义工资转化为实际工资。计算公式为:


\begin{equation*}
\text { 实际工资 }=\frac{\text { 名义工资 }}{\text { 居民消费价格指数 }} \tag{14.24}
\end{equation*}


\subsection*{14.4. 2 股票价格指数}
目前, 金融指数产品创新层出不穷, 指数期货、指数期权、指数存托凭证、指数债券、指数存款等极大丰富了金融市场,指数化投资逐渐成为证券市场的重要投资方式。作为金融指数的代表, 股票价格指数(简称股价指数)最为大众所熟悉和关注,国际上许多著名的股价指数都是由专业指数公司编制和发布的。虽然股价指数的编制原理相同, 但在\\
具体问题上不同指数有各自的处理方法, 这里仅以我国的上证指数为例, 简要介绍股价指数的编制。

上证股价指数是由上海证券交易所编制并发布的指数系列,包括上证综合指数、上证 180 指数、上证 50 指数、上证 380 指数等。其中编制最早也最具典型意义的是上证综合指数。该指数自 1991 年 7 月 15 日起正式发布, 以 1990 年 12 月 19 日为基日, 基日指数为 100 点, 以现有所有上市股票(包括 A 股和 B 股)为样本, 以报告期股票发行量为权数进行编制,计算公式为:


\begin{equation*}
\text { 今日股价指数 }=\frac{\text { 今日市价总值 }}{\text { 基日市价总值 }} \times 100 \tag{14.25}
\end{equation*}


市价总值等于收盘价乘以发行股数, 遇发行新股或扩股时, 需要进行修正。

上证综合指数在编制上有几个特点:

(1)该指数包括挂牌上市的所有股票, 其优点是: 能全面、准确地反映某个时点股票价格的全面变动情况, 考虑到行业分布和不同公司的规模, 具有广泛的代表性。但它同时也有一些缺点: 一是敏感性差, 不能及时反映主要上市公司股票价格对市场大势的影响;二是只要有新股上市就要计人指数, 使得指数内部结构变动频繁, 影响了结构的稳定性和指数前后的可比性。

(2)该指数以发行量为权数, 这也是国际上通行的做法, 好处是比较全面。但我国股票发行中的法人股占相当比重, 且不能上市流通, 这样, 指数所反映的只能是流通市场的潜在能量, 而不是现实市场股价的综合变动。

这说明, 任何指数都是有局限性的, 不可能依靠一个指数说明所有问题, 还需要其他一些数据做补充说明。所以, 我国的股票指数也是一个系列。比如在上证指数中, 除上证综合指数, 还有上证 180 指数、上证 50 指数、上证 380 指数等股票指数做补充。认识到这一点, 有助于我们科学地看待目前社会上发布的各种指数。

\section*{14. 4 . 3 消费者满意度指数}
对消费者满意度的研究始于 20 世纪 70 年代, 研究的背景是, 一直以来企业都以其收人 (产值)、成本、利润等指标衡量经营业绩, 各国的发展水平也是以人均 GDP 来表现的, 但这些指标在为人们提供衡量经济发展水平的客观手段的同时, 逐渐淡化了经济发展的初衷一一为了使人类生活更加幸福。现代社会中一切发展都离不开人, 企业要具有可持续的竞争力, 不仅要重视产值、利润等指标, 也要重视黄金顾客、最有价值顾客等以消费者为导向的指标, 要逐渐认识到, 人类幸福本身才是社会进步的驱动因素和终极目标。人们从对经济资源生产效率的关注, 逐步转向对经济资源产出质量的关注。而消费者满意度正是从最终消费者的角度来衡量产出质量的指标, 其理论和实践是在适应经济发展和多方面需要的过程中产生并逐步发展起来的。

世界上第一个满意度指数由瑞典于 1989 年建立, 编制者运用瑞典统计局提供的数据,分别编制全国、各行业和各类公司满意度指数, 它们已经成为瑞典最有价值的国民经济指\\
标之一。之后各国纷纷开始编制各自的满意度指数。中国质量协会从 2004 年起开始编制汽车、家电、房地产等行业的满意度指数, 总体来说, 我国在编制消费者满意度指数方面还有很大的发展空间。

消费者满意度是一个经济心理学概念, 要衡量它就必须建立模型, 将消费者满意度与一些相关变量(如价值、质量、投诉行为、忠诚度等)联系起来。虽然各国满意度模型不尽相同, 但有着共同的基本框架。模型的前导变量有两个: 消费者对产品/服务的价值感知; 消费者对产品/服务的期望。满意度的结果变量是消费者投诉和消费者忠诚度。忠诚度是模型中最终的因变量, 可以作为消费者留存和企业利润的指示器。模型的框架如图 14-2 所示。

\begin{center}
\end{center}

图 14-2 消费者满意度模型框架图

可以看出, 这是一个结构方程模型, 有多个因变量, 构成一个原因和结果的关系网,通过模型参数估计, 计算各要素的得分, 然后计算指数。参数估计可以使用的方法有偏最小二乘方法和结构方程方法, 这些是比较专门的统计方法。

各国编制的消费者满意度指数的内容略有区别, 以美国消费者满意度指数为例, 指数的编制有国家、经济领域、行业和公司四个层次。全国消费者满意度指数包括 7 个经济领域, 34 个行业, 约 200 家公司。这 7 个领域都是向居民家庭提供产品或服务的重要领域,其产值占国民生产总值的 $75 \%$ 。

经济领域确定之后, 从中选择有代表性的行业, 选择的标准是: 行业集中度高, 且该行业中大部分公司的财务数据公开。选定行业后, 再从每个行业选出 10 20 家公司, 对每个公司随机抽取大约 250 名消费者进行调查, 利用调查数据进行模型参数估计, 计算出公司的消费者满意度指数。

从中选公司抽取用户时, 不是根据公司提供的用户名单抽取, 而是先按地区进行分层, 在中选地区内采用计算机辅助电话调查, 由计算机随机拨号产生样本, 每月抽取 6000 个消费者样本实施调查, 通过调查中得知的商品品牌将消费者与公司对应起来。

计算出各公司的消费者满意度指数后, 需要确定上一层次(行业)指数计算的权重,进而计算经济领域指数和全国指数。在计算行业指数时,以各公司销售份额的比重作为权数; 计算经济领域指数时, 以各行业销售份额的比重作为权数; 计算全国指数时, 以各经济领域 GDP 的贡献百分比作为权数。

消费者满意度指数是社会学、心理学的研究成果在管理和营销领域应用的具体体现,用指数方式描述人的主观感受和心理活动是指数应用领域拓展的一个重要方向。指数理论\\
和其他统计方法相结合,为指数的应用开拓了更大的空间。

\section*{14.5 综合评价指数}
\subsection*{14.5.1 综合评价与综合评价指数}
统计的综合评价是针对研究的对象, 建立一个进行测评的指标体系, 利用一定的方法或模型, 对搜集的资料进行分析,对被评价的事物作出量化的总体判断。

许多评判都是以综合评价的方式体现的。例如, 跳水比赛中的评判, 裁判员需要对运动员跳水完成情况分别打分, 评判的指标有技术完成情况和动作配合情况, 根据裁判员打分情况, 结合动作难度系数, 运动员得到每一跳的得分, 将每一跳得分加总, 得到最终得分, 根据最终得分排出名次。

应该说这还是比较简单的综合评价, 如果对一个国家的科技竞争力进行评价, 情况要复杂得多。首先要研究可以通过哪些指标反映科技竞争力, 其次要把不同指标的内容融合在一起, 而怎样融合, 每项指标的权重有多大, 就成为不太容易说清也不太容易达成共识的技术问题。在综合评价中, 对于同样的数据, 采用不同的方法, 可能有不同的结论, 评价结果并不是唯一的, 这是综合评价的特点。例如, 对世界著名大学进行排名, 一些国际有名的机构排名的结果各不相同就是最好的例证。

综合评价指数是将评价结果数量化的一种技术处理, 它将多指标进行综合, 最后形成概括性的一个指数, 通过指数比较达到评价的目的。所以说, 综合评价指数是指数理论与方法在其他领域的进一步发展和应用。

综合评价指数的基础是单项指标, 不同的单项指标通常不能直接进行加减乘除的运算, 需要将数据处理技术与指数分析方法结合起来。构建综合评价指数的具体方法有很多, 不同构建方法会带来不同的结果, 但构造指数的原理是相同的。

构建综合评价指数一般需要如下步骤:

(1) 建立综合评价指标体系。

所建立的指标体系是否科学与合理, 直接关系到评价结果的科学性和准确性。首先应进行必要的定性研究, 对所研究的问题进行深人的分析, 尽量选择具有一定综合意义的代表性指标。研究结果显示, 采用适当的统计方法, 如运用多元统计的方法进行指标的笁选, 可以提高指标的客观性。

(2)评价指标的无量纲化处理。

综合评价需要运用由多个指标组成的指数体系, 而这些指标的性质不同, 计量单位不同, 具有不同的量纲, 因此需要对各指标的实际数据进行无量纲化处理, 使之具有可比性, 在此基础上才有可能进行综合。

(3) 确定各项评价指标的权重。

这是一项重要而又容易引起争议的工作。对于不同的指标, 不同的人从不同的角度审\\
视, 会有不同的看法和评价, 所以在综合评价中如何确定各项指标的权重, 一直是学术界讨论的问题。简单地说, 确定评价指标的权重有两类方法: 一类是主观确定权数, 可以采用多种方式, 但共同特征是由有关专家通过研究讨论决定, 特点是可以集中专家集体智慧, 工作效率比较高, 但很难找到客观的评价标准。另一类是客观确定权数, 也有许多不同的实现方法, 但共同特征是权数由实际数据确定。例如, 经济指标的权数可以是该指标反映的现象的结果 (如产值) 占总体 (总产值) 的比重。也有一些研究采用专门的统计方法, 通过模型或其他计算方式产生权数。这类方法的特点是依据数据, 客观性更强, 但有时难以反映评价的导向性。因为综合评价是目标性的, 提倡什么、强调什么、鼓励什么、突出什么, 都可以通过权数反映出来。

(4)计算综合评价指数。

有了各项指标的无量纲化处理结果, 有了各项指标的权重, 通过适当的方法, 就可以得到综合评价指数。当然, 综合评价指数的计算并不是最终目的, 指数只是一个综合性的描述, 重要的是对这个结果背后的东西进行深人分析, 综合评价才有意义, 这就是进一步的统计分析问题了。

\section*{14. 5.2 综合评价指数的构建方法}
在构建综合评价指数时, 一个问题是指标的无量纲化处理, 对此有不同的处理方法。

\section*{1. 统计标准化}
这是一种通用的将具体数值转变为标准模式的做法, 公式为:


\begin{equation*}
z_{i}=\frac{x_{i}-\bar{x}}{s} \tag{14.26}
\end{equation*}


式中, $z_{i}$ 为第 $i$ 个指标的标准化值; $\bar{x}$ 为 $x_{i}$ 的均值; $s$ 为标准差。

由第 4 章内容可知, 一个变量的分布, 核心指标是均值和标准差。式 (14.26)将该变量的均值和标准差结合在一起考虑, 消除了变量分布不同的影响, 最具有统计意义。

\section*{2. 相对标准化}
这种方法是先给一个评价指标确定一个标准值, 然后将实际值和标准值进行比较, 实现指标的相对化处理, 公式为:


\begin{equation*}
z_{i}=\frac{x_{i}}{x_{s}} \tag{14.27}
\end{equation*}


式中, $x_{s}$ 为进行标准化确定的对比标准, 通常可以选择最优值或平均值作为对比标准。定的标准不同, 标准值的含义也就不同。这种方法可以体现评价者进行评价的目标性。例如, 是以最优水平还是以平均水平进行评价比较。

\section*{3. 功效系数法}
功效系数法是对多目标规划原理中的功效系数加以改进, 从而把要评价的指标转化为可以度量的评判分数, 公式为:


\begin{equation*}
z_{i}=\frac{x_{i}-\min \left(x_{i}\right)}{\max \left(x_{i}\right)-\min \left(x_{i}\right)} \tag{14.28}
\end{equation*}


式中, $\max \left(x_{i}\right)$ 和 $\min \left(x_{i}\right)$ 分别为指标 $x_{i}$ 的最大值和最小值。按该方法计算, $z_{i}$ 的取值在 $0 \sim 1$ 之间。本章开头的统计应用专栏中, 中国人民大学中国发展指数 (RCDI) 就是按功效系数法对各指标值进行标准化处理的。

将功效系数法进行一些拓展, 会得到改进的功效系数法, 公式为:


\begin{equation*}
z_{i}=\frac{x_{i}-\min \left(x_{i}\right)}{\max \left(x_{i}\right)-\min \left(x_{i}\right)} \times 40+60 \tag{14.29}
\end{equation*}


这样得到的标准化分数在 $60 \sim 100$ 之间。这种处理的效果在于, 可以减小极端数值对计算结果的视觉影响, 接近人们对分数的一般看法。

有了各指标的无量纲标准化处理, 再结合各指标的权重, 就可以计算综合评价指数。通常采用加权平均方式进行处理, 公式为:


\begin{equation*}
I=\frac{\sum_{i=1}^{n} z_{i} w_{i}}{\sum_{i=1}^{n} w_{i}} \tag{14.30}
\end{equation*}


式中, $0 \leqslant w_{i} \leqslant 1, \sum_{i=1}^{n} w_{i}=1$ 。

综合评价指数方法将指数传统意义上的动态对比延伸到不同现象横截面的比较, 再次表明,指数的应用领域十分广阔。

\section*{思考与练习}
\section*{思考题}
14.1 什么是指数? 它有哪些性质?

14.2 什么是同度量因素? 同度量因素在编制加权综合指数中有什么作用?

14.3 拉氏指数与帕氏指数各有什么特点?

14.4 加权平均指数与加权综合指数有何区别与联系?

14.5 什么是指数体系? 它有什么作用?

14.6 试述平均数指数体系。

14.7 构建综合评价指数的步骤是什么?

\section*{练习题}
14.1 某企业生产甲、乙两种产品, 资料如下:

\begin{center}
\begin{tabular}{ccccccc}
\hline
\multirow{2}{*}{产品名称} & 计量单位 & \multicolumn{2}{c}{产量} &  & \multicolumn{2}{c}{单位成本 (元)} \\
\cline { 6 - 7 }
 &  & 基期 & 报告期 &  & 基期 & 报告期 \\
\hline
甲 & 台 & 2000 & 2200 &  & 12.0 & 12.5 \\
乙 & 吨 & 5000 & 6000 &  & 6.2 & 6.0 \\
\hline
\end{tabular}
\end{center}

(1)计算产量与单位成本个体指数。

(2)计算两种产品产量总指数以及由于产量增加而增加的生产费用。

(3)计算两种产品单位成本总指数以及由于成本降低而节约的生产费用。

14.2 某商场销售的三种商品的资料如下:

\begin{center}
\begin{tabular}{|c|c|c|c|c|c|}
\hline
\multirow{2}{*}{产品名称} & \multirow{2}{*}{计量单位} & \multicolumn{2}{|c|}{销售量} & \multicolumn{2}{|c|}{单价(元)} \\
\hline
 &  & 基期 $q_{0}$ & 报告期 $q_{1}$ & 基期 $p_{0}$ & 报告期 $p_{1}$ \\
\hline
甲 & 千克 & 100 & 115 & 100 & 100 \\
\hline
乙 & 台 & 200 & 220 & 50 & 55 \\
\hline
丙 & 件 & 300 & 315 & 20 & 25 \\
\hline
\end{tabular}
\end{center}

(1) 计算三种商品的销售额总指数。

(2)分析销售量和价格变动对销售额影响的绝对值和相对值。

14.3 试根据下列资料分别用拉氏指数和帕氏指数计算销售量指数及价格指数。

\begin{center}
\begin{tabular}{ccccccc}
\hline
 &  & \multicolumn{2}{c}{销售量} &  & \multicolumn{2}{c}{单价 (元)} \\
\cline { 6 - 7 }
商品名称 & 计量单位 & 基期 $q_{0}$ & 报告期 $q_{1}$ &  & 基期 $p_{0}$ & 报告期 $p_{1}$ \\
\hline
甲 & 支 & 400 & 600 &  & 0.25 & 0.2 \\
乙 & 件 & 500 & 600 &  & 0.4 & 0.36 \\
丙 & 个 & 200 & 180 & 0.5 & 0.6 &  \\
\hline
\end{tabular}
\end{center}

14.4 某公司三种产品的有关资料见下表, 试问三种产品产量平均增长了多少? 产量增长对产值有什么影响?

\begin{center}
\begin{tabular}{cccc}
\hline
产品名称 & 个体产量指数 & 基期产值 (万元) & 报告期产值(万元) \\
\hline
甲 & 1.25 & 100 & 120 \\
乙 & 1.10 & 100 & 115 \\
丙 & 1.50 & 60 & 85 \\
\hline
\end{tabular}
\end{center}

14.5 三种商品销售的资料见下表, 通过计算说明其价格总的变动情况。

\begin{center}
\begin{tabular}{|c|c|c|c|}
\hline
\multirow{2}{*}{商品名称} & \multicolumn{2}{|c|}{商品销售总额(万元)} & \multirow{2}{*}{}\begin{tabular}{c}
报告期价格 \\
比基期降低 (\%) \\
\end{tabular} \\
\hline
 & $q_{0} p_{0}$ & $q_{1} p_{1}$ &  \\
\hline
甲 & 80 & 86 & 10 \\
\hline
乙 & 20 & 34 & 5 \\
\hline
丙 & 160 & 144 & 15 \\
\hline
\end{tabular}
\end{center}

14.6 某商场上期销售收入为 525 万元,本期要求达到 556.5 万元。在规定销售价格下调 $2.6 \%$ 的条件下,该商店商品销售量要增加多少,才能使本期销售达到原定的目标?

14.7 某地区 2003 年平均职工人数为 229.5 万人, 比 2002 年增加 $2 \% ; 2003$ 年工资总额为 167076 万元, 比 2002 年多支出 9576 万元。试推算 2002 年职工的平均工资。

14.8 某电子生产企业 2003 年和 2002 年三种主要产品的单位成本和产量资料如下表所示。

\begin{center}
\begin{tabular}{|c|c|c|c|c|c|}
\hline
\multirow{2}{*}{产品名称} & \multirow{2}{*}{计量单位} & \multicolumn{2}{|c|}{产量} & \multicolumn{2}{|c|}{单位产品成本(元)} \\
\hline
 &  & 2002 年 & 2003 年 & 2002 年 & 2003 年 \\
\hline
高能电池 & 节 & 900 & 1000 & 8.5 & 9.0 \\
\hline
电路板 & 块 & 500 & 500 & 55.0 & 58.5 \\
\hline
录音机 & 台 & 700 & 800 & 100.0 & 115.0 \\
\hline
\end{tabular}
\end{center}

(1)计算三种产品的产值总指数及产值增减总额。

(2) 以 2003 年的产量为权数计算三种产品的加权单位产品成本综合指数, 以及产量变动所导致的产值增减额。

(3) 以 2002 年的单位产品成本为权数计算三种产品的加权产量综合指数, 以及单位产品成本变动所导致的产值增减额。

(4) 分析产量和单位产品成本变动对销售额影响的相对值和增减额。

14.9 某工厂有三个生产车间,基期和报告期各车间的职工人数和劳动生产率资料见下表。试分析该工厂劳动生产率的变动及原因。

\begin{center}
\begin{tabular}{|c|c|c|c|c|}
\hline
\multirow{2}{*}{车间} & \multicolumn{2}{|c|}{职工人数} & \multicolumn{2}{|c|}{劳动生产率 (万元/人)} \\
\hline
 & 基期 $f_{0}$ & 报告期 $f_{1}$ & 基期 $x_{0}$ & 报告期 $x_{1}$ \\
\hline
一车间 & 200 & 240 & 4.4 & 4. 5 \\
\hline
二车间 & 160 & 180 & 6.2 & 6.4 \\
\hline
三车间 & 150 & 120 & 9.0 & 9.2 \\
\hline
合计 & 510 & 540 & 6.32 & 6.18 \\
\hline
\end{tabular}
\end{center}

14.10 下表是某地 2005 年粮食类零售价格指数计算简表, 请以表中资料说明编制商品零售价格指数的一般步骤, 并计算出 2005 年粮食类零售价格指数。

\begin{center}
\begin{tabular}{|c|c|c|c|c|c|c|c|}
\hline
\multirow[b]{2}{*}{商品类别} & \multirow[b]{2}{*}{}\begin{tabular}{l}
规格 \\
等级 \\
\end{tabular} & \multirow[b]{2}{*}{}\begin{tabular}{l}
计量 \\
单位 \\
\end{tabular} & \multicolumn{2}{|c|}{平均价格(元)} & \multirow[b]{2}{*}{}\begin{tabular}{c}
权数 W \\
$(\%)$ \\
\end{tabular} & \multicolumn{2}{|c|}{以上年为基期} \\
\hline
 &  &  & \begin{tabular}{c}
2004 年 \\
$p_{0}$ \\
\end{tabular} & \begin{tabular}{c}
2005 年 \\
$p_{1}$ \\
\end{tabular} &  & \begin{tabular}{c}
指数 $i$ \\
$(\%)$ \\
\end{tabular} & \begin{tabular}{l}
$i W$ \\
$(\%)$ \\
\end{tabular} \\
\hline
粮食类总指数 &  &  &  &  & 100 & 107.92 & 107.92 \\
\hline
1. 细粮小类 &  &  &  &  & 82 & 108.74 & 89. 17 \\
\hline
(1) 面粉 & 标准粉 & 千克 & 2.00 & 2. 20 & 56 & 110.00 & 61.60 \\
\hline
(2) 粳米 & 一等 & 千克 & 2. 80 & 3.00 & 44 & 107.14 & 47. 14 \\
\hline
2. 粗粮小类 &  &  &  &  & 18 & 104.18 & 18. 75 \\
\hline
\end{tabular}
\end{center}

\section*{附录一 术语表}
B

备择假设 (alternative hypothesis) 与原假设逻辑相反的假设。

比例 (proportion) 一个样本(或总体)中各个部分的数据占全部数据之比。

比率 (ratio) 样本 (或总体) 中不同类别数据之间的比值。

必然事件 (certain event) 在同一组条件下, 每次试验一定出现的事件。

变量 (variable) 说明现象某种特征的概念。

标准差 (standard deviation) 方差的平方根。

标准分数(standard score) 也称标准化值或分数, 它是变量值与其平均数的离差除以标

准差后的值。

标准化残差 (standardized residual) 残差除以它的标准差后得到的数值。

不规则波动 (irregular variations) 也称为随机性 (randomness), 指序列中的偶然性波动。

不可能事件 (impossible event) 在同一组条件下, 每次试验一定不出现的事件。

C

参数 (parameter) 用来描述总体特征的概括性数字度量, 它是研究者想要了解的总体的某种特征值。

残差 (residual) 因变量的观测值 $y_{i}$ 与根据估计的回归方程求出的预测值 $\hat{y}_{i}$ 之差, 用 $e$

表示。对于第 $i$ 个观测值,残差为 $e_{i}=y_{i}-\hat{y}_{i}$ 。

抽样分布 (sampling distribution) 样本统计量的分布。

抽样框 (sampling frame) 用于抽选样本的总体单位信息, 是概率抽样中所不可缺少的。
\footnotetext{\begin{itemize}
  \item 按术语的汉语拼音字母排序。
\end{itemize}
}
抽样误差 (sampling error) 由抽样的随机性引起的样本结果与总体真值之间的差异。处理 (treatment) 又称水平, 指因素的不同表现。

D

单因素方差分析 (one-way analysis of variance) 研究一个分类型自变量同数值型因变量之间关系的一种统计方法。

第 II 类错误(type II error) 原假设为伪却在检验中未拒绝原假设, 又称取伪错误或 $\beta$ 错误 ( $\beta$ error),用 $\beta$ 表示其概率。

第 I 类错误(type I error) 原假设为真却在检验中拒绝原假设, 又称弃真错误或 $\alpha$ 错误 ( $\alpha$ error),用 $\alpha$ 表示其概率。

点估计 (point estimate) 用样本估计量 $\theta$ 的某个取值直接作为总体参数 $\theta$ 的估计值。独立样本 (independent sample) 一个样本中的元素与另一个样本中的元素相互独立。对照组 (control group) 随机抽选的实验对象的子集。在这个子集中,每个单位不接受实验组成员所接受的某种特别的处理。

多重比较方法 (multiple comparison procedures) 通过对总体均值之间的配对比较来检验哪些均值之间存在差异的方法。

多重共线性 (multicollinearity) 回归模型中两个或两个以上的自变量彼此相关。

多重判定系数 (multiple coefficient of determination) 回归平方和占总平方和的比例, 反映因变量 $y$ 取值的变差中, 能被估计的回归方程所解释的比例。

多元回归方程 (multiple regression equation) 描述因变量 $y$ 的期望值与自变量 $x_{1}, x_{2}, \cdots$, $x_{k}$ 之间关系的方程。一般形式为: $E(y)=\beta_{0}+\beta_{1} x_{1}+\beta_{2} x_{2}+\cdots+\beta_{k} x_{k}$ 。

多元回归模型 (multiple regression model) 描述因变量 $y$ 如何依赖于自变量 $x_{1}, x_{2}, \cdots$, $x_{k}$ 和误差项 $\varepsilon$ 的方程。一般形式为: $y=\beta_{0}+\beta_{1} x_{1}+\beta_{2} x_{2}+\cdots+\beta_{k} x_{k}+\varepsilon$ 。

\section*{F}
$\boldsymbol{F}$ 分布 ( $\boldsymbol{F}$ distribution) 设随机变量 $Y$ 与 $Z$ 相互独立, 且 $Y$ 和 $Z$ 分别服从自由度为 $m$ 和 $n$ 的 $\chi^{2}$ 分布, 则 $X=\frac{Y / m}{Z / n}$, 称 $X$ 服从第一自由度为 $m$ 、第二自由度为 $n$ 的 $F$ 分布,记为 $F(m, n)$ 。

方差 (variance) 各变量值与其平均数离差平方的平均数。

方差分析 (analysis of variance,ANOVA) 通过检验各总体的均值是否相等来判断分类型自变量对数值型因变量是否有显著影响。

方差分析表 (analysis of variance table) 用来汇总方差分析计算和结果的表。

非抽样误差 (non-sampling error) 除抽样误差以外的, 由其他原因引起的样本观测结果

与总体真值之间的差异。

非概率抽样 (non-probability sampling) 根据研究目的对数据的要求选择样本单位。

非平稳序列(non-stationary series) 包含趋势、季节性或周期性的序列, 它可能只含有其中一种成分,也可能含有几种成分。\\
分类变量 (categorical variable) 说明事物类别的一个名称,其取值是分类数据。

分类数据 (categorical data) 只能归于某一类别的非数字型数据, 它是对事物进行分类的结果, 数据表现为类别, 是用文字来表述的。

峰度 (kurtosis) 数据分布峰值的高低。

\section*{G}
概率 (probability) 随机事件出现可能性大小的数值。

概率抽样 (probability sampling) 遵循随机原则进行的抽样, 总体中每个单位都有一定的机会被选入样本。

估计标准误差 (standard error of estimate) 度量各实际观测点在直线周围的散布状况的一个统计量, 它是均方残差 (MSE) 的平方根, 用 $s_{e}$ 表示。

估计的多元回归方程 (estimated multiple regression equation) 利用最小二乘法, 根据样本数据求出的多元回归方程的估计。其一般形式为: $\hat{y}=\hat{\beta}_{0}+\hat{\beta}_{1} x_{1}+\hat{\beta}_{2} x_{2}+\cdots+$ $\hat{\beta}_{k} x_{k}$ 。

估计的回归方程 (estimated regression equation) 根据样本数据求出的回归方程的估计。对于一元线性回归, 估计的回归方程形式为 $\hat{y}=\hat{\beta}_{0}+\hat{\beta}_{1} x$ 。

估计量 (estimator) 用来估计总体参数的统计量的名称, 用 $\hat{\theta}$ 表示。样本均值、样本比例、样本方差等都可以是一个估计量。

估计值 (estimated value) 根据一个具体的样本计算出来的估计量的数值。

观测数据 (observational data) 通过调查或观测收集到的数据, 这类数据是在没有对事物人为控制的条件下得到的。

H

回归方程 (regression equation) 描述因变量 $y$ 的期望值如何依赖于自变量 $x$ 的方程。一元线性回归方程的形式为 $E(y)=\beta_{0}+\beta_{1} x$ 。

回归模型 (regression model) 描述因变量 $y$ 如何依赖于自变量 $x$ 和误差项 $\varepsilon$ 的方程。只涉及一个自变量的一元线性回归模型可表示为 $y=\beta_{0}+\beta_{1} x+\varepsilon$ 。

\section*{J}
基本事件 (elementary event) 如果一个事件不能分解成两个或更多个事件, 则这个事件称为基本事件。

几何平均数 (geometric mean) $n$ 个变量值乘积的 $n$ 次方根。

季节性 (seasonality) 也称季节变动 (seasonal fluctuation), 指时间序列在一年内重复出现的周期性波动。

假设检验 (hypothesis testing) 利用样本信息, 对提出的命题进行检验的一套程序和方法。

截面数据(cross-sectional data)在相同或近似相同的时间点上收集的数据, 用于描述现象在某一时刻的变化情况。\\
居民消费价格指数 (consumer price index, CPI) 度量居民消费品和服务项目价格水平随时间变动的相对数, 反映居民家庭购买的消费品和服务价格水平的变动情况。

均方 (mean square) 平方和除以相应的自由度的结果。

均值 (mean) 也称为平均数, 它是一组数据相加后除以数据的个数得到的结果。

K

卡方分布 (Chi-square distribution) 即 $\chi^{2}$ 分布, 设随机变量 $X_{1}, X_{2}, \cdots, X_{n}$ 相互独立,

且 $X_{i}(i=1,2, \cdots, n)$ 服从标准正态分布 $N(0,1)$, 则它们的平方和 $\sum_{i=1}^{n} X_{i}^{2}$ 服从自由度为 $n$ 的 $\chi^{2}$ 分布。

$\mathbf{L}$

拉氏指数 (Laspeyres index) 拉斯贝尔斯提出的一种指数计算方法, 它在计算综合指数时将作为权数的同度量因素固定在基期。

累积频数 (cumulative frequencies) 将各有序类别或组的频数逐级累加起来得到的频数。

离散系数 (coefficient of variation) 也称变异系数, 是一组数据的标准差与其相应的平均数之比, 是测度数据离散程度的统计量。

离散型随机变量 (discrete random variable) 如果随机变量 $X$ 的所有取值都可以逐个列举出来, 则称 $X$ 为离散型随机变量。

连续型随机变量 (continuous random variable) 如果随机变量 $X$ 的所有取值无法逐个列举出来, 则称 $X$ 为连续型随机变量。

列联表 (contingency table) 由两个或两个以上的变量交叉分类的频数分布表。

列联系数 (coefficient of contingency) 简称 $c$ 系数, 是描述列联表数据相关程度的系数,通常用于列联表大于 $2 \times 2$ 的情况。

$\mathbf{P}$

帕氏指数 (Paasche index) 帕舍提出的一种指数计算方法, 它在计算综合指数时将作为权数的同度量因素固定在报告期。

判定系数 (coefficient of determination) 回归平方和占总平方和的比例, 用 $R^{2}$ 表示, 它是对估计的回归方程拟合优度的度量。

匹配样本 (matched sample) 一个样本中的数据与另一个样本中的数据相对应。

偏度 (skewness) 数据分布的不对称性。

频数(frequency)落在某一特定类别或组中的数据个数。

频数分布 (frequency distribution) 各个类别及其相应的频数形成的分布。

平均差 (mean deviation) 也称平均绝对离差, 它是各变量值与其平均数离差绝对值的平均数。

平均增长率 (average rate of increase) 时间序列中逐期环比值(也称环比发展速度)的几何平均数减 1 后的结果, 用于描述现象在整个观察期内平均增长变化的程度。\\
平稳序列(stationary series) 基本上不存在趋势的序列, 该序列中的各观察值基本上在某个固定的水平上波动。

Q

区间估计 (interval estimate) 在点估计的基础上, 给出总体参数估计的一个区间范围,该区间通常由样本统计量加减抽样误差得到。

趋势(trend) 时间序列在长期内呈现出来的某种持续上升或持续下降的变动。时间序列中的趋势可以是线性的,也可以是非线性的。

全距 (range) 也称极差, 是一组数据的最大值与最小值之差。

$\mathbf{S}$

时间序列 (time series) 同一现象在不同时间的相继观察值排列而成的序列。

时间序列数据(time series data)在不同时间收集到的数据, 用于描述现象随时间变化的情况。

实验数据(experimental data)在实验中控制实验对象而收集到的数据。

实验组(experiment group)随机抽选的实验对象的子集。在这个子集中,每个单位接受某种特别的处理。

数值变量(metric variable) 说明事物数字特征的一个名称, 其取值是数值数据。

数值数据(metric data)按数字尺度测量的观察值, 其结果表现为具体的数值。

双因素方差分析(two-way analysis of variance)研究两个分类型自变量同数值型因变量之间关系的一种统计方法。

四分位差 (quartile deviation) 也称四分位距,它是上四分位数与下四分位数之差。

四分位数 (quartile) 也称四分位点, 它是一组数据排序后处于 $25 \%$ 和 $75 \%$ 位置上的值。

随机事件 (random event) 在同一组条件下, 每项试验可能出现也可能不出现的事件。

T

$t$ 分布 ( $t$ distribution) 设随机变量 $X \sim N(0,1), Y \sim \chi^{2}(n)$, 且 $X$ 与 $Y$ 独立, 则 $t=$

$X / \sqrt{\frac{Y}{n}}$ ,其分布称为 $t$ 分布, 记为 $t(n)$ 。

调整的多重判定系数 (adjusted multiple coefficient of determination) 用样本量 $n$ 和自变量的个数 $k$ 去调整 $R^{2}$ 的结果, 记为 $R_{\alpha}^{2}$ 。

统计量 (statistic) 描述样本特征的概括性数字度量。

统计学 (statistics) 收集、处理、分析、解释数据并从数据中得出结论的科学。

$\mathbf{V}$

$\boldsymbol{V}$ 相关系数(V correlation coefficient)一种描述列联表数据相关程度的系数, 可用于多种情况的列联表。在 $2 \times 2$ 列联表情况下, $V$ 相关系数与 $\varphi$ 相关系数相同。

W

无偏性 (unbiasedness) 估计量抽样分布的数学期望等于被估计的总体参数。\\
$\mathbf{X}$

显著性水平 (significance level) 在原假设正确的情况下,错误地拒绝原假设的概率。显著性水平用 $\alpha$ 表示。

相关关系 (correlation) 变量之间存在的一种不确定的数量关系, 一个变量的取值不能由另一个变量唯一确定。

相关系数 (correlation coefficient) 根据样本数据计算的度量两个变量之间线性关系强度的统计量。

Y

样本 (sample) 从总体中抽取的一部分元素的集合, 构成样本的元素的数目称为样本量。移动平均法 (moving average) 对时间序列逐期递移求得一系列平均数作为预测值的一种方法。

一致性 (consistency) 随着样本量的增大, 估计量的值越来越接近被估总体的参数。

因变量 (dependent variable) 被预测或被解释的变量, 用 $y$ 表示。

因素/因子 (factor) 检验的对象,所研究的分类型变量的另一个名称。

有效性 (efficiency) 用于估计同一总体参数的两个无偏估计量, 有更小标准差的估计量更有效。

预测区间估计(prediction interval estimate)对 $x$ 的一个给定值 $x_{0}$, 因变量 $y$ 的一个个别值的区间估计。

原假设(null hypothesis) 提出一个(或两个)参数是否等于(或大于、小于)某个特殊值的命题。

\section*{Z}
增长率 (growth rate) 也称增长速度, 是时间序列中报告期观察值与基期观察值之比减 1 后的结果, 用百分数表示。有环比增长率和定基增长率之分。

指数平滑法 (exponential smoothing) 对过去的观察值加权平均进行预测的一种方法。

置信区间 (confidence interval) 由样本统计量所构造的总体参数的估计区间。

置信区间估计 (confidence interval estimate) 对于 $x$ 的一个给定值 $x_{0}$, 因变量 $y$ 的平均值的区间估计。

置信水平 (confidence level) 也称为置信度或置信系数, 它是将构造置信区间的步骤重复多次,置信区间中包含总体参数真值的次数所占的比例。

中位数 (median) 一组数据排序后处于中间位置上的数值, 用 $M_{e}$ 表示。

中心极限定理 (central limit theorem) 设从均值为 $\mu$ 、方差为 $\sigma^{2}$ (有限) 的任意一个总体中抽取样本量为 $n$ 的样本, 当 $n$ 充分大时, 样本均值 $\bar{X}$ 的抽样分布近似服从均值为 $\mu$ 、方差为 $\sigma^{2} / n$ 的正态分布。

众数 (mode) 一组数据中出现次数最多的数值, 用 $M_{o}$ 表示。

周期性 (cyclicity) 也称为循环波动 (cyclical fluctuation), 指时间序列中呈现出来的围\\
绕长期趋势的一种波浪形或振荡式变动。

自变量 (independent variable) 用来预测或解释因变量的一个或多个变量, 用 $x$ 表示。

自由度 (degree of freedom) 附加给独立的观测值的约束或限制的个数。

总平方和 (sum of squares for total) 记为 $S S T$, 它是全部观测值 $x_{i j}$ 与总均值 $\bar{x}$ 的误差平方和。

总体 (population) 包含所研究的全部个体(或数据)的集合。

组间平方和 (sum of squares for factor A) 记为 $S S A$, 它是各组均值 $\bar{x}_{i}$ 与总均值 $\bar{x}$ 的误差平方和, 反映各样本均值之间的差异程度, 又称为因素平方和。

组距 (class width) 一个组的上限值与下限值之差。

组内平方和 (sum of squares for error) 记为 SSE, 它是每个水平或组的各样本数据与其组均值的误差平方和, 反映每个样本各观测值的离散状况, 又称为误差平方和。

组中值 (class midpoint) 每一组中下限值与上限值中间的值, 即组中值 $=$ (下限值 + 上

限值) $/ 2$ 。

最小二乘法 (method of least squares) 通过使因变量的观察值 $y_{i}$ 与估计值 $\hat{y}_{i}$ 之间的离差平方和达到最小来估计 $\beta_{0}$ 和 $\beta_{1}$ 的方法,也称为最小平方法。

\section*{附录二 用 Excel 生成概率分布表 }
\section*{附表 1}
\begin{center}
\end{center}

利用 Excel 提供的统计函数 “NORM. S. DIST”, 可以生成标准正态分布概率表, 即 $P(Z \leqslant x)$ 。生成标准正态分布概率表的具体操作步骤如下。

第 1 步: 将 $x$ 的值(可根据需要确定)输人到工作表的 A 列, 将 $x$ 取值的尾数输人到第 1 行, 形成标准正态分布表的表头, 如下表所示。

\begin{center}
\begin{tabular}{|c|c|c|c|c|c|c|c|c|c|c|c|}
 & $\mathrm{A}$ & $\mathrm{B}$ & $\mathrm{C}$ & $\mathrm{D}$ & $\mathrm{E}$ & $\mathrm{F}$ & $\mathrm{G}$ & $\mathrm{H}$ & $\mathrm{I}$ & $\mathrm{J}$ & $\mathrm{K}$ \\
\hline
1 & $\boldsymbol{x}$ & $\mathbf{0 . 0 0}$ & $\mathbf{0 . 0 1}$ & $\mathbf{0 . 0 2}$ & $\mathbf{0 . 0 3}$ & $\mathbf{0 . 0 4}$ & $\mathbf{0 . 0 5}$ & $\mathbf{0 . 0 6}$ & $\mathbf{0 . 0 7}$ & $\mathbf{0 . 0 8}$ & $\mathbf{0 . 0 9}$ \\
\hline
2 & $\mathbf{0 . 0}$ &  &  &  &  &  &  &  &  &  &  \\
\hline
3 & $\mathbf{0 . 1}$ &  &  &  &  &  &  &  &  &  &  \\
\hline
4 & $\mathbf{0 . 2}$ &  &  &  &  &  &  &  &  &  &  \\
\hline
5 & $\mathbf{0 . 3}$ &  &  &  &  &  &  &  &  &  &  \\
\hline
6 & $\mathbf{0 . 4}$ &  &  &  &  &  &  &  &  &  &  \\
\hline
7 & $\mathbf{0 . 5}$ &  &  &  &  &  &  &  &  &  &  \\
\hline
8 & $\mathbf{0 . 6}$ &  &  &  &  &  &  &  &  &  &  \\
\hline
9 & $\mathbf{0 . 7}$ &  &  &  &  &  &  &  &  &  &  \\
\hline
10 & $\mathbf{0 . 8}$ &  &  &  &  &  &  &  &  &  &  \\
\hline
11 & $\mathbf{0 . 9}$ &  &  &  &  &  &  &  &  &  &  \\
\hline
12 & $\mathbf{1 . 0}$ &  &  &  &  &  &  &  &  &  &  \\
\hline
\end{tabular}
\end{center}

第 2 步: 在 B2 单元格输人公式 “= NORM. S. DIST ( \$ A $2+B \$ 1) ”$, 然后将其向下、向右复制即可得到标准正态分布概率表, 部分结果如下表所示(读者可根据需要生成不同 $x$ 的标准正态分布概率表)。

\begin{center}
\begin{tabular}{|c|c|c|c|c|c|c|c|c|c|c|c|}
\hline
 & A & B & C & D & E & F & G & $\mathrm{H}$ & I & $\mathrm{J}$ & K \\
\hline
1 & $x$ & 0.00 & 0.01 & 0.02 & 0.03 & 0.04 & 0.05 & 0.06 & 0.07 & 0.08 & 0.09 \\
\hline
2 & 0 & 5000 & 0.5040 & 0.5080 & 0.5120 & 5160 & 0.5199 & 0.5239 & 5279 & .5319 & $\because$ \\
\hline
3 & 1 & 5398 & 5438 & 0.5478 & 0.5517 & .5557 & .5596 & .5636 & .5675 & .5714 & 5753 \\
\hline
4 &  & 5793 & 0.5832 & 0.5871 & .5910 & 0.5948 & 0.5987 & .6026 & 0.6064 & .6103 &  \\
\hline
5 & .3 & 6179 & 0.6217 & .6255 & .6293 & 0.6331 & 0.6368 & .6406 & .6443 & .6480 & 6517 \\
\hline
6 &  & 6554 & 0.6591 & 1.6628 & 0.6664 & 0.6700 & 0.6736 & .6772 & 08 & 44 & 9 \\
\hline
7 & 5 & 6915 & 0.6950 & 0.6985 & 0.7019 & 0.7054 & 0.7088 & .7123 & 0.7157 & 0.7190 &  \\
\hline
8 & 6 & 725 & 0.7291 & 0.7324 & 0.7357 & 0.7389 & 0.7422 & 7454 & 0.7486 & .7517 & 07 \\
\hline
9 & .7 & 7580 & 0.7611 & .7642 & 3 & 0.7704 & 34 & 64 & 0.7794 & .7823 &  \\
\hline
10 & 0.8 & 7881 & 0.79 & 0.79 & 0 & 5 & 0.80 &  &  & 106 &  \\
\hline
11 &  & 81 &  & 0 &  &  &  &  &  &  &  \\
\hline
12 & 1.0 & 34 & 0.8438 & 0.8 & 0 & 0.8508 & 0.8531 & 4 &  & 9 &  \\
\hline
13 &  &  &  &  &  &  &  &  &  &  &  \\
\hline
14 & 1.2 & 70 & 0.8869 & 0.8 & 0 & 0.8 &  &  &  &  &  \\
\hline
15 &  &  &  &  &  &  &  &  &  &  &  \\
\hline
16 & 1.4 &  &  &  &  &  &  &  &  &  &  \\
\hline
17 &  &  &  &  &  &  &  &  &  &  &  \\
\hline
18 & 1. &  &  &  &  &  &  &  &  &  &  \\
\hline
19 &  &  &  &  &  &  &  &  &  &  &  \\
\hline
20 & 1.8 &  & 0 &  &  &  &  &  &  &  &  \\
\hline
21 & 1.9 &  & 0.97 &  &  &  &  &  &  &  &  \\
\hline
22 & 2 &  &  &  &  &  &  &  &  &  &  \\
\hline
23 & 2.1 &  &  &  &  &  &  &  &  &  &  \\
\hline
24 & 0 &  & 0 & 0.98 & 0.98 &  &  &  &  &  &  \\
\hline
25 & 2.3 &  &  &  &  &  &  &  &  &  &  \\
\hline
26 & 2 &  & 99 & 0.98 & 0.99 &  &  &  &  &  &  \\
\hline
27 &  &  &  &  &  &  &  &  &  &  &  \\
\hline
28 & 6 &  & c &  &  &  &  &  &  &  &  \\
\hline
29 &  &  &  &  &  &  &  &  &  &  &  \\
\hline
30 & 2 &  &  &  &  &  &  &  &  &  &  \\
\hline
31 &  &  &  &  &  &  &  &  &  &  &  \\
\hline
32 & 3 &  &  &  &  &  &  &  &  &  &  \\
\hline
33 &  &  & 0.35 & 0.99 &  &  &  &  &  &  &  \\
\hline
34 &  &  &  &  &  &  &  &  &  &  &  \\
\hline
35 &  &  & -4 & 0.99 &  &  &  &  &  &  &  \\
\hline
36 &  &  &  &  &  &  &  &  &  &  &  \\
\hline
37 &  &  &  &  &  &  &  &  &  &  &  \\
\hline
38 &  &  &  &  &  &  &  &  &  &  &  \\
\hline
39 &  &  &  &  &  &  &  &  &  &  &  \\
\hline
40 &  &  &  &  &  &  &  &  &  &  &  \\
\hline
41 &  &  &  &  &  &  &  &  &  &  &  \\
\hline
42 & 0. & 0000 & 0000 & .0000 & 1.0000 & 1.0000 & 1.0000 & 1.0000 & 1.0 & 1. & 1.000 \\
\hline
\end{tabular}
\end{center}

利用 Excel 提供的统计函数 “NORM. S. INV”, 可以生成标准正态分布分位数表, 该表是根据标准正态分布随机变量分布的累积概率的值计算的相应的临界值。如果 $P(Z \leqslant x)=p$,则对于任意给定的 $p(0 \leqslant p \leqslant 1)$, 可以求出相应的 $x$ 。用 Excel 生成标准正态分布分位数表的具体操作步骤如下。

第 1 步: 将标准正态变量累积概率的值输人到工作表的 A 列, 其尾数输人到第 1 行, 形成标准正态分布分位数表的表头, 如下表所示。

\begin{center}
\begin{tabular}{|c|c|c|c|c|c|c|c|c|c|c|c|}
\hline
 & $\mathrm{A}$ & $\mathrm{B}$ & $\mathrm{C}$ & $\mathrm{D}$ & $\mathrm{E}$ & $\mathrm{F}$ & $\mathrm{G}$ & $\mathrm{H}$ & $\mathrm{I}$ & $\mathrm{J}$ & $\mathrm{K}$ \\
\hline
1 & $\boldsymbol{p}$ & $\mathbf{0 . 0 0 0}$ & $\mathbf{0 . 0 0 1}$ & $\mathbf{0 . 0 0 2}$ & $\mathbf{0 . 0 0 3}$ & $\mathbf{0 . 0 0 4}$ & $\mathbf{0 . 0 0 5}$ & $\mathbf{0 . 0 0 6}$ & $\mathbf{0 . 0 0 7}$ & $\mathbf{0 . 0 0 8}$ & $\mathbf{0 . 0 0 9}$ \\
\hline
2 & $\mathbf{0 . 5 0}$ &  &  &  &  &  &  &  &  &  &  \\
\hline
3 & $\mathbf{0 . 5 1}$ &  &  &  &  &  &  &  &  &  &  \\
\hline
4 & $\mathbf{0 . 5 2}$ &  &  &  &  &  &  &  &  &  &  \\
\hline
5 & $\mathbf{0 . 5 3}$ &  &  &  &  &  &  &  &  &  &  \\
\hline
6 & $\mathbf{0 . 5 4}$ &  &  &  &  &  &  &  &  &  &  \\
\hline
7 & $\mathbf{0 . 5 5}$ &  &  &  &  &  &  &  &  &  &  \\
\hline
8 & $\mathbf{0 . 5 6}$ &  &  &  &  &  &  &  &  &  &  \\
\hline
9 & $\mathbf{0 . 5 7}$ &  &  &  &  &  &  &  &  &  &  \\
\hline
10 & $\mathbf{0 . 5 8}$ &  &  &  &  &  &  &  &  &  &  \\
\hline
11 & $\mathbf{0 . 5 9}$ &  &  &  &  &  &  &  &  &  &  \\
\hline
12 & $\mathbf{0 . 6 0}$ &  &  &  &  &  &  &  &  &  &  \\
\hline
13 & $\mathbf{0 . 6 1}$ &  &  &  &  &  &  &  &  &  &  \\
\hline
14 & $\mathbf{0 . 6 2}$ &  &  &  &  &  &  &  &  &  &  \\
\hline
15 & $\mathbf{0 . 6 3}$ &  &  &  &  &  &  &  &  &  &  \\
\hline
16 & $\mathbf{0 . 6 4}$ &  &  &  &  &  &  &  &  &  &  \\
\hline
17 & $\mathbf{0 . 6 5}$ &  &  &  &  &  &  &  &  &  &  \\
\hline
18 & 0.66 &  &  &  &  &  &  &  &  &  &  \\
\hline
19 & $\mathbf{0 . 6 7}$ &  &  &  &  &  &  &  &  &  &  \\
\hline
20 & 0.68 &  &  &  &  &  &  &  &  &  &  \\
\hline
21 & 0.69 &  &  &  &  &  &  &  &  &  &  \\
\hline
22 & $\mathbf{0 . 7 0}$ &  &  &  &  &  &  &  &  &  &  \\
\hline
23 & $\mathbf{0 . 7 1}$ &  &  &  &  &  &  &  &  &  &  \\
\hline
\end{tabular}
\end{center}

第 2 步: 在 B2 单元格输人公式 “=NORM. S. INV $(\$ \mathrm{~A} 2+\mathrm{B} \$ 1)$, 然后将其向下、向右复制即可得到标准正态分布分位数表, 部分结果如下表所示 (读者可根据需要生成不同 $p$ 值的标准正态分 $\mathrm{F}$ 布分位数表)。

\begin{center}
\begin{tabular}{|c|c|c|c|c|c|c|c|c|c|c|c|}
\hline
 & $\rho$ & 0.000 & 0.001 & 0.002 & 0.003 & 0.004 & 0.005 & 0.006 & 0.007 & 0.008 & 0.009 \\
\hline
 & 0.50 & 0.0000 & 0.0025 & 0.0050 & 0.0075 & 0.0100 & 0.0125 & 0.0150 & 0.0175 & 0.0201 & 0.0226 \\
\hline
 & 0.51 & 0.0251 & 0.0276 & 0.0301 & 0.0326 & 0.0351 & 0.0376 & 0.0401 & 0.0426 & 0.0451 & 0.0476 \\
\hline
 & 0.52 & 0.0502 & 0.0527 & 0.0552 & 0.0577 & 0.0602 & 0.0627 & 0.0652 & 0.0677 & 0.0702 & 0.0728 \\
\hline
 & 0.53 & 0.0753 & 0.0778 & 0.0803 & 0.0828 & 0.0853 & 0.0878 & 0.0904 & 0.0929 & 0.0954 & 0.0979 \\
\hline
 & 0.54 & 0.1004 & 0.1030 & 0.1055 & 0.1080 & 0.1105 & 0.1130 &  & 0.1181 & 0.1206 & 0.1231 \\
\hline
 & 0.55 & 0.1257 & 0.1282 & 0.1307 & 0.1332 & 0.1358 & 0.1383 & 0.1408 & 0.1434 & 0.1459 & 0.1484 \\
\hline
 & 0.56 & 0.1510 & 0.1535 & 0.1560 & 0.1586 & 0.1611 & 0.1637 & 0.1662 & 0.1687 & 0.1713 & 0.1738 \\
\hline
 & 0.57 & 0.1764 & 0.1789 & 0.1815 & 0.1840 & 0.1866 & 0.1891 & 0.1917 & 0.1942 &  & 93 \\
\hline
 & 0.58 & 0.2019 & 0.2045 & 0.2070 & 0.2096 & 0.2121 & 0.2147 & 0.2173 & 0.2198 & 0.2224 & 0.2250 \\
\hline
 & 0.59 & 0.2275 & 0.2301 & 0.2327 & 0.2353 & 0.2378 & 0.2404 & 0.2430 & 0.2456 & 0.2482 & 0.2508 \\
\hline
 & 0.60 & 0.2533 & 0.2559 & 0.2585 & 0.2611 & 0.2637 & 0.2663 & 0.2689 & 0.2715 &  &  \\
\hline
 & 0.61 & 0.2793 & 0.2819 & 0.2845 & 0.2871 & 0.2898 & 0.2924 & 0.2950 & 0.2976 & 0.3002 & 0.3029 \\
\hline
 & 0.62 & 0.3055 & 0.3081 & 0.3107 & 0.3134 & 0.3160 & 0.3186 & 0.3213 & 0.3239 & 0.3 & 0.3292 \\
\hline
 & 0.63 & 0.3319 & 0.3345 & 0.3372 & 0.3398 & 0.3425 & 0.3451 & 0.3478 & 0.3505 & 3531 &  \\
\hline
 & 0.64 & 0.3585 & 0.3611 & 0.3638 & 0.3665 & 0.3692 & 0.3719 & 0.3745 & 0.3772 & 0.3799 & 0.3826 \\
\hline
 & 0.65 & 0.3853 & 0.3880 & 0.3907 & 0.3934 & 0.3961 & 0.3989 & 0.4016 & 0.4043 & 70 & 0.4 \\
\hline
 & 0.66 & 0.4125 & 0.4152 & 0.4179 & 0.4207 & 0.4234 & 0.4261 & 0.4289 & 0.4316 & 0.4344 & 0.4372 \\
\hline
 & 0.67 & 0.4399 & 0.4427 & 0.4454 & 0.4482 & 0.4510 & 0.4538 & 0.4565 & 0.4593 & 0.4621 & 04649 \\
\hline
 & 0.68 & 0.4677 & 0.4705 & 0.4733 & 0.4761 & 0.4789 & 0.4817 & 0.4 & 0.4874 & 22 & 30 \\
\hline
 & 0.69 & 0.4959 & 0.4987 & 0.5015 & 0.5044 & 0.5072 & 0.5101 & 0.5129 & 0.5158 & 0.5187 & 0. \\
\hline
 & 0.70 & 0.5244 & 0.5273 & 0.5302 & 0.5 & 0.5 & 0.5388 & 0.5417 & 0.5 & 76 & 04 \\
\hline
 & 0.71 & 0.5534 & 0.5563 & 0.5592 & 0.5622 & 0.56 & 0.5681 & 0.5 & 0.5 & 0 &  \\
\hline
 & 0.72 & 0.5828 & 0.5858 & 0.5888 & 0.5918 & 0.5948 & 0.5978 & \begin{tabular}{|l|l|l|}
 \\
\end{tabular} & 0.6038 & 0.6068 & 0. \\
\hline
 & 0.73 & 0.6128 & 0.6158 & 0.6189 & 0.6 & 0.6 & 0.6 & 0.6 &  & 2 & $0 \mathrm{E}$ \\
\hline
 & 0.74 & 0.6433 & 0.6464 & 0.6495 & 0.6526 & 0.6557 & 0.6588 & 0.6620 & 0.6 & 0 & 0. \\
\hline
 & 0.75 & 0.6745 & 0.6776 & 0.6808 & 0.6840 & 0.6871 & 0.6903 & 0.6935 & 0.6967 & 0.6999 & 0.7031 \\
\hline
 & 0.76 & 0.7063 & 0.7095 & 0.7128 & 0.71 & 0.7 & 0.7 & 0.7 & 0.7 &  & 0.7356 \\
\hline
2 & 0.77 & 0.7388 & 0.7421 & 0.7454 & 0.7488 & 0.7521 & 0.7554 & 0.7588 & 0.7 & 0 . & 0.7 \\
\hline
 & 0.78 & 0.7722 & 0.7756 & 0.7790 & 0.7824 & 0.7858 & 0.7892 & 0.7926 & 0.7961 & 0.7995 & 0.8 \\
\hline
 & 0.79 & 0.8064 & 0.8099 & 0.8134 & 0.8 & 0.8 & 0.8 & 0.8274 & 0.8 & 15 & 08 \\
\hline
32 & 0.80 & 0.8416 & 0.8452 & 0.8488 & 0.8524 & 0.8560 & 0.8596 & 0.8633 & 0.8 & 0.8 & 0.8 \\
\hline
 & 0.81 & 0.8779 & 0.8816 & 0.8853 & 0.8890 & 0.8927 & 0.8965 & 0.9002 & 0.9040 & 0.9078 & 0.9116 \\
\hline
 & 0.82 & 0.9154 & 0.9192 & 0.9230 & 0.9269 & 0.9307 & 0.9346 & 0.9385 & 0.9424 & 0. & 0.9502 \\
\hline
 & 0.83 & 0.9542 & 0.9581 & 0.9621 & 0.9661 & 0.9701 &  & 0.9782 & 0.9822 & 0. &  \\
\hline
 & 0.84 & 0.9945 & 0.9986 & 1.0027 & 1.0069 & 1.0110 & 1.0152 & 1.0194 & 1.0237 & 1.0279 & 1.0322 \\
\hline
 & 0.85 & 1.0364 & 1.0407 & 1.0450 & 1.0494 & 1.0537 & 1.0581 & 1.0625 & 1.0669 & 1.0714 & 1.0758 \\
\hline
 & 0.86 & 1.0803 & 1.0848 & 1.0893 & 1.0939 & 1.0985 & 1.1031 & 1.1077 & 1.1123 &  &  \\
\hline
 & 0.87 & 1.1264 & 1.1311 & 1.1359 & 1.1407 & 1.1455 & 1.1503 & 1.1552 & 1.1601 & 1.1650 & 1.1700 \\
\hline
4 & 0.88 & 1.1750 & 1.1800 & 1.1850 & 1.1901 & 1.1952 & 1.2004 & 1.2055 & 1.2107 & 1.2160 & 1.2212 \\
\hline
 & 0.89 & 1.2265 & 1.2319 & 1.2372 & 1.2426 & 1.2481 & 1.2536 & 1.2591 & 1.2646 & 1.2702 &  \\
\hline
 & 0.90 & 1.2816 & 1.2873 & 1.2930 & 1.2988 & 1.3047 & 1.3106 & 1.3165 & 1.3225 & 1.3285 & 1.3346 \\
\hline
43 & 0.91 & 1.3408 & 1.3469 & 1.3532 & 1.3595 & 1.3658 & 1.3722 & 1.3787 & 1.3852 & 1.3917 & 1.3984 \\
\hline
44 & 0.92 & 1.4051 & 1.4118 & 1.4187 & 1.4255 & 1.4325 & 1.4395 & 1.4466 & 1.4538 & 1.4611 &  \\
\hline
 & 0.93 & 1.4758 & 1.4833 & 1.4909 & 1.4985 & 1.5063 & 1.5141 & 1.5220 & 1.5301 & 1.5382 & 1.5464 \\
\hline
46 & 0.94 & 1.5548 & 1.5632 & 1.5718 & 1.5805 & 1.5893 & 1.5982 & 1.6072 & 1.6164 & 1.6258 & 1.6352 \\
\hline
47 & 0.95 & 1.6449 & 1.6546 & 1.6646 & 1.6747 & 1.6849 & 1.6954 & 1.7060 & 1.7169 &  &  \\
\hline
10 & 0.96 & 1.7507 & 1.7624 & 1.7744 & 1.7866 & 1.7991 & 1.8119 & 1.8250 & 1.8384 & 1.8522 & 1.8663 \\
\hline
49 & 0.97 & 1.8808 & 1.8957 & 1.9110 & 1.9268 & 1.9431 & 1.9600 & 1.9774 & 1.9954 & 2.0141 & 2.0335 \\
\hline
50 & 0.98 & 2.0537 & 2.0749 & 2.0969 & 2.1201 & 2.1444 & 2.1701 & 2.1973 & 2.2262 & 2.2571 & 2.2904 \\
\hline
\end{tabular}
\end{center}

\begin{center}
\end{center}

利用 Excel 提供的统计函数 “T. INV”, 可以构建 $t$ 分布临界值表, 该表是根据 $t$ 分布的右尾概率 $\alpha$ 计算的相应的临界值。如果 $P(t \geqslant x)=\alpha$, 则对于任意给定概率 $p(0 \leqslant \alpha \leqslant 1)$, 可以求出相应的 $x$ 。生成 $t$ 分布临界值表的具体操作步骤如下。

第 1 步: 将 $t$ 分布自由度 $d f$ 的值输人到工作表的 A 列, 将右尾概率 $\alpha$ 的取值输人到第 1 行, 形成 $t$ 分布临界值表的表头, 如下表所示。

\begin{center}
\begin{tabular}{|c|c|c|c|c|c|c|c|c|}
\hline
 & $\mathrm{A}$ & $\mathrm{B}$ & $\mathrm{C}$ & $\mathrm{D}$ & $\mathrm{E}$ & $\mathrm{F}$ & $\mathrm{G}$ & $\mathrm{H}$ \\
\hline
1 & $\mathbf{d f / \alpha}$ & $\mathbf{0 . 1 0 0}$ & $\mathbf{0 . 0 5 0}$ & $\mathbf{0 . 0 2 5}$ & $\mathbf{0 . 0 1 0}$ & $\mathbf{0 . 0 0 5}$ & $\mathbf{0 . 0 0 1}$ & $\mathbf{0 . 0 0 0 5}$ \\
\hline
2 & $\mathbf{1}$ &  &  &  &  &  &  &  \\
\hline
3 & $\mathbf{2}$ &  &  &  &  &  &  &  \\
\hline
4 & $\mathbf{3}$ &  &  &  &  &  &  &  \\
\hline
5 & $\mathbf{4}$ &  &  &  &  &  &  &  \\
\hline
6 & $\mathbf{5}$ &  &  &  &  &  &  &  \\
\hline
7 & $\mathbf{6}$ &  &  &  &  &  &  &  \\
\hline
8 & $\mathbf{7}$ &  &  &  &  &  &  &  \\
\hline
9 & $\mathbf{8}$ &  &  &  &  &  &  &  \\
\hline
10 & $\mathbf{9}$ &  &  &  &  &  &  &  \\
\hline
11 & $\mathbf{1 0}$ &  &  &  &  &  &  &  \\
\hline
12 & $\mathbf{1 1}$ &  &  &  &  &  &  &  \\
\hline
13 & $\mathbf{1 2}$ &  &  &  &  &  &  &  \\
\hline
14 & $\mathbf{1 3}$ &  &  &  &  &  &  &  \\
\hline
15 & $\mathbf{1 4}$ &  &  &  &  &  &  &  \\
\hline
16 & $\mathbf{1 5}$ &  &  &  &  &  &  &  \\
\hline
17 & $\mathbf{1 6}$ &  &  &  &  &  &  &  \\
\hline
18 & $\mathbf{1 7}$ &  &  &  &  &  &  &  \\
\hline
19 & $\mathbf{1 8}$ &  &  &  &  &  &  &  \\
\hline
20 & $\mathbf{1 9}$ &  &  &  &  &  &  &  \\
\hline
21 & $\mathbf{2 0}$ &  &  &  &  &  &  &  \\
\hline
\end{tabular}
\end{center}

第 2 步: 在 B2 单元格输人公式 “=T. INV ( $\left.\mathrm{B} \$ 1, \$ \mathrm{~A} 2^{*}-1\right)$, 然后将其向下、向右复制即可得到 $t$ 分布临界值表,部分结果如下表所示(读者可根据需要生成不同 $\alpha$和不同自由度的 $t$ 分布临界值表)。

\begin{center}
\begin{tabular}{|c|c|c|c|c|c|c|c|c|}
\hline
 & A & B & C & D & E & F & G & $\mathrm{H}$ \\
\hline
1 & df $/ \alpha$ & 0.100 & 0.050 & 0.025 & 0.010 & 0.005 & 0.001 & 0.0005 \\
\hline
2 & 1 & 3.0777 & 6.3138 & 12.7062 & 31.8205 & 63.6567 & 318.3088 & 636.6192 \\
\hline
3 & 2 & 1.8856 & 2.9200 & 4.3027 & 6.9646 & 9.9248 & 22.3271 & 31.5991 \\
\hline
4 & 3 & 1.6377 & 2.3534 & 3.1824 & 4.5407 & 5.8409 & 10.2145 & 12.9240 \\
\hline
5 & 4 & 1.5332 & 2.1318 & 2.7764 & 3.7469 & 4.6041 & 7.1732 & 8.6103 \\
\hline
$\underline{0}$ & 5 & 1.4759 & 2.0150 & 2.5706 & 3.3649 & 4.0321 & 5.8934 & 6.8688 \\
\hline
 & 6 & 1.4398 & 1.9432 & 2.4469 & 3.1427 & 3.7074 & 5.2076 & 5.9588 \\
\hline
8 & 7 & 1.4149 & 1.8946 & 2.3646 & 2.9980 & 3.4995 & 4.7853 & 5.4079 \\
\hline
$y$ & 8 & \begin{tabular}{l}
1.3968 \\
\end{tabular} & 1.8595 & 2.3060 & 2.8965 & 3.3554 & 4.5008 & 5.0413 \\
\hline
10 & 9 & 1.3830 & 1.8331 & 2.2622 & 2.8214 & 3.2498 & 4.2968 & 4.7809 \\
\hline
1 & 10 & 1.3722 & 1.8125 & 2.2281 & 2.7638 & 3.1693 & 4.1437 & 4.5869 \\
\hline
12 & 11 & 1.3634 & 1.7959 & 2.2010 & 2.7181 & 3.1058 & 4.0247 & 4.4370 \\
\hline
13 & 12 & 1.3562 & 1.7823 & 2.1788 & 2.6810 & 3.0545 & 3.9296 & 4.3178 \\
\hline
14 & 13 & 1.3502 & 1.7709 & 2.1604 & 2.6503 & 3.0123 & 3.8520 & 4.2208 \\
\hline
15 & 14 & 1.3450 & 1.7613 & 2.1448 & 2.6245 & 2.9768 & 3.7874 & 4.1405 \\
\hline
16 & 15 & 1.3406 & 1.7531 & 2.1314 & 2.6025 & 2.9467 & 3.7328 & 4.0728 \\
\hline
17 & 16 & 1.3368 & 1.7459 & 2.1199 & 2.5835 & 2.9208 & 3.6862 & 4.0150 \\
\hline
18 & 17 & 1.3334 & 1.7396 & 2.1098 & 2.5669 & 2.8982 & 3.6458 & 3.9651 \\
\hline
19 & 18 & 1.3304 & 1.7341 & 2.1009 & 2.5524 & 2.8784 & 3.6105 & 3.9216 \\
\hline
20 & 19 & 1.3277 & 1.7291 & 2.0930 & 2.5395 & 2.8609 & 3.5794 & 3.8834 \\
\hline
21 & 20 & 1.3253 & 1.7247 & 2.0860 & 2.5280 & 2.8453 & 3.5518 & 3.8495 \\
\hline
22 & 21 & 1.3232 & 1.7207 & 2.0796 & 2.5176 & 2.8314 & 3.5272 & 3.8193 \\
\hline
23 & 22 & 1.3212 & 1.7171 & 2.0739 & 2.5083 & 2.8188 & 3.5050 & 3.7921 \\
\hline
24 & 23 & 1.3195 & 1.7139 & 2.0687 & 2.4999 & 2.8073 & 3.4850 & 3.7676 \\
\hline
20 & 24 & 1.3178 & 1.7109 & 2.0639 & 2.4922 & 2.7969 & 3.4668 & 3.7454 \\
\hline
26 & 25 & 1.3163 & 1.7081 & 2.0595 & 2.4851 & 2.7874 & 3.4502 & 3.7251 \\
\hline
27 & 26 & 1.3150 & 1.7056 & 2.0555 & 2.4786 & 2.7787 & 3.4350 & 3.7066 \\
\hline
28 & 27 & 1.3137 & 1.7033 & 2.0518 & 2.4727 & 2.7707 & 3.4210 & 3.6896 \\
\hline
29 & 28 & 1.3125 & 1.7011 & 2.0484 & 2.4671 & 2.7633 & 3.4082 & 3.6739 \\
\hline
30 & 29 & 1.3114 & 1.6991 & 2.0452 & 2.4620 & 2.7564 & 3.3962 & 3.6594 \\
\hline
31 & 30 & 1.3104 & 1.6973 & 2.0423 & 2.4573 & 2.7500 & 3.3852 & 3.6460 \\
\hline
 & 31 & 1.3095 & 1.6955 & 2.0395 & 2.4528 & 2.7440 & 3.3749 & 3.6335 \\
\hline
33 & 32 & 3086 & 6939 & 2.0369 & 2.4487 & 2.7385 & 3.3653 & 3.6218 \\
\hline
34 & 33 & 1.3077 & 1.6924 & 2.0345 & 2.4448 & 2.7333 & 3.3563 & 3.6109 \\
\hline
35 & 34 & 1.3070 & 1.6909 & 2.0322 & 2.4411 & 2.7284 & 3.3479 & 3.6007 \\
\hline
36 & 35 & 1.3062 & 1.6896 & 2.0301 & 2.4377 & 2.7238 & 3.3400 & 3.5911 \\
\hline
37 & 36 & 1.3055 & 1.6883 & 2.0281 & 2.4345 & 2.7195 & 3.3326 & 3.5821 \\
\hline
38 & 37 & 1.3049 & 1.6871 & 2.0262 & 2.4314 & 2.7154 & 3.3256 & 3.5737 \\
\hline
39 & 38 & 1.3042 & 1.6860 & 2.0244 & 2.4286 & 2.7116 & 3.3190 & 3.5657 \\
\hline
40 & 39 & 1.3036 & 1.6849 & 2.0227 & 2.4258 & 2.7079 & 3.3128 & 3.5581 \\
\hline
41 & 40 & 1.3031 & 1.6839 & 2.0211 & 2.4233 & 2.7045 & 3.3069 & 3.5510 \\
\hline
42 & 41 & 1.3025 & 1.6829 & 2.0195 & 2.4208 & 2.7012 & 3.3013 & 3.5442 \\
\hline
43 & 42 & 1.3020 & 1.6820 & 2.0181 & 2.4185 & 2.6981 & 3.2960 & 3.5377 \\
\hline
44 & 43 & 1.3016 & 1.6811 & 2.0167 & 2.4163 & 2.6951 & 3.2909 & 3.5316 \\
\hline
45 & 44 & 1.3011 & 1.6802 & 2.0154 & 2.4141 & 2.6923 & 3.2861 & 3.5258 \\
\hline
46 & 45 & 1.3006 & 1.6794 & 2.0141 & 2.4121 & 2.6896 & 3.2815 & 3.5203 \\
\hline
\end{tabular}
\end{center}

\begin{center}
\end{center}

利用 Excel 提供的统计函数 “CHISQ. INV. RT”, 可以构建 $\chi^{2}$ 分布临界值表, 该表是根据 $\chi^{2}$ 分布的右尾概率 $\alpha$ 计算的相应的临界值。如果 $P\left(\chi^{2} \geqslant x\right)=\alpha$, 则对于任意给定的概率 $p(0 \leqslant \alpha \leqslant 1)$, 可以求出相应的 $x$ 。生成 $\chi^{2}$ 分布临界值表的具体操作步骤如下。

第 1 步: 将 $\chi^{2}$ 分布自由度 $d f$ 的值输人到工作表的 $\mathrm{A}$ 列, 将右尾概率 $\alpha$ 的取值输人到第 1 行, 形成 $\chi^{2}$ 分布临界值表的表头,如下表所示。

\begin{center}
\begin{tabular}{|c|c|c|c|c|c|c|c|c|c|c|c|}
\hline
 & $\mathrm{A}$ & $\mathrm{B}$ & $\mathrm{C}$ & $\mathrm{D}$ & $\mathrm{E}$ & $\mathrm{F}$ & $\mathrm{G}$ & $\mathrm{H}$ & $\mathrm{I}$ & $\mathrm{J}$ & $\mathrm{K}$ \\
\hline
1 & $\mathrm{df} / \alpha$ & $\mathbf{0 . 9 9 5}$ & $\mathbf{0 . 9 9 0}$ & $\mathbf{0 . 9 7 5}$ & $\mathbf{0 . 9 5 0}$ & $\mathbf{0 . 9 0 0}$ & $\mathbf{0 . 1 0 0}$ & $\mathbf{0 . 0 5 0}$ & $\mathbf{0 . 0 2 5}$ & $\mathbf{0 . 0 1 0}$ & $\mathbf{0 . 0 0 5}$ \\
\hline
2 & $\mathbf{1}$ &  &  &  &  &  &  &  &  &  &  \\
\hline
3 & $\mathbf{2}$ &  &  &  &  &  &  &  &  &  &  \\
\hline
4 & $\mathbf{3}$ &  &  &  &  &  &  &  &  &  &  \\
\hline
5 & $\mathbf{4}$ &  &  &  &  &  &  &  &  &  &  \\
\hline
6 & $\mathbf{5}$ &  &  &  &  &  &  &  &  &  &  \\
\hline
7 & $\mathbf{6}$ &  &  &  &  &  &  &  &  &  &  \\
\hline
8 & $\mathbf{7}$ &  &  &  &  &  &  &  &  &  &  \\
\hline
9 & $\mathbf{8}$ &  &  &  &  &  &  &  &  &  &  \\
\hline
10 & $\mathbf{9}$ &  &  &  &  &  &  &  &  &  &  \\
\hline
11 & $\mathbf{1 0}$ &  &  &  &  &  &  &  &  &  &  \\
\hline
12 & $\mathbf{1 1}$ &  &  &  &  &  &  &  &  &  &  \\
\hline
13 & $\mathbf{1 2}$ &  &  &  &  &  &  &  &  &  &  \\
\hline
14 & $\mathbf{1 3}$ &  &  &  &  &  &  &  &  &  &  \\
\hline
$\mathbf{1 5}$ & $\mathbf{1 4}$ &  &  &  &  &  &  &  &  &  &  \\
\hline
16 & $\mathbf{1 5}$ &  &  &  &  &  &  &  &  &  &  \\
\hline
17 & $\mathbf{1 6}$ &  &  &  &  &  &  &  &  &  &  \\
\hline
18 & $\mathbf{1 7}$ &  &  &  &  &  &  &  &  &  &  \\
\hline
19 & $\mathbf{1 8}$ &  &  &  &  &  &  &  &  &  &  \\
\hline
20 & $\mathbf{1 9}$ &  &  &  &  &  &  &  &  &  &  \\
\hline
21 & $\mathbf{2 0}$ &  &  &  &  &  &  &  &  &  &  \\
\hline
\end{tabular}
\end{center}

第 2 步: 在 B2 单元格输人公式 “=CHISQ. INV. RT (B\$1,\$ A2)”, 然后将其向下、向右复制即可得到 $\chi^{2}$ 分布临界值表, 部分结果如下表所示(读者可根据需要生成不同 $\alpha$ 和不同自由度的 $\chi^{2}$ 分布临界值表)。

\begin{center}
\begin{tabular}{|c|c|c|c|c|c|c|c|c|c|c|c|}
\hline
 & A & B & C & D & E & $\mathrm{F}$ & G & H & I & J & K \\
\hline
 & df $/ \alpha$ & 0.995 & 0.990 & 0.975 & .950 & 0.900 & 0.100 & 0.050 & 0.025 & 0.010 & 0.005 \\
\hline
2 & 1 & 0.0000 & 0.0002 & 0.0010 & 0.0039 & 0.0158 & 2.7055 & 3.8415 & 5.0239 & 6.6349 & 78794 \\
\hline
 & 2 & 0.0100 & 0.0201 & 0.0506 & 0.1026 & 0.2107 & 4.6052 & 5.9915 & 7.3778 & 9.2103 & 10.5966 \\
\hline
$\Delta$ & $x$ & 0.0717 & 0.1148 & 0.2158 & 0.3518 & 0.5844 & 6.2514 & 7.8147 & 9.3484 & 11.3449 & 12.8382 \\
\hline
5 & 4 & 0.2070 & 0.2971 & 0.4844 & .7107 & 1.0636 & 7.7794 & 9.4877 & 11.1433 & 13.2767 & 14.8603 \\
\hline
6 & 5 & 0.4117 & 0.5543 & 0.8312 & 1.1455 & 1.6103 & 9.2364 & 11.0705 & 12.8325 & 15.0863 & 16.7496 \\
\hline
 & 0 & 0.6757 & 0.8721 & 1.2373 & 1.6354 & 2.2041 & 10.6446 & 12.5916 & 14.4494 & 16.8119 & 18.5476 \\
\hline
8 & 7 & 8893 & 1.2390 & 1.6899 & 2.1673 & 331 & 12.0 & 14.0671 & 16.0128 & 18.4753 & 20.2777 \\
\hline
9 & 8 & 3444 & 1.6465 & 2.1797 & 2.7326 & 3.4895 & 13.3616 & 15.5073 & 17.5345 & 20.0902 & 21.9550 \\
\hline
10 & 9 & 1.7349 & 2.0879 & 2.7004 & 3.3251 & 4.1682 & 14.6837 & 16.9190 & 19.0228 & 21.6660 & 23.5894 \\
\hline
11 & 10 & 2.1559 & 2.5582 & 3.2470 & 3.9403 & 4.8652 & 15.9872 & 18.3070 & 20.4832 &  &  \\
\hline
12 & 11 & 2.6032 & 3.0535 & 3.8157 & 4.5748 & 5.5778 & 17.2750 & 19.6751 & 21.9200 & 24.7250 & 26.7568 \\
\hline
13 & 12 & 3.0738 & 3.5706 & 4.4038 & 5.2260 & 6.3038 & 18.5493 & 21.0261 & 23.3367 & 26.2170 & 28.2995 \\
\hline
 & 13 & 3.5650 & 4.1069 & 5.0088 & 5.8919 & 7.0415 & 19. & 22.3620 & 24.7356 & 27.6882 & 29.8195 \\
\hline
 & 14 & 4.0747 & 4.6604 & 5.6287 & 6.5706 & 7.7895 & 21.0641 & 23.6848 & 26.1189 & 29.1412 & 31.3193 \\
\hline
$\perp$ & 15 & 4.6009 & 5.2293 & 6.2621 & 7.2609 & 8.5468 & 22.3071 & 24.9958 & 27.4884 & 30.5779 & 32.8 \\
\hline
17 & 16 & 5.1422 & 5.8122 & 6.9077 & 7.9616 & 9.3122 & 18 & 26. & 28.8454 & 31.9999 & 34.2672 \\
\hline
18 & 17 & 5.6972 & 6.4078 & 7.5642 & 8.6718 & 10.0852 &  & 27.5871 & 30.1910 & 33.4087 & 35.7185 \\
\hline
19 & 18 & 6.2648 & 7.0149 & 8.2307 & 9.3905 & 10.8649 & 34 & 28.8693 & 31.5264 & 3053 & 37.1565 \\
\hline
20 & 19 & 6.8440 & 7.6327 & 8.9065 & 10.1170 & 6509 &  & 30.1435 & 32.8523 & 36.1909 & 5823 \\
\hline
21 & 20 & 7.4338 & 8.2604 & 9.5908 & 10.8508 & 12.4426 & 20 & 31.4104 & 34.1696 & 5662 & 39 \\
\hline
22 & 21 & 8.0337 & 8.8972 & 10.2829 & 11.5913 & 13.2396 & 29.6151 & 32.6706 & 35.4789 & 38.9322 & 41.4011 \\
\hline
23 & 22 & 8.6427 & 9.5425 & 10.9823 & 380 & 14.0415 & 30. & 33.9244 & 36.7807 & 40.2894 & 42.7957 \\
\hline
24 & 23 & 9.2604 & 10.1957 & 11.6886 & 13.0905 & 14.8480 & 32. & 35.1725 & 38.0756 & 41.6384 & 44 \\
\hline
25 & 24 & 9.8862 & .8564 & 12.4012 & 13.8484 & 15.6587 & 33. & 36.4150 & 39.3641 & 42.9798 & 45.5585 \\
\hline
$\angle$ & 25 & 10.5197 & .5240 & 13.1197 & 14.6114 & 16.4734 &  & 37.6525 & 40.6465 & 44.3141 & 46.9279 \\
\hline
2 & 26 & 11.1602 & 2.1981 & 13.8439 & 15.3792 & 17.2919 & 35.5632 & 38.8851 & 41.9232 & 45.6417 & 48 \\
\hline
28 & 27 & 11.8076 & .8785 & 14.5734 & 16.1514 & 18.1139 & 36.7412 & 40.1133 & 43.1945 & 46.9629 &  \\
\hline
29 & 28 & 12.4613 & 5647 & 15.3079 & 16.9279 & 18.9392 & 37.9159 & 41.3371 & 44.4608 & 48.2782 & 50.9934 \\
\hline
3 & 29 & 13.1211 & 2565 & 16.0471 & 17.7084 & 19.7677 & 39.0875 & 42.5570 & 45.7223 & 49.5879 & 52.3356 \\
\hline
 & 30 &  & 9535 & 16.7 & 327 & 20.5992 & 40.2560 & 43.7730 & 46.9792 & 50.8922 & 53.6720 \\
\hline
 & 31 & 78 & 5 & 1 &  &  &  &  &  & 14 & 27 \\
\hline
3 & 32 & 15.1340 & 3622 & 18.2 & .0719 & 36 & 47 & 46.1943 & 04 & 858 &  \\
\hline
 & 3 &  & .0735 &  & 20.8665 & 23.1102 & 52 & 47.3999 & 50.7251 & 54.7755 & 57. \\
\hline
 & 34 & 3 & 1 &  &  &  &  &  &  & 09 & 58.9639 \\
\hline
3 & 35 & 918 & 5089 & 20.5694 & 22.4650 & 67 &  & 49. & 33 & 421 & 60.2 \\
\hline
 & 3 &  & 7 &  & 23.2686 & 25.6433 & 47.2122 & 50.9985 & 54.4373 & 58.6192 & 61.5 \\
\hline
 & 3 & 58 &  &  &  &  &  &  &  & 59.8925 &  \\
\hline
 & 38 & 19.2889 & .6914 & 22.8785 & 839 & 27.3430 & 49 & 53.3 & 55 & 61.1621 &  \\
\hline
4 & 30 &  & $n^{-1}$ & 23.6 & 25.6954 & 28.1958 & 50.6598 & 54.5722 & 58.1201 & 62.4281 & 65.4756 \\
\hline
4 & 40 & 065 & 3 &  &  &  &  &  &  & 63.6907 & 66 \\
\hline
42 & 41 & 21.4208 & 9056 & 25.2145 & 27.3256 & 29.9071 & 52.9485 & 56.9424 & 506 & 64.9501 & 68 \\
\hline
 & 4 & 385 & 5501 & 5.9987 & 28.1440 & 30.7654 & 54.0902 & 58.1240 & 61.7768 & 66.2062 & 69.3360 \\
\hline
44 & 43 & 595 & 976 & E &  & .6255 &  & 59.3035 & 304 & 67.4593 & 70.6159 \\
\hline
$4 !$ & 44 & 23.5837 & 1480 & 27.5746 & 29.7875 & .4871 & 35 & 60.4 & 5 & 095 & 70 \\
\hline
46 & 4. & 3110 & 9013 & 3.3662 & 30.6123 & 3504 & 57.5053 & 61.6562 & 65.4102 & 69.9568 & 73.1 \\
\hline
1 & 46 & 5.0413 & 572 & 1601 & .4390 & 152 &  & 02.0200 & 100 & 2014 & 14. \\
\hline
\end{tabular}
\end{center}

\begin{center}
\end{center}

利用 Excel 提供的统计函数 “F. INV. RT”, 可以构建 $F$ 分布临界值表, 该表是根据 $F$ 分布的右尾概率 $\alpha$ 计算的相应的临界值。如果 $P(F \geqslant x)=\alpha$, 则对于任意给定的概率 $p(0 \leqslant \alpha \leqslant 1)$, 可以求出相应的 $x$ 。生成 $F$ 分布临界值表的具体操作步骤如下。

第 1 步: 在 B1 单元格输人 $F$ 分布右尾概率 $\alpha$ 的取值 (如 $\alpha=0.05$ ), 在第 2 行输人分子自由度 $d f 1$ 的值, 在第 1 列输人分母自由度 $d f 2$ 的值, 如下表所示。

\begin{center}
\begin{tabular}{|c|c|c|c|c|c|c|c|c|c|c|c|}
\hline
 & $\mathrm{A}$ & $\mathrm{B}$ & $\mathrm{C}$ & $\mathrm{D}$ & $\mathrm{E}$ & $\mathrm{F}$ & $\mathrm{G}$ & $\mathrm{H}$ & $\mathrm{I}$ & $\mathrm{J}$ & $\mathrm{K}$ \\
\hline
1 & $\alpha=$ & $\mathbf{0 . 0 5}$ &  &  &  &  &  &  &  &  &  \\
\hline
2 & df2/df1 & $\mathbf{1}$ & $\mathbf{2}$ & $\mathbf{3}$ & $\mathbf{4}$ & $\mathbf{5}$ & $\mathbf{6}$ & $\mathbf{7}$ & $\mathbf{8}$ & $\mathbf{9}$ & $\mathbf{1 0}$ \\
\hline
3 & $\mathbf{1}$ &  &  &  &  &  &  &  &  &  &  \\
\hline
4 & $\mathbf{2}$ &  &  &  &  &  &  &  &  &  &  \\
\hline
5 & $\mathbf{3}$ &  &  &  &  &  &  &  &  &  &  \\
\hline
6 & $\mathbf{4}$ &  &  &  &  &  &  &  &  &  &  \\
\hline
$\mathbf{7}$ & $\mathbf{5}$ &  &  &  &  &  &  &  &  &  &  \\
\hline
8 & $\mathbf{6}$ &  &  &  &  &  &  &  &  &  &  \\
\hline
9 & $\mathbf{7}$ &  &  &  &  &  &  &  &  &  &  \\
\hline
10 & $\mathbf{8}$ &  &  &  &  &  &  &  &  &  &  \\
\hline
11 & $\mathbf{9}$ &  &  &  &  &  &  &  &  &  &  \\
\hline
12 & $\mathbf{1 0}$ &  &  &  &  &  &  &  &  &  &  \\
\hline
13 & $\mathbf{1 1}$ &  &  &  &  &  &  &  &  &  &  \\
\hline
14 & $\mathbf{1 2}$ &  &  &  &  &  &  &  &  &  &  \\
\hline
15 & $\mathbf{1 3}$ &  &  &  &  &  &  &  &  &  &  \\
\hline
16 & $\mathbf{1 4}$ &  &  &  &  &  &  &  &  &  &  \\
\hline
17 & $\mathbf{1 5}$ &  &  &  &  &  &  &  &  &  &  \\
\hline
18 & $\mathbf{1 6}$ &  &  &  &  &  &  &  &  &  &  \\
\hline
19 & $\mathbf{1 7}$ &  &  &  &  &  &  &  &  &  &  \\
\hline
20 & $\mathbf{1 8}$ &  &  &  &  &  &  &  &  &  &  \\
\hline
21 & $\mathbf{1 9}$ &  &  &  &  &  &  &  &  &  &  \\
\hline
22 & $\mathbf{2 0}$ &  &  &  &  &  &  &  &  &  &  \\
\hline
\end{tabular}
\end{center}

第 2 步: 在 B3 单元格输人公式 “=F. INV. RT ( \$B\$1, B\$2, \$ A3)”, 然后将其向下、向右复制即可得到 $F$ 分布临界值表, $\alpha=0.05$ 时 $F$ 分布临界值表的部分结果如下表所示 (读者可根据需要生成不同 $\alpha$ 和不同自由度的 $F$ 分布临界值表)。

\begin{center}
\begin{tabular}{|c|c|c|c|c|c|c|c|c|c|c|c|}
\hline
 & A & B & C & D & $\mathrm{E}$ & $\mathrm{F}$ & G & $\mathrm{H}$ & I & $\mathrm{J}$ & $\mathrm{K}$ \\
\hline
 & $\alpha=$ & 0.05 &  &  &  &  &  &  &  &  &  \\
\hline
 & df2/df1 & 1 & 2 & 3 & 4 & 5 & 6 & 7 & 8 & 9 & 10 \\
\hline
 & 1 & 161.448 & 199.500 & 215.707 & 224.583 & 230.162 & 233.986 & 236.768 & 238.883 & 240.543 & 241.882 \\
\hline
 & 2 & 18.513 & 19.000 & 19.164 & 19.247 & 19.296 & 19.330 & 19.353 & 19.371 & 19.385 & 19.396 \\
\hline
 & 3 & 10.128 & 9.552 & 9.277 & 9.117 & 9.013 & 8.941 & 8.887 & 8.845 & 8.812 & 8.786 \\
\hline
 & 4 & 7.709 & 6.944 & 6.591 & 6.388 & 6.256 & 6.163 & 6.094 & 6.041 & 5.999 & 5.964 \\
\hline
 & 5 & 6.608 & 5.786 & 5.409 & 5.192 & 5.050 & 4.950 & 4.876 & 4.818 & 4.772 & 4.735 \\
\hline
 & 6 & 5.987 & 5.143 & 4.757 & 4.534 & 4.387 & 4.284 & 4.207 & 4.147 & 4.099 & 4.060 \\
\hline
y & 7 & 5.591 & 4.737 & 4.347 & 4.120 & 3.972 & 3.866 & 3.787 & 3.726 & 3.677 & 3.637 \\
\hline
10 & 8 & 5.318 & 4.459 & 4.066 & 3.838 & 3.687 & 3.581 & 3.500 & 3.438 & 3.388 & 3.347 \\
\hline
 & 9 & 5.117 & 4.256 & 3.863 & 3.633 & 3.482 & 3.374 & 3.293 & 3.230 & 3.179 & 3.137 \\
\hline
12 & 10 & 4.965 & 4.103 & 3.708 & 3.478 & 3.326 & 3.217 & 3.135 & 3.072 & 3.020 & 2.978 \\
\hline
13 & 11 & 4.844 & 3.982 & 3.587 & 3.357 & 3.204 & 3.095 & 3.012 & 2.948 & 2.896 & 2.854 \\
\hline
 & 12 & 4.747 & 3.885 & 3.490 & 3.259 & 3.106 & 2.996 & 2.913 & 2.849 & 2.796 & 2.753 \\
\hline
 & 13 & 4.667 & 3.806 & 3.411 & 3.179 & 3.025 & 2.915 & 2.832 & 2.767 & 2.714 & 2.671 \\
\hline
 & 14 & 4.600 & 3.739 & 3.344 & 3.112 & 2.958 & 2.848 & 2.764 & 2.699 & 2.646 & 2.602 \\
\hline
 & 15 & 4.543 & 3.682 & 3.287 & 3.056 & 2.901 & 2.790 & 2.707 & 2.641 & 2.588 & 2.544 \\
\hline
8 & 16 & 4.494 & 3.634 & 3.239 & 3.007 & 2.852 & 2.741 & 2.657 & 2.591 & 2.538 & 2.494 \\
\hline
y & 17 & 4.451 & 3.592 & 3.197 & 2.965 & 2.810 & 2.699 & 2.614 & 2.548 & 2.494 & 2.450 \\
\hline
 & 18 & 4.414 & 3.555 & 3.160 & 2.928 & 2.773 & 2.661 & 2.577 & 2.510 & 2.456 & 2.412 \\
\hline
 & 19 & 4.381 & 3.522 & 3.127 & 2.895 & 2.740 & 2.628 & 2.544 & 2.477 & 2.423 & 2.378 \\
\hline
 & 20 & 4.351 & 3.493 & 3.098 & 2.866 & 2.711 & 2.599 & 2.514 & 2.447 & 2.393 & 2.348 \\
\hline
 & 21 & 4.325 & 3.467 & 3.072 & 2.840 & 2.685 & 2.573 & 2.488 & 2.420 & 2.366 & 2.321 \\
\hline
 & 22 & 4.301 & 3.443 & 3.049 & 2.817 & 2.661 & 2.549 & 2.464 & 2.397 & 2.342 & 2.297 \\
\hline
 & 23 & 4.279 & 3.422 & 3.028 & 2.796 & 2.640 & 2.528 & 2.442 & 75 & 2.320 & 2.275 \\
\hline
 & 24 & 4.260 & 3.403 & 3.009 & 2.776 & 2.621 & 2.508 & 2.423 & .355 & 2.300 & 255 \\
\hline
 & 25 & 4.242 & 3. & 2.991 & 2.759 & 3 & 2.490 & 2.405 & 37 & 2.282 & 236 \\
\hline
 & 26 & 4.225 & 3.369 & 2.975 & 2.743 & 2.587 & 2.474 & 2.388 & 21 & 2.265 & 2.220 \\
\hline
 & 27 & 4.210 & 3.354 & 2.960 & 2.728 & 2.572 & 2.459 & 2.373 & 2.305 & 2.250 & 2.204 \\
\hline
 & 28 & 4.19 & 3. & 2.947 & 2.714 & 2.558 & 2.445 & 2.359 & 2.291 & 2.236 & 2.190 \\
\hline
 & 29 & 4.183 & 8 &  & 2.701 & 2.545 & 2.432 & 2.346 & 8 & 2.223 & 2.177 \\
\hline
 & 30 & 4.171 & 3.316 & 2.922 & 2.690 & 2.534 & 2.421 & 2.334 & 266 & 2.211 & 2.165 \\
\hline
 & 31 & 4.16 & 3.305 & 04 & 2.679 & 2.523 & 2.409 & 2.323 & 55 & 2.199 & 2.153 \\
\hline
 & 32 & 4.149 & 3.29 &  & 2.668 &  & 399 & 2.313 & 2.244 & 2.189 & 142 \\
\hline
 & 33 & 39 & 3.2 & 2.892 & 2.659 &  & 9 & 2.303 & 5 & 2.179 & 133 \\
\hline
 & 34 & 4.130 & 3.276 & 2.883 & 2.650 & 2 & 380 & 2.294 & 5 & 2.170 & 123 \\
\hline
 & 35 &  & 3.2 & 2. & 2.641 & 185 & 372 & 2.285 & 7 & 2.161 & .114 \\
\hline
 & 36 & 4.1 &  &  &  &  & 4 & 2.277 & 2.209 & .153 & 2.106 \\
\hline
 & 37 & 4.105 & 3.252 & 2.859 & 2.626 &  & 6 & 2.270 & 2.201 & 2.145 & 2.098 \\
\hline
 & 38 & 4.098 & 3.2 & 2.852 & 2.619 & 2.463 & 2.349 & 2.262 & 4 & 2.138 & 2.091 \\
\hline
4 & 39 & 4. &  &  & 2.612 & 2.456 & 2.342 & 2.255 & 2.187 & 2.131 & 2.084 \\
\hline
4 & 40 & 4.085 & 3.232 & 2.839 & 2.606 &  & 2.336 & 2.249 & 2.180 & 2.124 & 2.077 \\
\hline
4 & 41 & 4.079 & 3.226 & 2.833 & 2.600 & 2.443 & 2.330 & 2.243 & 2.174 & 18 & 2.071 \\
\hline
4 & 42 & 4.073 & 3.2 & 2.827 & 2.594 & 2.438 & 2.324 & 2.237 & 2.168 & 2.112 & 2.065 \\
\hline
4 & 43 & 4.067 & 3.2 &  & 89 & 2.432 & 2.318 & 2.232 & 2.163 & 2.106 & 2.059 \\
\hline
46 & 44 & 4.062 & 3.209 & 2.816 & 2.584 & 2.427 & 2.313 & 2.226 & 2.157 & 2.101 & 2.054 \\
\hline
47 & 45 & 4.057 & 3.204 & 2.812 & 2.579 & 2.422 & 2.308 & 2.221 & 2.152 & 2.096 & 2.049 \\
\hline
\end{tabular}
\end{center}


\end{CJK*}
\end{document}